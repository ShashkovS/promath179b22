\documentclass[a4paper,11pt]{article}
\usepackage[mag=1000]{newlistok}

\ВключитьКолонтитул

\УвеличитьШирину{1.3cm}
\УвеличитьВысоту{2.5cm}

\Заголовок{Вероятностные пространства. Геометрические вероятности}
\НомерЛистка{37}
\renewcommand{\spacer}{\vspace{1.1pt}}
\ДатаЛистка{16.09 -- 25.09.2019}
% 54 задач
\Оценки{18/13/8}


\long\def\решение#1\крешение{}\long\def\ответ#1\кответ{} % Скрыть решения
\begin{document}


\СоздатьЗаголовок
{\small
\опр \выд{Вероятностным пространством} называется тройка $(\Omega ,\A, P)$, где
%\vspace{-1mm}
\begin{items}{-5}
\item[$\Omega $] --- некоторое множество, элементы которого называются \выд{элементарными событиями} или \выд{исходами};
\item[$\A$] --- совокупность подмножеств множества $\Omega$, называемых \выд{событиями}, такая что:
%\vspace{-2mm}
\begin{nums}{-4}
\item[с1.] $\es \in \A$, $\Omega \in \A$;
\item[с2.] если $A \in \A$, то событие $\overline {A}$, \выд противоположное событию $A$,
%         (происходящее тогда и только тогда, когда не происходит событие $A$)
лежит в $\A$;
\item[с3.] если $A,B \in \A$, то \выд сумма событий $A\cup B$
%         (происходящее тогда и только тогда, когда происходит хотя бы одно из событий $A, B$)
лежит в $\A$;
\item[с4.] если $A,B \in \A$, то \выд произведение событий $A\cap B$
%         (происходящее тогда и только тогда, когда происходит и $A$, и $B$)
лежит в $\A$;
% Возможно, пункт 3 нужно убрать
\item[с5.]  если $\Omega$ бесконечно и $A_i \in \A$ при $i=1,2,\dots $, то
    $\bigcup\limits_{i=1}^{\infty} A_i \in \A $  и
    $\bigcap\limits_{i=1}^{\infty} A_i \in \A $.
\end{nums}
%\vspace{-1mm}
\item[$P$] --- числовая функция $P:\A\to {\Bbb R}$ (называемая \выд{вероятностью} или \выд{вероятностной мерой}), такая что
%\vspace{-2mm}
\begin{nums}{-4}
\item[в1.] $P(\es)=0$, $P(\Omega)=1$, $P(A)\ge 0$ для любого $A \in \A$;
\item[в2.] (\выд{аддитивность\/}) %\выд{(аддитивность вероятностной меры)}
если $A\cap B=\es$ (то~есть~события $A$ и $B$ \выд{несовместны}), то $P(A\cup B)=P(A)+P(B)$;
% Возможно, пункт 3 нужно убрать
\item[в3.] (\выд{счётная аддитивность\/})
если $\Omega$ бесконечно, $A_i \in \A$ при $i=1,2,\dots $ и $A_i\cap A_j=\es$ при $i\ne j$,\\ то
$P(\bigcup\limits_{i=1}^{\infty}A_i)=
% \sum\limits_{i=1}^{\infty}P(A_i)=
\lim\limits_{n\to\bes}\sum\limits_{i=1}^{n}P(A_i)$.
\end{nums}
\end{items}
\копр
}

\vspace*{-3mm}
% это если нет п.3





\задача
Пусть $(\Omega ,\A, P)$ --- вероятностное пространство, $A$ и $B$ — события.
Докажите, что:
\\
\пункт
$P(\overline{A})=1-P(A)$;
\пункт
$P(A)\le 1$;
\пункт
если $A\subset B$, то $P(A)\le P(B)$;
\\
\пункт
$P(A\cup B) = P(A) + P(B) - P(A\cap B)$;
\пункт
% если $P(C)\ne 0$, то
$P(A\cup B\mid C) = P(A\mid C) + P(B\mid C) - P(A\cap B\mid C)$;
\\
\пункт
Переформулируйте в терминах множеств исходов утверждение: «событие $A$ влечёт событие~$B$».
\кзадача
\решение
\textbf{а})
$P(\Omega)=1$, события $\overline {A}$ и $A$ несовместны, а их объединение — как раз $\Om$.
\textbf{б})
\textbf{в})
\textbf{г})
\textbf{д})
\крешение
















\задача
Рассмотрим любое
конечное множество $\Omega $ из $k$ элементов.
Пусть $\A$ --- множество $2^\Omega$ всех подмножеств $\Omega$.
Для каждого $X\in\A$ положим $P(X)=|X|/k$.
Докажите, что тройка $(\Omega ,\A, P)$ образует вероятностное пространство.
Найдите все такие вероятностные пространства в листке 35.
\кзадача
\решение

\крешение












%\задача
% \пункт
% Обязательно ли элементарное событие является событием?
% \пункт
%Докажите, что если все элементарные исходы являются событиями, то функция $P$ полностью определяется вероятностями элементарных исходов.
%\кзадача
% \ответ
% \textbf{а}) Нет.
% \кответ
% \решение
% \textbf{а})
% \textbf{б})
% \крешение












\задача
Пусть $(\Omega ,\A, P)$ --- конечное вероятностное пространство.
\пункт
Чему равны минимальное и максимальное значения $|\A|$?
\пункт
Докажите, что $|\A|$ — всегда степень двойки.
\кзадача
\ответ
\textbf{а})
\кответ
\решение
\textbf{а})
\textbf{б})
\крешение












\пзадача[Схема Бернулли]
Проводятся $n$ независимых опытов, в каждом опыте может произойти определенное событие (<<успех>>) с вероятностью $p$ (или не произойти --- <<неудача>> --- с вероятностью $q=1-p$), после чего подсчитывается количество успехов.
С какой вероятностью будет ровно $i$ успехов?
%То есть исход — число от $0$ до $n$.
%Постройте вероятностное пространство, соответствующее этому эксперименту.
% Проводится $n$ независимых опытов (независимые в совокупности события\break $A_1, \ldots, A_n$), и вероятность успеха в каждом из них равна $p$.
% \пункт Дайте пример вероятностного пространства и таких событий $A_1,\ldots, A_n$.
% \пункт Найдите вероятность того, что ровно $i$ из этих событий произошли.
\кзадача
\ответ

\кответ
\решение

\крешение












\задача
На неудачном перекрёстке авария происходит с вероятностью 0{,}01 в день.
Какое число аварий произойдёт на нём за год с наибольшей вероятностью?
\кзадача
\ответ
3.
\кответ
\решение
Вероятности примерно: $0{,}026, 0{,}094, 0{,}173, 0{,}211, 0{,}193, 0{,}141, ...$
\крешение












% \сзадача
% Приведите пример вероятностного пространства без аксиомы в3, в котором задача 3б неверна.
% \кзадача





\решение
Счётные пересечения и объединения:
Если $\Omega$ бесконечно, и если $A_i \in \A, i=1,2,\dots $, то $\bigcup\limits_{i=1}^{\infty} A_i \in \A $  и $\bigcap\limits_{i=1}^{\infty} A_i \in \A $.

Счетная аддитивность вероятностной меры:
в случае бесконечного $\Omega$ если $A_i \in \A, i=1,2,\dots $ и $A_i\cap A_j=\es,i\ne j$,
(т.е.  $\{A_i\}$ --- попарно несовместны), то $P(\bigcup\limits_{i=1}^{\infty}A_i)=\sum\limits_{i=1}^{\infty}P(A_i)$.
\крешение












\пзадача[Геометрическое распределение]
Проводится сколь угодно длинная серия независимых опытов, в каждом из которых может произойти событие (<<успех>>) с вероятностью $p$, или событие <<неудача>> (с вероятностью $q=1-p$), до тех пор, пока не произойдёт успех.
Подсчитывается количество испытаний до наблюдения первого <<успеха>>.
Найдите вероятности каждого исхода.
\кзадача
\ответ

\кответ
\решение

\крешение











\задача
Пусть вероятность попасть под машину, переходя улицу в неположенном
месте, равна 0{,}01. Какова вероятность остаться целым, сто раз
перейдя улицу в неположенном месте?
%Как связана эта вероятность с числом $e$ (см.~листок 19)?
\кзадача
\ответ

\кответ
\решение

\крешение







% \ЛичныйКондуит{0mm}{7mm}
% \ОбнулитьКондуит
% \newpage




\раздел{Геометрические вероятности}

{\small
Иногда множество исходов естественно представлять как какую-то фигуру на прямой, на плоскости или в пространстве.
В этом случае событиями считают любые подмножества фигуры, которые имеют длину, площадь или объём.
А~вероятность события --- как долю от полной длины, площади или объёма.
\par}


\пзадача
В мишень радиуса $1$ стреляют точечной пулей.
С какой вероятностью пуля попадёт в круг радиуса $1/2$ с тем же центром?
Попробуйте придумать рассуждения с разными ответами.
\кзадача
\ответ

\кответ
\решение

\крешение











\задача [Парадокс Бертрана]
С какой вероятностью случайная хорда окружности больше
стороны правильного треугольника, вписанного в эту окружность?
Попробуйте придумать рассуждения с разными ответами.
\кзадача
\ответ

\кответ
\решение

\крешение













% \сзадача
% С какой вероятностью случайный треугольник на плоскости остроугольный?
% Придумайте рассуждения с разными ответами.
% \кзадача
% \ответ

% \кответ
% \решение
%
% \крешение













\задача
Юра ежедневно в случайное время между 16 ч и 18 ч едет ужинать к маме или невесте, которые живут по той же линии метро,
но в разных концах. Юра садится в первый пришедший поезд (в любом направлении).
Он считает, что его шансы ужинать у мамы или невесты равны,
но за 20 дней был у мамы лишь дважды. Как это могло быть?
\кзадача
\ответ

\кответ
\решение

\крешение












%
%\noindent
%При решении требуется построить соответствующее бесконечное
%вероятностное пространство.


\задача
На отрезок $[0, L]$ бросают 3 точки.
С какой вероятностью третья окажется между первыми двумя?
\кзадача
\ответ

\кответ
\решение

\крешение












\пзадача
Палку %случайным образом
случайно ломают на 3 части. С какой вероятностью
из них можно сложить треугольник?
\кзадача
\ответ

\кответ
\решение

\крешение











\задача
В течение часа к станции в случайные моменты времени
подходят два поезда. Какова вероятность того, что удастся перебежать
из поезда в поезд, не ожидая на платформе, если оба стоят по 5 минут?
\кзадача
\ответ

\кответ
\решение

\крешение











% \сзадача [Задача Бюффона]
% На плоскость, разлинованную параллельными прямыми (на расстоянии
% $1$ друг от друга), брошена игла длины $\lambda<1$. Найдите
% вероятность пересечения иглы с какой-нибудь прямой.
% \кзадача
% \ответ

% \кответ
% \решение
%
% \крешение












\задача
На шахматную доску случайно кладут прямоугольник со сторонами, параллельными сторонам доски.
С какой вероятностью в нём будет поровну чёрного и белого?
\кзадача
\ответ

\кответ
\решение

\крешение













\задача
На окружности случайным образом выбирают \пункт 3; \спункт $n$ полуокружностей.
С какой вероятностью они покрывают всю окружность?
\кзадача
\ответ
\textbf{а}) $1/4$; \textbf{б}) $1 - 2^{-n+1}n$.
\кответ
\решение
\textbf{а})
\textbf{б})
\крешение



\ЛичныйКондуит{0mm}{7mm}
% \GenXMLW
\end{document}
