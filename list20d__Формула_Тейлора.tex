\documentclass[a4paper, 11pt]{article}
\usepackage{newlistok}
%\documentstyle[11pt, russcorr, listok]{article}
\newcommand{\0}[1]{\overline{#1}}
\def\C{\mbox{$\Bbb C$}}

\УвеличитьШирину{1.2truecm}
\УвеличитьВысоту{3.4truecm}


\newcommand{\RpR}{{\cal R}([a,b])}
\newcommand{\intab}{\int\limits_a^b}
\def\U{{\mathcal U}}

\renewcommand{\spacer}{\vspace{2pt}}

\begin{document}

%\scalebox{1}{\vbox{%
%\ncopy{1}{
\Заголовок{Формула Тейлора}
\НомерЛистка{20д}
\ДатаЛистка{сентябрь 2021}
\Оценки{27/20/13}

\СоздатьЗаголовок

%\раздел{Первообразная}



%\опр Функция $f$, определенная на $M\in\R$, называется $n$ раз дифференцируемой,
%если на $M$ существуют ее производные $f',f'',\dots,f^{(n)}$. Множество
%таких функций обозначается $\D^{(n)}(M)$.



\опр Говорят, что функция $f(x)$ есть "о маленькое " от функции $g(x)$
(при $x\rightarrow a$), если существует такая функция $\alpha(x)$, что
$\lim\limits_{x\rightarrow a}\alpha(x)=0$ и
$f(x)=g(x)\cdot\alpha(x)$.
Обозначение: $f(x)=o(g(x))$.
\копр

\задача Докажите, что
$\sin x=o(1)$ и $x^2=o(x)$
при $x\rightarrow 0$;
$x=o(x^2)$ при $x\rightarrow \infty$.
\кзадача

\опр Пусть $M\subseteq\R$ --- открытое множество.
Функцию $f:M\rightarrow\R$ называют {\em $n$ раз
непрерывно дифференцируемой},
если на $M$ существуют
и непрерывны производные $f',f'',\dots,f^{(n)}$. Множество
таких функций обозначают $C^{n}(M)$.
Множество функций, дифференцируемых на $M$
любое число раз, обозначают $C^{\infty}(M)$.
\копр

\задача Пусть $f\in C^{n}(\U_\varepsilon(x_0))$,
$f(x_0)=f'(x_0)=\dots=f^{(n)}(x_0)=0$. Докажите, что
\сНовойСтроки
\пункт для любого $k<n$ и для любого $x\in\U_\varepsilon(x_0)$ существует
такое $\alpha\in\U_\varepsilon(x_0)$, что
$f^{(k)}(x)=(x-x_0)f^{(k+1)}(\alpha);$
\пункт для любого $x\in\U_\varepsilon(x_0)$ существуют такие $x_1,x_2,\dots,x_n\in\U_\varepsilon(x_0)$,
что $$f(x)=(x-x_0)(x_1-x_0)(x_2-x_0)\dots(x_{n-1}-x_0)f^{(n)}(x_n);$$
\пункт $f(x)=o((x-x_0)^n)$.
\кзадача

\задача Пусть $f\in C^n(\U_\varepsilon(x_0))$. Докажите, что первые $n$ производных
в точке $x_0$ многочлена
$$P(x)=f(x_0)+\frac{f'(x_0)}{1!}(x-x_0)+\frac{f''(x_0)}{2!}(x-x_0)^2+\dots+\frac{f^{(n)}(x_0)}{n!}(x-x_0)^n$$
совпадают с первыми $n$ производными в точке $x_0$ функции $f(x)$.
\кзадача

%\пункт Докажите, что для каждой функции из п.~а) такой многочлен единственен.

\задача
\пункт
Пусть $f\in C^n(\U_\varepsilon(x_0))$. Докажите, что
при любом $x\in\U_\varepsilon(x_0)$
справедливо следующее равенство:
$$f(x)=f(x_0)+\frac{f'(x_0)}{1!}(x-x_0)+\frac{f''(x_0)}{2!}(x-x_0)^2+\dots+\frac{f^{(n)}(x_0)}{n!}(x-x_0)^n+o((x-x_0)^n)$$
(оно называется формулой Тейлора с остаточным членом в форме Пеано).\\
\пункт
Докажите, что выполнение равенства п.~а) при любом $x\in\U_\varepsilon(x_0)$
означает в случае $n=0$
непрерывность функции $f(x)$ в точке $x_0$,
а в случае $n=1$ --- дифференцируемость функции $f(x)$ в точке $x_0$.
\кзадача

\задача
%\сНовойСтроки
\пункт
Пусть $f\in C^{n+1}(\U_\varepsilon(x_0))$.
Докажите, что для любого $x\in\U_\varepsilon(x_0)$
существует такое $\alpha\in\U_\varepsilon(x_0)$,
что справедливо следующее равенство:
$$f(x)=f(x_0)+\frac{f'(x_0)}{1!}(x-x_0)+%\frac{f''(x_0)}{2!}(x-x_0)^2+
\dots+
\frac{f^{(n)}(x_0)}{n!}(x-x_0)^n
+
\frac{f^{(n+1)}(\alpha)}{(n+1)!}(x-x_0)^{n+1}$$
(оно называется формулой Тейлора с остаточным членом в форме Лагранжа).\\
\пункт
Пусть $f\in C^{\infty}(\U_\varepsilon(x_0))$.
Пусть $x\in\U_\varepsilon(x_0)$ и существует такое число $c>0$, что
при любом $\alpha\in\U_\varepsilon(x_0)$
и при любом $n\in\N$
выполнено неравенство
$|f^{(n)}(\alpha)|<c$. Докажите, что тогда
$$f(x)=f(x_0)+\frac{f'(x_0)}{1!}(x-x_0)+\frac{f''(x_0)}{2!}(x-x_0)^2+\dots$$
(т.~е.~стоящий справа ряд
(называемый рядом Тейлора
с центром в $x_0$
функции $f$)
сходится к $f(x)$.)
\кзадача

\задача Напишите ряд Тейлора с центром в $\pi/6$  функции $\sin x$.
Сходится ли он к $\sin x$ при $x\in\R$?
\кзадача


\задача Напишите ряды Тейлора с центром в нуле для следующих функций:\quad
\вСтрочку
\пункт $e^x$;
\пункт $a^x$ ($a>0$);
\пункт $\sin x$;
\пункт $\cos x$;
\пункт $\frac1{1-x}$;
\пункт $\ln(1+x)$;
\пункт $(1+x)^\alpha$ $(\alpha\in\R)$;
\пункт $\arctg x$;
\пункт $\arcsin x$;
\пункт $\frac1{1+x^2}$.
%\пункт $\displaystyle{\frac1{1+x^2}}$.
%\пункт В пунктах а) --- г) докажите, что ряд Тейлора сходится к
%соответствующей функции при любом $x\in\R$.\\
%\спункт Исследуйте сходимость ряда Тейлора к соответствующей функции в
%пунктах д)~---~з).
\кзадача


\задача
\пункт Исследуйте сходимость полученного ряда Тейлора к соответствующей функции
при $x\in\R$ в каждом из пунктов задачи 7.
\пункт Нарисуйте в любой удобной программе графики нескольких функций и их многочленов Тейлора с центром в нуле (и посмотрите ролик: youtu.be/3d6DsjIBzJ4).
\кзадача

\задача
\вСтрочку
Докажите, что \пункт $\tg x=x+\frac13x^3+o(x^3)$ при $x\rightarrow0$;
\пункт
$e^x-(1+x+\frac{x^2}{2!}+\dots+\frac{x^n}{n!})<\frac3{(n+1)!}$
при $0\leqslant x\leqslant 1$ и вычислите $e$ с точностью до~$10^{-5}$;
\пункт  $|\sin x -(x-\frac{x^3}6)|<10^{-5}$ при $|x|<1/4$.
\кзадача




\задача Вычислите пределы:
\вСтрочку
\пункт
$\displaystyle\lim\limits_{x\rightarrow 0}\frac{\cos x -e^{-x^2/2}}{x^4}$;
\пункт
$\displaystyle\lim\limits_{x\rightarrow 0}\frac{e^x\sin x -x(x+1)}{x^3}$;
\пункт
$\displaystyle\lim\limits_{x\rightarrow 0}\frac{\arctg x -\sin x}{\tg x-\arcsin x}$.
\кзадача

\задача
Пусть $f(x)=e^{-1/x^2}$ при $x\ne0$, $f(0)=0$.
Найдите для $f$ %этой функции
ряд Тейлора с центром в нуле.
%чтобы для некоторого~$x$ равенство из п.~б) предыдущей задачи не выполнялось.
\кзадача

\задача
Пусть $f\in C^{\infty}(\U_\varepsilon(0))$.
Верно ли, что ряд Тейлора с центром в нуле для функции $f$\\
\вСтрочку
\пункт
сходится при всех $x$ из $\U_\varepsilon(0)$?
\пункт
если сходится при некотором $x$ из $\U_\varepsilon(0)$,
то обязательно к $f(x)$?
\кзадача

\сзадача
Перестановка $(x_1,x_2,\dots,x_n)$ чисел $1,2,\dots,n$ называется
\выд{змеей} (длины $n$), если выполнены неравенства
$x_1<x_2>x_3<x_4>\dots$ (Например,
при $n=2$ есть только одна змея $1<2$, при $n=3$ две:
$1<3>2$ и $2<3>1$.)
Пусть $k_n$ --- число змей длины $n$.
\вСтрочку
\пункт Найдите рекуррентную формулу для вычисления~$k_n$.
\пункт Пусть $K(x)=\sum\limits_{n=0}^{\infty}k_{n}\frac{x^{n}}{n!}$.
Докажите, что $2K'(x)=1+K^2(x)$ и найдите $K(x)$, решив это
уравнение.
\пункт
Докажите, что ряд Тейлора тангенса есть
$
\tg x=1\frac{x}{1!}+2\frac{x^3}{3!}+16\frac{x^5}{5!}+\dots=
\sum\limits_{n=1}^{\infty}k_{2n-1}\frac{x^{2n-1}}{(2n-1)!}.
$
\кзадача

%\СделатьКондуит{6.3mm}{9mm}


\ЛичныйКондуит{.1mm}{5mm}

% \GenXMLW

%}}}

%\СделатьКондуит{4.5mm}{9mm}

\end{document}

\задача
\пункт
\пункт
\кзадача
