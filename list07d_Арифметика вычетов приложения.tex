\documentclass[a4paper,12pt]{article}
\usepackage[mag=1000]{newlistok}
\usepackage{tikz}
\usetikzlibrary{calc}

%\УвеличитьШирину{.1truecm}
%\УвеличитьВысоту{2truecm}

\Заголовок{Арифметика вычетов: приложения}
\НомерЛистка{7д}
\newcommand{\ov}{\overline}
%\renewcommand{\spacer}{\vfill}
\ДатаЛистка{10/2018}
\Оценки{22/17/12}

\begin{document}

%\scalebox{.93}{\vbox{
%\ncopy{1}{
\СоздатьЗаголовок


%\раздел{Многочлены}

%\раздел{Теорема Эйлера}


\раздел{Представимость чисел в виде суммы двух квадратов}

%\задача
%Пусть $a\in\Z_p$ и $a\neq0$. Известно, что среди остатков $a$, $a^{-1}$, $-a$, $(-a)^{-1}$ есть совпадающие.
%Докажите, что $a^2+1=0$ или $a+1=0$ или $a-1=0$.
%\кзадача

%\задача
%Пусть $p > 2$~--- простое. Сколько из чисел $1,
%2,\ldots, p - 1$ удовлетворяют сравнению $x^{\frac{p - 1}2} - 1
%\equiv 0 \pmod{p}$, а сколько~--- сравнению $x^{\frac{p - 1}2} + 1
%\equiv 0 \pmod{p}$?
%\кзадача

\задача Пусть $p$~--- простое вида $4k + 1$, и пусть $x=(2k)!$. Докажите, что
%найдется такое целое число $x$, что
$x^2 \equiv -1 \pmod{p}$.
\кзадача

\задача Пусть $p$~--- простое вида $4k + 1$, и пусть $x$ удовлетворяет сравнению $x^2 \equiv -1 \pmod{p}$. Докажите, что
\сНовойСтроки
\пункт $(a + xb)(a - xb)\equiv a^2 + b^2 \pmod{p}$ при $a,b\in\Z$;
\пункт среди чисел вида $m + xn$, где $m,n\in\Z$, $0 \leq m,n
\leq [\sqrt p]$, найдутся два с равными остатками от деления на $p$;
\пункт найдётся ненулевое число $a + bx$, делящееся на
$p$, где $a,b\in\Z$, причём $|a|<\sqrt p$ и $|b|<\sqrt p$;
% не равны одновременно нулю и по абсолютной величине оба меньше $\sqrt p$;
\пункт $p$ представимо в виде суммы двух квадратов целых чисел.
\кзадача

\задача Пусть $p$~--- простое число вида $4k+3$, числа $a$ и $b$ целые и $a^2 + b^2$ делится на $p$.
Докажите, что $a$ делится на $p$ и $b$ делится на $p$. %{\it Указание:} воспользуйтесь задачей 11, а).
\кзадача

\задача Докажите, что произведение чисел, представимых в виде суммы
двух квадратов целых чисел, само представимо в виде суммы двух
квадратов целых чисел. \кзадача

\задача Сформулируйте и докажите теорему о том,
как по разложению числа на простые множители узнать, представимо ли это число
в виде суммы двух квадратов целых чисел.
\кзадача

\раздел{Функция Эйлера и китайская теорема об остатках}


%\задача \пункт  Найдите $\varphi({p}^{\alpha})$, где $p$ простое, $\alpha\in\N$.\\
%\пункт Докажите, что $\varphi(ab)=\varphi(a)\varphi(b)$, если $(a,b)=1$.
%; \пункт $\varphi({p_1}^{\alpha_1})\dots\varphi({p_k}^{\alpha_k})$.
%\кзадача

%\опр
\smallskip
\noindent
{\bf Определение.} Определим функцию Эйлера $\varphi(m)$ как количество обратимых элементов в $\Z_m$.
%\копр

\smallskip
\задача
Докажите, что это определение согласуется с данным в задаче 17 листка $23$.
\кзадача

%\опр
%Определим множество $\Z_k\times\Z_l$ как множество всех пар, в которых первый элемент принадлежит $\Z_k$, а второй принадлежит $\Z_l$.\\
%Суммой и произведением пар $(\alpha,\beta)$ и $(\gamma,\delta)$ из $\Z_k\times\Z_l$ будем считать пары $(\alpha+\gamma, \beta+\delta)$ и $(\alpha\gamma,\beta\delta)$ соответственно.\\ Нулем в $\Z_k\times\Z_l$ будем называть пару $([0],[0])$, а единицей --- пару $([1],[1])$.\\ Тогда в $\Z_k\times\Z_l$ можно (аналогично листку $15\frac12$) определить делители нуля, обратимые элементы.
%\копр

\задача
Пусть $k$ и $l$ --- взаимно простые натуральные числа. Для каждого натурального $n$
сопоставим элементу $\ov{n}_{kl}$ из $\Z_{kl}$ пару элементов $(\ov{n}_k,\ov{n}_l)$ из $\Z_k\times\Z_l$ (то есть, остатку от деления $n$ на $kl$ сопоставляем пару --- остатки от деления $n$ на $k$ и на $l$). Докажите, что
\\
\пункт паре $(\ov{0},\ov{0})$ соответствует только $\ov{0}$;\\
\пункт это сопоставление является биекцией между $\Z_{kl}$ и $\Z_k\times\Z_l$;\\
\пункт $\ov{n}_{kl}$ --- обратимый элемент тогда и только тогда, когда $\ov{n}_k$ и $\ov{n}_l$ --- обратимые элементы;\\
\пункт $\varphi(kl) = \varphi(k)\varphi(l)$.
\кзадача

\задача Пусть $p$ --- простое, $k,m$ --- произвольные натуральные числа. Найдите\\ \пункт $\varphi(1)$; \пункт $\varphi(p)$; \пункт $\varphi(p^k)$; \пункт $\varphi(m)$.
\кзадача

\задача [Китайская теорема об остатках]\\
\пункт Пусть натуральные $m_1, \dots, m_k$ попарно взаимно просты.
Докажите, что для любых целых $b_1,\dots,b_k$ существует такое
целое $x$, что
$x\equiv b_1\!\pmod{m_1}$, \dots,
$x\equiv b_k\!\pmod{m_k}$,
и это $x$ можно единственным образом выбрать так, что
%такое $x$ найд\"ется на отрезке
$0\leq x< m_1\cdot m_2\cdot\ldots\cdot m_k$.\\
\пункт Используя функцию Эйлера, явно укажите такое $x$.
\кзадача

\задача
 Укажите все целые числа, которые удовлетворяют системе

\таа
{\пункт $  \left\{
\begin{array}{l}
x \equiv 3 \pmod{5};  \\[4pt]
x \equiv 7 \pmod{17}.  \\[4pt]
\end{array}
\right. $}
{\пункт $  \left\{
\begin{array}{l}
x \equiv 2 \pmod{13};  \\[4pt]
x \equiv 4 \pmod{19}.  \\[4pt]
\end{array}
\right. $}


\кзадача

\задача
Найдите такое натуральное число $a$, что $a/2$ --- точный квадрат, $a/3$ --- точный куб, $a/5$~--- точная 5-я степень.
\кзадача


\сзадача
Существует ли
\вСтрочку
\пункт
сколь угодно длинная;
\пункт
бесконечная арифметическая прогрессия, каждый член которой --- степень
натурального числа с целым показателем, большим~1?
%\пункт бесконечной арифметической прогрессии с таким
%свойством не существует.
\кзадача

\ЛичныйКондуит{0mm}{8mm}
% \GenXMLW

\end{document}

