\documentclass[a4paper, 12pt]{article}
\usepackage{newlistok}

\УвеличитьШирину{1.2truecm}
\УвеличитьВысоту{1.5truecm}


%\renewcommand{\spacer}{\vspace{1pt}}

\begin{document}

%\scalebox{1}{\vbox{%
%\ncopy{1}{
\Заголовок{Суммирование рядов}
\НомерЛистка{69}
\ДатаЛистка{04.10 -- 08.10.2021}
\Оценки{14/11/8}

\СоздатьЗаголовок

\noindent

\опр
Пусть $(a_n)$~--- любая последовательность чисел.
Формальное выражение\break
$a_1+a_2+a_3+\ldots=\sum\limits_{n=1}^{+\infty} a_n$
называют {\em рядом}.
Число $s_n=a_1+a_2+\dots+a_n$ называют {\em  $n$-й частичной
суммой} ряда.
Говорят, что ряд $\sum\limits_{n=1}^{+\infty} a_n$
{\em сходится}, если
существует конечный предел $\lim\limits_{n\to{+\infty}}s_n$, и тогда этот предел называют {\em суммой ряда};
% Тогда пишут $\sum\limits_{n=1}^{+\infty} a_n=A$.
% Если предел  $\lim\limits_{n\to{+\infty}}s_n$
% не существует, то говорят, что ряд $\sum\limits_{n=1}^{+\infty} a_n$
иначе говорят, что ряд {\em расходится}.
\копр

% \задача
% Пусть %$(a_n)$ --- последовательность,
% $a_n\geq0$ при $n\in\N$.
%%из неотрицательных чисел.
% Докажите, что ряд $\sum\limits_{n=1}^\infty a_n$ сходится
% тогда и только тогда, когда
% ограничено множество его частичных
% сумм $\{s_n\ |\ n\in\N\}$, причём в этом случае
% $\sum\limits_{n=1}^\infty a_n=\sup\{s_n\ |\ n\in\N\}$.
% \кзадача

\пзадача
Какие из следующих рядов сходятся? Найдите их суммы.\\
\вСтрочку
%Сходятся ли следующие ряды? Для каждого сходящегося ряда
%найдите его сумму.
\пункт
$\sum\limits_{n=1}^{+\infty} (-1)^n$;
%\пункт
%$\sum\limits_{n=1}^\infty \frac1{2^n}$;
\пункт
%({\em геометрическая прогрессия})
$\sum\limits_{n=1}^{+\infty} \frac1{q^n}$, $q\in\R,\ q\ne0$;
\пункт
({\em гармонический ряд})
$\sum\limits_{n=1}^{+\infty} \frac1n$;
\пункт
$\sum\limits_{n=1}^{+\infty} \frac{n}{2^n}$;
\спункт
$\sum\limits_{n=1}^{+\infty} \frac{n^2}{2^n}$;\\
\пункт
$\sum\limits_{n=1}^{+\infty} \frac1{n(n+1)}$;
\пункт
$\sum\limits_{n=1}^{+\infty} \frac1{n(n+2)}$;
%\пункт
%$\sum\limits_{n=1}^\infty \frac{n}{(n+1)!}$;
\спункт
$\sum\limits_{n=1}^{+\infty} \frac1{n(n+1)\dots(n+k)}$.
\кзадача

\пзадача
\пункт
Докажите, что если ряд $\sum\limits_{n=1}^{+\infty} a_n$ сходится,
то $\lim\limits_{n\to{+\infty}}a_n=0$. \пункт Верно ли обратное?\\
\пункт
({\em Критерий Коши сходимости ряда.})
Докажите, что ряд $\sum\limits_{n=1}^{+\infty} a_n$ сходится тогда и только
тогда, когда для любого $\varepsilon>0$ существует такое $N$, что
из $n\geqslant m>N$
(где $n,m\in\N$) следует $|a_m+a_{m+1}+\dots+a_n|<\varepsilon$.
\кзадача

%\задача
%Cходится ли ряд $\sum\limits_{n=1}^\infty a_n$, если
%для каждого $p=1, 2, 3,\dots$ выполнено $\lim\limits_{n\to\infty}
%(a_{n+1}+a_{n+2}\hm+\dots+a_{n+p})=0$?

\задача
Пусть ряд $\sum\limits_{n=1}^{+\infty} a_n$ расходится, но
$\lim\limits_{n\to{+\infty}}a_n=0$. Верно ли, что
$\lim\limits_{n\to{+\infty}} s_n={\infty}$?
\кзадача

\задача
%\пункт
%Пусть ряды
%$\sum\limits_{n=1}^\infty a_n$ и $\sum\limits_{n=1}^\infty b_n$
%сходятся. Докажите, что тогда ряд $\sum\limits_{n=1}^\infty (\alpha
%a_n+\beta b_n)$ сходится, причём выполнено равенство
%$\sum\limits_{n=1}^\infty (\alpha a_n+\beta b_n)=
%\alpha\sum\limits_{n=1}^\infty a_n+\beta\sum\limits_{n=1}^\infty b_n$.
%\пункт
%Пусть ряд $\sum\limits_{n=1}^\infty a_n$ сходится,
%а ряд $\sum\limits_{n=1}^\infty b_n$ расходится.
%Докажите, что тогда ряд $\sum\limits_{n=1}^\infty (a_n+b_n)$
%расходится.
%\пункт
Верно ли, что если ряды
$\sum\limits_{n=1}^{+\infty} a_n$ и $\sum\limits_{n=1}^{+\infty} b_n$
сходятся, то сходится и ряд $\sum\limits_{n=1}^{+\infty} a_nb_n$?
\кзадача

%\сзадача
%Пусть $(a_n)$, $(b_n)$~--- монотонно стремящиеся к нулю последовательности
%положительных чисел, причём $\sum a_n$ и $\sum b_n$ расходятся.
%Всегда ли ряд $\sum \min(a_n, b_n)$ расходится?


\задача
Сходится ли ряд $\sum\limits_{n=1}^{+\infty} \frac{a_n}{n}$, где
$a_n=1$, если в десятичной записи числа
$n$ нет цифры 9,\break и $a_n=0$ в противном случае?
\кзадача

\пзадача
Пусть $a_n\geq0$  при $n\in\N$.
\пункт Докажите, что
если %ряд
$\sum\limits_{n=1}^{+\infty} a_n$
%из неотрицательных членов
сходится, то
$\sum\limits_{n=1}^{+\infty} a_n^2$ %также
сходится.
\пункт Верно ли обратное?
\кзадача

% \задача
% Докажите:
% \вСтрочку
% \пункт
% ряд $\sum\limits_{n=1}^{+\infty} \frac1{n!}$ сходится;
% \пункт
% $\sum\limits_{n=1}^{+\infty} \frac1{n!}=e$;
% \пункт
% $e-\sum\limits_{n=1}^m \frac1{n!}<\frac1{m!\,m}$;
% \пункт
% $e$ иррационально.
% \кзадача

\пзадача
Пусть $a_n\geqslant0$ при всех $n\in\N$ и $\sigma\colon\N\to\N$~---
биекция (перестановка натурального ряда). Тогда $\sum\limits_{n=1}^{+\infty}a_n=
\sum\limits_{n=1}^{+\infty}a_{\sigma(n)}$ (то есть,
ряд слева от знака равенства сходится тогда и только тогда, когда и ряд справа, причём их суммы равны).
\кзадача


\сзадача
Пусть $p_n$ --- $n$-е простое число, $n\in\N$.\\
\вСтрочку
\пункт
Докажите, что
$\lim\limits_{n\rightarrow{+\infty}}
\left(\frac1{1-1/p_1^2}\cdot\ldots\cdot\frac1{1-1/p_n^2}\right)=
\sum\limits_{n=1}^{+\infty} \frac1{n^2}$.\\
\пункт Существует ли предел
$\lim\limits_{n\rightarrow{+\infty}}
\left(\frac1{1-1/p_1}\cdot\ldots\cdot\frac1{1-1/p_n}\right)?$
\пункт Сходится ли ряд
$\sum\limits_{n=1}^{+\infty} \frac1{p_n}$?
\кзадача


\сзадача
\вСтрочку
\пункт
Пусть $\gamma_k$ --- сумма ряда
$\sum\limits_{n=2}^{+\infty}\frac1{n^k}$.
Найдите сумму $\sum\limits_{k=2}^{+\infty}\gamma_k$.\\
\пункт [Эйлер.]
Пусть $A$ --- множество всех целых
чисел, представимых в виде $n^k$, где $n,k$ --- %любые
целые числа, большие 1.
Найдите сумму~\hbox{$\sum\limits_{a\in A}\frac1{a-1}$.}
\кзадача

\ЛичныйКондуит{0mm}{6.5mm}
% \GenXMLW


\end{document} 