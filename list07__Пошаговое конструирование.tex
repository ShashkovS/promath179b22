% !TeX encoding = windows-1251
\documentclass[12pt,a4paper]{article}
\usepackage[mag=1000]{newlistok}

\УвеличитьШирину{.8cm}
\УвеличитьВысоту{1.5cm}
\renewcommand{\spacer}{\vspace{3pt}}

\ВключитьКолонтитул
\parindent=0mm

\newthm{теория}{}{}{\it}{\it}{}

\begin{document}


\Заголовок{Пошаговое конструирование}
\НадНомеромЛистка{179 школа, 7Б.}
\Оценки{13/10/7}
\НомерЛистка{7}
\ДатаЛистка{14.10 -- 21.10 /2017}
\СоздатьЗаголовок

\теория
Многоэтажные здания строят, ставя по очереди следующий этаж на предыдущий.
Начав с первого этажа, можно построить 100-этажный небоскрёб.
То же с серийными примерами: их получают последовательно, сравнивая с предыдущими и достраивая по мере необходимости.
\ктеория

\задача
Как отметить 100 таких точек, чтобы никакие 3 не лежали на одной прямой?
%и никакие 4 не лежали на одной окружности?
\кзадача


\задача
От прямоугольника с неравными сторонами отрезают квадрат со стороной, равной меньшей стороне прямоугольника;
если оставшаяся часть не квадрат, процесс повторяют.
Докажите, что можно выбрать такой начальный прямоугольник, для которого ровно на сотом шаге получится квадрат,
причем все отрезанные квадраты будут разного размера (оставшаяся часть не в счёт).
\кзадача

\пзадача
В компании из $n$ человек $(n\geq4)$ у каждого появилась новость, известная
лишь ему одному. За один телефонный разговор двое сообщают друг другу
все известные им новости. Докажите, что за $2n-4$ разговора все они
могут узнать все новости.
\кзадача

% \пзадача
% В компании из $n>4$ человек каждый узнал по новости.
% Созвонившись, двое рассказывают друг другу все известные им новости.
% Как за $2n-4$ звонка все смогут узнать все новости?
% \кзадача

\теория
Иногда приходится начинать с построения нескольких этажей сразу.
\ктеория

\пзадача
Найдутся ли 50 различных натуральных чисел, сумма которых кратна каждому из~них?
\кзадача



\теория
Когда предыдущие примеры серии построены, можно для следующего шага использовать не один, а несколько из них, быть может, даже все.
Также серия может распадаться на несколько подсерий. Шаги идут внутри подсерий, часто они однотипны, но начала у серий разные.
\ктеория


\пзадача
Как построить строку из 100 натуральных чисел, где каждое число при делении на
любое из предыдущих даёт в остатке 1?
\кзадача

%
% \задача
% Докажите, что для любого $n$ найдется
% \\\ввпункт
% $n$ различных простых чисел;
% \спункт
% $n$ различных простых чисел вида $4k-1$.
% \кзадача


% \задача
% Докажите, что для любого $n>5$ равносторонний треугольник можно разрезать на $n$ меньших равносторонних треугольников.
% \кзадача


\задача
Докажите, что квадрат можно разрезать на любое число квадратов, большее 5.
\кзадача


\пзадача
Докажите, что если после очередной реформы в обращение введут монеты в 5 и 26 рублей,
то, пользуясь только ими, можно будет уплатить без сдачи любую сумму в целое число рублей, начиная со 100.
\кзадача



\сзадача
Первоклассник Сёма пока умеет писать только цифры 1 и 7.
Докажите, что для любого $n\geq50$ он может написать кратное 7 число с суммой цифр $n$.
\кзадача




\теория
Можно шагать не по всем числам, а только по числам избранного вида.
\ктеория

\задача
Из клетчатого квадрата $64\times64$ вырезали одну клетку. Докажите, что полученную фигуру
можно разрезать на «уголки» из трёх клеток. («Уголок» — это квадрат $2 \times 2$ без одной клетки.)
\кзадача

\пзадача
На столе стоят 1024 стакана с водой. Разрешается взять один из стаканов и перелить
из него часть воды в стакан, где воды меньше так, чтобы воды стало поровну. Укажите,
как такими операциями добиться, чтобы во всех стаканах стало поровну воды.
\кзадача


\задача
Есть 100 газовых баллонов. Разрешается выбрать от 2 до 5 баллонов и на время
соединить их вместе: тогда давление в каждом из них станет равно среднему
арифметическому их давлений. Как такими операциями добиться, чтобы во всех баллонах
давление стало одинаковым?
\кзадача



% \теория
% Чтоб добавить новый этаж, иногда придется сначала переделать некоторые (или даже все) старые.\hspace*{-3cm}
% \ктеория
%
% \задача
% В шахматном турнире каждый с каждым сыграли по разу.
% Докажите, что можно так занумеровать участников, чтобы каждый не проиграл участнику со следующим номером.
% \кзадача
%
%
% \сзадача
% Можно ли выписать в ряд 100 различных чисел вида $1/k$ так, чтобы каждое следующее было больше предыдущего на одно и то же число?
%%Докажите, что для любого $n$ найдется убывающая арифметическая прогрессия из $n$ членов вида~$1/k$.
% \кзадача


% \теория
% Индуктивное построение нередко может быть при необходимости преобразовано в явный алгоритм.\hspace*{-3cm}
% \ктеория




\теория
%Чтобы решать задачу <<шаг за шагом>>, иногда надо самому организовать эти шаги бывает полезно организовать некоторый процесс.
%Часто в задаче имеется процесс, следуя которому можно шаг за шагом
Шаги естественно возникают, если в задаче есть какой-то процесс.
Если процесса нет, его бывает полезно организовать.
\ктеория


\задача
Плоскость разбита на части несколькими
\пункт прямыми;
\пункт прямыми и окружностями.
Докажите, что эти части можно раскрасить в два цвета так, чтобы части одинакового цвета не имели общего участка границы ненулевой длины.
\кзадача


% \теория
% Индуктивное построение похоже на подъем по лестнице шаг за шагом.
% А если влезаешь по ветвящемуся дереву?
% Надежнее строить лесенку сверху вниз.
% Смотрим на нужный объект и ищем, с каким нижним он связан или из какого нижнего его можно получить добавкой.
% А от найденного спускаем ещё одну ступеньку вниз, и так пока не упрёмся во что-то твёрдое.
% \ктеория


\задача
На клетчатой доске $100\times100$ стоят несколько \пункт слонов; \пункт ладей.
Докажите, что их можно раскрасить в 3 цвета так, чтобы фигуры одинакового цвета друг друга не били.
\кзадача


% \задача
% В Зазеркалье все дороги между городами – односторонние, и, выехав из города, вернуться в него нельзя.
% Докажите, что города можно занумеровать по порядку так, чтобы при проезде по любой дороге номер города уменьшался.
% \кзадача


\ЛичныйКондуит{0mm}{6mm}
% \GenXMLW
\end{document} 