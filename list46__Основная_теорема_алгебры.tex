% !TEX encoding = Windows Cyrillic

\documentclass[a4paper, 12pt]{article}
\usepackage{newlistok}

\Заголовок{Основная теорема алгебры}
\НомерЛистка{46}
\ДатаЛистка{11.03.2020 -- 01.04.2020}
\Оценки{22/17/12}




\УвеличитьШирину{1.5truecm}
\УвеличитьВысоту{2.5truecm}

 \long\def\решение#1\крешение{}\long\def\ответ#1\кответ{} % Скрыть решения
%\renewcommand{\spacer}{\vspace{10pt}}
\begin{document}



\СоздатьЗаголовок

\опр
Говорят, что многочлен $P(z)$ имеет корень $\alpha$ кратности $k$, если $P(z)$ делится на $(z-\alpha)^k$, но не делится на $(z-\alpha)^{k+1}$.
\копр


{\narrower
\теорема
Произвольный многочлен степени $n>0$ с комплексными коэффициентами имеет
ровно~$n$ комплексных корней (считаемых со своими кратностями).
\ктеорема

}

\опр
Комплексная функция $f\colon\Cbb\to\Cbb$ комплексной переменной \выд непрерывна в точке $z_0\in\Cbb$,
если для всякого $\ep>0$ найдётся такое $\de>0$, что при всех $z$, где $|z-z_0|<\de$, верно $|f(z)-f(z_0)|<\ep$.
\копр












\пзадача
Пусть $f\colon \Cbb \to \R$ — вещественная функция на комплексной плоскости.
Дайте определение того, что
\пункт $f$ ограничена на $\Cbb$;
%\пункт $f$ не ограничена на $\Cbb$;
%\пункт функция непрерывна на $\Cbb$;
\пункт $\lim\limits_{|z|\to+\infty}f(z) = +\infty$.
% \\\пункт Дайте определение непрерывности комплексной функции $f\colon\Cbb\to\Cbb$ комплексной переменной.
\кзадача
\решение
\textbf{а})
Функция $f$ ограничена на $\Cbb$, если существует такая константа $C\in\R$, что $|f(z)|<C$ при любом $z\in\Cbb$.
\\\textbf{б})
Будем писать $\lim\limits_{|z|\to+\infty}f(z) = +\infty$, если для любого вещественного $C>0$ найдётся такой вещественный радиус $R>0$, что для любого $z$ с $|z|>R$ выполнено неравенство $|f(z)|>C$.
\marginnote{\hbox to 0cm{\hss\rule[-8pt]{\textwidth}{0.5pt}}}
\крешение














\задача
Пусть $F(z) = f(z) + i g(z)$, где $f$ и $g$ — функции из $\Cbb$ в $\R$.
Докажите, что функция $F$ непрерывна в точке $z_0$ тогда и только тогда, когда функции $f$ и $g$ непрерывны в точке $z_0$.
\кзадача
\решение
Пусть $F(z)$ непрерывна в точке $z_0\in\Cbb$.
Тогда для всякого $\ep>0$ найдётся такое $\de>0$, что при всех $z$, где $|z-z_0|<\de$, верно $|F(z)-F(z_0)|<\ep$.
Но $|F(z)-F(z_0)| = \sqrt{|f(z)-f(z_0)|^2+|g(z)-g(z_0)|^2}$, а значит $|f(z)-f(z_0)|\leq|F(z)-F(z_0)|\leq\ep$.
Следовательно, $f(z)$ непрерывна в $z_0$. Аналогично  $g(z)$ непрерывна в точке $z_0$.

Пусть теперь $f$ и $g$ непрерывны в точке $z_0\in\Cbb$.
Докажем, что $F(z)$ непрерывна в точке $z_0$.
Пусть $\ep>0$.
Тогда по определению существует такое $\de_1$, что при всех $z$, где $|z-z_0|<\de_1$, верно $|f(z)-f(z_0)|<\ep/2$.
И существует такое $\de_2$, что при всех $z$, где $|z-z_0|<\de_2$, верно $|g(z)-g(z_0)|<\ep/2$.
Возьмём $\de = \min(\de_1, \de_2)$.
Тогда при всех $z$, где $|z-z_0|<\de$, выполнено
$$|F(z)-F(z_0)| \leq |f(z)-f(z_0)|+|g(z)-g(z_0)|\leq 2\cdot\ep/2 = \ep.$$
\marginnote{\hbox to 0cm{\hss\rule[-8pt]{\textwidth}{0.5pt}}}
\крешение


\noindent {\bf Замечание.}
Далее $P(z)$ --- произвольный многочлен степени $n>0$ от комплексной
переменной $z$ с~комплексными коэффициентами.
$P(z)$ задаёт функцию из $\Cbb$ в $\Cbb$, %комплексной переменной $z$ с комплексными значениями,
а $|P(z)|$ --- функцию из $\Cbb$ в $\R$.
% можно рассматривать как функцию комплексной переменной
% $z$ с вещественными значениями.















\пзадача [Поведение многочлена на бесконечности] Докажите, что
$|P(z)|\rightarrow+\infty$ при $|z|\rightarrow+\infty$.
\кзадача
\решение
Пусть $P(z) = c_n z^n + c_{n-1} z^{n-1} + \ldots + c_1 z + c_0$.
Обозначим <<хвост>> $c_{n-1} z^{n-1} + \ldots + c_1 z + c_0$ через $Q(z)$.
Тогда $P(z) = c_n z^n + Q(z)$.

Пусть дана константа $C>0$.
Найдём явно такой радиус $R>0$, что для любого z с $|z|>R$ выполнено неравенство $|P(z)|>C$.

Будем рассматривать только $z$ c $|z|>1$.
Возьмём $M = \max(|c_0|, |c_1|, \ldots, |c_{n-1}|)$.
Тогда выполнено $|Q(z)|\leq M+M|z|+\ldots M |z|^{n-1} \leq n M |z|^{n-1}$.
Дальше
$$|P(z)| \geq |c_n z^n| - |Q(z)| \geq |c_n z^n| - n M |z|^{n-1} = |z|^{n-1} (|c_n| |z| - nM) \geq (|c_n| |z| - nM).$$
Если взять $R = nMC/|c_n|$, то при $|z|>R$ будет выполнено неравенство $|P(z)|>C$.
\marginnote{\hbox to 0cm{\hss\rule[-8pt]{\textwidth}{0.5pt}}}
\крешение












\задача \пункт [Непрерывность многочлена]
Докажите, что функция $P(z)$ непрерывна на $\Cbb$.\\
\пункт [Непрерывность модуля многочлена]
Докажите, что функция $|P(z)|$ непрерывна на $\Cbb$.
\кзадача
\решение
\textbf{а}) \textbf{Через сумму и произведение}.
Докажем, что константа и $f(z)=z$ непрерывны, а также что сумма и произведение непрерывных функций непрерывны.
Из этого будет сразу следовать непрерывность многочлена.

\bigskip

\textbf{Константа и $f(z)=z$.}
Итак, пусть $f(z)$ — константа или $f(z)=z$.
Если нам дали точку $z_0\in\Cbb$ и $\ep>0$, то возьмём $\de=\ep$.
Если вдруг $|z-z_0|<\de=\ep$, то очевидно, что $|(f(z) - f(z_0)|<\ep$.

\bigskip

\textbf{Сумма.}
Пусть функции $f$ и $g$ непрерывны в точке $z_0$. Докажем, что и сумма $f+g$ непрерывна в точке $z_0$.
Пусть $\ep>0$.
Тогда по определению существует такое $\de_1$, что при всех $z$, где $|z-z_0|<\de_1$, верно $|f(z)-f(z_0)|<\ep/2$.
И существует такое $\de_2$, что при всех $z$, где $|z-z_0|<\de_2$, верно $|g(z)-g(z_0)|<\ep/2$.
Возьмём $\de = \min(\de_1, \de_2)$.
Тогда при всех $z$, где $|z-z_0|<\de$, выполнено
$$|(f+g)(z)-(f+g)(z_0)| \leq |f(z)-f(z_0)|+|g(z)-g(z_0)|\leq 2\cdot\ep/2 = \ep.$$

\bigskip

\textbf{Произведение.}
Пусть функции $f$ и $g$ непрерывны в точке $z_0$. Докажем, что и произведение $f\cdot g$ непрерывна в точке $z_0$.
Возьмём $0<\ep<1$ и странную константу $M=2(|f(z_0)| + 1)\cdot(|g(z_0)| + 1)$.

Тогда по определению существует такое $\de_1$, что при всех $z$, где $|z-z_0|<\de_1$, верно $|f(z)-f(z_0)|<\ep/M<1$.
И существует такое $\de_2$, что при всех $z$, где $|z-z_0|<\de_2$, верно $|g(z)-g(z_0)|<\ep/M<1$.
Возьмём $\de = \min(\de_1, \de_2)$.
Теперь пусть $|z-z_0|<\de$. Считаем и преобразуем:

\noindent
\begin{tabular}{p{8.6cm}c}
  & \rule{0pt}{20pt}
  $|(fg)(z) - (fg)(z_0)| = |f(z)g(z) - f(z_0)g(z_0)| = $
  \\
  вычтем и прибавим $f(z)g(z_0)$
  &  \rule{0pt}{20pt}
  $ = |f(z)g(z) - f(z)g(z_0) + f(z)g(z_0) - f(z_0)g(z_0)| = $
  \\
  вынесем за скобку $f(z)$ и $g(z_0)$
  &  \rule{0pt}{20pt}
  $ = \Bm{f(z)\br{g(z)-g(z_0)} - g(z_0) \br{f(z)-f(z_0)}} \leq $
  \\
  модуль разности не больше суммы модулей
  &  \rule{0pt}{20pt}
  $ \leq \Bm{f(z)\br{g(z)-g(z_0)}} + \Bm{g(z_0) \br{f(z)-f(z_0)}} \leq $
  \\
  $|f(z)-f(z_0)|<\ep/M, |g(z)-g(z_0)|<\ep/M$
  & \rule{0pt}{20pt}
  $ \leq |f(z)|\ep/M + |g(z_0)| \ep/M \leq $
  \\
  $|f(z)-f(z_0)|<1 \Rightarrow |f(z)|<|f(z_0)+1|$
  &  \rule{0pt}{20pt}
  $ \leq (|f(z_0)|+1)\ep/M + |g(z_0)|\ep/M \leq \ep/2 + \ep/2 = \ep.$
\end{tabular}

\bigskip

\noindent
\textbf{Напрямую}.
Пусть $P(z) = c_n z^n + c_{n-1} z^{n-1} + \ldots + c_1 z + c_0$.
Возьмём произвольную точку $z_0\in\Cbb$ и $0<\ep<1$.
$$P(z)-P(z_0) = c_n (z^n - z_0^n) + c_{n-1} (z^{n-1}  - z_0^{n-1} )+ \ldots + c_1 (z - z_0) + с_0 ( 1 - 1).$$
Заметим, что $z^k-z_0^k = (z-z_0)(z^{k-1} + z^{k-2}z_0^1+\ldots+z^1 z_0^{k-2} + z_0^{k-1})$.
Свободные члены сокращаются, во всех оставшихся можно вынести за скобку $(z-z_0)$.
В результате получим:
$$P(z)-P(z_0) = (z-z_0)\cdot Q(z), \text{ где $Q(z)$ — какой-то многочлен степени $n-1$.}$$

Обозначим через $C$ — максимальный из модулей коэффициентов многочлена $Q(z)$.
Тогда при $|z-z_0|<1$ выполнено $Q(z)\leq n C (|z_0|+1)^n$. Обозначим эту странную константу через $M$.
Значит, при $|z-z_0|<\ep/M$ выполнено
$$
|P(z)-P(z_0)| = |z-z_0|\cdot |Q(z)| \leq \ep/M \cdot M = \ep.
$$


\textbf{б})


\marginnote{\hbox to 0cm{\hss\rule[-8pt]{\textwidth}{0.5pt}}}
\крешение











\задача [Поведение многочлена в круге]
Докажите, что
$|P(z)|$ ограничен в любом круге (конечного радиуса) и достигает в нём
своих максимума и минимума.\qquad
{\small (Вместо круга разрешается решить эту задачу для квадрата со сторонами, параллельными осям координат — этого достаточно для дальнейшего.)}

\кзадача
\решение
\marginnote{\hbox to 0cm{\hss\rule[-8pt]{\textwidth}{0.5pt}}}
\крешение












\задача [Разложение Тейлора]
Докажите, что для любого $z_0\in\Cbb$ существуют такое $k\in\N$ и такие $c_k,c_{k+1},\dots,c_n\in\Cbb$,
что $c_k\ne0$ и для любого $z\in\Cbb$ справедливо равенство
$$
P(z)=P(z_0)+c_k(z-z_0)^k+c_{k+1}(z-z_0)^{k+1}+\dots+c_n(z-z_0)^n\eqno (*)
$$
Представление $P(z)$ в таком виде называется \выд{разложением Тейлора многочлена $P(z)$ в точке~$z_0$}.
\кзадача
\решение
Возьмём $w = z - z_0$.
Тогда $z = w+z_0$.
Подставим $z = w+z_0$ в многочлен, раскроем все скобки и приведём подобные.
Получится многочлен степени $n$ уже от $w$.
Осталось подставить $w = z - z_0$.

% \textbf{Второе решение}.
% Пусть $P(z) = c_n z^n + c_{n-1} z^{n-1} + \ldots + c_1 z + c_0$.
% Заметим, что $z^k-z_0^k = (z-z_0)(z^{k-1} + z^{k-2}z_0+\ldots+z z_0^{k-2} + z_0^{k-1})$.
% Тогда
% $$P(z)-P(z_0) = c_n (z^n - z_0^n) + c_{n-1} (z^{n-1}  - z_0^{n-1} )+ \ldots + c_1 (z - z_0) \dv .$$
\marginnote{\hbox to 0cm{\hss\rule[-8pt]{\textwidth}{0.5pt}}}
\крешение












\пзадача
Напишите разложение Тейлора для многочленов
\\
\пункт $P(z) = z^3 - 3z - 2$ в точке $z_0 = -1$;
% Ответ: -3 (z+1)^2 + (z+1)^3
\пункт $P(z) = iz^3 + 2z^2 - iz + 179$ в точке $z_0 = i$;
% Ответ: 179 - (z-i)^2 + i*(x-i)^3
\кзадача
\решение
\textbf{а})
$-3 (z+1)^2 + (z+1)^3$;
\textbf{б})
$179 - (z-i)^2 + i(x-i)^3$.
\marginnote{\hbox to 0cm{\hss\rule[-8pt]{\textwidth}{0.5pt}}}
\крешение













\пзадача [Вспомогательная лемма для следующей задачи]
Пусть $(*)$ --- разложение Тейлора многочлена $P(z)$ в точке $z_0\in\Cbb$, и пусть
${\Bbb D}(z_0,r)$ --- круг с центром в $z_0$ радиуса $r$.
%т.~е. ${\Bbb D}(z_0, r) = \{z\in\Cbb:|z-z_0|\le r\}$.
Докажите, что  существует такое $r>0$, что для любого $z\in{\Bbb D}(z_0, r)$,
$z\ne z_0$ выполнено $|P(z)|<|P(z_0)+c_k(z-z_0)^k|+|c_k(z-z_0)^k|.$
\кзадача
\решение





\marginnote{\hbox to 0cm{\hss\rule[-8pt]{\textwidth}{0.5pt}}}
\крешение











\задача [Поведение многочлена в малой окрестности точки]
Пусть $P(z_0)\ne0$.
Докажите, что  существует такое $z_1$, что $|P(z_1)|<|P(z_0)|$.
\кзадача
\решение
По предыдущей задаче существует такое $r_1>0$, что для любого $z\in{\Bbb D}(z_0, r_1)$,
$z\ne z_0$ выполнено $|P(z)|<|P(z_0)+c_k(z-z_0)^k|+|c_k(z-z_0)^k|.$

Так как $|P(z)|>0$, то найдётся такой $r_2>0$, что при $|z-z_0|<r_2$ выполнено $|c_k(z-z_0)^k| < |P(z_0)|$.
Возьмём $\phi = (-\Arg(P(z_0)) - \Arg(c_k))/k$.
Тогда если $\Arg(z-z_0) = \phi$, то
$$\Arg(c_k(z-z_0)^k) = \Arg(c_k) + k\cdot \Arg(z-z_0) = -\Arg(P(z_0)).$$
А значит аргументы у $P(z_0)$ и $c_k(z-z_0)^k$ противоположны.
Так как $|c_k(z-z_0)^k| < |P(z_0)|$, то модуль их суммы просто равен разности модулей:
$$
|P(z_0)+c_k(z-z_0)^k|+|c_k(z-z_0)^k| =  |P(z_0)| - |c_k(z-z_0)^k|+|c_k(z-z_0)^k|  = |P(z_0)|.
$$

Теперь возьмём $r=\min(r_1, r_2)/2$ и $z_1 = z_0 + r e^{i\phi}$.
Тогда $\Arg(z_1-z_0) = \phi$ и $z_1 \in {\Bbb D}(z_0, r_1)$, следовательно
$$
|P(z_1)|<|P(z_0)+c_k(z_1-z_0)^k|+|c_k(z-z_0)^k| = |P(z_0)|.
$$
\marginnote{\hbox to 0cm{\hss\rule[-8pt]{\textwidth}{0.5pt}}}
\крешение











\пзадача [Поведение многочлена на плоскости]
\сНовойСтроки
\пункт
Докажите, что  $|P(z)|$ достигает на плоскости своего минимума: существует такое
$\mu\ge0$, что $|P(z)|\ge\mu$ при любом $z\in\Cbb$, причём
найдётся такое $z_0\in\Cbb$, что~$|P(z_0)|=\mu$.
\пункт Пусть $\mu$ такое, как в п.~а). Докажите, что  $\mu=0$.
\кзадача
\решение
\textbf{а}) Пусть $|P(0)|=A$. По задаче 3 и 1б) найдётся такой вещественный радиус $R$, что при $z > |R|$ выполнено неравенство $|P(z)| > A$. При этом в круге радиуса $R$ по задаче 5 в этом круге найдётся минимум $\mu \leq A$. Значит $\mu$ --- минимум значения $|P(z)|$ как в круге радиуса $R$, так и вне него, то есть на всей плоскости.
\textbf{б}) По задаче 9, если $|P(z_0)|=\mu \neq 0$, то найдётся точка $z_1$, в которой значение $|P(z_1)|$ строго меньше $\mu$, что противоречит определению $\mu$. Поэтому $\mu =0$.

\marginnote{\hbox to 0cm{\hss\rule[-8pt]{\textwidth}{0.5pt}}}
\крешение











\задача \пункт Докажите, что  всякий %произвольный
многочлен ненулевой степени с комплексными коэффициентами
имеет хотя бы один комплексный корень. \пункт Выведите из пункта а) основную теорему
алгебры.
\кзадача
\решение
\textbf{а}) В задаче 10 доказано, что для любого многочлена $P(z)$ найдётся такое $z_0$, что $|P(z_0)|=0$, а значит, и $P(z_0)=0$.
\textbf{б}) Доказываем утверждение по индукции. База: $\deg P(z)=1$, то $P(z)=az+b$, и есть корень $z_0= -\frac{b}{a}$. Пусть для всех многочленов степени $n-1$ это верно. Рассмотрим многочлен $P(z)$ степени $n$.

По предыдущему пункту у него найдётся хотя бы один корень $z_0$. В силу разложения Тейлора (или задачи 8 листка 25) $P(z)=(z-z_0)Q(z)$.  Если $\deg Q(z) \geq 1$, то по предположению индукции у $Q(z)$ ровно $n-1$ корень с учётом кратностей, а значит, у $P(z)$ ровно $n$ корней с учётом кратностей.

\marginnote{\hbox to 0cm{\hss\rule[-8pt]{\textwidth}{0.5pt}}}
\крешение











\пзадача Разложите в произведение многочленов не более чем второй
степени с вещественными коэффициентами многочлены
\пункт $x^4+3x^2+2$;
\пункт $x^4+4$;
\пункт $x^6+8$;
\пункт $x^n-1$;
\пункт $x^{2n}-\sqrt3x^n+1$.
\кзадача
\решение
\textbf{а}) $(x^2+1)(x^2+2)$;\\
\textbf{б}) $(x^2-2x+2)(x^2+2x+2)$;\\
\textbf{в}) $x^6+8 = (x^2-\sqrt6+2)(x^2+\sqrt6+2)(x^2+2)$;\\
\textbf{г}) $(x-1)\prod_{k=1}^{\frac{n-1}{2}} (x^2-2(\cos\frac{2 \pi k}{n})x+1)$ при нечётных $n$,

$(x+1)(x-1)\prod_{k=1}^{\frac{n-2}{2}} (x^2-2(\cos\frac{2 \pi k}{n})x+1)$ при чётных $n$. \\

\textbf{д}) $x^{2n}-\sqrt3x^n+1 = (x^n-\frac{\sqrt3}{2}+\frac{i}{2})(x^n-\frac{\sqrt3}{2}-\frac{i}{2}) = \prod_{k=1}^n(x^2-2 (\cos \frac{2\pi k+\frac{\pi}{6}}{n})x+1$
\marginnote{\hbox to 0cm{\hss\rule[-8pt]{\textwidth}{0.5pt}}}
\крешение












\задача
Докажите, что произвольный
многочлен  с вещественными
коэффициентами раскладывается в произведение многочленов не более чем
второй степени с вещественными коэффициентами.
\кзадача
\решение
\marginnote{\hbox to 0cm{\hss\rule[-8pt]{\textwidth}{0.5pt}}}
\крешение












\пзадача Многочлен $P(x)\in\R[x]$ таков, что $P(x)\geq0$ при всех $x\in\R$. Докажите, что его можно представить в виде
суммы \пункт квадратов многочленов из $\R[x]$; \пункт двух таких квадратов.
\кзадача
\решение
\textbf{а}) Заметим, что если два многочлена представлены в виде суммы квадратов, то и их произведение также можно представить в виде суммы квадратов: достаточно раскрыть скобки в произведении.

Пусть $\alpha$ --- вещественный корень многочлена $P(x)$ кратности $r$, то есть $P(x) = (x-\alpha)^r Q(x)$, где $Q(\alpha)\neq 0$. Предположим, что $r$ нечётно. Тогда в окрестности $\alpha$, не содержащей других корней, при переходе через $\alpha$ значение $P(x)$ меняет знак, что противоречит условию. Значит, $r=2k$ --- чётно, и $P(x) = ((x-\alpha)^k)^2$.

Если же $\alpha \in \mathbb{C}$ , $\alpha \not \in \mathbb{R}$ --- комплексный и не вещественный корень $P(x)$, то $P(x)$ содержит множитель $(x-\alpha)(x-\overline{\alpha}) = x^2-(\alpha+\overline{\alpha})x+\alpha\overline{\alpha}=((x-\frac{\alpha+\overline{\alpha}}{2})^2+(\mathrm{Im} \alpha)^2)$ --- сумма двух квадратов.


\textbf{б}) По аналогии с предыдущим пунктом, достаточно проверить, что произведение многочленов, представимых как сумма двух квадратов само является суммой двух квадратов. Пусть $P(x)=p_1(x)^2 + p_2(x)^2$, $Q(x) = q_1^2(x) + q_2^2(x)$. Рассмотрим комплексные многочлены $f(x) = p_1(x)+ip_2(x)$, $g(x) = q_1(x)+iq_2(x)$. Тогда $P(x)=|f(x)|$, $Q(x)=|g(x)|$, а $P(x)Q(x)=|f(x)g(x)|$, то есть также является суммой квадратов.



\marginnote{\hbox to 0cm{\hss\rule[-8pt]{\textwidth}{0.5pt}}}
\крешение











\задача
Докажите, что максимум функции $|P(z)|$ в фиксированном круге достигается в некоторой точке
граничной окружности этого круга.
\кзадача
\решение
\marginnote{\hbox to 0cm{\hss\rule[-8pt]{\textwidth}{0.5pt}}}
\крешение

\ЛичныйКондуит{0mm}{8mm}
% \GenXMLW




\end{document}


%\СделатьКондуит{11mm}{9mm}

%\сзадача Многочлен $P(x)\in\R(x)$ при всех неотрицательных
%$x$ принимает только неотрицательные значения.
%Докажите, что его можно представить в виде  $Q(x)/R(x)$, где
%$Q(x)$ и  $R(x)$ --- многочлены с неотрицательными коэффициентами.
%\кзадача






%
% \опр
% Пусть $A_n=(x_n,y_n)$ --- последовательность точек на плоскости, $A=(x,y)$ — некоторая точка.
% Последовательность точек $A_n$ сходится к точке $A$, если выполнено любое из 3 утверждений:
% \\1. $d(A_n,A)\to0$ при $n\to\infty$ (где $d(M, N)$ — расстояние между точками $M$ и $N$);
% \\2. $|x_n-x|+|y_n-y|\to 0$ при $n\to\infty$;
% \\3. $x_n\to x$ и $y_n\to y$ при $n\to\infty$.
% \копр
%
%
% \задача
% Докажите, что эти условия эквивалентны.
% \кзадача
%
%
% \опр
% Функция $f\colon \R^2 \to \R^2$ называется \выд непрерывной в точке $A$, если для любой последовательности точек $A_n$, сходящейся к $A$, последовательность $f(A_n)$ сходится к $f(A)$.
% \копр
%
% \опр
% \выд Кривой называется любая непрерывная функция из отрезка в плоскость $\R^2$.
% \\ Часто имеют в виду отрезок $[0,1]$ и пишут $\gamma\colon[0,1]\to\R^2$.
% Кривая называется \выд петлёй, если её значения в концах совпадает.
% \копр
%
%
% \задача
% Доказать, что следующие условия для функции $f\colon \R^2 \to \R^2$ эквивалентны:
% \\1. $f$ непрерывна;
% \\2. Для любого открытого круга $W$ с центром в $f(x)$ существует такой открытый круг $V$ с центром в $x$, что $f(U)\subset V$;
% \\3. Для любого открытого прямоугольника $W$ с центром в $f(x)$ существует такой открытый прямоугольник $V$ с центром в $x$, что $f(U)\subset V$.
% \кзадача
%
%
%
% \задача
%   Доказать, что из любого покрытия замкнутого квадрата открытыми прямоугольниками можно выбрать конечное подпокрытие.
% \кзадача
%
%
% \задача
%   Доказать, что любая ограниченная последовательность комплексных чисел имеет сходящуюся подпоследовательность.
% \кзадача
%
%
%
% \задача
%   Дана непрерывная кривая $\gamma\colon [0,1] \to \R^2$, причём $\gamma(0) = (0, 0)$ и $\gamma(1) = (2, 0)$.
%   Докажите, что кривая пересекает прямую $x=1$, то есть найдётся такая точка $t\in[0,1]$, что $\gamma(t) = (1, y)$ для некоторого $y$.
% \кзадача
%
%
%
% \задача
% Пусть одна непрерывная кривая соединяет левую сторону квадрата с правой, а другая — правую с левой. Доказать, что они пересекаются.
% \кзадача
%
%
% \задача
%   Существует ли непрерывное взаимно однозначное отображение квадрата в отрезок?
% \кзадача
%
%
% *Пусть отображение f из а) квадрата в себя, б) плоскости в себя, непрерывно и взаимно однозначно. Верно ли, что обратное отображение непрерывно?
%
% Определение. Если f — петля, то функцией угла называется функция, которая точке t сопоставляет одно из значений угла между осью Ox и Of(t).
%
% Доказать, что существует непрерывная функция угла. Доказать, что две непрерывные функции угла отличаются прибавлением 2\pi k, k целое.
%
% Определение. Если f с волной — функция угла для f, то f с волной (1)-f с волной(0) называется числом оборотов f вокруг 0.
%
% Доказать, что если образ f целиком содержится в некотором угле с вершиной в 0, то его число оборотов вокруг 0 равно 0.
%
% Определение. Непрерывное семейство кривых — это то же самое, что непрерывное отображение из квадрата в R^2.
%
% Доказать, что любые две кривые из непрерывного семейства кривых, не проходящих через 0, имеют одно и то же число оборотов.
%
% (Дама с собачкой) Доказать, что если образ f находится на расстоянии больше R от 0, а для всех t: d(f(t),g(t))<R, то f и g имеют одно и то же число оборотов вокруг 0.
%
% Доказать основную теорему алгебры. (Указание: рассмотреть кривые f_R(t)=P(Re^(it)))
%
% (Теорема о барабане) Доказать, что не существует такого непрерывного отображения из диска на его граничную окружность, которая переводит все точки граничной окружности в себя.
%
% (Теорема Брауэра о неподвижной точке) Доказать, что любое непрерывное отображение из диска в себя имеет неподвижную точку.
%
% Пусть f: D^2 -> R^2 — непрерывное отображение, и если f ограничить на граничную окружность, получится кривая с ненулевым числом оборотов вокруг нуля. Доказать, что существует такое x, что f(x)=0
%
% Пусть доска n на n покрашена в два цвета. Доказать, что можно либо пройти по белым клеткам от верхнего конца до нижнего, либо по чёрным клеткам от левого конца до правого.
%
% Пусть одна непрерывная кривая соединяет левую сторону квадрата с правой, а другая — правую с левой. Доказать, что они пересекаются.
% \раздел{***}
