\documentclass[a4paper,12pt]{article}
\usepackage[mag=1000]{newlistok}
\usepackage{tikz}
\usetikzlibrary{calc}

\УвеличитьШирину{1truecm}
\УвеличитьВысоту{2.5truecm}

\Заголовок{Алгоритм Евклида и линейные уравнения}
\НомерЛистка{21}
\renewcommand{\spacer}{\vfill}
\ДатаЛистка{03.09 -- 12.09/2018}
\Оценки{26/21/16}

\newcommand{\0}[1]{\overline{#1}}

%\documentstyle[11pt, russcorr, listok]{article}
%\newcommand{\del}{\mathrel{\raisebox{-.3 ex}{${\vdots}$}}}

\begin{document}

\СоздатьЗаголовок

\ввзадача %[Алгоритм Евклида]
Даны целые числа $a>b>0$.
\выд{Алгоритм Евклида} можно описать так: делим $a$ на $b$,
получаем остаток $r_1<b$,
затем делим $b$ на $r_1$,
получаем остаток $r_2<r_1$,
%затем
делим $r_1$ на $r_2$,
получаем остаток $r_3<r_2$, и т.~д.
Докажите, что %в конце концов
какой-то остаток $r_{n-1}$ разделится нацело на $r_n$,
и $r_n=(a,b)$.
\кзадача

\пзадача Найдите %, не раскладывая
%на множители  %С помощью алгоритма Евклида найдите
\вСтрочку
\пункт
$(525,231)$;
\пункт
$(7\,777\,777,7\,777)$;
\пункт
$(10946,17711)$;
\пункт
$(2^m-1,2^n-1)$.
\кзадача

%\сзадача Докажите, что $(2^m-1,2^n-1)=2^{(m,n)}-1$.
%\кзадача



\задача
\пункт
В обозначениях задачи 1 докажите, что каждое из чисел $r_1$, $r_2$, $\dots$
можно представить в виде $ax+by$, подобрав подходящие целые $x$ и $y$.\\
\пункт
Как с помощью алгоритма Евклида найти такие целые числа $x$ и $y$,
что $ax+by=(a,b)$?\\
\пункт Докажите, что $(a,b)$ делится на любой
общий делитель чисел $a$ и $b$.
\кзадача

\пзадача
С помощью пункта б) предыдущей задачи докажите, что если $(a,b)=1$ и $ac\del b$, то $c\del b$.
\кзадача

\задача
%Имеются два шаблона: длины $a$ см и длины $b$ см, $(a,b)=d$.
Какие расстояния можно отложить от данной точки на прямой,
пользуясь двумя шаблонами (без делений) длины $a$~см и $b$~см (где $(a,b)=d$)?
%\пункт $a$~см и $b$~см, где $(a,b)=d$?
\кзадача

% \задача
% Пусть целые числа $a$ и $b$ \выд{взаимно просты}/ %(то есть $(a,b)=1$).
% Докажите, что\\
% \вСтрочку
% \пункт найдутся такие целые числа $x$ и $y$, что $ax+by=1$;
% \пункт если число $c$ целое и $ac\del b$,  то $c\del b$.
%\вСтрочку
%\пункт $(ac,b)=(c,b)$;
%\кзадача

%\задача Числа $a$, $b$ и $c$ целые, $(a,b)=1$. Докажите, что
%если $ac\del b$,  то $c\del b$.
%%\вСтрочку
%%\пункт $(ac,b)=(c,b)$;
%\кзадача



\пзадача
Решите в целых числах $x$, $y$ уравнения
\вСтрочку
\пункт
$12x=42y$;
\пункт
$ax+by=0$, где $(a,b)=d$;
\пункт $2x + 3y = 1$;
\пункт $4x + 6y = 2$;
\пункт $4x + 6y = 5$;
\пункт $20x + 19y = 2019$.
\кзадача

%\задача
%\вСтрочку
%\пункт Докажите, что уравнение
%$ax+by=c$ имеет решение в целых числах~$x$,~$y$~тогда
%и только тогда, когда %$c$ делится на $(a,b)$.
%$c$ делится на $(a,b)$,
%и в этом случае
%найдется целое решение $x,\ y$, где $0\leq x<b$.
%\пункт Как найти одно из решений (укажите %какой-нибудь
%способ)?
%\пункт Как, зная одно решение,
%найти остальные? %решения?
%\кзадача

\ввпзадача
\вСтрочку
\пункт Докажите, что уравнение
$ax+by=c$ имеет решение в целых числах~$x$,~$y$~если
и только если %$c$ делится на $(a,b)$.
$c\del(a,b)$.
%и в этом случае найдется целое решение $x,\ y$, где $0\leq x<b$.
\пункт Как найти одно из решений? % (укажите какой-нибудь способ)?
\пункт Зная одно решение $(x_0,y_0)$, докажите, что остальные получаются по формуле $\left(x_0+\frac{b}{(a,b)}t,y_0-\frac{a}{(a,b)}t\right)$, когда $t$ пробегает все целые числа.
\кзадача

\задача
Рассмотрим на координатной плоскости множество $\{(x,y)\ | \ x,\ y\ -\ \text{целые},\ ax+by=c\}$,
то есть все целые точки $(x,y)$, дающие решения уравнения $ax+by=c$.
\пункт Докажите, что все эти точки лежат на одной прямой и делят её на равные отрезки.
\пункт Найдите длину этих отрезков.
\кзадача

\ввзадача[Китайская теорема об остатках] \пункт Пусть натуральные числа $a$ и $b$ взаимно просты, $r_1$ и $r_2$ --- целые неотрицательные числа, меньшие $a$ и $b$ соответственно. Докажите, что найдётся число, дающее при делении на $a$ остаток $r_1$, а при делении на $b$ --- остаток $r_2$.
\пункт Как найти остальные такие числа?
\спункт Обобщите теорему на случай, когда надо найти все числа, дающее данные остатки $r_1,\ldots,r_n$ при делении на данные попарно взаимно простые натуральные числа $a_1,\dots,a_n$.
\кзадача

\пзадача
Решите в целых числах уравнение $2x+3y+5z=1$.
\кзадача

\задача
\пункт В фирме 28 служащих с большим стажем и 37 --- с маленьким. Хозяин фирмы выделил некую сумму для подарков служащим на Новый год. Бухгалтер подсчитал, что есть только один способ разделить деньги так, чтобы все служащие с большим стажем получили поровну и все с маленьким --- тоже поровну (все получают целое число рублей, большее 0). Какую наименьшую и какую наибольшую сумму мог выделить хозяин на подарки?
\спункт А если ещё требуется, чтобы служащий с большим стажем получил больше денег, чем служащий с маленьким стажем?
\кзадача

% \ввзадача[Китайская теорема об остатках] \пункт Пусть натуральные числа $a$ и $b$ взаимно просты, $r_1$ и $r_2$ --- целые неотрицательные числа, меньшие $a$ и $b$ соответственно. Докажите, что найдётся число, дающее при делении на $a$ остаток $r_1$, а при делении на $b$ --- остаток $r_2$.
% \пункт Как найти остальные такие числа?
% \спункт Обобщите теорему на случай, когда надо найти все числа, дающее данные остатки $r_1,\ldots,r_n$ при делении на данные попарно взаимно простые натуральные числа $a_1,\dots,a_n$.
% \кзадача

\задача
Натуральные числа $a$ и $b$ взаимно просты. %, $c$ --- целое.
Докажите, что уравнение $ax+by=c$
\сНовойСтроки
\пункт
при любом целом $c$
имеет такое решение в целых числах $x$ и $y$, что $0\leq x<b$;
\пункт
имеет решение в {\em целых неотрицательных} числах $x$ и $y$, если
$c$ целое, большее $ab-a-b$;
\спункт
при целых $c$ от 0 до $ab-a-b$ ровно в половине случаев
имеет целое неотрицательное решение,
прич\"ем если для %какого-то
$c=c_0$
такое решение есть, то для
$c=ab-a-b-c_0$ таких решений нет.
\кзадача

\сзадача
{\em Слонопотам типа $(p,q)$} ходит по бесконечной клетчатой доске, сдвигаясь за ход на $p$ клеток по любому направлению <<горизонталь-вертикаль>> и на $q$ клеток по оставшемуся. (Шахматный конь --- слонопотам типа (1,2).) Какие слонопотамы могут попасть на соседнее с собой поле?
\кзадача

\сзадача
Натуральные числа $m$ и $n$ взаимно просты.
Известно, что дробь $\displaystyle\frac{m + 179n}{179m+n}$
можно сократить на число $k$. Каково наибольшее
возможное значение $k$?
\кзадача

\сзадача
Есть  шоколадка  в  форме  равностороннего  треугольника со  стороной $n$,  разделенная  бороздками  на  равносторонние  треугольники  со  стороной 1.  Играют  двое.  За  ход  можно  отломить  от  шоколадки  треугольный  кусок  вдоль  бороздки,  съесть  его,  а  остаток  передать  противнику.  Тот,  кто  получит  последний  кусок  —  треугольник со  стороной 1,  —  победитель.  Тот,  кто  не  может  сделать  ход,  досрочно проигрывает.  Кто  выигрывает  при  правильной  игре?
\кзадача

% \задача
% Решите в целых числах $x$, $y$: \вСтрочку
%\пункт $5x+7y=11$;
% \пункт $17x+23y=36$; \пункт $nx+(2n-1)y=3$; \пункт~\hbox{$525х-231у=42$.}
% \кзадача
%
%\задача Пусть $a$ и $b$ --- натуральные числа, $(a,b)=d$.
%По окружности длины $a$ см катится колесо длины $b$ см.
%В колесо вбит гвоздь, который, ударяясь об окружность, оставляет
%на ней отметки.\\
%\вСтрочку
%%\сНовойСтроки
%\пункт
%Сколько всего таких отметок оставит гвоздь на окружности?
%\пункт
%Сколько раз прокатится колесо по окружности, %прежде чем
%пока гвоздь не попад\"ет в уже отмеченную %ранее
%точку?
%\кзадача



% \задача На плоскости дана фигура, которая
% при повороте вокруг точки $O$ на угол $48^\circ$
% переходит в себя.
% Обязательно ли эта фигура переходит в себя при повороте
% вокруг $O$ на угол
% \вСтрочку
% \пункт $72^\circ$;
% \пункт $90^\circ$?
% \кзадача




%\задача Пусть натуральные числа $a$ и $b$ взаимно просты
%и не равны одновременно 1. Докажите, что существуют
%такие целые числа $x$ и $y$, что $|x|<b$, $|y|<a$, и $ax+by=1$.
%\кзадача






\ЛичныйКондуит{0mm}{6mm}
% \GenXMLW


\end{document}

