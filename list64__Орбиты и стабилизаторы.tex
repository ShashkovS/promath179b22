% !TEX encoding = Windows Cyrillic
\documentclass[a4paper,12pt]{article}
\usepackage{newlistok}

\ВключитьКолонитул

\УвеличитьВысоту{2cm}
\УвеличитьШирину{1.5cm}
\renewcommand{\spacer}{\vfil}

\Заголовок{Орбиты и стабилизаторы}
\НомерЛистка{64}
\ДатаЛистка{05.04 -- 16.04.2021}
\Оценки{26/21/16}


\begin{document}
\СоздатьЗаголовок

{\footnotesize Цель этого листка --- научиться считать количество комбинаторных объектов с учётом различных симметрий.
Например, количество раскрасок {\it каруселей} или {\it ожерелий}.

{\it Раскраску карусели} из $n$ вагончиков в $k$ цветов
можно представлять как раскраску вершин правильного $n$-угольника в $k$ цветов. Не обязательно использовать все цвета (это касается всех раскрасок в этом листке).  При этом раскраски, совмещающиеся поворотом, считаются неотличимыми. Более научное названия для раскрасок каруселей --- {\it циклические последовательности}.

Две {\it раскраски ожерелья} из $n$ бусин в $k$ цветов считаются одинаковыми, если их можно перевести друг в друга поворотом или поворотом и <<переворачиванием>>.
\par}

\задача Найдите число раскрасок карусели из $n$ вагончиков в $k$ цветов для \пункт $n=5$; \пункт $n=4$; \пункт $n=6$.
\кзадача

\задача Найдите число замкнутых ориентированных связных $p$-звенных ломаных с вершинами в вершинах данного правильного $p$-угольника (где $p$-простое). Ломаные, отличающиеся поворотом, неотличимы. Выведите отсюда критерий Вильсона для простого числа  $p$ (см. задачу 23.12г)).
\кзадача

\опр Пусть дана группа преобразований $G$ множества $X$. Говорят также, что группа $G$ {\it действует} на множестве $X$. Введём на $X$ отношение эквивалентности $\sim$:

$a \sim b \Leftrightarrow \exists g: \,\,\, g(a)=b$.

Классы эквивалентности по данному отношению $\sim$ называются {\it орбитами}. Орбита элемента $x \in X$ обозначается так: $Gx$.
Количество орбит обозначается через $C(G,X)$ или $C(G)$.
\копр

\пзадача Докажите, что $\sim$ --- отношение эквивалентности.
\кзадача

\задача\label{GX} Объясните, что количество раскрасок каруселей и ожерелий является количеством орбит для некоторой группы преобразований $G$ некоторого множества $X$ и найдите $G$ и $X$.
\кзадача

\задача
\пункт
Опишите группу движений единичного круга;
\пункт
Найдите орбиту каждой точки при действии этой группы;
\пункт
Найдите преобразование, не имеющее конечного порядка.
\кзадача

\пзадача Сколько элементов может быть в орбите карусели, состоящей из $n$ вагончиков и раскрашенной в $k$ цветов при \пункт $n=4$; $n=6$; \пункт произвольном $n$?
\кзадача

\задача  Ответьте на пункты предыдущей задачи для ожерелья из $n$ бусин, раскрашенного в $k$ цветов.
\кзадача

\опр Неподвижными точками преобразования $g \in G$ называется те $x \in X$, для которых $g(x)=x$. {\it Множество неподвижных точек} преобразования $g$ обозначается через $X^g$.
\копр

\опр \выд{Стабилизатором\/} элемента $x \in X$ при действии группы преобразований $G$ называется
множество $\{g \mid g(x)=x\} \subset G$.
\выдд Обозначение: $G_x$.
\копр

\задача\label{X^g} Найдите количество неподвижных точек $|X^g|$ для группы преобразования $G$ множеств $X$ из задачи \ref{GX}, соответствующей \пункт раскраскам карусели; \пункт раскраскам ожерелий.
\кзадача

\задача
Найдите стабилизаторы каждой из точек следующих множеств при действии их групп движений:
\пункт
квадрата;
\пункт
правильного $m$-угольника.
\спункт
куба;
\кзадача

\задача
Рассмотрим группу движений куба $G$. Эта группа также является группой преобразований следующих множеств:
\пункт множества вершин куба;
\пункт множества диагоналей куба;
\пункт множества граней куба;
\спункт множества пар вершин куба.
Опишите орбиты и стабилизаторы во всех случаях.
\кзадача

\пзадача\label{formula} Докажите следующие формулы:
\пункт $C(G) = \sum\limits_{x\in X}\frac{1}{|Gx|}$.
\пункт $\sum\limits_{x\in X} |G_x| = \sum_{g\in G}|X^g|$.
\кзадача

\замечание Как мы видим из задач \ref{X^g} и \ref{formula}, множество $|X^g|$ вычислять относительно просто, а чтобы связать её с $C(G)$ достаточно связать количество элементов в орбите и стабилизаторе. Это будет сделано в следующей задаче.
\кзамечание

\пзадача
\пункт Пусть $g(a) = b$. Как найти все элементы в $G_b$, зная все преобразования из $G_a$?
\\\пункт Докажите, что $|G| = |Gx|\cdot |G_x|$ для любого $x\in X$.
\\\пункт[Лемма Бернсайда] Докажите, что число орбит равно среднему числу неподвижных точек, \\
то есть $C(G) = \frac{1}{|G|}\sum_{g\in G}|X^g|$.
\кзадача

\задача
\пункт Сколько существует раскрасок каруселей из $n$ вагончиков в $k$ цветов?
\\\пункт Сколько существует раскрасок ожерелий из $n$ бусин в $k$ цветов?
\\\пункт Сколькими способами можно раскрасить вершины куба в $k$ цветов? Раскраски, совмещающиеся вращением куба (то есть движением, сохраняющим ориентацию, считаются одинаковыми.
\кзадача

\ЛичныйКондуит{0mm}{6mm}


% \GenXMLW

\end{document}




\задача
Пусть группа $G$ конечна. Докажите, что для любых двух элементов одной орбиты $a,b \in Gx$ выполнено $|G_a| = |G_b|$.
\кзадача

\задача
Пусть группа $G$ конечна. Докажите, что для любого $x \in X$ верно $|G| = |Gx| \cdot |G_x|$.
\кзадача



\задача
Образует ли группу множество преобразований плоскости, переводящих прямые в прямые? Что это за преобразования?
\кзадача


\newpage
\раздел{Изоморфизмы групп}
\опр
\label{homo}
Пусть $G$ --- группа преобразований множества $X$, а $H$ --- группа преобразований множества $Y$. Группы $G$ и $H$ называются \выд{изоморфными\/}, если найдётся биекция $\ph \from G \to H$, при которой тождественное преобразование переходит в тождественное, обратное --- в обратное, а композиция преобразований --- в композицию преобразований, то есть:\\
(\emph{i}\/) $\ph(\id_X) = \id_Y$;\\
(\emph{ii}\/) для каждого $g \in G$ верно: $\ph(g^{-1}) = (\ph(g))^{-1}$;\\
(\emph{iii}\/) для любых $g_1,g_2\in G$ верно: $\ph(g_1\circ g_2)=\ph(g_1)\circ\ph(g_2)$.\\
Отображение $\ph$ в этом случае называется \выд{изоморфизмом}.
\выдд Обозначение: $G\isom H$, $G\stackrel{\ph}{\isom} H$.
\копр

\задача
Правда ли, что если $G\isom H$, то
\пункт $\#G = \#H$;
\пункт $\#X = \#Y$?
\кзадача

\задача
Пусть $\ph \from G \to H$ --- биекция, такая что выполнено условие (\emph{iii}\/) определения~\ref{homo}. Докажите, что $\ph$ является изоморфизмом.
\кзадача


\задача
Докажите, что следующие группы изоморфны:\\
\пункт группа вращений правильной 4-угольной призмы (не являющейся кубом) и группа движений квадрата;\\
\пункт группа движений куба и группа движений октаэдра;\\
\пункт группа вращений правильного $n$-угольника и группа вычетов по модулю $n$ (см. задачу \ref{gr}в). Эта группа обозначается $\Z_n$ или $\Z/n\Z$;\\
\спункт группа движений тетраэдра и группа вращений куба.
\кзадача


\задача
Пусть $\ph\from G\to H$ --- изоморфизм. Докажите, что
для любого элемента $g\in G$ верно: $\ord(g) = \ord(\ph(g))$;
\кзадача

\задача
Какие из следующих групп изоморфны:\\
1) группа вращений правильного 24-угольника;\\
2) группа движений правильного 12-угольника;\\
3) группа движений правильной 6-угольной призмы;\\
4) группа движений правильного тетраэдра;\\
5) группа $S_4$?
\кзадача


\раздел{Абстрактные группы}


\опр
\выд Абстрактной \выд группой (или просто \emph{группой}) называется множество $G$ с операцией умножения, обладающей следующими свойствами:\\
(\emph{i}\/) $(ab)c = a(bc)$ для любых $a,b,c\in G$ (\emph{ассоциативность});\\
(\emph{ii}\/) существует такое элемент $e\in G$ (\emph{единица}), что $ae=ea=a$ для любого $a\in G$;\\
(\emph{iii}\/) для всякого элемента $a\in G$ существует такой элемент $a^{-1}\in G$ (\emph{обратный элемент}), что $a a^{-1} = a^{-1} a = e$.
\копр

\задача
Докажите, что всякая группа преобразований с операцией композиции является абстрактной группой.
\кзадача

\задача
\label{gr}
Являются ли  следующие множества с указанными операциями группами:\\
\пункт $(\Z, + )$; \пункт $(\R\setminus \hc{0}, \cdot)$; \пункт $(\text{остатки по модулю }5, +)$; \пункт $(\text{остатки по модулю }5, \cdot)$;\\ \пункт $(\text{ненулевые остатки по модулю }5, \cdot)$; \пункт то же самое по модулю 10.
\кзадача

\задача
\пункт
Пусть $G$ --- группа преобразований множества $X$, и $h\in G$.
Докажите, что отображение $L_h \from G \to G$, $g\corr{L_h} h\circ g$ является преобразованием $G$ (такое преобразование называется \выд{левым сдвигом});\\
\пункт
Реализуйте произвольную абстрактную группу как группу преобразования некоторого множества.
\кзадача

%\задача
%Пусть $G$ --- группа преобразований множества $X$. Обозначим через $\wt{G} = \hc{L_h\mid h\in G}$ множество всех левых сдвигов группы $G$. Докажите, что $\wt{G}$ образует группу, изоморфную $G$.
%\кзадача

\задача
Докажите, что в группе может быть только одна единица, только один обратный элемент.
\кзадача

\задача
Докажите, что группы 1) вращений окружности; 2) комплексных чисел, по модулю равных $1$ c операцией умножения; и 3) группа матриц вида $\rbmat{\cos \ph & - \sin \ph\\\sin \ph &\phantom{-}\cos\ph}$ с операцией умножения $\hr{\begin{smallmatrix}a&b\\c&d\end{smallmatrix}}\cdot\hr{\begin{smallmatrix}x&y\\z&t\end{smallmatrix}}=
\hr{\begin{smallmatrix}ax+bz&ay+bt\\cx+dz&cy+dt\end{smallmatrix}}$ изоморфны.
\кзадача



\end{document}




