\documentclass[a4paper,12pt]{article}
\usepackage[mag=1000]{newlistok}
\usepackage{tikz}
\usetikzlibrary{calc}

\УвеличитьШирину{1.3truecm}
\УвеличитьВысоту{2.5truecm}

\Заголовок{Многочлены. Корни многочленов.}
\НомерЛистка{25}
\renewcommand{\spacer}{\vfill}
\ДатаЛистка{17.10 -- 27.10/2018}
\Оценки{30/25/20}

\newcommand{\0}[1]{\overline{#1}}

%\documentstyle[11pt, russcorr, listok]{article}
%\newcommand{\del}{\mathrel{\raisebox{-.3 ex}{${\vdots}$}}}

\begin{document}

\СоздатьЗаголовок


\опр \выд{Многочленом степени $n$ от одной переменной $x$}
называется любое выражение вида\break
$
a_nx^n + a_{n-1}x^{n-1} + \dots  + a_0,  %\eqno (*)
$
где
$n\in\N\cup\{0\}$,
а \выд{коэффициенты}
$a_n,\dots ,a_0$ --- любые числа, прич\"ем $a_n\ne0$.
Краткое обозначение: $A(x)$ или $A$. Коэффициент $a_n$ называют
\выд{старшим}, $a_0$ --- {\em свободным}.\\
Число 0 называют \выд{нулевым} многочленом, его степень не определена.
Степень ненулевого многочлена $A$ обозначают $\deg A$.
Множества всех многочленов с целыми, рациональными, действительными коэффициентами обозначаются соответственно $\Z[x]$, $\Q[x]$, $\R[x]$.
Многочлен разрешается записывать в виде суммы и/или произведения нескольких многочленов: выражение $x-2x+(x-1)^5$ мы тоже считаем многочленом (чтобы его найти, надо раскрыть скобки и привести подобные).
\копр

\задача
Определите \выд{сумму} и \выд{произведение} многочленов.
% --- это сумма и
%Докажите, что сумма и произведение многочленов также
%являются многочленами.
%Как найти их коэффициенты?
\кзадача

\пзадача \вСтрочку
\пункт Пусть $\deg A=10$, $\deg B=\deg C=7$. Какими могут быть
$\deg(A+B)$ и $\deg(B+C)$?\\
\пункт Докажите, что $\deg AB=\deg A+\deg B.$
%Как связана $\deg (A+B)$ с $\deg A$ и $\deg B~?$
\пункт Докажите, что $\deg A(B(x))=\deg A\cdot \deg B$.
\кзадача

\задача Может ли произведение нескольких ненулевых многочленов
быть нулевым многочленом?
\кзадача



%\задача
%\пункт  Докажите, что  для любого натурального числа $n$ и
%любого действительного
%числа $C$ найд\"ется такое число $x$, что $x^n>C(1+x+\dots +x^{n-1})$.
%\пункт  Докажите, что  если  многочлен зада\"ет нулевую функцию, то
%он нулевой.
%\пункт  Докажите, что разные многочлены задают разные  функции.
%\кзадача

%\noindent {\bf Замечание.}
%Таким образом, можно не различать многочлен и задаваемую им функцию.

%\опр \выд{Сумма} и \выд{произведение} многочленов --- это сумма и
%произведение соответствующих функций.
%\копр


%\задача Найдите сумму  и произведение многочленов \\ \вСтрочку
%\пункт $a_3x^3+a_2x^2+a_1x+a_0$  и  $b_2x^2+b_1x+b_0;$
%\пункт  $x^{19}-9x+7$ и  $x^7+99x+1.$
%\кзадача


\noindent
%{\narrower

%}

%\задача
%Старший коэффициент многочлена $A$ положителен.
%Пусть $\deg A>0$.
%Докажите, что последовательность $A(1)$, $A(2)$,  \dots \
%\hbox{бесконечно большая.}
%$\lim\limits_{n\rightarrow+\infty}A(n)=+\infty$.
%\кзадача


%\задача Может ли ненулевой многочлен задавать нулевую функцию?
%\кзадача

\опр %{\bf Замечание.}
Многочлен $A(x)$ зада\"ет функцию, которая сопоставляет каждому числу $s$ %\in\R$
число $A(s)$ (результат подстановки в выражение $A(x)$ числа $s$ вместо переменной $x$).
\копр

\пзадача
Найдите сумму всех коэффициентов многочлена:
\вСтрочку
\пункт $(x-1)^{n}$;
\пункт $(x+1)^{n}$;
\пункт $(x-2)^{n}$;
\пункт $(x+2)^{n}$;
\пункт $(1-x+x^4)^{1000}.$
\пункт Найдите сумму коэффициентов при неч\"етных степенях
в пункте~д).
\кзадача


\раздел{Число корней многочлена}


%\опр Многочлен $A$ \выд{делится} на многочлен $B,$ если
%найд\"ется такой многочлен~$C,$ что $A=BC.$
%\копр

%\опр %{\bf Замечание.}
%Многочлен $A(x)$ зада\"ет функцию $A:\R\rightarrow\R,$
%которая сопоставляет каждому действительному
%числу $s$ %\in\R$
%число $A(s)$ (результат
%подстановки в выражение $A(x)$ числа $s$ вместо переменной $x$).
%\копр

\опр
Число  $s$  называется \выд{корнем} многочлена  $A,$  если  $A(s)=0.$
\копр

%\опр
%Если в результате подстановки числа $r$ в многочлен получился 0,
%то $r$ называют \выд{корнем} этого многочлена.
%\копр

\задача Докажите, что если многочлен $A$ \выд{делится} на многочлен $B,$
то есть существует такой многочлен~$C,$ что $A=BC$, то
все корни $B$ являются корнями $A.$ Верно ли
обратное утверждение?
\кзадача

\задача Делится ли многочлен $x^9-1$ на многочлен $x$?
А на многочлен $x^2-1$?
\кзадача

\задача  Произвольный многочлен $A(x)$ домножили на $(x-1)$. Могут ли у
получившегося многочлена все коэффициенты быть положительными?
\кзадача

\пзадача
Докажите, что $s$ --- корень многочлена $A(x)$
если и только если $A(x)$ делится на $x-s$.
%$s$ --- корень $A(x)$.
\кзадача


\пзадача Пусть $A(1)=A(2)=0.$  Докажите, что  $A(x)$ делится на $(x-1)(x-2).$
\кзадача

\пзадача Докажите, что число различных корней многочлена $A$ не больше
$\deg A.$
\кзадача

\задача
Могут ли разные многочлены задавать одну и ту же функцию?
\кзадача

\пзадача Пусть многочлен $A(x)$ таков, что $A(x)=A(-x)$ при любом $x$.
Докажите, что  существует такой многочлен $P(x),$ что
$A(x)=P(x^2)$ при любом $x$.
\кзадача


% \задача Можно ли задать многочленом функцию $\sin x$?
% \кзадача

\задача Пусть значения многочленов $A$ и $B$
совпадают при $n$ различных значениях переменной, и степени
этих многочленов меньше $n$. Докажите, что тогда $A=B.$
\кзадача

\задача
В скольких точках прямая может пересекать параболу?
\кзадача


%\задача
%%\пункт  Может ли ненулевой многочлен задавать нулевую функцию?
%%\пункт
%Могут ли разные многочлены задавать одинаковые функции?
%\кзадача

\задача
%\пункт  Докажите, что  любой квадратный тр\"ехчлен
%можно представить в виде $a+bx+cx(x-1).$\\
%\пункт Найдите уравнение параболы, проходящей через точки
%$(0;1)$, $(1;2)$ и $(2;4)$.\\
\пункт  Докажите, что  любой многочлен степени 3
представляется в виде
$$a+bx+cx(x-1)+dx(x-1)(x-2).$$
\пункт Найдите такой многочлен $P(x)$ степени 3,
что $P(0)=-8$, $P(1)=5$, $P(2)=6$, $P(3)=1$.
\кзадача

\пзадача
Даны различные числа $a_1,a_2,\dots ,a_n$ и любые числа $b_1,b_2,\dots ,b_n$.\\
\пункт
Найдите многочлен степени $n-1$, который равен $b_1$ при $x=a_1$ и равен 0 при $x\in\{a_2,\dots,a_n\}$.
\пункт
Докажите, что существует единственный многочлен $P(x)$ степени
меньше $n$ такой, что $P(a_1)=b_1,$  \dots , $P(a_n)=b_n.$
\кзадача



%\ЛичныйКондуит{.1mm}{6mm}{8}
%\ЛичныйКондуит{0mm}{6mm}

%\СделатьКондуит{5mm}{7.7mm}

%\GenXMLW


\раздел{Корни многочленов с целыми коэффициентами}


\задача  Докажите, что  если многочлен $A(x)$ с целыми коэффициентами
принимает при $x=0$ и $x=1$ неч\"етные значения, то уравнение $A(x)=0$
не имеет целых решений.
\кзадача

\пзадача \пункт Ненулевая несократимая дробь $p/q$ --- корень многочлена
$A(x)=a_nx^n + \dots + a_0$ с целыми коэффициентами.
Докажите, что  тогда $a_n$ делится на $q$
и $a_0$ делится на $p$.\\
\пункт Пусть в пункте а) дано $a_n=1$. Докажите, что
все рациональные корни $A$ --- целые числа.
\кзадача

\пзадача Найдите все рациональные корни у
\пункт $x^3-6x^2+15x-14$;
\пункт $6x^4+19x^3-7x^2-26x+12$.
\кзадача

\пзадача
Пусть $A(x)\in\Q[x]$, $A(\sqrt2)=0$. Докажите, что $A(-\sqrt2)=0.$
\кзадача

\пзадача
\пункт
Найдите ненулевой
многочлен $P$ с целыми коэффициентами и корнем
$\sqrt2+\sqrt3$.\\
\пункт Найдите все корни многочлена $P$ из пункта а).
\кзадача

\ЛичныйКондуит{0mm}{6mm}
% \GenXMLW


\end{document}

