\documentclass[12pt,a4paper]{article}
\usepackage[mag=960, tikz]{newlistok}

\УвеличитьШирину{1.3cm}
\УвеличитьВысоту{2.2cm}
\renewcommand{\spacer}{\vspace{1.5pt}}

\ВключитьКолонтитул

\begin{document}

\Заголовок{Графы. Компоненты связности.}
\НадНомеромЛистка{179 школа, 7Б.}
\Оценки{23/19/15}
\НомерЛистка{10}
\ДатаЛистка{29.11 -- 13.12/2017}


\СоздатьЗаголовок

\опр
Пусть даны граф и его вершина $A$. \выд{Компонента связности} вершины $A$ --- это подграф, состоящий из всех вершин, связанных с $A$  путём, и всех рёбер, входящих в эти пути. %--- он называется \выд{компонентой связности}.
Обозначение: $K(A)$.
\копр

\задача
Пусть вершины $A$ и $B$ графа связаны путём. Докажите, что $K(A)=K(B)$.
\кзадача

\задача
На сколько компонент связности распадается граф слона?
\кзадача

\пзадача
Из единичных спичек сложена клетчатая доска $8\times8$. Жук хочет, чтобы с любой клетки можно было доползти до любой другой, не переползая через спички.
%Какое наименьшее число спичек
Сколько спичек минимум придётся убрать?
%, если граничные спички убирать \пункт нельзя; \пункт можно.
% \пункт убирать граничные спички можно, и жук может выползать за пределы доски (но жук хочет только иметь возможность посещать все 64 клетки)?
\кзадача

\задача
Сколько рёбер куба (максимум) можно перекусить (посередине), чтобы он не распался на~части?
\кзадача


\пввзадача
Дан граф с $n$ вершинами. Докажите, что
\пункт если граф связен, то в нём не менее $n-1$ рёбер;
\пункт если граф распадается на $k$ компонент связности, то в нём не менее $n-k$ рёбер;
\пункт если в связном графе больше $n-1$ рёбер, то одно ребро можно удалить (оставив концы) так, что граф останется связным.
\кзадача

\пзадача %Можно ли рёбра куба покрасить в два цвета так, чтобы от любой вершины до любой можно было добраться по рёбрам каждого цвета?
Можно ли рёбра куба покрасить в два цвета так, чтобы по рёбрам каждого цвета можно было пройти из любой вершины в любую?
\кзадача


% \задача
% Дан связный граф, степень каждой его вершины чётна. Одно ребро удалили (оставив концы). Докажите, что граф остался связным.
% \кзадача




% \задача
% Докажите, что в самодополнительном графе
% \пункт число вершин имеет вид либо $4n$, либо $4n+1$ (где $n$ --- целое);
% \пункт от любой вершины до любой другой можно добраться не более чем по трём рёбрам.
% \кзадача



\задача
В аудитории %на Турнире городов
сидят $50$ школьников: кто-то из 8-го класса (возможно, все или никто), а остальные из 9-го. За один вопрос можно выбрать двоих и узнать, одинаковый у них номер класса или нет. После какого наименьшего числа вопросов можно будет разделить школьников на две группы так, чтобы
в одной группе оказались все школьники одного класса, а в другой --- другого? % (какой именно класс в какой из групп узнавать не нужно).
%рассадить школьников в две аудитории так, чтобы
%школьники разных классов были в разных аудиториях?
%в одной сидели школьники одного класса, а в другой --- другого.
\кзадача

\задача
\пункт
Дан клетчатый прямоугольник $m\times n$. Каждую его клетку разрезали по одной из диагоналей. На какое наименьшее число частей мог распасться многоугольник?
\пункт Какое наибольшее число клеток доски $9\times9$ можно разрезать по обеим диагоналям, чтобы доска не распалась на части?
\кзадача

\задача Каждую клетку большого клетчатого прямоугольника покрасили в один из 179 цветов (все цвета присутствуют). Пару различных цветов назовём хорошей, если найдутся две соседние (по стороне) клетки этих цветов. Каково минимально возможное число хороших пар?
\кзадача

\опр
Пусть дан простой граф $G$. Рассмотрим полный граф с теми же вершинами и сотрём в нём все рёбра, которые есть в $G$. Полученный граф называется \выд{дополнительным} к графу $G$.
%\выд{Самодополнительным} называется граф, изоморфный своему дополнительному графу.
%--- это простой граф с теми же вершинами, что и у $G$, а рёбра проведены там, где у $G$ они не проведены
\копр

\пзадача
\пункт Докажите, что граф, дополнительный к несвязному графу, связен. \пункт Верно ли обратное?
\пункт Каких графов на $n$ данных вершинах больше: связных или несвязных?
Дайте ответ для всех~$n$.
\кзадача

\задача
В стране любые два города соединены либо железной дорогой, либо авиалинией. Докажите, что:
\пункт одним из этих видов транспорта можно добраться (напрямую или с пересадками)  из любого города в любой другой;
\пункт для каждого города можно выбрать свой вид транспорта так, чтобы при помощи него можно было бы добраться из этого города до любого другого, совершив не более одной~пересадки;\\
\пункт %одним из этих двух видов транспорта можно добраться из любого города в любой другой, совершив не более двух пересадок.
в пункте а) можно выбрать вид транспорта так, чтобы количество пересадок было не больше двух.
\кзадача

%\раздел{***}

\опр
Связный граф без циклов называется \выд{деревом}. Ребро связного графа, при удалении которого (без удаления концов) граф перестаёт быть связным, называется \выд {мостом}.
\копр

%Ребро, при выкидывании которого граф перестает быть связным, называется мостом.

\пввзадача
Докажите, что
\пункт граф является деревом, если и только если каждые
две его вершины соединены ровно одним путём с различными рёбрами;
\пункт в дереве более чем с одной вершиной есть две \выд{висячие} вершины (т.е.~степени 1);
\пункт в дереве с $n$ вершинами $n-1$ рёбер; \пункт любое ребро дерева~---~мост.
\кзадача

\задача
$N$-угольник разбит на треугольники несколькими диагоналями, не
пересекающимися нигде, кроме вершин.
Построим граф, соответствующий этому разбиению: отметим внутри каждого
треугольника точку (это будут вершины
графа), соединяя две точки ребром ровно в том случае,
когда соответствующие точкам треугольники имеют общую сторону. Докажите, что
%\сНовойСтроки
\вСтрочку
\пункт этот граф будет деревом;
\пункт %Докажите, что
хотя бы у двух
треугольников разбиения две стороны
совпадают со сторонами $N$-угольника (при $N>3$).
\кзадача



\опр
Граф $O$ называется {\it остовом} связного графа $G$,
если $О$ имеет те же вершины, что и $G$,
получается из $G$ удалением некоторых р\"ебер и является деревом.
\копр

\пзадача
\пункт Всякий ли связный граф имеет остов? \пункт Может ли граф иметь несколько остовов?
\кзадача

% \задача Волейбольная сетка имеет вид прямоугольника размером $50\times600$
% клеток. Какое наибольшее число вер\"евочек можно перерезать так, чтобы
% сетка не распалась на куски?
% \кзадача

\пввзадача
Всегда ли в связном графе можно удалить некоторую вершину
вместе со всеми выходящими из не\"е р\"ебрами так, чтобы граф
остался связным?
\кзадача

%\задача Сколько остовов у полного графа на 4 вершинах? А на 5 вершинах?
%\кзадача

\ссзадача
\выд{(Теорема Кели)} Докажите, что полный граф с $n$ вершинами имеет $n^{n-2}$ остовов.
\кзадача

% \задача
% У каждого депутата есть ровно один друг и ровно один враг. Докажите, что депутатов можно разделить на две нейтральные палаты (в каждой палате нет ни друзей, ни врагов).
% \кзадача
%
% \задача
% Каждая из девочек до завтрака не более двух раз поболтала по телефону. Докажите, что их можно разбить на три группы так, чтобы в каждой группе не было болтавших между собой девочек.
% \кзадача



% \задача
% Возможна ли компания, где у каждого ровно 5 друзей, а у любых двух --- ровно 2 общих друга?
% \кзадача

%\задача
%Докажите, что из каждого связного графа можно удалить одну вершину и все выходящие из нее ребра так, что останется связный граф.
%\кзадача

% \задача
% Из столицы выходит 101 авиалиния, из
% города Дальний --- одна, а из остальных городов по 100. Докажите,
% что из столицы можно долететь в Дальний (возможно, с пересадками).
% \кзадача
%


% \GenXMLW
\ЛичныйКондуит{0mm}{5mm}


\end{document}



\задача
В стране 2017 городов, каждый соединён
дорогами не менее, чем с 1008 другими.
Докажите, что из любого
города можно проехать в любой другой напрямую или
через один промежуточный город.
%\пункт найдутся 4 города, соединённые дорогами по циклу.
\кзадача

% \задача
% Группа островов соединена мостами.
%%так, что от каждого острова можно добраться до любого другого.
% Турист обошёл
% все острова, пройдя по каждому мосту %ровно
% один раз. На острове
% Светлом он побывал трижды. Сколько мостов ведёт со
% Светлого, если турист
% \вСтрочку
% \пункт
% не с него начал и не на нём закончил?
% \пункт
% с него начал, но не на нём закончил?
% \пункт
% с него начал и на нём закончил?
% \кзадача




% \задача
% \пункт Можно ли обойти всю шахматную доску конём по циклу?
% \пункт На шахматной доске стоит конь. Двое играют в игру, по очереди делая конём шахматный ход. Проиграет тот, кто поставит коня на клетку, где он уже был. Кто может обеспечить себе победу?
% \кзадача

\сзадача
% Известно, что в графе от любой вершины до любой другой можно добраться, пройдя суммарно не более 100 рёбер, а также можно добраться, пройдя суммарно чётное число рёбер (при подсчёте каждое ребро учитывается проходится несколько раз.
В~стране между некоторыми парами городов осуществляются двусторонние беспосадочные авиарейсы.
Известно, что из~любого города в~любой другой можно долететь, как сделав не~более 100 перелётов, так и сделав чётное число перелетов.
При каком наименьшем натуральном $d$ из~любого города можно гарантированно
долететь в~любой другой, сделав чётное число перелётов, не~превосходящее~$d$?
(Разрешается посещать один и~тот~же город или совершать один и~тот~же перелет
 более одного раза.)
\кзадача




% \задача
% \вСтрочку
% В некой стране $N$ городов, некоторые из которых соединены дорогами.
% Из любого города можно добраться в любой другой
% ровно одним способом (двигаясь по дорогам и нигде не разворачиваясь назад).
% \\
% \пункт Докажите, что в стране есть город, из которого ведёт ровно
% одна дорога.
% \пункт
% Сколько дорог в этой стране?
% \пункт
% Одну дорогу закрыли на ремонт. Можно ли теперь попасть из любого
% города в любой другой?
% \кзадача

ОРГРАФЫ

\опр
Граф называется \выд{ориентированным}, если на каждом ребре указано
направление.
%Пара вершин может при этом соединяться двумя рёбрами
%разных направлений.
\копр

\задача
В турнире каждая команда сыграла с каждой по разу.
Ничьих не было. Всегда ли можно расположить команды в таком
порядке, чтобы 1-я команда выиграла у 2-й,
2-я --- у 3-й, и т.~д.?
\кзадача

\задача
На новом сайте «Разговоры.ru» зарегистрировалось 2000 человек. Каждый из них пригласил к себе в друзья по 1000 человек. Два человека объявляются друзьями тогда и только тогда, когда каждый из них пригласил другого в друзья. Какое наименьшее количество пар друзей могло образоваться?
\кзадача


\задача
\вСтрочку
\пункт
Строка из 36 нулей и единиц начинается с 5 нулей.
Среди пятёрок подряд стоящих цифр~встречаются все 32 возможные комбинации.
Найти 5 последних цифр строки.
\пункт Почему такая строка есть?
\пункт ГИРЛЯНДЫ
\кзадача

\задача
Каждый из 450 депутатов дал пощёчину ровно одному своему
коллеге. Докажите, что из них можно выбрать 150
человек,  среди которых никто никому не давал пощёчины.
\кзадача

\сзадача
Схема проезда по городу представляет собой граф (рёбра --- улицы,
вершины --- перекрёстки). Назовём ребро $AB$ \выд{перешейком}, если
любой путь, соединяющий $A$ и $B$, содержит ребро $AB$.
%от $A$ до $B$ можно проехать единственным путём --- по ребру $AB$.
Докажите, что можно ввести на всех улицах, кроме перешейков, одностороннее
движение (а на перешейках --- двустороннее) так, чтобы от любого
перекрёстка можно было доехать по правилам до любого другого. %, не нарушая правил.
\кзадача

РАМСЕЙ

\задача
На плоскости отметили 17 точек и соединили каждые две из них
цветным отрезком: красным, желтым или зел\"еным.
Докажите, что
%\вСтрочку
%\пункт из каждой отмеченной точки
%выходит не меньше 6 одноцветных отрезков;
%\пункт
найдутся три точки в вершинах одноцветного треугольника.
\кзадача




\задача
У царя Гвидона было три сына (и больше детей не было).
Из его потомков сто имело по два сына, а остальные умерли бездетными.
Сколько потомков было у царя Гвидона?
\кзадача

%\vspace*{-5truemm}
\задача
%\вСтрочку
\пункт
На чаепитие пришли 27 школьников. Каждый принес по 2 пирожных.
Все пирожные раз\-ло\-жи\-ли на 27 тарелок (по 2 на тарелку).
Докажите, что, как бы ни были размещены пирожные,
можно так раздать тарелки школьникам, что каждому
достанется хотя бы одно пирожное, которое он сам принес.
\спункт  А если каждый принёс по 10 пирожных (и их разложили
по 10 штук на тарелку)?
\кзадача

\сзадача [Теорема Холла] В некоторой компании $n$ юношей.
При каждом $k$ от 1 до $n$ верно утверждение:
для любых $k$ юношей в компании число девушек,
знакомых хотя бы с одним из этих $k$ юношей, не меньше $k$.
Можно ли женить всех юношей на знакомых девушках?
\кзадача


\задача
На какое наименьшее число частей надо разделить проволоку длиной 12 см, чтобы
из них можно было сделать каркас куба со стороной 1 см? Части можно %только
изгибать и скреплять друг с другом.
%Можно ли из куска проволоки длиной 120 см
%сделать каркас куба со стороной 10 см, не ломая проволоки?
\кзадача
