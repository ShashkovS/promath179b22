\documentclass[a4paper, 12pt]{article}
\usepackage{newlistok}
%\documentstyle[11pt, russcorr, listok]{article}
\newcommand{\0}[1]{\overline{#1}}
\def\C{\mbox{$\Bbb C$}}

\УвеличитьШирину{1truecm}
\УвеличитьВысоту{2.5truecm}
%\hoffset=-2.5truecm
%\voffset=-3.7truemm
%\documentstyle[11pt, russcorr, ll]{article}

\pagestyle{empty}
\begin{document}

\Заголовок{Предел функции}
\НомерЛистка{59}
\ДатаЛистка{18.01.2021 -- 29.01.2021}
\Оценки{35/27/21}
\СоздатьЗаголовок


\опр \label{lim2} {\small\sc (Предел функции по Гейне)} Пусть функция $f$ определена
в некоторой окрестности $\cal U$ точки $a$ кроме, быть может, самой точки $a$.
Число $b$ называется пределом $f$ в точке $a$, если
для каждой сходящейся к $a$ последовательности $(x_n)$,
элементы которой отличны от $a$ и принадлежат~$\cal U$, верно равенство $\lim\limits_{n \to \infty} f(x_n)=b$.\\
{\sl Обозначения:}
$b=\lim\limits_{x \to a} f(x)$ или $f(x)\to b$ при $x\to a$
(\лк $f(x)$ стремится к $b$ при $x$, стремящемся~к~$a$\пк).
%\noindent
%{\sl Замечание:}
\копр

\задача
\пункт Зависит ли определение 1 от выбора окрестности $\cal U$?
%А если эта точка --- предельная для области определения функции?
\пункт Влияет ли значение $f$ в точке $a$ на существование предела $f$ в $a$ и его значение?
\пункт Может ли функция иметь два предела в точке?
\кзадача

\пзадача Дайте определение того, что функция $f$ не имеет предела в точке $a$.
\кзадача

\опр \label{epsilon-delta}  {\small\sc (Предел функции по Кош\'и.)}
Пусть функция $f$ определена в некоторой окрестности $\cal U$ точки $a$ кроме, быть может, самой точки $a$.
Число $b$ называется пределом $f$ в точке $a$, если для любой окрестности $\cal V$ точки $b$
найдется такая окрестность $\cal W$ точки $a$, что при всех $x\ne a$ из $\cal W$
число $f(x)$ лежит в $\cal V$.
%Обозначение: $f\in C(а)$.
%(Неформально говоря, непрерывная функция переводит близкие точки в близкие.)
%$$
%\forall \varepsilon > 0\quad \exists \delta > 0:
%\qquad f\left(U_{\delta}(a)\cap M\right)\subset
%U_{\varepsilon}\left(f(a)\right).
%$$
\копр

\задача
Докажите эквивалентность определений 1 и 2.
\кзадача

%\задача
%Пусть точка $a$ принадлежит области определения функции $f$ и
%является предельной для этой области.
%Докажите, что $f$  непрерывна в точке $a$ %тогда и только тогда, когда
%если и только если е\"е предел в точке $a$ равен $f(a)$.
%\кзадача

\пзадача\label{example}  Найдите следующие пределы (если они существуют):\\
\vspace*{-19pt}
\\
\вСтрочку
\пункт $\lim\limits_{x\to1}\{ x\}$;
\пункт $\lim\limits_{x\to1}[x]$;
\пункт $\lim\limits_{x \to 3} \frac{x^3-6x^2+9x}{x-3}$;
%\пункт $\lim\limits_{x \to 3} \frac{x^4-2x^3-2x^2-9}{x^3+3x^2-54}$;
\пункт $\lim\limits_{x \to -1} \frac{x^2+4x+1}{x^2+2x+1}$;
\пункт $\lim\limits_{x \to 0} x\sin \frac1x$;
\пункт $\lim\limits_{x \to +\infty} \frac{\sqrt{x+\sqrt{x+\sqrt x}}}{\sqrt{x+1}}$.
\кзадача


%\задача\label{example} Найдите пределы (если они существуют) при
%$x \to 1$
%следующих функций $f(x)$:\\
%\сНовойСтроки
%\вСтрочку\пункт $c$;
%\пункт $ax+b$;
%\пункт $x^2$;
%\пункт $(x^2-1)/(x-1)$;
%\пункт $1/x$;
%\пункт $\{ x \}$; % ($\{ x\}$ --- дробная часть $x$);
%\пункт $[x]$. % ($[x]$ --- целая часть $x$).
%\кзадача

\пзадача Дайте определение
\вСтрочку
%\сНовойСтроки
%\пункт того, что $+\infty$ является предельной
%точкой множества $M\subseteq\R$;
\пункт предела функции при $x \to +\infty$;\\
\пункт того, что $f(x)$ стремится к $+\infty$, при $x \to a$
(где $a\in \R$ или $a=+\infty$).
\кзадача

\задача Найдите пределы (если они существуют) при
$x \to +\infty$ функций из задачи~\ref{example}, а)--г).
\кзадача


%\опр Число $a\in \R$ называется \выд{предельной точкой}
%множества $M\subset \R$, если в любой окрестности точки $a$
%найд\"ется элемент множества $M$, не равный $a$.
%\копр




%\задача Докажите, что функция, имеющая предел в точке $a$,
%ограничена в некоторой окрестности этой точки.
%\кзадача

%\задача Пусть $f:M\rightarrow\R$ и существует $\lim\limits_{x \to a} f(x)>0$.
%Докажите, что найд\"ется такая
%окрестность $U(a)$, что при всяком $x\in \dot U(a)\cap M$
%будет выполнено неравенство $f(x)>0$.
%\кзадача



\пзадача
Сформулируйте и докажите
\вСтрочку
\пункт теоремы о пределе суммы, разности,
произведения и отношения двух функций;
\пункт
\лк принцип двух милиционеров\пк\ для функций
%(по аналогии с задачей 14 листка 14).
\кзадача



%\задача %По аналогии с задачей 14 листка 14
%Сформулируйте и докажите \лк принцип двух милиционеров\пк\ для функций.
%\кзадача

\задача Найдите пределы при $x \to \pm \infty$ функции $f(x)=\frac{P(x)}{Q(x)}$,
где $P(x),Q(x)$ --- многочлены. % с действительными коэффициентами.
\кзадача

\задача
\вСтрочку
\пункт Пусть функции $f$ и $g$ определены на $\R$, прич\"ем
$\lim\limits_{x \to a}f(x)=A$ и $\lim\limits_{x \to A}g(x)=B$.
Обязательно ли тогда $\lim\limits_{x \to a}g(f(x))=B$?
\пункт А если $g(A)=B$?
\кзадача

%\УстановитьГраницы{0cm}{5cm}
%\опр\label{sinus} Отложим дугу $\stackrel{\smile}{OA}$ длины $x\in [0;2\pi)$
%на единичной окружности против часовой стрелки ($2\pi$ есть по определению
%длина единичной окружности), см.~рис.
%Положим по определению $\sin x$ равным ординате
%точки $A$, $\cos x$ равным абсциссе точки $A$.
%Доопределим $\sin x$ и $\cos x$
%на всю прямую по формулам:
%$\sin (x+2\pi n)\stackrel{\mbox{\small\it def}}{=\!\!\!=}\sin x$,\,
%$\cos (x+2\pi n)\stackrel{\mbox{\small\it def}}{=\!\!\!=}\cos x\ (n\in\Z)$.

%\копр
%\ВосстановитьГраницы

%\задача
%Докажите, что для каждого $x\in [0;2\pi)$ существует дуга длины $x$.
%\кзадача

%\задача Докажите формулы
%для $\sin{(\alpha\pm\beta)}$ и $\cos{(\alpha\pm\beta)}$.
%\кзадача

\задача
Докажите неравенства:
\вСтрочку
\пункт $\sin x<x$ при $x>0$;
\пункт $x<{\rm tg} x$ при $0<x<\pi/2$.
\кзадача

\пзадача[Первый \лк замечательный\пк{} предел] Докажите, что
$\lim\limits_{x\to0}\frac{\sin x}x=1$.
\кзадача

\пзадача Найдите:
\вСтрочку
\пункт $\lim\limits_{x\to0}\frac{\sin{\alpha x}}{x}$;
\пункт $\lim\limits_{x\to0}\frac{1-\cos x}x$;
\пункт $\lim\limits_{x\to a}\frac{\sin x-\sin a}{x-a}$;
\пункт $\lim\limits_{x\to a}\frac{\cos x-\cos a}{x-a}$;
\пункт $\lim\limits_{x\to +\infty}\frac{\log_2 x}{x}$.
\кзадача

\пзадача Найдите:
\вСтрочку
\пункт $\lim\limits_{x \to 0} \frac{\sqrt{1+x}-1}x$;
\пункт $\lim\limits_{x \to 0} \frac{\sqrt[n]{1+x}-1}x$ ($n\in \N$);
\пункт $\lim\limits_{x \to 1} \frac{\sqrt[m]{x}-1}{\sqrt[n]{x}-1}$
($m,n\in \N$).
\кзадача

\задача Докажите, что:
\вСтрочку
\пункт $\lim\limits_{x \to +\infty} \left(1+\frac1x\right)^x=e$;
\пункт $\lim\limits_{x \to -\infty} \left(1+\frac1x\right)^x=e$;
\кзадача

\пзадача[Второй \лк замечательный\пк{} предел] Докажите, что
$\lim\limits_{x \to 0} (1+x)^{1/x}=e$.
\кзадача

%\vspace*{-5pt}
%\раздел{Дополнительные задачи}

%\vspace*{-5pt}

\пзадача Определите предел слева $\lim\limits_{x \to a-0} f(x)$ и предел справа $\lim\limits_{x \to a+0} f(x)$ функции $f$ в точке $a$.
\кзадача

%\задача
%\вСтрочку
%\пункт Пусть функция $f(x)$ имеет предел в точке $a$. Докажите, что
%$f(x)$ имеет совпадающие пределы справа и слева в точке $a$.
%\пункт Верно ли обратное?
%\кзадача

%\задача Найдите пределы  при $x \to \pm 0$,\,
%$x \to 1 \pm 0$
%функций из задачи~\ref{example}.
%\кзадача

\задача Приведите пример функции, которая в точке $a$
\вСтрочку
%\пункт не имеет предела;
%\пункт имеет предел, не равный $f(a)$;
\пункт имеет разные пределы слева и справа;
\пункт имеет предел слева, но не имеет предела справа;
\пункт не имеет предела ни справа, ни слева.
\кзадача

\задача Докажите, что функция, монотонная на некотором интервале,
имеет предел как слева, так и справа в каждой точке этого интервала.
\кзадача

\задача
Докажите, что монотонная функция, определённая на отрезке, \\
\пункт непрерывна хотя  бы в одной его точке (может, в конце --- тогда непрерывна <<слева>> или <<справа>>);\\
\спункт непрерывна во всех его точках, за исключением не более чем счётного числа точек.
\кзадача

%\сзадача[М.~Прасолов]
%Пусть функция $f$ определена на отрезке (или на $\R$) и в каждой точке
%этого отрезка ($\R$) имеет конечный предел, не обязательно совпадающий со
%значением в точке.
%Насколько $f$ может отличаться от непрерывной? Более точно, каким
%может быть у $f$ множество точек разрыва?
%\кзадача

\сзадача Приведите пример функции, определенной на $\R$, не равной тождественно
нулю ни на каком интервале, но имеющей в каждой точке нулевой предел.
\кзадача

\сзадача
%\вСтрочку
%\пункт Существует
Может ли функция, определенная на $\R$, иметь в каждой
точке бесконечный предел?
%\пункт Тот же вопрос для функции, определенной на $\R$.
%Существует ли функция, определенная на $\R$, имеющая в каждой
%действительной точке бесконечный предел?
%\пункт Пусть $M\subseteq\R$ \выд{плотно} в $\R$ (то есть каждая действительная точка
%является предельной для~$M$). Когда существует функция, определенная на
%$M$, имеющая в каждой действительной точке бесконечный предел?
\кзадача


\ЛичныйКондуит{0mm}{8mm}

%\СделатьКондуит{4.5mm}{7.5mm}

% \GenXMLW

\end{document}

%\задача
%Рассмотрим условие
%$\quad\exists \varepsilon >0\quad \forall m\in\N\quad
%\exists n\ge m\ : \ |x_n-a|<\varepsilon$. Выясните, эквивалентно ли оно
%одному из следующих условий:
%\вСтрочку
%\пункт
%$a$ --- предельная точка $(x_n)$;
%\пункт
%$(x_n)$ ограничена;
%\пункт
%$(x_n)$ имеет предельную точку.
%\кзадача
%
%\задача
%Рассмотрим условие
%$\quad\forall \varepsilon >0\quad \exists m\in\N\quad
%\forall n\ge m\ : \ |x_n-a|>\varepsilon$. Выясните, эквивалентно ли оно
%одному из следующих условий:
%\вСтрочку
%\пункт
%$a$ не является пределом $(x_n)$;
%\пункт
%$(x_n)$ неограничена;
%\пункт
%$(x_n)$ не имеет предельных точек.
%\кзадача

