\documentclass[a4paper,12pt]{article}
\usepackage{newlistok}
\usepackage[matrix,arrow]{xy}
%\documentstyle[11pt, russcorr, listok]{article}
%\renewcommand{\spacer}{\vspace{1.8pt}}

\УвеличитьШирину{1truecm}
\УвеличитьВысоту{1truecm}
%\hoffset=-2.5truecm
%\voffset=-27.3truemm
%\documentstyle[11pt, russcorr, ll]{article}

\pagestyle{empty}

\Заголовок{Повторение}
\Подзаголовок{}
\НомерЛистка{50}
\ДатаЛистка{май 2020}
\Оценки{9/6/3}

\begin{document}


\СоздатьЗаголовок

\задача
Пусть $k$ и $n$ --- натуральные числа, большие 1.
Верно ли, что множество корней степени $n$ из комплексного числа $z$
совпадает с множеством корней степени $kn$ из числа $z^k$?
\кзадача

% \задача
% Найдите все комплексные числа $z$, по модулю равные 1, при которых
% $z^2+(1+i)z$ принимает чисто мнимые значения. Изобразите соответствующее
% геометрическое место точек на плоскости.
% \кзадача

\задача
Докажите, что произведение расстояний от вершины правильного $N$-угольника, вписанного в окружность радиуса 1, до остальных его вершин равно $N$.
\кзадача

\задача
Докажите, что точки плоскости, соответствующие комплексным числам
$z_1$, $z_2$, $z_3$ лежат на одной проямой тогда и только тогда,
когда существуют вещественные числа $\lambda_1,\lambda_2,\lambda_3$,
не все равные нулю, такие, что $\lambda_1z_1+\lambda_2z_2+\lambda_3z_3=0$
и $\lambda_1+\lambda_2+\lambda_3=0$.
\кзадача

\сзадача
Пусть $c_1,c_2,\dots,c_n,z$ --- такие комплексные числа, что
$$
\frac1{z-c_1}+\frac1{z-c_2}+\dots+\frac1{z-c_n}=0,
$$
причем точки $c_1,c_2,\dots,c_n$ являются вершинами выпуклого
многоугольника. Докажите, что точка $z$ лежит внутри
этого многоугольника.
\кзадача


\задача
Докажите, что из всякого покрытия отрезка интервалами можно выбрать конечное
число интервалов, которые покрывают этот отрезок, причем каждая точка
покрыта не более чем двумя интервалами.
\кзадача


\задача
Функция $f$ непрерывна на отрезке $[a;b]$.
Для $x\in[a;b]$ пусть $g(x)=\min\{f(t)\ |\ t\in[a;x]\}.$
Обязательно ли $g(x)$ непрерывна на $[a;b]$?
\кзадача

\задача
Функция $f$ непрерывна на отрезке $[a;b]$  и принимает на нём некоторое значение
дважды. Докажите, что для любого $\varepsilon>0$
найдутся такие точки $x_1,x_2\in[a;b]$, что $|x_1-x_2|<\varepsilon$
и $f(x_1)=f(x_2)$.
\кзадача

\задача
Найдите все непрерывные на $\R$ функции $f$ такие, что $f(x)=f(2x)$
при всех $x\in\R$.
\кзадача

\задача
Для каждого $x\in\R$ пусть
$f(x)=\lim\limits_{m\rightarrow\infty}\left(\lim\limits_{n\rightarrow\infty}
\cos^{2n}(2\pi x m!)\right)$. В каких точках из $\R$ непрерывна функция $f$?
\кзадача

\задача
Пусть $\Delta^1_hf(x)=f(x+h)-f(x)$,
$\Delta^{m+1}_hf(x)=\Delta^1_h\left(\Delta^{m}_hf(x)\right)$.
Докажите, что непрерывная функция $f(x)$ является многочленом степени
не выше $m$ тогда и только тогда, когда
$\Delta^{m+1}_hf(x)=0$ при любых $x,h\in\R$.
\кзадача

\задача
Постройте такую функцию $f(x,y)$, что при любом $b\in\R$ функция $f(x,b)$
будет непрерывной на $\R$ как функция от $x$, при любом $a\in\R$
функция $f(a,y)$ будет непрерывной на $\R$ как функция от $y$,
но $f(x,y)$ как функция двух переменных будет разрывна в точке $(0,0)$.
\кзадача


\сзадача
Дана бесконечная последовательность функций
$f_1$, $f_2$, \ldots \ (все функции определены на $\R$ и принимают
действительные значения).
Всегда ли существует конечный набор
функций $g_1$, $g_2$, \ldots, $g_N$ (также определенных на $\R$ и принимающих
действительные значения),
композициями которых можно записать любую из функций исходной
последовательности (например, $f_1(x)=g_2(g_1(g_2(x)))$ при всех $x\in\R$)?
\кзадача


\сзадача
Докажите, что существует ровно два отображения из перестановок в числа, таких что
$f(e)=1$ и $f(ab)=f(a)f(b)$. А именно, $f(\sigma) =1$ и $f(\sigma) = \text{знак}\ \sigma$.
\кзадача

\сзадача
Каждому из $N$ мудрецов написали на лбу число и выдали две варежки:
одну черную и одну белую. По сигналу все мудрецы одновременно надевают варежки.
После чего их строят в шеренгу в порядке возрастания написанных на их лбах чисел
и просят соседей взяться за руки.
Как мудрецам надевать варежки, чтобы в результате каждая белая варежка взялась
за белую, а каждая черная — за черную? (Мудрец видит все числа, кроме своего.)
\кзадача

\ЛичныйКондуит{0mm}{5mm}
% \GenXMLW

\end{document} 