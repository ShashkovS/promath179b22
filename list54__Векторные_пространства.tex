% !TeX encoding = windows-1251

\documentclass[a4paper,12pt]{article}
\usepackage{newlistok}
%\usepackage{mathtools}

\УвеличитьШирину{1truecm}
\УвеличитьВысоту{2truecm}


\Заголовок{Векторные пространства, базис, размерность}
\НомерЛистка{54}
\ДатаЛистка{02.10.2020 -- 23.10.2020}
\Оценки{30/23/16}
\ВключитьКолонтитул

\begin{document}
	\СоздатьЗаголовок
	
Пусть $K$ --- любое поле (можно считать, что это любое поле из $\Q, \R, \mathbb{C}, \Z/p\Z$).
		
%\раздел{Векторные пространства}

\опр \выд{Векторным\/ {\rm или} линейным пространством} над полем
$K$ называется любое множество $V$, элементы которого (векторы)
можно складывать друг с другом и умножать на элементы поля так, что снова получаются векторы из этого пространства, причём
выполнены следующие аксиомы:
\begin{enumerate}
  \item $u + v = v + u$ для любых $u,v\in V$;
  \item $u + (v + w) = (u + v) + w$ для любых $u, v, w \in V$;
  \item существует \выд{нулевой} вектор $0$ такой, что $u + 0 = u$
  для любого $u \in V$;
  \item для любого $v \in V$ существует \выд{противоположный}
  вектор $-v$ такой, что $v + (-v) = 0$;
  \item $1v = v$ для любого $v \in V$;
  \item $(\lambda\mu)v = \lambda(\mu v)$ для любых $\lambda,\mu \in K$ и $v \in V$;
  \item $(\lambda+\mu)v = \lambda v + \mu v$ для любых $\lambda,\mu \in K$ и $v \in V$;
  \item $\lambda(u + v) = \lambda u + \lambda v$ для любых $\lambda \in K$ и $u, v \in V$.
\end{enumerate}
\копр

	\задача	Являются ли линейными пространствами
	\пункт $\R$ над $\Q$; \пункт $\Q$ над $\R$;\\
    \пункт множество всех векторов на плоскости над $\R$;	\\
	\пункт многочлены с коэффициентами из поля $K$ (обозначение: $K[x]$) над $K$;\\
	\пункт многочлены степени не выше $n$ над $\R$; ровно степени $n$ и нулевой многочлен, над $\R$;\\
	\пункт многочлены над $\R$, равные в точке $x=1$ нулю; единице;\\
	%\пункт Функции $f:\R \to \Q$ над $\Q$;  $f:\R \to \R$ над $\Q$; непрерывные функции $f:\R \to \R$ над $\R$;\\
	\пункт строки (или столбцы) из $n$ элементов поля $K$ (обозначение: $K^n$);\\
	\пункт бесконечные последовательности действительных чисел;\\
%	\пункт ограниченные последовательности; неограниченные последовательности;\\
	\пункт сходящиеся последовательности действительных чисел;\\
%	\пункт арифметические прогрессии; геометрические прогрессии;\\
	\пункт последовательности Фибоначчи (последовательности, удовлетворяющие условию $x_{n+1} = x_{n-1}+x_n)$;
	\пункт множество решений однородной системы линейных уравнений; неоднородной;\\
    \пункт поле характеристики $p$ над полем $\Z/p\Z$?
	\кзадача

\пзадача Докажите, что элементы $0$ и $-v$ в третьей и четвертой
аксиомах определены однозначно. \кзадача

\опр Вектор $b\in V$ {\it линейно выражается} через векторы
$a_1,\dots,a_m \in V$, если существуют такие $\mu_1,\dots,\mu_m\in
K$, что $b = \mu_1 a_1 + \dots + \mu_m a_m$. (Вектор $0$ линейно
выражается через пустую систему векторов.) \копр

\опр Система векторов $(a_1,\dots,a_m)$ называется {\it линейно
зависимой}, если выполняется одно из следующих эквивалентных
условий:\\
1) существуют такие $\lambda_1,\dots,\lambda_m$, не все равные нулю,
что $\lambda_1 a_1 + \dots + \lambda_m a_m = 0$;\\
2) хотя бы один из векторов $a_1,\dots,a_m$ линейно выражается через
остальные.

\noindent Пустая система векторов считается линейно независимой.
\копр


\задача
Являются ли линейно независимыми векторы следующих множеств:\\
\пункт $\{(1, -1, 0), (-1, 0, 1), (0, 1, -1)\} \subset \R^3$; \пункт $\{(1,1,0),(1,0,1),(0,1,1)\} \subset \R^3$?
\кзадача

\пзадача Если система векторов $\{a_1,\dots,a_m\}$ линейно
независима, а система векторов $\{a_1,\dots,a_m,b\}$ линейно
зависима, то вектор $b$ линейно выражается через векторы
$a_1,\dots,a_m$.
\кзадача

\задача
В задаче 18,{\em в} предыдущего листка выберем из весов коров минимальное количество так, чтобы любой оставшийся вес линейно выражался над $\Q$ через выбранные. Докажите, что все равенства из условия задачи будут тогда выполняться покомпонентно (отдельно для коэффициентов при первом выбранном весе, отдельно --- при втором, ...) и получите ещё одно решение задачи про коров.
\кзадача


\опр Векторное пространство $V$ называется {\it бесконечномерным},
если в нем существуют линейно независимые системы из сколь угодно
большого числа векторов. В~противном случае пространство $V$
называется {\it конечномерным}. \копр

\задача Приведите примеры конечномерных и бесконечномерных векторных пространств. \кзадача

\опр {\it Базисом} (конечномерного) векторного пространства $V$
называется всякая линейно независимая система векторов, через
которую выражаются все векторы пространства $V$. \копр

\задача Докажите, что если $\{e_1,\dots,e_n\}$~--- базис
пространства $V$, то всякий вектор $x\in V$ однозначно выражается
через $e_1,\dots,e_n$. Коэффициенты этого выражения называются {\it
координатами} вектора~$x$ в базисе $\{e_1,\dots,e_n\}$. \кзадача

\опр Всякая система векторов (не обязательно линейно независимая),
через которую линейно выражаются все векторы пространства $V$,
называется {\it порождающей}. \копр

\пзадача Докажите, что в конечномерном векторном пространстве
\сНовойСтроки \пункт всякая линейно независимая система векторов
может быть дополнена до базиса (в частности, существует хотя бы один
базис); \пункт из всякой порождающей системы векторов можно выбрать
базис. \кзадача

\пзадача Докажите, что все базисы конечномерного векторного
пространства $V$ содержат одно и то же число векторов. \кзадача

\опр Число векторов в базисе конечномерного пространства $V$
называется \выд{размерностью} пространства $V$ и обозначается $\dim
V$. \копр

\задача Пусть $n$ --- размерность конечного поля $F$ характеристики $p$ как векторного пространства над полем $\Z/p\Z$. Сколько элементов в $F$? \кзадача

%\раздел{Линейность над $\Z_2$}

\задача \пункт На табло расположены лампочки. Есть несколько кнопок.
Каждая кнопка меняет состояние соединенных с ней лампочек. Докажите,
что число узоров, которые можно получить, нажимая на кнопки, есть
степень двойки. \спункт Пусть для любого набора лампочек существует
кнопка, соединенная с нечётным числом из них. Докажите, что все
лампочки можно погасить. \пункт Пусть лампочки образуют квадрат
$4\times 4$ и рядом с каждой лампочкой есть кнопка,
соединенная со всеми лампочками в том же столбце и в той же строке.
Как изменить состояние ровно одной лампочки? \кзадача

\задача \пункт Докажите, что в дереве нет непустых подграфов, у
которых степень каждой вершины четна. \пункт Пусть $a$~--- число
подграфов данного графа, у которых степень каждой вершины четна.
Докажите, что число $a$~--- степень двойки. \пункт На ребрах дерева 
стоят знаки $+$ и $-$. Разрешается менять знак на всех ребрах,
выходящих из одной вершины. Докажите, что из любого узора можно
получить любой другой. \пункт Пусть $b$~--- наибольшее количество
узоров на данном графе, ни один из которых нельзя получить из
другого операциями, описанными в предыдущем пункте. Докажите, что
число $b$~--- степень двойки. \спункт Докажите, что для любого графа
$a = b$. \кзадача



\задача \пункт Докажите, что через любые четыре точки плоскости
проходит бесконечное количество кривых второго порядка (линий, заданных уравнениями вида $ax^2 + bxy + cy^2 + dx + ey + f =
0$, где не все $a$, $b$, $c$ равны нулю). \пункт Докажите, что через
любые пять точек плоскости проходит хотя бы одна кривая второго
порядка. \пункт Существуют такие пять точек, через которые проходит
ровно одна кривая второго порядка; бесконечное количество кривых
второго порядка. \кзадача

	
	\ЛичныйКондуит{0mm}{5mm}
% \GenXMLW
	
\end{document}


\задача[Числа Фибоначчи] Пусть последовательности
$(a_n)$ и $(b_n)$ удовлетворяют соотношению $x_{n + 2} = x_{n + 1} + x_n$.
Докажите, что \\
\пункт при любом $\lambda\in\R$ последовательность
$\lambda (a_n) = (\lambda a_1,\lambda a_2,\dots)$
удовлетворяет этому соотношению; \\
\пункт при любых $\lambda,\mu \in\R$ последовательность $\lambda(a_n)
+ \mu(b_n) = (\lambda a_1 + \mu b_1,\lambda a_2 + \mu b_2,\dots)$
также удовлетворяет этому соотношению.\\
\пункт Пусть вектора $(a_1,a_2)$ и $(b_1,b_2)$ неколлинеарны.
Докажите, что любая последовательность, удовлетворяющая соотношению
пункта~а), представляется в виде $(\lambda a_n) +
(\mu b_n)$ при некоторых $\lambda,\mu \in \R$.\\
\пункт Найдите
геометрические прогрессии,  удовлетворяющие соотношению пункта~а).\\
\пункт Найдите явную формулу последовательности Фибоначчи с
начальными условиями $a_0 = a_1 = 1$. \кзадача

\задача Найдите явные формулы для следующих (вещественных)
последовательностей: \сНовойСтроки \пункт $a_{n + 3} = 2a_{n + 2} +
a_{n + 1} - 2a_n$, если $a_0 = 1$, $a_1 = 2$, $a_2 = 3$; \пункт
$a_{n + 2} = -2a_{n + 1} - a_n$, в случаях $a_0 = 1$, $a_1 = -1$;
$a_0 = 0$, $a_1 = -1$; $a_0 = 1$, $a_1 = 2$. \пункт $a_{n + 2} =
7a_{n + 1} - 12{,}5a_n$, если $a_0 = a_1 = 1$. \пункт $a_{n + 2} =
a_{n + 1} + a_n + 1$, если $a_0 = a_1 = 1$. \кзадача

\задача Опишите все решения рекуррентного уравнения $a_{n + k} =
c_{k - 1}a_{n + k - 1} + \cdots + c_0a_n + c$, где $c_0,\dots,c_{k -
1} \in \R$, при любых начальных условиях, если \вСтрочку \пункт $c =
0$; \пункт $c \in \R$. \кзадача
