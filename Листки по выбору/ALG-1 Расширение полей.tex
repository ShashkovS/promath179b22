% !TEX encoding = Windows Cyrillic
\documentclass[a4paper, 12pt]{article}
\usepackage[tikz]{newlistok}

% \УвеличитьШирину{1.5truecm}
% \УвеличитьВысоту{2.5truecm}

\begin{document}

\Заголовок{Расширения полей I: Алгебраические числа}
\НомерЛистка{ALG-1}
\ДатаЛистка{2022.01}
%\Оценки{99/99/99}
\СоздатьЗаголовок


% rel 10.12.2011
% не продвинулся серьезно никто, кажется
\задача
Приведите пример ненулевого многочлена с~рациональным коэффициентами, корнем которого является
\пункт $1+\sqrt[3]2$;\quad
\пункт $\sqrt2+\sqrt3$;\quad
\спункт $\sqrt[3]2+\sqrt3$;\quad
\спункт $(1+\sqrt[3]2)\sqrt3$.
\кзадача






\опр
Действительное число называется \emph{алгебраическим}, если оно является корнем ненулевого многочлена с~рациональными коэффициентами, и~\emph{трансцендентным} в~противном случае.
\копр




\сзадача
\пункт Трансцендентные числа существуют.

\спункт Приведите конкретный пример трансцендентного числа.
\кзадача






\сзадача
Алгебраические числа образуют поле.
\кзадача






\опр
\emph{Минимальным многочленом} алгебраического числа~$\alpha$ называется неприводимый многочлен $m_\alpha\in\Q[x]$, такой что $m_\alpha(\alpha)=0$. \emph{Степенью} алгебраического числа называется степень его минимального многочлена.
\копр




\задача
\пункт Любое алгебраическое число степени~2 может быть представлено в~виде $a\pm\sqrt d$, где числа $a$ и~$d$ рациональные. (Верно~ли аналогичное утверждение для алгебраических чисел степени~4?)

\пункт Если $\alpha=a+\sqrt d$ (числа $a$ и~$d$ рациональные), то $m_\alpha=(x-\alpha)(x-\bar\alpha)$, где $\bar\alpha=a-\sqrt d$.
\кзадача






\задача
\пункт $\{P\in\Q[x] : P(\alpha)=0\}=(m_\alpha)$.

\пункт Минимальный многочлен алгебраического числа~$\alpha$ существует и~единственен (с~точностью до умножения на ненулевую константу).
\кзадача






\задача
Если $\alpha$~--- алгебраическое действительное число, то внутри действительных чисел есть подполе~$\Q(\alpha)$, изоморфное полю~$\Q[x]/(m_\alpha)$.
\кзадача






\опр
Пусть $L$ поле, $K$ его подполе (<<$L/K$\footnote{Читается <<$L$ над $K$>>, не путать с~фактором.}~--- расширение полей>>). Говорят, что элемент поля $L$ \emph{алгебраичен} над $K$, если он является корнем ненулевого многочлена с~коэффициентами в~$K$. (Таким образом, выше шла речь об алгебраических элементах в~расширении $\R/\Q$.)

Расширение $L/K$ называется \emph{алгебраическим}, если любой его элемент алгебраичен.
\копр




\задача
Любое конечное поле характеристики~$p$ является алгебраическим расширением поля~$\mathbb F_p$.
\кзадача






\задача
Любое расширение конечных полей получается последовательностью расширений вида $K\subset L\cong K[x]/(P)$.
\кзадача

\задача
Если конечное поле имеет характеристику~$p$, то количество элементов в~нем является степенью числа~$p$.
\кзадача






\задача
Для любого поля~$K$ и~любого многочлена~$P$ над этим полем найдется расширение, в~котором многочлен~$P$
\пункт имеет корень; \пункт раскладывается на линейные множители.
\кзадача






\задача
\пункт Если~$L$~--- поле из $q=p^n$ элементов, то любой его элемент является корнем многочлена $x^q-x$.

\пункт Для любого $q$ вида $p^n$ существует поле из~$q$ элементов.

\спункт Единственно~ли такое поле?
\кзадача











\ЛичныйКондуит{0mm}{6.5mm}
% \GenXMLW
\end{document} 