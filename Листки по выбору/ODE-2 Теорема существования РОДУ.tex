% !TEX encoding = Windows Cyrillic
\documentclass[a4paper, 12pt]{article}
\usepackage{newlistok}

\УвеличитьШирину{1.5truecm}
\УвеличитьВысоту{2.5truecm}

\newcommand{\cF}{\ensuremath{\mathcal F}}
\newcommand{\cM}{\ensuremath{\mathcal M}}
\newcommand{\uF}{\text{$\hbox{$\cF$}\mkern6mu\raise2pt
                 \hbox{$\hbox{}^\uparrow$}$}}
\newcommand{\dF}{\text{$\hbox{$\cF$}\mkern-5mu\raise-2pt
                 \hbox{$\hbox{}_\downarrow$}$}}
                 \def\bl{\!\in\!}
\def\fain#1#2{\;\,\forall\,{#1}\bl{#2}\;}
\def\bydef{\stackrel{\rm def}{=}}


\sloppy

\begin{document}

\Заголовок{Теорема существования РОДУ}
\НомерЛистка{ODE-2}
\ДатаЛистка{2022.01}
%\Оценки{99/99/99}
\СоздатьЗаголовок









\опр[Постановка задачи.]
  На плоскости $XOY$ заданы: прямоугольник
  $$П=\{\,(x,y)\,|\;a\le x\le b\,,\;c\le y\le d\,\}\;,
  $$
  точка $(x_0,y_0)$, лежащая строго внутри него, и дифференциальное
  уравнение $y'=F(x,y)$, правая часть которого $П\to^F\R$ является
  непрерывной функцией на $П$. Мы докажем, что существует
  $\ep$-окрестность $U_\ep$ точки $x_0$ и дифференцируемая функция
  ${U_\ep}{\to^f}{[c,d]}$, такие что $f'(x)=F(x,f(x))$ $\fain x{U_\ep}$
  и $f(x_0)=y_0$.
\копр





\задача
  Зададимся некоторой $\ep$-окрестностью $U_\ep$ точки
  $x_0$ и рассмотрим следующие два множества дифференцируемых функций
  ${U_\ep}{\to^{\phi}}{[c,d]}$, заданных на этой окрестности:
  \begin{align*}
      \uF&\bydef\{\,\phi\,|\;\fain x{U_\ep}\;\phi'(x)>F(x,\phi(x))\,\}\\
      \dF\;&\bydef\{\,\phi\,|\;\fain x{U_\ep}\;\phi'(x)<F(x,\phi(x))\,\}
  \end{align*}
  Докажите, что $\exists\;\ep$ : оба множества $\uF$, $\dF$
  непусты, и справа от $x_0$ график любой функции из $\uF$ лежит
  выше графика любой функции из $\dF$, а слева --- наоборот.
\кзадача





\задача
  Определим функцию $f(x)$ справа от $x$ как
  $\inf\limits_{\phi\in\,\uF}\phi(x)$, а слева от $x$ как
  $\sup\limits_{\phi\in\,\uF}\phi(x)$. Докажите, что $f$ существует,
  непрерывна, дифференцируема и удовлетворяет уравнению $y'=F(x,y)$.
\кзадача





\опр[Обозначения.]
  Пусть $C=\sup\limits_П|F(x,y)|$. Обозначим через $D_\de$ отрезок
  $[x_0-\de,x_0+\de]$, где $\de$ выбрано так, чтобы \лк бабочка\пк\
  $B_\de\bydef\{\,(x,y)\,|\;x\in D_\de,|y-y_0|\le C|x-x_0|\}$ лежала целиком
  внутри $П$. Обозначим через $\cM_\de$ множество всех непрерывных
  функций ${D_\de}{\to^{\phi}}{[c,d]}$, график которых содержится в $B_d$.
\копр





\задача
  Докажите, что $\cM_\de$ является полным метрическим пространством
  с расстоянием
  $\rho(\phi,\psi)=\sup\limits_{x\in D_\de}|\phi(x)-\psi(x)|$.
\кзадача





\задача[лемма Асколи--Арцела]
  Дано некоторое множество $\cF$ непрерывных функций на отрезке.
  Докажите, что любая ограниченная последовательность функций из $\cF$
  содержит поточечно сходящуюся подпоследовательность\footnote
{предел которой не обязан принадлежать $\cF$}
  тогда и только тогда, когда все функции в $\cF$ ограничены общей
  константой и {\it в равной степени\/} непрерывны (\те
  $\forall\;\ep>0$ $\exists\;\de>0$ : $|x_1-x_2|<\de$ $\Rightarrow$
  $|\phi(x_1)-\phi(x_2)|<\ep$ сразу для всех $\phi\in\cF$\;).
\кзадача





\задача[ломаные Эйлера]\label{lomeul}
  Разобъём $D_\de$ на $2n$ равных частей длины $h=\de/n$ и определим
  непрерывную функцию $\phi_n(x)$, полагая $\phi(x_0)=y_0$, и далее
  продолжая её влево и вправо индуктивным правилом: над отрезком
  $[x_0+kh,x_0+(k+1)h]$ (где $k=0,1,2,\dots$) и над отрезком
  $[x_0+(k-1)h,x_0+kh]$ (где $k=0,-1,-2,\dots$) $\phi(x)$ есть прямая с
  угловым коэффициентом $F(x_0+kh,\phi(x_0+kh))$ (значение $\phi(x_0+kh)$
  определено по индуктивному предположению). Докажите, что все $\phi_n$
  лежат в $\cM_\de$ и из них можно выбрать подпоследовательность,
  имеющую поточечный предел, также лежащий в $\cM_\de$.
\кзадача





\задача
  Явно опишите последовательность ломаных Эйлера для уравнения $y'=y$ с
  начальным условием $y(0)=1$ и шагом $h=1/n$, и честно найдите её
  предел при $n\to\infty$.
\кзадача





\задача
  Докажите, что поточечный предел любой сходящейся последовательности
  ломаных Эйлера из \ref{lomeul} является дифференцируемой функцией,
  удовлетворяющий уравнению $y'=F(x,y)$ (мы ещё вернёмся к этой задаче
  в следующем листке).
\кзадача






\ЛичныйКондуит{0mm}{6.5mm}
% \GenXMLW
\end{document}
