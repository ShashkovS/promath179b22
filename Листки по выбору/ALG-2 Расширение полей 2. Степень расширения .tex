% !TEX encoding = Windows Cyrillic
\documentclass[a4paper, 12pt]{article}
\usepackage[tikz]{newlistok}

\УвеличитьШирину{1.5truecm}
\УвеличитьВысоту{2.5truecm}

\begin{document}

\Заголовок{Расширения полей II: Степень расширения}
\НомерЛистка{ALG-2}
\ДатаЛистка{2022.01}
%\Оценки{99/99/99}
\СоздатьЗаголовок


\опр
Пусть $L/K$~--- расширение полей (т.\,е. $K$~--- подполе поля~$L$). Тогда $L$ можно рассматривать как векторное пространство над $K$. Размерность $[L:K]$ этого пространства называется \emph{степенью расширения}.
%
Расширение, имеющее конечную степень, называется \emph{конечным}.
\копр




%
\задача
Чему равна \пункт степень $[\Cbb:\R]$; \пункт степень $[\mathbb F_4:\mathbb F_2]$?
\кзадача






\задача
\пункт Если поле из $p$~элементов вложено в~поле из $q$ элементов, то число~$q$~--- степень числа $p$.
\quad
\пункт Количество элементов конечного поля~--- степень простого числа.
\кзадача






\задача
\пункт Расширение $K(\sqrt d)/K$ имеет степень 2.

\пункт Если $P$~--- неприводимый многочлен степени~$n$, то $[K[x]/(P):K]=n$.
\кзадача






\задача
\пункт Если есть башня из трех полей $F\subset K\subset L$, то $[L:F]=[L:K]\cdot [K:F]$.

\пункт Если $L/F$~--- расширение полей степени~$n$, то степень любого промежуточного расширения $K/F$ делит число~$n$.
\кзадача






\задача
Найдите\quad
\пункт $[\Q(\sqrt2,\sqrt3):\Q(\sqrt3)]$;\quad
\пункт $[\Q(\sqrt2,\sqrt3):\Q]$;\quad
\пункт $[\Q(\sqrt2+\sqrt3):\Q]$.
\кзадача






% Как-нибудь про независимость радикалов?
\опр%
Пусть на плоскости введена система координат. Будем сопоставлять каждому набору $\mathcal K$ точек подполе~$K$ действительных чисел, порожденное всеми координатами этих точек.
\копр




\задача
Коэффициенты уравнения\\
\пункт прямой, проходящей через пару точек из $\mathcal K$;\\
\пункт окружности с~центром в~точке из $\mathcal K$ и~проходящей через точку из $\mathcal K$\\
лежат в~$K$.
\кзадача






\задача
Пусть $\mathcal L$ получается из $\mathcal K$ добавлением точки пересечения\\
\пункт двух прямых; \пункт прямой и~окружности; \пункт двух окружностей с~коэффициентами из~$K$.\\
Чему может равняться степень расширения $L/K$?
\кзадача






\задача
Если число $\alpha$ можно получить из элементов поля $K\subset\R$ при помощи циркуля и~линейки, то $[K(\alpha):K]$~--- степень двойки.
% Переписать подробнее?
\кзадача






\задача
Циркулем и~линейкой нельзя построить отрезок в~$\sqrt[3]2$ длиннее данного (то есть задача об удвоении куба не имеет решения).
\кзадача






\задача
Найдите минимальный многочлен числа
\пункт $\cos\smash{\dfrac\pi9}$; \пункт $\cos\smash{\dfrac\pi5}$; \спункт $\cos\smash{\dfrac\pi7}$.

\noindent
\small{\textsc{Указание.} Используйте равенства вида $\cos n\varphi=\cos m\varphi$.}
\кзадача






\задача
Задача о~трисекции угла не имеет решения.
\кзадача







\задача
\пункт Конечное расширение алгебраично\footnote{Определение можно найти в~листке <<Расширения полей~I>>.}. (Верно ли обратное?)

\пункт Если расширение порождено (как поле) конечным набором алгебраических элементов, то оно конечно и~его степень не превосходит произведения степеней этих элементов.
\кзадача






\задача
Если $L/K$~--- произвольное расширение, то множество его элементов, алгебраичных над $K$, образует поле (в~частности, алгебраические числа образуют поле).
\кзадача











 \ЛичныйКондуит{0mm}{6.5mm}
% \GenXMLW
\end{document} 