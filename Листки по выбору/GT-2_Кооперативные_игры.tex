% !TEX encoding = Windows Cyrillic
\documentclass[a4paper,12pt]{article}
\usepackage[mag=980]{newlistok}

\ВключитьКолонтитул
\УвеличитьШирину{1.7cm}
\УвеличитьВысоту{2.5cm}
\renewcommand{\spacer}{\vspace{1pt}}
%\vspace*{1.0cm}


\begin{document}

\НомерЛистка{GT-2}
\Заголовок{Кооперативные игры}
\ДатаЛистка{2022.01}

\СоздатьЗаголовок
%{\footnotesize
\раздел{Постановка задачи}

Предположим, что имеется $N$ игроков. Для удобства занумеруем их числами $1,2,\ldots,N$. Подмножество $S \subset \{1,\ldots,N\}$  будем называть
\textit{коалицией} игроков.

\выд{Коалиционная игра в характеристической форме} (coalitional game или cooperative game in characteristic form) --- это множество игроков
$\{1,\ldots,N\}$ с заданной характеристической функцией $v : 2^N \to \R$. Для краткости будем называть такую игру <<игра, заданная функцией $v$>> или просто <<игра $v$>>.

Для  коалиции $S$ значение $v(S)$ обозначает сумму денег, которую эта коалиция
может заработать самостоятельно. Характеристическая функция может принимать отрицательные значения (например, при дележе расходов). Будем считать, что пустая коалиция (куда никто не входит), не может заработать денег и никому ничего не должна, т.е. $v(\es)=0$.

В такой игре у игроков есть интерес в создании большой коалиции и дележе
полученного $v(N)$. Вопрос в том, как поделить $v(N)$? Мы обсудим две концепции решения --- устойчивые решения (т.н. \выд{ядро}) и справедливое решение
(т.н. \выд{вектор Шепли}).

\раздел{Ядро игры}

Предположим, что большая коалиция решила каким-то образом разделить $v(N)$. Такой делёж устойчив, если никакая коалиция $S \subset N$ не захочет отделиться и более выгодным для себя способом переделить имеющийся у них запас средств $v(S)$.

\опр Будем называть {\it дележом} игры, заданной функцией $v$ любой набор $(x_1,\ldots,x_N) \in \R^N$, для которого $x_1+\ldots+x_N = v(N)$.

{\it Ядерным дележом}  или {\it устойчивым дележом} называется такой делёж игры $v$, что для любой непустой коалиции $S \subset \{1,\ldots,N\}$  выполнено неравенство $\sum_{i \in S}x_i \geq S$.

{\it Ядром игры} $v$ называется множество устойчивых дележей.
\копр

%}
{\bf В  Задачах \ref{100baksov}-\ref{pirog} необходимо задать характеристическую функцию и найти ядро игры.}

\задача[Делёж богатства]\label{100baksov} Трое нашли на дороге $100$ долларов. Деньги можно делить произвольным образом между игроками. Решение о дележе принимается большинством, то есть если двое договорились, то третий не может возразить.
\кзадача

\задача  (Простая игра или модель голосования)
 Предположим, что есть $N$ игроков (членов парламента) и задана {\it ключевая коалиция} $K$, то есть такая, что для принятия закона необходимо и достаточно того, чтобы все члены коалиции $K$ проголосовали за закон. То есть можно положить $v(S) = \begin{cases} 1, \text{ если } K \subset S \\ 0, \text{ иначе.} \end{cases}$.
\кзадача


\задача[Музыканты]  Оркестр из трех музыкантов
($A$,~$B$,~$C$) играет в подземном переходе. Поодиночке они
могли бы заработать, соответственно, $6$, $18$ и $30$
рублей в час. Играя по двое, они бы получили:
$A$~и~$B$~---~36, $A$~и~$C$~---~48 , $B$~и~$C$~---~54 рубля в час.
А~вместе они имеют~$72$ рубля в час.
\кзадача

\задача[Простая модель рынка] Имеется множество продавцов $K$ и множество покупателей $L$. Каждый продавец имеет 1 единицу товара, каждый из покупателей хочет купить 1 единицу товара. Выигрыш коалиции $S \subset K\cup L$ равен количеству проданных товаров.
\кзадача

\задача[Семейные вечера]
По вечерам семья любит слушать музыку. Мама обожает Аллу Пугачеву (точка 0 на прямой), сын больше всего любит группу Чайф (точка 0.5 на той же прямой), а папа тащится от Шнура из группы <<Ленинград>> (точка 1 на прямой). Удовольствие от музыки равно 1 минус расстояние от нее до своей любимой музыки.
Установка воспроизводящего устройства требует одну секунду издержек. Любая пара членов семьи или любой член семьи в отдельности могут уйти в одну из свободных комнат, самостоятельно установить проигрывающее устройство и слушать музыку там. То есть выигрыш коалиции --- это суммарное удовольствие от прослушивания музыки (естественно, мы выбираем музыку так, чтобы максимизировать это удовольствие) за вычетом издержек на установку проигрывающего устройства. 
\кзадача

\задача[Делёж пирога]\label{pirog} Остался последний кусочек пирога. Три брата не знают, как его поделить. Папа спрашивает: <<Сколько стоил пирог в магазине?>> Сначала отвечает старший сын, затем средний, затем младший. Называть одинаковые цены нельзя. Для простоты будем считать, что деньги бесконечно делимы, а стоимость пирога --- случайная величина, равномерно распределенная на отрезке $[0;1]$. Тот из братьев, чья версия ближе всего к правильной, получает пирог (если таких двое, кусок делится пополам). Будем считать, что цена коалиции --- это супремум достижимых коалицией платежей.
\кзадача

\задача \пункт Может ли ядро быть пустым?
\пункт Докажите, что если ядро непусто, то оно является выпуклым множеством.
\пункт Пусть $N=3$. Может ли ядро быть точкой? отрезком? треугольником? 4-угольником? 6-угольником? Приведите примеры в каждом из случаев.
\кзадача

\newpage
\раздел{Вектор Шепли}

С каждой перестановкой игроков $\tau: \{1,\dots,n\} \to  \{1,\dots,n\}$ свяжем следующий делёж $x^{\tau}$: игрок $\tau(i)$ получает полезность $v(\tau\{1,\dots,i\})-v(\tau\{1,\dots,i-1\})$. Другими словами, игрок получает тот прирост, который он привносит своим присоединением к коалиции предыдущих игроков.

\textit{Вектором Шепли} игры $v$ называется вектор $\Phi(v) \in \R^N$, задаваемый формулой: $$\Phi(v) = \frac{1}{n!} \sum_{\tau} x^{\tau}.$$

\задача\label{Shapl:examp} Найдите вектор Шепли для игры \пункт делёж богатства; \пункт простой игры с ключевой коалицией $K$; \пункт музыканты; \пункт семейные вечера; \пункт делёж  пирога. \пункт В каких пунктах вектор Шепли лежит в ядре?
\кзадача


\задача Докажите, что вектор Шепли $\Phi(v)$ игры $v$ обладает следующими свойствами:

(1) ({\it эффективность}) $\sum_{i=1}^N \Phi(v)_i = v(N)$.

(2) ({\it симметричность}) Для любой перестановки $\sigma : N \to N$ выполнено $\Phi[\sigma v] = \sigma \Phi[v]$.

(3) ({\it линейность})   Множество игр в характеристической форме имеет структуру векторного пространства с естественными операциями: сумма игр определяется как  $(v+w)(S) = v(S)+w(S)$, умножение на коэффициент как $(\alpha v)(S) = \alpha v(S)$.  Линейность означает линейность вектора Шепли как оператора на этом пространстве, то есть для любых игр $v_1,v_2,v$ и коэффициента $\alpha \in \R$ выполнены равенства

$\Phi(v_1) + \Phi(v_2) = \Phi(v_1+v_2)$; $\alpha \Phi(v) = \Phi(\alpha v)$.

(4) ({\it болваны не получают ничего}) Игрок $i$ называется \textit{болваном}, если для любой коалиции $K$ выполнено $v(K \cup \{i\}) = v(K)$. Для любого болвана соответствующая координата вектора Шепли равна $0$.
\кзадача

\сзадача Пусть есть \textit{правило} (нахождения справедливого дележа), то есть отображение $\Phi$ из множества игр (с побочными платежами и множеством игроков $N$) в $\R^N$; значение $\Phi$ на игре $v$ обозначим через $\Phi(v)$.
Докажите, что если для $\Phi$ выполнены свойства (1)-(4), то $\Phi(v)$ --- вектор Шепли.
\кзадача

Из последних двух задач следует то, что вектор Шепли --- разумный способ определить <<справедливый>> делёж в игре $v$. А если вектор Шепли ещё и лежит в ядре, то мы с уверенностью можем прогнозировать, что реализуется именно этот делёж. Однако, как показывает задача \ref{Shapl:examp}, вектор Шепли не всегда лежит в ядре, даже когда ядро непусто. Определим класс игр, для которых <<всё хорошо>>: и ядро непусто, и вектор Шепли там лежит. Более того, ядро образует многогранник, центром которого и является вектор Шепли.


Игру $v$ будем называть \textit{супемодулярной} или \text{выпуклой}, если для любых двух коалиций $K$ и $K'$ выполняется неравенство $v(K)+v(K') \leq v(K\cap K') + v(K \cup K')$.

\задача Докажите, что игра $v$ супермодулярна тогда и только тогда, когда для любых коалиций $K \subset L$ и игрока $i$ выполнено неравенство $v(K\sqcup {i}) - v(K) \leq v(L \sqcup {i})-v(L)$.
\кзадача

\задача Какие из игр \ref{100baksov}-\ref{pirog} являются супермодулярными?
\кзадача


\задача Илья Муромец, Алеша Попович и Добрыня Никитич охотятся на Змеев-Горынычей. В одиночку никто из них не может одолеть ни одного Змея-Горыныча. Втроем --- могут одолеть одного Змея-Горыныча за час, вдвоем --- одолевают $\alpha\in (0;1)$ Змеев-Горынычей в час.

\пункт Найдите ядро и вектор Шепли в зависимости от $\alpha$.
\пункт При каких $\alpha$ игра будет супермодулярной?
\кзадача

\задача От шоссе до деревни Малое Гадюкино идет грунтовая дорога. Осенью дорога приходит в ужасное состояние, поэтому Малые Гадюкинцы на общем собрании решили заасфальтировать ее. При распределении затрат необходимо учесть тот факт, что деревня растянута вдоль дороги, и фактически Гадюкинцы живут на разных расстояниях от шоссе.

Всего в Малом Гадюкино обитает  $n$  семей, на расстояниях от шоссе, равных  $x_{1}$,  $x_{2}$,\ldots  $x_{n}$ метров. За 1 рубль можно заасфальтировать 1 метр.
\\\пункт Сформулируйте данную игру как игру в характеристической форме, найдите в ней вектор Шепли.
\пункт Будет ли игра супермодулярной?
\кзадача



\задача \пункт Докажите, что вектор $x^\tau$ из определения вектора Шепли является ядерным дележом в случае супермодулярной игры.
\пункт [Теорема Шепли] Докажите, что если игра $v$ супермодулярна, то ядро непусто, а вектор Шепли лежит в ядре.
\спункт Докажите, что ядро состоит из выпуклых линейных комбинаций $x^\tau$. Выведите отсюда, что вектор Шепли является центром ядра.
\кзадача

\ЛичныйКондуит{0mm}{5.5mm}
% \GenXMLW


\end{document}





Критерий непустоты ядра. Назовём набор неотрицательных чисел $\{\lambda_K\}$ по непустым подмножествам $K \subset N$ \textit{сбалансированным}, если для каждого $i$ выполнено равенство $\sum_{K: i \in K} \lambda_K=1$.

\задача[Теорема Бондаревой-Шепли] Ядро игры непусто тогда и только тогда, когда для любого сбалансированного набора $\{\lambda_K\}$ выполнено неравенство $$\sum_{S \in 2^N \setminus \{\es\}}\lambda_S v(S) \leq v(N).$$
\кзадача


\раздел{Вектор Шепли}

С каждой нумерации игроков $\tau: \{1,\dots,n\} \to N$ свяжем следующий делёж
$x^{\tau}$: игрок $\tau(i)$ получает полезность $v(\tau\{1,\dots,i\})-v(\tau\{1,\dots,i-1\})$. Другими словами, игрок получает тот прирост, который он привносит своим присоединением к коалиции предыдущих игроков.

\textit{Вектором Шепли} игры $v$ называется вектор $\Phi(v) \in \R^N$, задаваемый формулой: $$\Phi(v) = \frac{1}{n!} \sum_{\tau} x^{\tau}.$$

\задача Докажите, что вектор Шепли $\Phi(v)$ игры $v$ обладает следующими свойствами:

(1) ({\it эффективность}) $\sum_{i=1}^N \Phi(v)_i = v(N)$.

(2) ({\it симметричность}) Для любой перестановки $\sigma : N \to N$ выполнено $\Phi[\sigma v] = \sigma \Phi[v]$.

(3) ({\it линейность}) $\Phi(v_1) + \Phi(v_2) = \Phi(v_1+v_2)$; $\alpha \Phi(v) = \Phi(\alpha v)$.

(4) ({\it болваны не получают ничего}) Игрок $i$ называется \textit{болваном}, если для любой коалиции $K$ выполнено $v(K \cup \{i\}) = v(K)$. Для любого болвана соответствующая координата вектора Шепли равна $0$.
\кзадача

\задача Пусть есть \textit{правило} (нахождения справедливого дележа), то есть отображение $\Phi$ из множества игр (с побочными платежами и множеством игроков $N$) в $\R^N$; значение $\Phi$ на игре $v$ обозначим через $\Phi(v)$.
Если для $\Phi$ выполнены свойства (1)-(4), то $\Phi(v)$ --- вектор Шепли.
\кзадача





\end{document} 