% !TeX encoding = windows-1251
\documentclass[a4paper,12pt]{article}
\usepackage{newlistok}

% \УвеличитьШирину{1.5cm}
%\renewcommand{\spacer}{\vfill}

\ВключитьКолонитул
\Заголовок{Метрические пространства: непрерывность}
\НомерЛистка{MS-3}
\ДатаЛистка{2022.01}


\begin{document}
\СоздатьЗаголовок

\noindent
Всюду в этом листке, где упоминается пространство $\R^n$, имеется в виду, что оно снабжено евклидовой метрикой $d_2$.

\noindent
Под словом \лк функция\пк подразумевается отображение в $\R$.

\опр
Точка $a$ метрического пространства $M$ называется \emph{предельной точкой} множества $X \subset M$, если в любой $\ep$-окрестности точки $a$ найдётся точка из $X$.
\копр

\опр
Подмножество $U$ метрического пространства $M$ называется \emph{открытым}, если вместе с каждой своей точкой оно содержит какую-нибудь её $\ep$-окрестность.
\копр

\опр
Подмножество $B$ метрического пространства $M$ называется \emph{замкнутым}, если оно содержит все свои предельные точки.
\копр

\взадача
Докажите, что $U \subset M$ открыто тогда и только тогда, когда $M \setminus U$ замкнуто.
\кзадача

\задача
Пусть $M$ снабжено дискретной метрикой. Опишите все его открытые подмножества.
\кзадача

\задача
Множество $X$ на плоскости обладает таким свойством, что его пересечение с любой прямой есть открытое подмножество этой прямой. Обязательно ли $X$ открытое? Тот же вопрос, если все слова \лк открытое\пк заменить на \лк замкнутое\пк.
\кзадача

\опр
\label{cont1}
Отображение $f \from M \to N$ \выд{непрерывно в точке $m \in M$}, если для любой последовательности $(x_i)$, сходящейся к $m$, последовательность $(f(x_i))$ сходится к $f(m)$. Если $f$ непрерывно во всех точках множества $M$, то говорят, что $f$ \выд{непрерывно на $M$}.
\копр

\опр
\label{cont2}
Отображение $f \from M \to N$ \выд{непрерывно на $M$} (или просто \emph{непрерывно}), если прообраз любого открытого множества открыт.
\копр

\ввзадача
Докажите эквивалентность определений \ref{cont1} и \ref{cont2}.
\кзадача

\задача
Рассмотрим на $\R^2$ функции вычисления суммы, разности, произведения и частного координат. Докажите, что они непрерывны на своей области определения.
\кзадача

\взадача
Докажите, что композиция непрерывных отображений непрерывна.
\кзадача

\задача
Докажите, что сумма и произведение непрерывных функций непрерывны.
\кзадача

\задача
Докажите, что отображение непрерывно тогда и только тогда, когда прообраз любого замкнутого множества замкнут.
\кзадача

\задача
Верно ли, что при непрерывном отображении открытые множества переходят в открытые? А замкнутые в замкнутые?
\кзадача

\задача
Пусть пространство $M$ таково, что для любого метрического пространства $N$ любое отображение $f \from M \to N$ непрерывно. Что можно сказать об $M$?
\кзадача

\задача
Пусть пространство $N$ таково, что для любого метрического пространства $M$ любое отображение $f \from M \to N$ непрерывно. Что можно сказать об $N$?
\кзадача

\ЛичныйКондуит{-0.3mm}{6mm}
\ОбнулитьКондуит
\newpage

\опр
Множество $X$ называется \выд связным, если из того, что $X$~принадлежит объединению двух открытых непересекающихся множеств, следует, что оно принадлежит одному из~этих множеств.
\копр

\опр
Множество $X$ называется \выд линейно-связным, если для любых двух его точек $x_0$ и $x_1$ существует путь из $x_0$ в $x_1$ (то есть непрерывное отображение $f \from [0, 1] \to X$ такое, что $f(0) = x_0$ и $f(1) = x_1$).
\копр

\взадача
Докажите, что образ связного множества при непрерывном отображении связен.
\кзадача

\взадача
Докажите, что образ линейно-связного множества при непрерывном отображении линейно-связен.
\кзадача

\задача
Верно ли, что прообраз связного множества при непрерывном отображении связен?
\кзадача

\задача
Докажите, что если множество линейно-связно, то оно связно.
\кзадача

\задача
Пусть $U \subset \R^n$ открыто и связно. Докажите, что оно линейно-связно.
\кзадача

\задача[задача-шутка]
Множество $X$ делит плоскость на две части (то есть его дополнение является несвязным объединением двух связных множеств). Обязательно ли $X$ связно?
\кзадача

\сзадача
Приведите пример связного, но не линейно-связного подмножества в $\R^n$ для какого-нибудь $n$.
\кзадача

\задача
Пусть $f \from M \to N$ непрерывное взаимно-однозначное отображение. Верно ли, что~$f^{-1}$ тоже непрерывно?
\кзадача

\опр
Непрерывное взаимно-однозначное отображение $f \from M \to N$ называется \выд гомеоморфизмом, если отображение $f^{-1}$ непрерывно. В этом случае говорят, что $M$ \выд гомеоморфно $N$ (обозначение: $M \cong N$).
\копр

\задача
Какие из следующих пар множеств гомеоморфны между собой:\\
\пункт
прямая и парабола;
\пункт
прямая и гипербола;
\пункт
прямая и интервал;\\
% \пункт
% открытый круг и открытый квадрат;
\пункт
открытый круг и плоскость;
\пункт
сфера с выколотой точкой и плоскость;\\
\пункт
интервал и отрезок;
\пункт
прямая и окружность;
\пункт
прямая и плоскость?
\кзадача

\сзадача
Пусть множества $M$ и $N$ таковы, что существуют непрерывное взаимно-однозначное отображение $f \from M \to N$ и непрерывное взаимно-однозначное отображение $g \from N \to M$. Верно~ли, что $M \cong N$?
\кзадача

\опр
Множество называется \emph{компактным} (или просто \emph{компактом}), если из любого его покрытия открытыми множествами можно выделить конечное подпокрытие.
\копр

\взадача
Докажите, что компактное множество замкнуто и ограничено. Верно ли обратное?
\кзадача

\ввзадача
Докажите, что образ компакта при непрерывном отображении\т компакт.
\кзадача

\задача
Докажите, что непрерывная функция достигает на компакте своего максимума и минимума.
\кзадача

\задача
Выполняется ли принцип вложенных компактов для произвольного метрического пространства?
\кзадача

\задача
Известно, что $f \from [0, 1] \to M$ непрерывно и взаимно-однозначно. Докажите, что $f$\т гомеоморфизм.
\кзадача


\ЛичныйКондуит{-0.3mm}{6mm}

% \GenXMLW


\end{document}
