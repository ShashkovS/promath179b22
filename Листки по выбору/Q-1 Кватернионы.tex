% !TEX encoding = Windows Cyrillic
\documentclass[a4paper, 12pt]{article}
\usepackage[tikz]{newlistok}

\УвеличитьШирину{1.5truecm}
\УвеличитьВысоту{2.5truecm}
\ВключитьКолонтитул

\sloppy
\def\mtrx#1#2#3#4{\begin{pmatrix}#1 & #2 \\ #3 & #4\end{pmatrix}}

\begin{document}

\Заголовок{Кватернионы и вращения}
\НомерЛистка{Q-1}
\ДатаЛистка{2022.01}
%\Оценки{99/99/99}

\СоздатьЗаголовок


\опр
	\emph{Алгеброй} (более точно, алгеброй над $\mathbb{R}$) называется множество $A$, содержащее в себе множество $\mathbb{R}$, на котором заданы операции сложения и умножения, удовлетворяющие следйющим свойствам:
	\begin{itemize}
		\item $A$ является абелевой группой по сложению, в которой $0 \in \mathbb{R} \subseteq A$ выступает нулём.
		\item Умножение дистрибутивно относительно сложения, а элемент $1 \in \mathbb{R} \subseteq A$ выступает единицей.
		\item Операции над вещественными числами в $A$ такие же, как обычно, и $at=ta$ для всех $a \in A$, $t \in \mathbb{R}$.
	\end{itemize}
	Алгебра называется ассоциативной (коммутативной), если умножение в ней ассоциативно (коммутативно). Алгебра называется алгеброй с делением, если для любых $a,b \in A, a \neq 0$ существуют и единственны левые и правые частные, то есть такие элементы $b/a$ и $a \setminus b$, что $(b/a)a=b$ и $a(a \setminus b) =b$.
\копр

\задача
	\пункт Докажите, что всякая алгебра является векторным пространством над $\mathbb{R}$, если задать умножение на число с помощью умножения в алгебре. \пункт Придумайте какую-нибудь двумерную алгебру, кроме $\mathbb{C}$. \пункт Придумайте алгебру произвольной конечной размерности.
\кзадача

\опр
	\emph{Кватернионы} -- это выражения вида $a+bi+cj+dk$, $a,b,c,d \in \mathbb{R}$ которые складываются покоординатно, а умножаются по следующей таблице:
	\[
	\begin{tabular}{c|cccc}%{>{$}l<{$}|*{4}{>{$}l<{$}}}
		   & 1   & $i$   & $j$ & $k$   \\
		\hline\vrule height 12pt width 0pt
		1   & 1   & $i$   & $j$ & $k$     \\
		$i$   & $i$   & $-1$ & $k$   & $-j$    \\
		$j$ & $j$ & $-k$   & $-1$  & $i$    \\
		$k$  & $k$  & $j$   & $-i$   & $-1$   \\
	\end{tabular}
	\]
	Запомнить её можно, например, так: расположите мнимые единицы $i \to j \to k \to i$ по кругу, тогда произведение каждой мнимой единицы на следующую равно третьей, а произведение мнимой единицы на предыдущую равно третьей со знаком минус. Множество всех кватернионоы обозначается $\mathbb{H}$. Число $a$ называется \emph{скалярной частью} кватерниона, а выражение $bi+cj+dk$ -- \emph{векторной частью}. Кватернион с нулевой вещественной частью называется \emph{чисто мнимым}. Если выбрать в трёмхерном пространстве базис, состоящий из трёх единичных взаимно перпендикулярных векторов $i,j,k$, то векторную часть кватерниона можно рассматривать как вектор в трёхмерном пространстве.
\копр

\задача
	\пункт Докажите, что кватернионы образуют ассоциативную, но не коммутативную алгебру. \пункт Дадим альтернативное определение кватернионов: кватернионы -- это формальные записи от двух переменных $i,j$ (например, $\sqrt{2\pi}-7ijiji+j^{17}$), которые можно преобразовывать по правилам $i^2=j^2=-1, ij=-ji$. Докажите, что если положить $k=ij$, то любой кватернион приводится к виду $a+bi+cj+dk$, и что это определение эквивалентно предыдущему.
\кзадача

\опр
%Напомним, что
\emph{Векторным произведением} двух векторов $u$ и~$v$ в~$\R^3$ называется вектор $[u,v]$, перпендикулярный плоскости векторов $u$ и~$v$ и~имеющий длину $|u|\cdot|v|\cdot\sin\varphi$.
\копр

\задача
	\пункт Вычислите $(a+bi+cj+dk)^2$ и убедитесь, что квадрат кватерниона вещественный тогда и только тогда, когда сам кватернион чисто мнимый. \пункт Пусть $u$ и $v$ -- два чисто мнимых кватерниона. Докажите, что $uv=-(u,v)+[u,v]$, где $({-},{-})$~--- скалярное произведение, а~$[{-},{-}]$~--- векторное произведение.
\кзадача

\задача
Высните, в~какие тождества для скалярного и~векторного произведения превращается ассоциативность кватернионного умножения $(uv)w=u(vw)$ и~докажите тождество Якоби, $[u,[v,w]]+[v,[w,u]]+[w,[u,v]]=0$.
\кзадача

\задача
	Найдите все такие $z \in \mathbb{H}$, что $zq=qz$ для любого $q \in \mathbb{H}$.
\кзадача

\задача
	Может ли многочлен с вещественными коэффициентами иметь бесконечно много корней в $\mathbb{H}$?
\кзадача


\ЛичныйКондуит{0mm}{6mm}
\ОбнулитьКондуит
\newpage


\опр
	Пусть $q \in \mathbb{H}$, $q=a+bi+cj+dk$. Кватернион $a-bi-cj-dk$ называется \emph{сопряжённым} к $q$, и обозначается $\overline{q}$. Число $\sqrt{a^2+b^2+c^2+d^2}$ называется \emph{модулем} кватерниона $q$ и обозначается $|q|$.
\копр

\задача
	Докажите, что: \пункт $\overline{(q_1+q_2)}=\overline{q_1}+\overline{q_2}$, $\overline{q_1q_2}=\overline{q_2}\overline{q_1}$. \пункт $q \overline{q} = \overline{q} q = |q|^2$, $|q_1q_2|=|q_1||q_2|$, \пункт $\mathbb{H}$ является алгеброй с делением.
\кзадача

\задача
	\пункт Докажите, что не существует такого многочлена $P$ с комплексными коэффициентами, что $P(z)=\overline{z}$ для любого комплексного $z$. \пункт Найдите такой "некоммутативный многочлен" от $q$, то есть выражение, использующее только операции сложения и умножения $q$ на фиксированные кватернионы, которое выражает $\overline{q}$ через $q$.
\кзадача

\задача
	\пункт Докажите, что если два целых числа представимы в виде суммы четырёх квадратов, то и их произведение тоже. \пункт Для сумм трёх квадратов это неверно.
\кзадача

\small{С помощью кватернионов мы доказали, что существует представление вида \\ $(x_1^2+x_2^2+x_3^2+x_4^2)(y_1^2+y_2^2+y_3^2+y_4^2) = P_1^2+P_2^2+P_3^2+P_4^2$, где $P_1,P_2,P_3,P_4$ -- линейные функции от $x_1,x_2,x_3,x_4,y_1,y_2,y_3,y_4$. Аналогичное представление для сумм $n$ квадратов возможно только при $n=1,2,4,8$. Последнее происходит аналогичным образов из неассоциативной и некоммутативной восьмимерной алгебры с делением $\mathbb{O}$}.

\опр
	Кватернион $a+bi+cj+dk$ называется \emph{рациональным}, если $a,b,c,d \in \mathbb{Q}$, и \emph{целым гурвицевым}, если числа $a,b,c,d$ либо одновременно целые, либо одновременно полуцелые.
\копр

\задача
	\пункт Докажите, что $q$ -- целый гурвицев кватернион тогда и только тогда, когда его \emph{след} $q+\overline{q}$ и \emph{норма} $q\overline{q}$ -- целые числа.
	\пункт Проверьте, что аналогичное условия для комплексных чисел задаёт в точности целые гауссовы числа. \пункт Сколько существует целых гурвицевых кватернионов, по модулю равных единице? \пункт Проверьте, что целые гурвицевы кватернионы, по модулю равные единице, образуют группу по умножению. Эта группа называется \emph{бинарной группой тетраэдра} b обозначается $T^*$. Какие порядки бывают у элементов этой группы? (Напомним, что порядок элемента $g$ -- это наименьшее такое число $n$, что $g^n=1$.)
\кзадача

\задача
	\пункт Пусть $\theta = \frac{1+i}{\sqrt{2}}$. Докажите, что $O^* = T^* \cup  \{\theta t : t \in T^* \}$ -- конечная группа. Она называется \emph{бинарной группой октаэдра}. Какие порядки могут быть у её элементов? \пункт Пусть $\zeta = \frac{1}{2}(\frac{1-\sqrt{5}}{2}+i+\frac{1+\sqrt{5}}{2}j)$. Докажите, что $I^* = \{\zeta^k t : k \in \mathbb{Z}, t \in T^* \}$ -- конечная группа. Она называется \emph{бинарной группой икосаэдра}. Найдите количество элементов в $I^*$ и все встречающиеся в ней порядки элементов.
\кзадача

\задача
	\пункт Докажите, что для любого кватерниона $q$ найдётся такой целый гурвицев кватернион $\alpha$, что $|q-\alpha|<1$. \пункт Докажите, что если $\alpha,\beta$ -- целые гурвицевы, то найдётся такое целое гурвицево $\gamma$, что $|\beta - \alpha \gamma| < |\alpha|$ (и, вообще говоря, другое $\gamma'$, такое, что $|\beta - \gamma' \alpha|< |\alpha|$). \пункт Целое гурвицево число $\pi$ назовём \emph{неприводимым}, если его нельзя представить в виде $\pi = \gamma \delta$, где $|\gamma|, |\delta| > 1$. Докажите \emph{лемму Евклида} для целых гурвицевых чисел: если произведение $\alpha \beta$ делится на $\pi$ \emph{слева} (\emph{справа}), то есть $\alpha \beta = \pi \varrho$ ($\alpha \beta = \varrho \pi$), то и одно из чисел $\alpha$ или $\beta$ делится на $\pi$ слева (справа).
\кзадача

\задача
	\пункт Пусть $p$ -- нечётное простое число. Докажите, что если $a^2 \equiv b^2 (\mathrm{mod} \ p)$, то либо $a \equiv b (\mathrm{mod} \ p)$, либо $a \equiv -b (\mathrm{mod} \ p)$. Сколько элементов в $\mathbb{Z}/p\mathbb{Z}$ являются квадратами? \пункт Докажите, что сравнение  $x^2 \equiv 1-y^2 (\mathrm{mod} \ p)$ имеет решение в целых числах. \пункт Докажите, что $p$ не является простым в целых гурвицевых числах (\emph{указание:} если $x,y$ как выше, то $p$ и $1-xi-yj$ не взаимно просты). \пункт Докажите, что любое целое число представляется в виде суммы четырёх квадратов.
\кзадача

%\задача (теорема Фробениуса)
%	Пусть $A$ -- конечномерная ассоциативная алгебра с делением размерности $n$. Назовём её элемент $I$ \emph{чисто мнимым}, если $I^2$ -- неположительное вещественное число, и \emph{мнимой единицей}, если $I^2 = -1$. \пункт Докажите, что в $A$ нет \emph{делителей нуля}, то есть если $ab =0$, то либо $a=0$, либо $b=0$. \пункт Докажите, что для любого невещественного $x \in A$ существуют такие $a,b \in \mathbb{R}$, что $a + bx$ -- мнимая единица. (\emph{указание}: так как алгебра конечномерна, то элементы $1,x,x^2,x^3,\ldots,x^n$ линейно зависимы). \пункт Докажите, что множество чисто мнимых элементов $A$ образует векторное подпространство размерности $n-1$. \пункт Докажите, что не существует трёхмерной ассоциативной алгебры с делением. \пункт Докажите, что если $A$ либо одномерна, либо двумерна, и в $A$ есть мнимая единица, либо $A$ четырёхмерна, и в $A$ есть три линейно независимых мнимых единицы $I,J,K$, которые умножаются как кватернионы. Говоря проще, $\mathbb{R}, \mathbb{C}, \mathbb{H}$ -- единственные ассоциативные алгебры с делением.
%\кзадача

%{\small Имеет место обобщение теоремы Фробениуса, согласно которому алгебры с делением существуют только в размерностях $1,2,4$ и $8$. Единственное известное доказательство этого факта существенно опирается на алгебраическую топологию.}

%\опр
%	Пусть $s \in \mathbb{H}, s \neq 0$. Операция \emph{сопряжения с помощью} $s$ определяется по формуле $q \mapsto {}^sq = sqs^{-1}$.
%\копр

%\задача
%	Проверьте, что ${}^s(q_1 + q_2) = {}^sq_1+{}^sq_2$, ${}^s(q_1q_2) = {}^sq_1 {}^sq_2$, ${}^sq^{-1} = ({}^sq)^{-1}$, $\overline{{}^sq} = {}^{\overline{s}^{-1}}\overline{q}$.
%\кзадача

%\задача
%	Пусть $s = \alpha + t$, где $\alpha \in \mathbb{R}$, а $t$ -- чисто мнимый. Докажите, что опреация сопряжения с помощью $s$ оставляет на месте пространство чисто мнимых кватернионов, и что если $q$ -- чисто мнимый, то ${}^sq$ -- это результат поворота вектора $q$ вокруг оси вектора $t$ на угол $2 \ \mathrm{arctg} \frac{|t|}{|\alpha|}$.
%\кзадача

%\задача
%	Отождествим трёхмерное пространство с пространством чисто мнимых кватернионов. Докажите, что любое вращение трёхмерного пространства, сохраняющее начало координат, и сохраняющее ориентацию, имеет вид $q \mapsto {}^sq$ для некоторого $s$, по модулю равного единице, причём $s$ определён однозначно с точностью до знака.
%\кзадача

%\задача
%	Проверьте, что в условиях предыдущей задачи сопряжения с помощью элементов групп \пункт $T^*$, \пункт $O^*$, \пункт $I^*$ задают в точности группу симметрий тетраэдра, откаэдра и икосаэдра, соответственно. \пункт Почему в нашем списке нет бинарной группы куба и бинарной группы додекаэдра?
%\кзадача

%\задача
%	Пусть $\pi$ -- плоскость в пространстве чисто мнимых кватернионов, и пусть $s$ -- единичный вектор нормали к $\pi$. Запишите формулу для отражения относительно плоскости $\pi$.
%\кзадача

%\задача
%	Пусть $q_1,q_2,q_3$ -- чисто мнимые кватернионы. Запишите формулу для площади прямоугольника, натянутого на $q_1$ и $q_2$ и объёма параллелепипеда, натянутого на $q_1,q_2$ и $q_3$.
%\кзадача

%\задача (расслоение Хопфа)
%	Будем рассматривать множество кватернионов $x+yi+zj+wk$ как единичную сферу в четырёхмерном пространстве, заданную уравнением $x^2+y^2+z^2+w^2=1$. Обозначим его через $S^3$. \пункт Докажите, что ${s \in S^3 : {}^sk = k}$ -- это окружность, то есть пересечение $S^3$ с двумерной плоскостью в $\mathbb{R}^4$. \пункт Пусть $q$ -- чисто мнимый кватернион, по модулю равный единице. Докажите, что множества $S_q = \{s \in S^3 : {}^sk=q \}$ -- попарно непересекающиеся окружности, на которые разбивается (или, как ещё говорят, расслаивается) трёхмерная сфера. Стереографическая проекция $S^3 \to \mathbb{R}^3$ определяется так: вложим $\mathbb{R}^3$ в $\mathbb{R}^4$ как гиперплоскость, заданную уравнением $w=0$, соединим точку $s \in S^3$ с северным полюсом $(0,0,0,1)$, и продлим полученную прямую до пересечения с $\mathbb{R}^3$. \пункт Напишите формулу для стереографической проекции. \пункт Докажите, что окружности, не проходящие через северный полюс, при стереографической проекции переходят в окружности, а проходящие через северный полюс -- в прямые. Нарисуйте несколько окружностей из расслоения Хопфа после стереографической проекции и проверьте, что они \emph{зацеплены}, то есть круг, ограничивающийся одной из окружностей, пересекает все другие, причём ровно в одной точке.
%\кзадача

%{\small С помощью восьмимерной алгебры $\mathbb{O}$ можно построить аналог расслоения Хопфа, разбив семимерную сферу на зацепленные трёхмерные сферы. Несуществование расслоений такого типа для $n \neq 1, 3, 7$ и позволяет доказать теорему о размерностях алгебр с делением.}


\ЛичныйКондуит{0mm}{6mm}
% \GenXMLW
%\СделатьКондуит{5.4mm}{7mm}

\end{document}
