% !TEX encoding = Windows Cyrillic
\documentclass[a4paper, 12pt]{article}
\usepackage[tikz]{newlistok}
\usetikzlibrary{matrix}

\УвеличитьШирину{1.5truecm}
\УвеличитьВысоту{2.5truecm}
\newcommand{\icomment}[1]{{\small #1}}

\def\mtrx#1#2#3#4{\begin{pmatrix}#1 & #2 \\ #3 & #4\end{pmatrix}}
\def\vctr#1#2{(#1\;#2)}

\begin{document}

\Заголовок{Приближение действительных чисел рациональными:\par
Цепные дроби}
\НомерЛистка{NT-1}
\ДатаЛистка{2022.01}
%\Оценки{99/99/99}
\СоздатьЗаголовок



% limited rel 13.10.2012 public rel 01.12.2012
% Дима скорее знает, для других, кажется, не взлетело (впрочем, Сережа П. начал, кажется)
%
% Все знают, что бывают числа рациональные и иррациональные, а также алгебраические (явл. корнями...) и трансцендентными. И все слышали, что, например, число $\pi$... или число $e$ иррационально (и даже трансцендентно), но как вообще такие факты можно было бы доказывать? Ну ладно, про pi это действительно непросто, но как вообще привести хоть какие-нибудь примеры? Ладно, \sqrt2 -- пример иррационального числа, <...>. Ну вот Лиувилль придумал как -- и про это листок.
% Грубо говоря, идея в следующем. Вот все знают, как доказать, что число нецелое: если расстояние от нашего числа до целого меньше 1, то наше число с неизбежностью нецелое. Рациональные числа уже всюду плотны, но тем не менее <...> если число в некотором смысле хорошо приближается рациональными, то получается доказать, что оно иррационально (или даже трансцендентно).
% Такой небольшой (8 задач, чуть больше половины страницы), но совершенно замечательный сюжет.

%\задача
%Если число $\alpha$ иррационально, то числа вида $\{n\alpha\}$ (скобки обозначают дробную часть)
%\quad
%\пункт бывают сколь угодно малы;
%\quad
%\пункт всюду плотны на отрезке $[0;1]$.
%\кзадача






% Охотник стреляет? Конь прыгает?
\задача
Охотник стоит в~точке плоскости с~координатой~$(0,0)$, а~в~остальных точках с~целыми координатами сидят одинаковые зайцы. Докажите, что в~каком~бы направлении ни выстрелил охотник, он обязательно попадет в~зайца.
\кзадача






\задача
Найдите $\sup\left(\sin x+\sin\sqrt2x\right)$.
\кзадача






\задача
Десятичная запись числа $2^n$ может начинаться с~любой последовательности цифр.
\кзадача






\опр
Будем говорить, что дробь $\dfrac pq$ приближает число~$\alpha$ с~\emph{коэффициентом качества}~$\delta$,
если
\[
0<\bigl|\alpha-\frac pq\bigr|\leqslant\frac\delta q.
\]
\копр




\задача
Число $\alpha$ может быть сколь угодно качественно приближено дробью тогда и~только тогда, когда оно иррационально.
\кзадача






\задача
Докажите, что число $e=\smash{\sum\dfrac1{i!}}$ иррационально.
\кзадача






%\задача
%Если $\alpha$ рациональное число, то существует бесконечно много дробей $p/q$, таких что
%\[
%\bigl|\alpha-\frac pq\bigr|\leqslant\frac1{q^2}.
%\]
%\кзадача






\опр
Число~$\alpha$ будем называть \emph{$k$-приближаемым}, если для любого $\delta>0$ существует такая дробь~$\dfrac pq$, что
\vspace*{-5mm}
$$
0<\bigl|\alpha-\frac pq\bigr|\leqslant\frac\delta{q^k}.
$$
Если же такой дроби для некоторого $\delta>0$ не существует, будем называть число \emph{$k$-непри\-бли\-жа\-емым.}
\копр




\задача%[теорема Лиувилля]
\пункт Число $\sqrt2$ является 2-неприближаемым.

\пункт Алгебраическое число степени~$k$ является $k$-непри\-бли\-жа\-емым \emph{(теорема Лиувилля).}%\\
% Нужна производная? Да нет, достаточно делить многочлены, видимо
%\small{Указание. Если $P$~--- минимальный многочлен числа~$\alpha$, то $|P(\dfrac pq)|\approx C|\alpha-\dfrac pq|$}.
\кзадача






\задача
Число $\smash{\sum\dfrac1{10^{i!}}}$ трансцендентно.
\кзадача






\задача
Любое иррациональное число обладает бесконечным число $2$-приближений с~коэффициентом~1 (в~частности, является $(2-\varepsilon)$-приближаемым).
\кзадача






\сзадача
Множество всех $(2+\varepsilon)$-приближаемых чисел \emph{имеет меру ноль}%
\footnote{Т.\,е. для каждого положительного~$\delta$ существует покрытие этого множества не~более чем счетным числом интервалов, сумма длин которых не превосходит~$\delta$.}.
\кзадача














% limited rel 17.06.2013 (Арсен, Даня, Вася); public rel 04.09.2013
% очередная попытка, типа
\enlargethispage{\baselineskip}
\опр
Пусть $a_0$~--- целое число, $a_i$~--- натуральные числа. Выражение вида
\[
\abovedisplayskip=8pt
\belowdisplayskip=8pt
[a_0;a_1;\ldots]:=a_0+\frac1{a_1+\frac{\mathstrut1}{a_2+\ldots}}
\]
называется \emph{цепной дробью}; число $\smash{\dfrac{p_n}{q_n}=[a_0;\ldots;a_n]}$ называется $n$-й \emph{подходящей дробью} или \emph{конвергентой}.
\копр




\задача
\пункт Вычислите $[3;7;15;1]$ (с~точностью до 7 знаков после запятой) и~$[1;1;\ldots]$;

\пункт разложите в цепную дробь числа $10/7$, $\sqrt3$, $\sqrt5$.
%
%\пункт найдите первые две подходящие дроби для числа~$\pi$.
\кзадача






\задача
Для любой бесконечной цепной дроби $[a_0;\ldots]$ последовательность конвергент сходится к~некоторому действительному числу.
\кзадача






\задача
\пункт Ненулевое рациональное число может разложено в~цепную дробь (``алгоритм Евклида''), причем ровно двумя способами: вида $[a_0;\ldots;a_n]$ и $[a_0;\ldots;a_n-1;1]$.

\пункт Иррациональное число может разложено в~цепную дробь ровно одним способом.
\кзадача






\задача
\пункт $[a_0;a_1;\ldots;a_n;z]$~--- дробно-линейная функция от $z$.

\vspace{-6pt}

\спункт Функция $\frac{az+b}{cz+d}$ ($a,b,c,d\in\Z$) представима в~виде $[a_0;\ldots;a_n;z]$ $\iff$ % $[a_0;\ldots;a_n;z]$, если и~только если
$\det\mtrx abcd=\pm1$.
\кзадача


\ЛичныйКондуит{0mm}{6.5mm}
\newpage
\ОбнулитьКондуит



\задача
Если разложение иррационального числа в~цепную дробь периодично, то это квадратичная иррациональность\footnote{Как мы увидим позже, верно и~обратное (``теорема Лагранжа'').}.
\кзадача





\УстановитьГраницы{0mm}{70mm}
\задача
\rightpicture{0mm}{0mm}{70mm}{confrac}
Пусть $\alpha$~--- положительное число. Рассмотрим последовательность векторов $(e_i)$: $e_1=\vctr10$, $e_2=\vctr01$; $e_{i+1}=e_{i-1}+a_{i-2}e_i$, где в~качестве $a_{i-2}$ берется наибольше натуральное число, при котором $e_{i+1}$ остается с~той же стороны от прямой $y=\alpha x$, что и~$e_{i-1}$ (``алгоритм вытягивания носов'').
\\\пункт Пара векторов $(e_i,e_{i+1})$~--- базис целочисленной решетки $\Z^2$.
й\\\пункт Вектора $(e_{2k-1})$ и~$(e_{2k})$ являются вершинами выпуклой оболочки части $\Z^2$ под и~над прямой $y=\alpha x$ соответственно.
\\\пункт $\alpha=[a_0;a_1;\ldots]$, $e_{n+2}=\vctr{q_n}{p_n}$.
\\\пункт $n$-я подходящая дробь является наилучшим (в~смысле коэффициента качества приближения $q|\alpha-\frac pq|$) приближением к~$\alpha$ cреди дробей со знаменателем, не превосходящим $q_n$.
\кзадача





\vspace{-6pt}
\задача
\пункт $\det\mtrx{q_n}{q_{n+1}}{p_n}{p_{n+1}}=(-1)^{n+1}$.

%\hangindent=-15em
%\hangafter=-1
\пункт У~любого иррационального числа~$\alpha$ бесконечно много приближений, таких что $|\alpha-\frac pq|<\frac1{2q^2}$. %\\[-.5mm]
\кзадача






%\сзадача
%Если $\frac{p_n}{q_n}$~--- $n$-я подходящая дробь для $\alpha=[a_0;\ldots;a_n]$ и~$q_n^2\bigl|\alpha-\frac{p_n}{q_n}\bigr|=\frac1{\lambda_n}$,
%то $\vphantom{\int}\lambda_n=a_{n+1}+[0;a_{n+2};a_{n+3};\ldots]+[0;a_n;a_{n-1};\ldots a_1]$ (в~частности, $\lambda_n>a_{n+1}$).
%\кзадача







\сзадача
У~любого иррационального числа~$\alpha$ бесконечно много приближений, таких что $\bigl|\alpha-\frac pq\bigr|<\frac1{\sqrt5q^2}$, причем константу $\sqrt5$ нельзя улучшить (``теорема Гурвица--Бореля'').
\кзадача

\ВосстановитьГраницы





 %оммаж Flajolet'у, да
\задача
Числитель и~знаменатель подходящей дроби для $[1;1;\ldots;1]$~--- два последовательных числа Фибоначчи (в~частности, $\lim\frac{F_{n+1}}{F_n}=[1;1;\ldots]$).
%б) -- найти предел?
\кзадача






\сзадача
Последовательность $(a_i)$ удовлетворяет некоторой линейной рекурренте
тогда и~только тогда, когда
ее производящая функция $a_0+a_1t+a^2t^2+\ldots$ рациональна.
\кзадача





\опр
\emph{Пути Дика}~--- это пути из точки $(0,0)$ в~точку $(2n,0)$, состоящие из шагов $(1,1)$ и~$(1,-1)$ и~не опускающиеся ниже прямой $y=0$. Количество таких путей~--- это $n$-е число Каталана.

\emph{Пути Моцкина}~-- это пути из точки $(0,0)$ в~точку $(n,0)$, состоящие из шагов $(1,1)$, $(1,0)$ и~$(1,-1)$ и~не опускающиеся ниже прямой $y=0$. Количество таких путей называется $n$-м \emph{числом Моцкина}.
\копр




%\vspace{-2mm}
%\begin{figure}[h]
%\centering
%\begin{tikzpicture}
%\matrix[matrix of math nodes,row sep=-\pgflinewidth,column sep=-\pgflinewidth,
%style={nodes={rectangle,draw,text depth=.75ex,text height=2ex,minimum width=3em}},
%column 1/.style={nodes={font=\boldmath}}]
%{
%n         &0&1&2&3&4& 5 & 6& 7    &  8  &9    &10\\
%C_{n/2}&1& &1& &2&      & 5&      & 14&   &42\\
%M_n     &1&1&2&4&9&21&51&127&323&835&2188\\
%};
%\end{tikzpicture}
%\end{figure}
%\vspace{-5mm}
%
%%% C_n<M_{2n}<C_{2n} (второе неравенство -- из скобочных структур без [[...]], например)
\задача
\пункт Производящая функция для чисел Каталана равна (обобщенной) цепной дроби
\[
\abovedisplayshortskip=-5pt
\frac1{1-\frac{\mathstrut t}{1-\frac{\mathstrut t}{1-\ldots}}}.
\]

\пункт Ее $k$-я конвергента дает производящую функцию для путей Дика, не поднимающихся выше прямой $y=k$...
% Еще не сказано явно про связь с многочленами Чебышева...

\пункт ...и она же равна производящей функции для плоских корневых деревьев\footnote{Ср. с~задачей 7 листка <<Числа Каталана>>.}, имеющих высоту не более $k$.
\кзадача






% Нельзя ли связать с Фибоначчи явно?
\задача
\пункт Производящая функция для чисел Моцкина равна (обобщенной) цепной дроби
\[
\abovedisplayshortskip=-5pt
\frac1{1-t-\frac{t^2}{1-t-\frac{t^2}{1-\ldots}}}.
\]
% с t вместо t^2 в числителе получатся числа Шрёдера

\пункт Ее $k$-я конвергента дает производящую функцию для путей Моцкина, не поднимающихся выше прямой $y=k$.

\smallskip

\noindent
(Упражнение: придумайте несколько комбинаторных интерпретаций чисел Моцкина, аналогичных вашим любимым интерпретациям чисел Каталана; попробуйте описать подмножества этих объектов, соответствующие конвергентам цепной дроби.)
% Пример: непересекающиеся хорды на n точках (если потребовать, чтобы все точки были заняты, получится n/2-е число Каталана).
% (Ср. тж. с путями Шредера -- там вместо свободных точек появляются фигуры с большим числом углов, типа.)
%
% Мораль еще: конвергенты приближают все производящую функцию (иррациональную) рациональными (задающимися линейными рекуррентами с постоянными коэффициентами) производящими функциями.
\кзадача











\ЛичныйКондуит{0mm}{6.5mm}
% \GenXMLW
\end{document} 