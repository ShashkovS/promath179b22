% !TeX encoding = windows-1251
\documentclass[a4paper,12pt]{article}
\usepackage{newlistok}
\usepackage{tikz}
\usetikzlibrary{arrows}
\usetikzlibrary{decorations.markings}
\usetikzlibrary{decorations.pathreplacing}

\УвеличитьШирину{1.5cm}
\УвеличитьВысоту{1.5cm}
%\renewcommand{\spacer}{\vfill}

\ВключитьКолонитул
\Заголовок{Метрические пространства: разные задачи}
\НомерЛистка{MS-4}
\ДатаЛистка{2022.01}



\begin{document}
\СоздатьЗаголовок

\задача[Описание открытых подмножеств $\R$]
\невСтрочку
\пункт
Докажите, что любое открытое множество в $\R$ можно представить как объединение интервалов.
\пункт
Докажите, что любое открытое множество в $\R$ является объединением непересекающихся интервалов и лучей.
\кзадача


\задача[Принцип вложенных шаров]
\невСтрочку
\пункт
Докажите, что метрическое пространство полно тогда и только тогда, когда любая последовательность вложенных замкнутых шаров, радиусы которых стремятся к нулю, имеет общую точку.
\спункт
Докажите, что стремление радиусов к нулю существенно, то есть существует полное пространство и последовательность вложенных шаров, имеющих пустое пересечение.\\
\help{Можно построить соответствующую метрику метрику на счётном множестве}
\кзадача

\задача
Докажите, что подмножество компакта компактно тогда и только тогда, когда оно замкнуто.
\кзадача

\задача[Описание компактов в $\R^n$]
\невСтрочку
\пункт
Докажите, что единичный куб в $\R^n$ является компактом.
\пункт
Докажите, что подмножество $\R^n$ является компактным тогда и только тогда, когда оно замкнуто и~ограничено.
\кзадача

\сзадача
Приведите пример замкнутого ограниченного множества в $C[0,1]$, не являющегося компактом.
\кзадача

% \задача
% Являются ли связными или линейно-связными следующие метрические пространства?
% \невСтрочку
% \пункт Множество рациональных чисел.
% \пункт $n$-мерная сфера (множество точек $\R^{n+1}$, задаваемых уравнением $x_1^2 + \dots + x_{n+1}^2 = 1$).
% \пункт Подмножество $\R^4$, задаваемое условием $x_1x_2 \neq x_3x_4$.
% \пункт Подмножество $\R^4$, задаваемое условием $x_1x_2 > x_3x_4$.
% \кзадача

% \сзадача
% Докажите, что если из $\R^n$ ($n \ge 2$) выбросить конечное или счётное число точек, то оставшееся множество будет связным.
% \кзадача

\опр
Рассмотрим семейство множеств $\{K_i \mid i \in \N\}$, каждое из которых является объединением непересекающихся отрезков:
\begin{items}{-3}
\item
$K_{1} = [a,b]$.
\item
Если $K_{i} = \bigcup\limits_{j}[a_{ij}, b_{ij}]$, то $K_{i+1} = \bigcup\limits_{j}\br{[a_{ij}, \frac23 a_{ij} + \frac13 b_{ij}] \cup [\frac13 a_{ij} + \frac23 b_{ij}, b_{ij}]}$.
\end{items}
\vskip -2mm
Положим $K[a,b] = \bigcap\limits_{i\in\N}K_i$. Полученное множество называется \emph{множеством Кантора} (на отрезке $[a,b]$).
\копр

\vskip -4mm
\задача
\невСтрочку
\пункт
Докажите, что множество Кантора замкнуто.
\пункт
Докажите, что множество Кантора континуально.
\пункт
Найдите рациональное число, принадлежащее $K[0,1]$, знаменатель которого не является степенью тройки.
\кзадача

\tikzset{
  % style to apply some styles to each segment of a path
  on each segment/.style={
    decorate,
    decoration={
      show path construction,
      moveto code={},
      lineto code={
        \path [#1]
        (\tikzinputsegmentfirst) -- (\tikzinputsegmentlast);
      },
      curveto code={
        \path [#1] (\tikzinputsegmentfirst)
        .. controls
        (\tikzinputsegmentsupporta) and (\tikzinputsegmentsupportb)
        ..
        (\tikzinputsegmentlast);
      },
      closepath code={
        \path [#1]
        (\tikzinputsegmentfirst) -- (\tikzinputsegmentlast);
      },
    },
  },
  % style to add an arrow in the middle of a path
  mid arrow/.style={postaction={decorate,decoration={
        markings,
        mark=at position .5 with {\arrow{stealth}}
      }}},
}
\putthere{175mm}{-35mm}{%
\begin{tikzpicture}[
   scale=0.8
  ,vertex/.style={
     circle
    ,draw=blue!50
    ,fill=blue!20
    ,thick
    ,inner sep=0
    ,minimum size=0.3mm
  }
  ,line/.style={
     color=blue!80
    ,thin
  }
  ,path/.style={
     color=black
    ,thick
    ,postaction={on each segment={mid arrow}}
  }
]
    % f_1
    \def\x{0}
    \def\y{0}
    \node at (\x,\y+1.5) [anchor=east]{$f_1$:};
    % grid
    \draw[thin,step=3cm,xshift=\x cm,yshift=\y cm] (0,0) grid +(3,3);
    \draw[path] (\x+0,\y+0) -- (\x+3,\y+3);

    % f_2
    \def\x{0}
    \def\y{-3.5}
    \node at (\x,\y+1.5) [anchor=east]{$f_2$:};
    % grid
    \draw[very thin,step=1cm,xshift=\x cm,yshift=\y cm] (0,0) grid +(3,3);
    \draw[thin,step=3cm,xshift=\x cm,yshift=\y cm] (0,0) grid +(3,3);
    % path
    \draw[path] (\x+0,\y+0) -- (\x+1,\y+1) -- (\x+2,\y+0) -- (\x+3,\y+1) -- (\x+2,\y+2) -- (\x+1,\y+1) -- (\x+0,\y+2) -- (\x+1,\y+3) -- (\x+2,\y+2) -- (\x+3,\y+3);
    \path
        node at (\x+1, \y+0) [anchor=south east,font=\footnotesize]{$1$}
        node at (\x+1, \y+0) [anchor=south west,font=\footnotesize]{$2$}
        node at (\x+3, \y+0) [anchor=south east,font=\footnotesize]{$3$}
        node at (\x+3, \y+2) [anchor=north east,font=\footnotesize]{$4$}
        node at (\x+1, \y+2) [anchor=north west,font=\footnotesize]{$5$}
        node at (\x+1, \y+2) [anchor=north east,font=\footnotesize]{$6$}
        node at (\x+1, \y+2) [anchor=south east,font=\footnotesize]{$7$}
        node at (\x+1, \y+2) [anchor=south west,font=\footnotesize]{$8$}
        node at (\x+3, \y+2) [anchor=south east,font=\footnotesize]{$9$}
    ;
\end{tikzpicture}
}{7cm}{}

\УстановитьГраницы{0mm}{32mm}
\задача[Кривая Пеано]
Положим $I = [0,1]$. Рассмотрим последовательность отображений $f_n \from I \to I^2$.\\
Первая функция строится как диагональ квадрата: $f_1(t) = (t,t)$.\\
Для построения второй функции необходимо разделить квадрат на девять маленьких квадратиков и обойти их диагонали в указанном порядке.\\
Для построения $f_3$ возьмём $f_2$ и проход по каждой диагонали заменим на проход по такой же \лк букве Ф\пк (соответствующим образом уменьшенной и повёрнутой).\\
И так далее.\\
Движение по всем ломаным происходит с постоянной скоростью.
% Дальше следует много картинок и текста
\невСтрочку
\пункт
Докажите, что последовательность $(f_n)$ имеет предел в пространстве непрерывных отображений из $I$ в $I^2$. Обозначим этот предел через $f$.\\
\help{Упомянутое пространство полное}
\ВосстановитьГраницы
\пункт
Докажите, что для любого $x \in I^2$ и для любого $\ep>0$ пересечение $U_{\ep}(x) \cap f_n(I)$ не пусто при $n\gg0$.
\пункт
Докажите, что для любой точки $x \in I^2$ и для любого $\ep>0$ пересечение $U_{\ep}(x) \cap f(I)$ не пусто.
\пункт
Докажите, что $f(I) = I^2$.
\вСтрочку
\пункт
Вычислите $f\hr{\frac14}$.
\кзадача
% \ВосстановитьГраницы

\ЛичныйКондуит{-0.3mm}{6mm}

% \GenXMLW


\end{document}
