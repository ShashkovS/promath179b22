% !TEX encoding = Windows Cyrillic
\documentclass[a4paper, 12pt]{article}
\usepackage{newlistok}

\УвеличитьШирину{1.5truecm}
\УвеличитьВысоту{2.5truecm}

\newcommand{\cF}{\ensuremath{\mathcal F}}
\newcommand{\cM}{\ensuremath{\mathcal M}}
\newcommand{\uF}{\text{$\hbox{$\cF$}\mkern6mu\raise2pt
                 \hbox{$\hbox{}^\uparrow$}$}}
\newcommand{\dF}{\text{$\hbox{$\cF$}\mkern-5mu\raise-2pt
                 \hbox{$\hbox{}_\downarrow$}$}}
                 \def\bl{\!\in\!}
\def\fain#1#2{\;\,\forall\,{#1}\bl{#2}\;}
\def\bydef{\stackrel{\rm def}{=}}


\sloppy
\begin{document}

\Заголовок{Теорема единственности РОДУ}
\НомерЛистка{ODE-3}
\ДатаЛистка{2022.01}
%\Оценки{99/99/99}
\СоздатьЗаголовок









\задача\label{nonun}
  Найдите все решения уравнения $y'=y^{2/3}$ и укажите два различных
  решения, удовлетворяющие начальному условию $y(0)=0$.
\кзадача





\опр[Постановка задачи.]
  Всюду в этом листке константа $C$, прямоугольник $П$, \лк бабочка\пк\
  $B_\de\subset П$, отрезок $D_\de\ni x_0$ и пространство $\cM_\de$
  непрерывных функций на $D_\de$ с графиками внутри $B_\de$ будут те же
  самые, что и в предыдущем листке. Мы докажем, что если правая часть
  дифференциального уравнения $y'=F(x,y)$ удовлетворяет дополнительному
  условию:
  \begin{equation}\label{lipcon}
     \exists\;L\in\R\,:\,|F(x,y_1)-F(x,y_2)|<L\cdot|y_1-y_2|\quad
     \fain x{[a,b]}\;\&\;\forall\,y_1,y_2\in[c,d]
  \end{equation}
  то любые два решения дифференциального уравнения $y'=F(x,y)$, графики
  которых проходят через точку $(x_0,y_0)$ совпадают над некоторой
  $\de$-окрестностью точки $x_0$.
\копр





\задача[приближения Пикара]
  Будем строить последовательные приближения $\psi_k(x)\in\cM_\de$ (с
  $k=0,1,2,\dots$) к решению уравнения $y'=F(x,y)$, взяв
  $\psi_0(x)\equiv y_0$ и подбирая в качестве $\psi_{k+1}$ такую
  дифференцируюмую функцию, производная от которой равна значениям
  функции $F$ на графике предыдущего приближения $\psi_k$, \те
  удовлетворяющую при $x\in D_\de$ уравнению
  $\psi'_{k+1}(x)=F(x,\psi_k(x))$ и такую, что
  $\psi_{k+1}(x_0)=y_0$. Докажите, что
  $\displaystyle\psi_{k+1}(x)=y_0+\int\limits_{x_0}^xF(t,\psi_k(t))\,dt$ и
  проверьте, что все $\psi_k\in\cM_\de$.
\кзадача





\задача
  Явно вычислите все приближения Пикара для уравнения
  $y'=y$ с начальным условием $y(0)=1$ и честно найдите их предел.
\кзадача





\задача
  Пусть функция $F$ удовлетворяет условию \eqref{lipcon}.
  Докажите, что при достаточно малом $\de$ правило\quad
  $\displaystyle P:\psi(x)\longmapsto P\psi(x)=
    y_0+\int\limits_{x_0}^xF(t,\psi(t))\,dt$\quad
  определяет сжимающее отображение $\cM_\de\to^P\cM_\de$.
\кзадача





\задача
  Докажите, что функция $\psi\in\cM_\de$ тогда и только
  тогда является решением уравнения $y'=F(x,y)$, когда $P\psi=\psi$.
\кзадача





\задача
  Докажите сформулированную в начале листка теорему единственности.
  Как она уживается с примером из \ref{nonun}?
\кзадача





\задача
  Пусть отображение $\cM\to^P\cM$ (в произвольном метрическом
  пространстве) является сжимающим с константой $0<\l<1$ (\те
  $\rho(P\phi,P\psi)\le\l\rho(\phi,\psi)$ $\forall\;\phi,\psi\in\cM$). Докажите,
  что расстояние от произвольной точки $\psi\in\cM$ до неподвижной
  точки $\psi_0$ отображения $P$ удовлетворяет неравенству
  $\rho(\psi,\psi_0)\le\dfrac{\rho(\psi,P\psi)}{1-\l}$.
\кзадача





\задача
  Докажите, что если функция $F$ удовлетворяет условию \eqref{lipcon},
  то {\it вся\/} последовательность ломаных Эйлера из \ref{lomeul}
  {\it равномерно\/} (\те по метрике $\cM_\de$, а не поточечно) сходится
  к решению уравнения $y'=F(x,y)$.
\кзадача





\сзадача[теорема о непрерывной зависимости от начальных условий]
  Пусть функция $F$ удовлетворяет условию \eqref{lipcon}. Докажите, что
  у точки $(x_0,y_0)$ существует окрестность $\widetilde П\subset П$, такая
  что при некотором фиксированном $\de>0$ и произвольных $(\widetilde x_0,\widetilde
  y_0)\in\widetilde П$ уравнение $y'=F(x,y)$ будет обладать единственным
  решением $y=f(x)$, определённым всюду на $D_\de$ и удовлетворяющим
  начальному условию $f(\widetilde x_0)=\widetilde y_0$, и более того,
  сопоставление точке $(\widetilde x_0,\widetilde y_0)\in\widetilde П$ такого решения
  будет непрерывным отображением из $\widetilde П$ в пространство непрерывных
  функций на $D_\de$.
\кзадача


\ЛичныйКондуит{0mm}{6.5mm}
% \GenXMLW
\end{document}
