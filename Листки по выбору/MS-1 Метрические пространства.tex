% !TeX encoding = windows-1251
\documentclass[a4paper,12pt]{article}
\usepackage{newlistok}

\УвеличитьШирину{1.5cm}
%\renewcommand{\spacer}{\vfill}

\Заголовок{Метрические пространства: примеры}
\НомерЛистка{MS-1}
\ДатаЛистка{2022.01}


\begin{document}
\СоздатьЗаголовок

\опр
\выд{Метрическим пространством} $(M, d)$ называется пара, состоящая из множества $M$ и~функции \лк расстояния\пк\ (\emph{метрики}) $d\colon M\times M \to \R$, удовлетворяющей следующим аксиомам:
\begin{items}{-3}
\item[(M1)]
$d(x,y) = 0$ тогда и только тогда, когда $x=y$;
\item[(M2)]
$d(x,y) = d(y,x)$ (\выд симметричность);
\item[(M3)]
$d(x,z)\le d(x,y)+d(y,z)$ (\выд{неравенство треугольника}).
\end{items}

Подмножество $N$ метрического пространства $M$, рассматриваемое как метрическое пространство (с~той же метрикой), называется \выд{подпространством} пространства $M$.
\копр

\задача
Пусть $(M, d)$\т метрическое пространство. Докажите, что $d(x,y)\ge0$ для любых $x,y\in M$.
\кзадача

\задача
Поездка на московском метрополитене от станции $A$ до станции $B$ требует времени, которое зависит от выбранного маршрута, времени ожидания поездов
\итп Выберем из всех возможных случаев тот, при котором затраченное время окажется наименьшим, и назовём это время расстоянием от станции~$A$ до станции $B$. Является ли такое расстояние метрикой на множестве станций московского метро? Если нет, предложите дополнительные условия, при которых введённое
расстояние будет метрикой.
\кзадача

\опр
Множество последовательностей $x=(x_1,x_2,\dots,x_n)$ длины $n$, состоящих из действительных чисел, называется \выд{$n$-мерным арифметическим пространством $\R^n$}. (Обычные прямая, плоскость и пространство\т это $\R^1$,
$\R^2$ и $\R^3$ соответственно).
\копр

% \УвеличитьПробелы{-4mm}{-2mm}
\задача\label{R^n}
Является ли метрическим пространством $\R^n$ с метрикой\\
\пункт
$d_1(x,y)=\sum\limits_{k=1}^{n}|y_k-x_k|$;\\
\пункт[евклидова метрика]
$d_2(x,y)=\sqrt{\sum\limits_{k=1}^{n}(y_k-x_k)^2}$;\\
\пункт
$d_\infty(x,y)=\max\limits_{1\le k\le n}|y_k-x_k|$?
\кзадача
% \ВосстановитьПробелы

\задача [Дискретная метрика]
Пусть $M$\т любое множество. Положим $d(x,y)=0$, если $x=y$ и~$d(x,y)=1$, если $x\ne y$. Докажите, что таким образом получается метрика (называемая \emph{дискретной}). Метрическое пространство $(M, d)$ также называется \выд дискретным.
\кзадача

\задача[Метрика Хэмминга]
Пусть $M$\т множество слов некоторого алфавита, состоящих из какого-то фиксированного числа букв. Расстоянием $d (x,y$) между словами $x$ и $y$ назовём количество букв, в~которых эти слова отличаются, если написать их одно под другим. Например, $d(\text{нос},\text{сон}) = 2$. Докажите, что $d$ является метрикой.
\кзадача

\задача[$p$-адическая метрика]
Пусть $p$\т простое число. Для $x, y \in \N$ положим $d_p(x,y) = 0$, если~$x = y$, и $d_p(x,y) = p^{-n}$, если~$x \ne y$ и $n$\т наибольший показатель степени числа $p$, при котором разность $x-y$ делится на $p$. Проверьте, что $(\N, d_p)$\т метрическое пространство.
\кзадача

\задача[Равномерная метрика]\label{C}
Пусть $M$\т множество ограниченных функций $f\colon [a,b] \to \R$. Положим $d (f,g) = \sup\limits_{x\in [a,b]}|f(x) - g(x)|$. Проверьте, что это метрика.
\кзадача

\ЛичныйКондуит{-0.3mm}{6mm}
\ОбнулитьКондуит
\newpage

\опр
Пусть $M$\т метрическое пространство, $x_0\in M$\т произвольная точка, $\ep > 0$\т вещественное число. Множество $U_\ep(x_0) = \{ x\in M \mid d(x, x_0) < \ep\}$ называется $\ep$-окрестностью точки $x_0$ (или \выд{открытым шаром} с центром $x_0$ и радиусом $\ep$).
Множество $B_\ep(x_0) = \{ x\in M \mid d(x,x_0)\le\ep\}$ называется \выд{замкнутым шаром} с центром $x_0$ и радиусом $\ep$.
\копр

\задача
Как выглядят шары в пространствах $\R^2$ и $\R^3$ относительно метрик из
задачи \ref{R^n}?
\кзадача

\задача[Хаусдорфовость метрического пространства]
Пусть $x_1$, $x_2$\т различные точки метрического пространства $M$. Докажите, что существует такое $\ep > 0$, что $U_\ep(x_1) \cap U_\ep(x_2) = \nothing$.
\кзадача

\задача
Докажите, что если два открытых шара метрического пространства имеют общую точку, то существует шар, лежащий в их пересечении.
\кзадача

\задача
Докажите, что если $U_{\ep}(x) \cap U_{\ep}(y) \ne \nothing$, то $d (x,y) < 2\ep$. Верно ли обратное (в произвольном метрическом пространстве)?
\кзадача

\задача
Докажите, что если $d(x,y) < \ep$, то $U_\ep(x) \subset U_{2\ep}(y)$.
\кзадача

\задача
Шары с радиусами $r_1$ и $r_2 = 57 r_1$ пересекаются. Радиусы шаров увеличили вдвое, не меняя их центров. Докажите, что один из полученных шаров содержится в другом.
\кзадача

\задача
Могут ли в метрическом пространстве существовать два шара разных радиусов, таких что шар большего радиуса содержится в шаре меньшего радиуса и не совпадает с ним?
\кзадача

\задача
\невСтрочку
\пункт
Сколько элементов содержит замкнутый шар радиуса $1$ на множестве слов длины $n$ с метрикой Хэмминга для алфавита $\{0,1\}$? А если в алфавите $m$ букв?
\пункт
Написано несколько последовательностей из нулей и единиц длины $n$, причём любые две из них отличаются по крайней мере в трех местах. Докажите, что их число не превосходит $\frac{2^n}{n+1}$.
\кзадача

\опр
Два метрических пространства $(M_1, d_1)$ и $(M_2, d_2)$ называются \выд изометричными, если существует взаимно однозначное отображение $f\colon M_1 \to M_2$, такое что для любых точек $x_1, x_2 \in M_1$ выполняется равенство $d_1(x_1,x_2)=d_2(f(x_1),f(x_2))$. Отображение $f$ в этом случае называется \выд изометрией.
\копр

\задача
Придумайте такую метрику на прямой $\R$, чтобы прямая относительно этой метрики и интервал $(0;1)$ относительно стандартной метрики были изометричны.
\кзадача

\задача
Изометричны ли $(\R^n, d_2)$ и $(\R^n, d_\infty)$?
\кзадача

\опр
Говорят, что метрическое пространство $N$ \выд вкладывается в метрическое пространство $M$, если $N$ изометрично некоторому подпространству в $M$.
\копр

\задача
Докажите, что $(\R^n, d_2)$ вкладывается в $(\R^N, d_2)$ при $n \le N$.
\кзадача

\задача
Докажите, что $(\R^n, d_\infty)$ вкладывается в метрическое пространство из задачи~\ref{C}.
\кзадача

\задача
Верно ли, что любое конечное метрическое пространство $M$ вкладывается в $(\R^n, d_2)$ при~$n \gg 0$? Если да, то как можно оценить $n$, зная $|M|$?
\кзадача



\ЛичныйКондуит{0.3mm}{6.5mm}
%\СделатьКондуит{6mm}{8mm}
% \GenXML

\end{document}
