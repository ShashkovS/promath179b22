% !TeX encoding = windows-1251
\documentclass[a4paper,12pt]{article}
\usepackage[mag=980]{newlistok}

\УвеличитьВысоту{2.5cm}
\УвеличитьШирину{1.9cm}

\newcommand{\wa}{\overrightarrow}
\newcommand{\smat}[1]{\hr{\begin{smallmatrix}#1\end{smallmatrix}}}

% \global\abovedisplayskip=-11pt
% \global\belowdisplayskip=-11pt
% \global\abovedisplayshortskip=-11pt
% \global\belowdisplayshortskip=-11pt


\Заголовок{Линейные преобразования и матрицы}
\НомерЛистка{STO-3}
\ДатаЛистка{2022.01}


\sloppy

\begin{document}
\СоздатьЗаголовок

\опр
Набор векторов $\hc{e_1\sco e_n}\subset\R^m$ называется базисом, если для любого вектора $w\in\R^m$ найдётся единственный набор чисел $\hc{w^1\sco w^n}$ (который называется координатами вектора $w$ в этом базисе) такой, что
$w = w^1 v_1 + \ldots + w^n v_n$.
\копр

\опр
    \выд Матрицей называется произвольная прямоугольная таблица действительных чисел.
\копр

\соглашение[Правило суммирования Эйнштейна]
В физике и механике часто рассматривают различные суммы произведений.
Например, если $w^i$ --- координаты вектора $w$ в базисе $\hc{e_i}$, то $w = \sum\limits_{\al=1}^{m}w^\al e_\al$.
В тех случаях, когда суммирование ведётся по одному нижнему и одному верхнему индексу, знак суммы может быть опущен:
пишут просто $w^\al e_\al$, имея в виду сумму по всем осмысленным значениям параметра $\al$.
Таким образом могут быть коротко записаны довольно длинные суммы:
$$
T^{\mu\nu} = \frac{1}{\mu_0} \left[ F^{\mu \alpha}F^\nu{}_{\alpha} - \frac{1}{4} \eta^{\mu\nu}F_{\alpha\beta} F^{\alpha\beta}\right] \text{--- тензор натяжений Максвелла}$$

\noindent
Координаты вектора $w\in\R^m$ в стандартном базисе $\hc{e_i}$ будем обозначать через $w^1\sco w^m$.
Будем записывать координаты вектора $w$ матрицей $m\times 1$ (то есть столбцом) его координат:
    $$ w = \hr{\begin{smallmatrix}w^1\\\vdots\\w^m\end{smallmatrix}} = w^1 e_1 + w^2 e_2 + \ldots +  w^m e_m = w^\al e_\al.$$
\vspace*{-3mm}
\ксоглашение



\задача
    Докажите, что набор векторов $\hc{e_1\sco e_m}\in\R^m$, где $e_i=(0\sco1_i\sco0)$, образует базис~$\R^m$.
    Будем называть его \выд{стандартным}.
\кзадача

\задача
    \пункт
    Пусть $f\colon\R^m\to\R^m$ --- биективное линейное отображение. Докажите, что набор векторов $\hc{e_{1'}\sco e_{m'}}$, где $e_{i'}=f(e_i)$, образует базис $\R^m$.\\
    \пункт
    Пусть линейное преобразование $f$ переводит базис $\hc{e_1\sco e_m}$ в базис. Докажите, что оно биективно.
\кзадача


\задача
    Пусть $f\colon\R^m\to\R^m$ --- произвольное линейное отображение.
    Запишем координаты вектора $e_{i'} = f(e_i)$ в столбец: $\hr{\begin{smallmatrix}c^1_\textbf{i}\\c^2_\textbf{i}\\\vdots\\c^m_\textbf{i}\end{smallmatrix}}$, а из этих столбцов составим квадратную таблицу
    \scalebox{0.8}{
    $C:=
    \begin{pmatrix}
    C^1_\textbf{1}&C^1_2&\cdots&C^1_\textbf{m}\\
    C^2_\textbf{1}&C^2_2&\cdots&C^2_\textbf{m}\\
    \vdots&\vdots&\ddots&\vdots\\
    C^m_\textbf{1}&C^m_2&\cdots&C^m_\textbf{m}
    \end{pmatrix}
    $}
    \\
    Эта таблица называется \выд{матрицей преобразования} $f$ в базисе $\hc{e_i}$.
\кзадача


\задача
  Выразите вектор $e_{i'}$ через базисные векторы $\hc{e_{\al}}$ и элементы матрицы $\hc{C^p_\textbf{q}}$ в короткой нотации.
\кзадача



\задача
    \пункт
    Найдите координаты вектора $f(w)$ в базисе $\hc{e_{1'}\sco e_{m'}}$.\\
    \пункт
    Найдите координаты вектора $f(w)$ в базисе $\hc{e_{1}\sco e_{m}}$.\\
    \пункт
    Придумайте привило умножения матрицы $C$ на вектор-столбец координат вектора $w$ так,
    чтобы работало правило $f(w) = C\cdot w$.
\кзадача


\задача
    Вычислите:
    \пункт
    $\smat{1&1\\1&1}\cdot\smat{1\\1}$;
    \пункт
    $\smat{\phm\cos\ph&\sin\ph\\-\sin\ph&\cos\ph}\cdot\smat{1\\0}$;
    Что это за линейное преобразование?\\
    \пункт
    $\smat{1&2\\2&4}\cdot\smat{\phm2\\-1}$;
    \пункт
    $\smat{1&2&3\\1&4&9\\1&8&27}\cdot\smat{\phm1\\-1\\\phm0}$;
    \пункт
    $\smat{0&0&1\\0&1&0\\1&0&0}\cdot\smat{\phm3\\\phm7\\-5}$.

\кзадача


\задача
    Пусть $f$ и $g$ --- два биективных линейных отображения $\R^m\to\R^m$,
    а $C$ и $D$  --- матрицы преобразований $f$ и $g$ в базисе $\hc{e_i}$ соответственно.
    \\
    \пункт
    Найдите координаты $g(f(e_1))$ в базисе $\hc{e_i}$.
    \пункт
    Найдите координаты $g(f(w))$ в базисе $\hc{e_i}$.\\
    \пункт
    Придумайте правило умножения матриц так, чтобы для каждого вектора $w$
    $$g(f(w)) = (D\cdot C) \cdot w.$$
\vspace{-6mm}
\кзадача
\задача
    Придумайте две матрицы $C$ и $D$ такие, что:
    \пункт
    $C D = D$;
    \пункт
    $C D = D C$;
    \пункт
    $C D \ne D C$.
\кзадача

\задача
    Постройте биекцию между множеством всех линейных отображений (операторов) $\R^m\to\R^m$ и множеством матриц размера $m$ на $m$.
\кзадача


\задача
    \пункт
    Найдите такую матрицу $E$, что для любой матрицы $C$ верно: $EC = CE = C$.\\
    \пункт
    Докажите, что такая матрица единственна.\\
    \пункт
    Пусть $C$ --- матрица биективного линейного оператора. Докажите, что найдётся матрица $D$ такая, что $CD=DC=E$.\\
    \пункт
    Докажите, что множество матриц биективных линейных операторов образуют группу относительно операции умножения.
\кзадача


\задача
    Придумайте, как описать аффинные преобразования с помощью матриц, векторов, их сложения и умножения.
\кзадача

\ЛичныйКондуит{0mm}{6mm}

%\СделатьКондуитИз{6.2mm}{6.2mm}{sp_STO.tex}
% \GenXMLW


\end{document}



{\footnotesize
    В этом листочке мы получим описание всех замен координат для перехода от одной инерциальной системы отсчёта к другой. Такие преобразования называются преобразованиями Галилея.

    Навсегда зафиксируем некоторую инерциальную систему отсчёта и будем называть её \выд{системой отсчёта лаборатории}.

}

\опр
    \выд Классической \выд заменой координат мы будем называть биекцию $f\colon\R^4\to\R^4$, для которой существует константа $t_0$, что для любой точки $(t,x,y,z)\in\R^4$ верно: $f(t,x,y,z) = (t+t_0,\ldots)$ (В классической теории при замене координат время может только сдвигаться на константу.) Таким образом, в каждый момент времени имеется биекция $f_t\colon\R^3\to\R^3$. Координатой точки $(x,y,z)$ в момент времени $t$ будем называть тройку чисел $f_t(x,y,z)$.
\копр



\textbf{Соглашение 1:} далее считаем, что любая точка двигается в новой системе координат равномерно и прямолинейно тогда и только тогда, когда она двигается равномерно и прямолинейно в системе отсчёта лаборатории.

\задача
    Докажите, что центр новой системы координат (то есть образ точки $(0,0,0)$) двигается равномерно и прямолинейно (то есть $f_t(0,0,0)=(p_0 + tp, q_0 + tq, r_0 + tr)$).
\кзадача

\задача
    \пункт
    Докажите, что для любого $t$ отображение $f_t$ является аффинным.\\
    \пункт
    Положим $g_t(x,y,z) = f_t(x,y,z) - f_t(0,0,0)$. Докажите, что преобразование $g_t$ при каждом $t$ линейно.\\
    \пункт
    Докажите, что преобразование $g_t$ не зависит от $t$. Обозначим его через $G$.
\кзадача

\textbf{Соглашение 2:} далее считаем, что преобразование координат сохраняет расстояние между любыми двумя точками (то есть расстояние между $f_t(x_1,y_1,z_1)$ до $f_t(x_2,y_2,z_2)$ не зависит от $t$).





\задача
    Докажите, что
\кзадача

\end{document}

\задача
Как перелётным птицам проще лететь: по ветру или против ветра? (в каком смысле \лк проще\пк следует понять самостоятельно)
\кзадача


{\footnotesize
Для изучения сложных вопросов необходимо изучить преобразования, связывающие две различные инерциальные системы отсчёта.

Навсегда зафиксируем некоторую инерциальную систему отсчёта и будем называть её \выд{системой отсчёта лаборатории}. Другая система\footnote{}, называется \выд инерциальной, если любая точка двигается в ней равномерно и прямолинейно тогда и только тогда, когда она двигается равномерно и прямолинейно в системе отсчёта лаборатории.

\noindent{\bf Обозначения:}
Для того, чтобы не путать разные системы координат, мы будем использовать следующие обозначения: центр координат системы отсчёта \лк лаборатории\пк (\лк неподвижной\пк) будем обозначать через $O$, вектора, вдоль которых измеряются координаты --- $(e_1,e_2,e_3)$ или $(e_x,e_y,e_z)$ в зависимости от ситуации, и, собственно, значения координат --- через $(x^1,x^2,x^3)$ или через $(x,y,z)$ соответственно. Для системы координат \лк ракеты\пк точно такие же обозначения, но со штрихами: $O'$, $(e_{1'},e_{2'},e_{3'})$, $(e_{x'},e_{y'},e_{z'})$, $(x^{1'},x^{2'},x^{3'})$ и $(x',y',z')$ соответственно.

Преобразования, превращающие данную инерциальную (то есть систему координат \лк лаборатории\пк) систему отсчёта в другую инерциальную называются преобразованиями Галилея.



}



\задача
Как может выражаться координаты центра системы отсчёта ракеты через координаты лаборатории и время?
\кзадача

\задача
Докажите, что если центр системы отсчёта ракеты неподвижен, то и вся система отсчёта ракеты неподвижна.
\кзадача

\опр
Преобразование $f$ называется \выд линейным, если для любых двух векторов $v$ и $w$ и любых двух чисел $\la$  и $\mu$ верно: $f(\la v + \mu w) = \la f(v) + \mu f(w)$.
\копр

\задача
Предположим, что $O'=O$. Докажите, что преобразование координат линейно.
\кзадача

\задача
Докажите, что множество всех преобразований Галилея




\end{document}
