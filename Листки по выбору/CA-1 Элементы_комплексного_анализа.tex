% !TEX encoding = Windows Cyrillic
\documentclass[a4paper, 12pt]{article}
\usepackage[tikz]{newlistok}

\УвеличитьШирину{1.5truecm}
\УвеличитьВысоту{2.5truecm}
\ВключитьКолонтитул

\begin{document}

\Заголовок{Элементы комплексного анализа}
\НомерЛистка{CA-1}
\ДатаЛистка{2022.01}
%\Оценки{99/99/99}
\СоздатьЗаголовок


% pre-rel 21.12.2013 (Кирилл, Дима); v2: 29.12.2013, upd 30.12
%
% Написано для Кирилла, в сущности (и для Паши П., быть может?). В большой степени по мотивам v08, но добавлены связи с вычетом и индексом (и ОТА).
\опр
Пусть $f$~--- отображение из $\R^n$ в~$\R^m$.
Если в~окрестности точки~$x_0$ для некоторого линейного отображения $A\colon\R^n\to\R^m$ верно, что
\[
f(x)=f(x_0)+A(x-x_0)+o(x-x_0),
\]
то говорят, что функция~$f$ (вещественно) \emph{дифференцируема} в~точке~$x_0$. Линейное отображение~$A$ называется \emph{дифференциалом} функции~$f$.
\копр




% Нужно еще какое-то утверждение про интегрирование и деформацию контура


% I. Дифференцируемость и Коши--Риман
\опр
Пусть $f$~--- отображение из $\Cbb$ в~$\Cbb$.
Если в~окрестности точки~$z_0$ для некоторого комплексного числа~$a$ верно, что
\[
f(z)=f(z_0)+a(z-z_0)+o(z-z_0),
\]
то говорят, что функция~$f$ \emph{комплексно дифференцируема} в~точке~$z_0$ и~пишут $f'(z_0)=a$.

Функция называется \emph{голоморфной} на некотором открытом множестве, если она комплексно дифференцируема в~каждой его точке; говорят, что функция \emph{голоморфна в~точке}, если она голоморфна в~некоторой окрестности этой точки.
\копр




\задача
Найдите (комплексные) производные (если они есть) следующих функций

\пункт $z$;\quad
\пункт $\bar z$;\quad
\пункт $\operatorname{Re}z+2i\operatorname{Im}z$;\quad
\пункт $z^n$;\quad
\пункт $\dfrac1{1+z}$;\quad
\пункт $\dfrac1{\bar z}$;\quad
\пункт $|z|$;\quad
\пункт $\dfrac{|z|^2}{\bar z}$;\quad
\пункт $\sqrt z$;\quad
\пункт $\operatorname{Arg}z$.
\кзадача






\задача
Какие из аффинных преобразований голоморфны?
\кзадача






\задача
Если функция имеет ненулевую комплексную производную в~точке, то она сохраняет углы между кривыми в~этой точке (``является конформным отображением''; ср., например, с~сохранением углов при инверсии).
\кзадача






\задача
Вещественно-дифференцируемая функция $x+iy\mapsto u(x,y)+iv(x,y)$ комплексно дифференцируема тогда и~только тогда, когда выполнены \emph{условия Коши--Римана}:
\[
\frac{\partial u}{\partial x}=\frac{\partial v}{\partial y},
\quad
\frac{\partial u}{\partial y}=-\frac{\partial v}{\partial x}.
\]
\кзадача






\vspace{-3mm}
\опр
Положим
$\displaystyle\frac\partial{\partial z}=
\frac12\left(\frac\partial{\partial x}-i\frac\partial{\partial y}\right)$,
$\displaystyle\frac\partial{\partial\bar z}=
\frac12\left(\frac\partial{\partial x}+i\frac\partial{\partial y}\right)$.
\копр




%\vspace{-3mm}
\задача
\пункт Как операторы $\dfrac\partial{\partial z}$ и~$\dfrac\partial{\partial\bar z}$ действуют на $z^n$ и~$\bar z^n$?

\vspace{1mm}

\пункт Вещественно-дифференцирумая функция~$f$ комплексно дифференцируема, тогда и~только тогда, когда $\dfrac\partial{\partial\bar z}f=0$; в~этом случае ее (комплексная) производная равна $\dfrac\partial{\partial z}f$.

\vspace{1mm}

\пункт Функция $x+iy\mapsto u(x,y)+iv(x,y)$, где $u$ и~$v$~--- (вещественные) многочлены, голоморфна тогда и~только тогда, когда может быть представлена в~виде $z\mapsto P(z)$ для некоторого (уже комплексного) многочлена~$P$.
\кзадача









% \clearpage
% II. Интеграл
\опр%
Аналогично определению интеграла Римана вещественной функции по отрезку можно определить интеграл $\int\limits_\gamma f(z)\,dz$ комплексной функции по кривой как предел интегральных сумм вида $f(\xi_i)(z_{i}-z_{i-1})$.
\копр




\vspace{-1mm}
\задача
Вычислите интеграл $\smash{\displaystyle\int\limits_{|z|=r}z^n\,dz}$ (для всех целых $n$; обход совершается против часовой стрелки).
\кзадача






\задача
\пункт Если $f(z)=az+b$, то интеграл $\smash{\displaystyle\int f(z)\,dz}$ по границе любого треугольника равен нулю.

\спункт Пусть функция~$f$ голоморфна внутри области~$\Omega$, ограниченной гладкой кривой $\partial\Omega$, и~непрерывна на $\Omega\cup\partial\Omega$. Тогда
\[
\int\limits_{\partial\Omega}f(z)\,dz=0.
\]
\emph{(Последним утверждением можно далее пользоваться без доказательства.)}


\пункт Интеграл голоморфной функции не меняется при деформации контура.
\кзадача






\опр
Пусть функция~$f$ голоморфна в~\emph{проколотой} окрестности точки~$z_0$. Интеграл
\[
\frac1{2\pi i}\int\limits_{\partial U(z_0)}\!f(z)\,dz=:\operatorname{res}_{z_0}f(z)\,dz
\]
($\partial U(z_0)$~--- маленькая кривая, обходящая один раз вокруг точки~$z_0$; в~силу предыдущей задачи от выбора конкретной кривой интеграл не зависит)
называется \emph{вычетом} в~точке~$z_0$.
\копр




\задача
\пункт Найдите вычет в~нуле функции $P(z)=\sum_{n=-N}^{N}a_nz^n$.

\спункт Для аналитических функций определение вычета выше согласовано с~определением (формального) вычета из листка <<Формальные ряды II>>.
\кзадача






\задача
\пункт Индекс особой точки векторного поля, задаваемого голоморфной функцией~$f$, равен вычету $\operatorname{dlog}f:=\frac{f'}f\,dz$ в~этой точке.

\спункт Как обобщить последнее утверждение на произвольные (гладкие) векторные поля?
\кзадача






% Нельзя ли еще как-то потаптаться на вычетах? Какое-то применение, что ли?
\задача
Если функция~$f$ голоморфна в~точке $z_0$, то
\[
\abovedisplayskip=9pt
\belowdisplayskip=0pt
f(z_0)=
\operatorname{res}_{z_0}\frac{f(z)}{z-z_0}dz=
\frac1{2\pi i}\int\limits_{\partial U(z_0)}\frac{f(z)}{z-z_0}\,dz
\]
(``интеграл Коши'').
\кзадача






\задача
\пункт Если две голоморфные функции равны на границе диска, то они равны и~внутри диска.

\пункт Любую ли бесконечно гладкую функцию на границе диска можно продолжить до голоморфной функции на диске?
\кзадача







% \clearpage
\задача
Модуль голоморфной на открытом множестве функции не имеет локальных максимумов на этом множестве (``принцип максимума'').
\кзадача






\задача
Найдите все двоякопериодические (имеющие два линейно независимых над $\R$ периода) голоморфные на всей плоскости функции.
\кзадача






\задача
Выведите из принципа максимума основную теорему алгебры.
% рассматриваем 1/P
\кзадача








%\clearpage
\задача
\пункт Если функция~$f$ голоморфна в~точке~$z_0$, то
\[
\abovedisplayskip=9pt
\belowdisplayskip=7pt
f^{(n)}(z_0)=
\frac{n!}{2\pi i}\int\limits_{\partial U(z_0)}
\!\frac{f(z)}{(z-z_0)^{n+1}}\,dz.
\]

\пункт Любая голоморфная функция является бесконечно комплексно-дифференцирумой.
\кзадача






\задача
Голоморфная на~$\Cbb$ ограниченная функция постоянна (``теорема Лиувилля'').
% можно и без аналитичности, просто доказывается, что производная равна нулю
\кзадача






\задача
Голоморфная в~точке функция аналитична в некоторой окрестности этой точки.
%
\help{Докажите, что $\displaystyle f(z)=f(z_0)+(z-z_0)f'(z_0)+\ldots+(z-z_0)^n\,\frac{f^{(n)}(z_0)}{n!}+\cdots$.}
%Если функция $f$ голоморфна в~точке~$z_0$, то в~некоторой окрестности этой точки
%\[
%f(z)=f(z_0)+(z-z_0)f'(z_0)+\ldots+(z-z_0)^n\,\frac{f^{(n)}(z_0)}{n!}+\ldots
%\]
%(т.\,е. любая голоморфная функция является аналитической).
\кзадача






\задача
Существует не более одного способа продолжить данную фукнцию на вещественной прямой до голоморфной функции на $\Cbb$ (``аналитическое продолжение'').
\кзадача






\задача
Аналитическое продолжение экспоненты дается формулой
\[
\abovedisplayskip=6pt
\exp(\rho+i\varphi)=e^\rho(\cos\varphi+i\sin\varphi).
\]
\кзадача






% ан. прод. четной/нечетной функции?
\задача
Существует~ли голоморфная в~нуле функция~$f$, такая что $f(1/n)=2^{-n}$?
\кзадача











\ЛичныйКондуит{0mm}{6.5mm}
% \GenXMLW
\end{document} 