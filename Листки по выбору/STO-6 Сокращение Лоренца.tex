% !TeX encoding = windows-1251
\documentclass[a4paper,12pt]{article}
\usepackage{newlistok}

\УвеличитьВысоту{2.1cm}
\УвеличитьШирину{1.5cm}
\renewcommand{\spacer}{\vfil}
\newcommand{\wa}{\overrightarrow}
\newcommand{\smat}[1]{\hr{\begin{smallmatrix}#1\end{smallmatrix}}}
\newcommand{\мв}{\,м$_в$}

%\newcommand{\help}[1]{({\small Подсказка: }{\reflectbox{\hbox{\footnotesize{#1}}}})}

\Заголовок{Сокращение Лоренца и другие парадоксы}
\НомерЛистка{STO-6}
\ДатаЛистка{2022.01}

\sloppy

\begin{document}
\СоздатьЗаголовок

\noindent{\small Пока мы по-прежнему считаем мир одномерным.
Все ракеты в этом листке летят вдоль оси $x$ со скоростью $u$.

}


\задача[сокращение Лоренца]
\пункт
Пусть ракета снабжена метровым стержнем, который наблюдается из лаборатории.
Какова его наблюдаемая длина в лаборатории?
\пункт
В начале и в конце стержня закреплены часы, которые в ракете показывают одинаковое время.
Какое время будет наблюдаться в лаборатории?
\кзадача

\задача[замедление времени]
Пусть ракета снабжена настенными часами, которые наблюдаются из лаборатории.
С какой скоростью идут эти часы при наблюдении из лаборатории?
\кзадача

\задача[сложение скоростей]
Пусть ракета снабжена табуреткой, двигающейся со скоростью $v$ относительно ракеты.
С какой скоростью летит табуретка относительно лаборатории?
\кзадача

\задача[парадокс шеста и сарая]
Возьмём шест длины 10\,м и сарай длиной также 10\,м.
Запустим шест так, чтобы в системе отсчёта сарая из-за лоренцева сокращения он имел длину 2\,м.
Тогда в некоторый момент он полностью поместится в сарае.
С другой стороны, с системе шеста сарай имеет длину 2\,м, и шест туда никак не может поместиться. Парадокс.
\кзадача


\задача[бетон и машина времени]
Зальём бетоном сарай так, чтобы оставался только вход.
Возьмём шест длиной 20\,м и разгоним до сокращения до 2\,м.
Как только шест влетит в сарай, зальём бетоном вход (у нас на это будет время точно не меньшее $\frac{8м}{3\cdot10^8 м/с}$).
В системе шеста вся наша затея выглядит комично. Парадокс.
\кзадача



\задача[непригодность ньютоновской механики для космических полётов]
Как должна зависеть скорость ракеты и параметр скорости от времени, чтобы наблюдатель в ракете всё время испытывал ускорение $g$?
Через какое время ракета разгонится до $0{,}9$ скорости света? А что предсказывает Ньютоновская механика?
\кзадача

\задача[парадокс близнецов]
\пункт
Близнецы А и Б расстались в тот день, когда им было по 21 году.
А двигался от Земли со скоростью $0{,}96$ скорости света в течении 7 лет (своего времени) в одну сторону и столько же обратно.
Насколько моложе он будет своего брата Б по возвращении?
\пункт
Однако в системе отчёта близнеца Б Земля улетала от него со скоростью $0{,}96$,
а потом летела к нему со скоростью $0{,}96$.
Поэтому по возвращению близнец А должен быть моложе.
Парадокс.
\кзадача


\раздел{Этот трёхмерный мир.}

% Достаточно изучив двумерный мир, перейдём к трёхмерному.

\задача[Лоренцево сокращение 2]
Придумайте мысленный опыт, подтверждающий, что шест, расположенный перпендикулярно направлению движения, не изменяет своей наблюдаемой длины.
\кзадача

\задача[инвариантный интервал]
Придумайте мысленный опыт, подтверждающий, что во всех системах отсчёта сохраняется число $(\Delta t)^2 - (\Delta x)^2 - (\Delta y)^2$ (время в \мв). Это число называется \выд{интервалом} между событиями.
\кзадача

\задача[преобразование углов]
Метровый стержень в ракете прибит под углом $\ph'$ к оси $x'$.
Под каким углом  к оси $x$ стержень наблюдается в лаборатории?
\кзадача

\задача[преобразование направлений движения]
Пусть табуретка летит со скоростью $v'$ под углом $\ph'$ к оси $x'$ внутри ракеты. Под каким углом  к оси $x$ наблюдается движение табуретки в лаборатории? Чем эта задача отличается от предыдущей?
\кзадача

\задача[эффект \лк прожектора\пк]
\пункт
Табуретка из предыдущей задачи на проверку оказалась фотоном, то есть двигается со скоростью света. Под каким углом к оси $x$ распространяется этот фотон в лаборатории?\\
\пункт
Частица, двигающая с большой скоростью, испускает свет в переднюю полусферу с своей системе отсчёта. Покажите, что в системе лаборатории свет сконцентрируется в узкий конус.
\кзадача

\задача[сложение скоростей 2]
Фотон из предыдущей задачи на проверку оказался табуретом, летящим в ракете вдоль оси $y'$ со скоростью $v'$. С какой скоростью он движется в системе лаборатории?
\кзадача
%
\ЛичныйКондуит{0mm}{6mm}
% \GenXMLW


\end{document}


