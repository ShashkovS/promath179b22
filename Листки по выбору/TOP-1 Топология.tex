% !TeX encoding = windows-1251
\documentclass[a4paper,12pt]{article}
\usepackage{newlistok}

\ВключитьКолонитул
\УвеличитьВысоту{5mm}
\УвеличитьШирину{2mm}



\Заголовок{Топологические пространства и компакты}
\НомерЛистка{TOP-1}
\ДатаЛистка{2022.01}

\begin{document}

\СоздатьЗаголовок

\опр
    Топологическим пространством называется пара $(X,\Tc)$, состоящая из множества $X$ и набора его подмножеств $\Tc=\hc{U_\al}$, которые называются \выд открытыми, так что:\\
    ({\it i\/}) $\es,X\in\Tc$, то есть пустое множество и всё множество $X$ открыты;\\
    ({\it ii\/}) если $U,V\in\Tc$, то $U\cap V\in\Tc$, то есть пересечение двух открытых открыто;\\
    ({\it iii\/}) если $\hc{V_\be}$ --- произвольный набор открытых множеств, то $\bigcup_\be V_\be\in\Tc$, то есть произвольное объединение открытых множеств открыто.

    Также говорят, что $X$ снабжено структурой топологического пространства, или что на $X$ задана топология.
\копр


\задача
    Придумайте несколько топологий на множестве $X=\hc{a,b}$.
\кзадача

\опр
    \выд Окрестностью точки называется любое открытое множество, эту точку содержащее.
\копр

\задача
    \пункт
    Может ли у точки в топологическом пространстве не быть окрестностей?\\
    \пункт
    Докажите, что на любом множестве можно задать топологию.
\кзадача

%\задача
%    Постройте естественную топологию на $\R$.
%\кзадача

\задача
    Пусть $(M,d)$ --- метрическое пространство. Положим открытыми те множества, которые вместе с каждой своей точкой содержат какую-нибудь её $\ep$-окрестность. Докажите, что таким образом  $M$ наделяется структурой топологического пространства.
\кзадача

Таким образом, любое метрическое пространство является топологическим. В частности $\R^n$ с метрикой $d_2$ является топологическим пространством.

\задача
    Докажите, что метрики $d_1$, $d_2$ и $d_\bes$ индуцируют на $\R^n$ одну и ту же топологию.
\кзадача


\задача
    \пункт
    Пусть $Y$ --- некоторое непустое подмножество метрического пространства $(X,d)$. Докажите, что метрика на $X$ индуцирует на $Y$ структуру метрического пространства.\\
    \пункт
    Пусть $Y$ --- некоторое непустое подмножество топологического пространства $(X,\Tc)$. Положим открытыми в $Y$ пересечения открытых в $X$ множеств с $Y$. Докажите, что это правило задаёт на $Y$ топологию, которая называется \выд индуцированной.
\кзадача


\задача
    \пункт
    Докажите, что любое открытое множество в $\R$ можно представить как объединение интервалов (то есть множество интервалов является \выд базой топологии)
    \пункт
    Докажите, что любое открытое множество в $\R$ является объединением непересекающихся интервалов и лучей.
\кзадача


\опр
Точка $x\in X$ называется \выд{предельной точкой множества $M$}, если любая окрестность точки $x$ содержит хотя бы одну точку из $M$ (отличную от $x$).
\копр

\опр
Множество $M$ в топологическом пространстве называется \выд замкнутым, если оно содержит все свои предельные точки.
\копр

\задача
    \пункт
    Докажите, что объединение и пересечение двух замкнутых множеств замкнуто;
    \\\пункт
    Докажите, что произвольное пересечение замкнутых также замкнуто;
    \\\пункт
    Покажите, что произвольное объединение замкнутых подмножеств не обязано быть замкнутым.
    \\\пункт
    Докажите, что множество замкнуто тогда и только тогда, когда его дополнение открыто.
\кзадача

\vfill
\ЛичныйКондуит{0mm}{6mm}
\ОбнулитьКондуит
\newpage

\задача
    Докажите, что шар $B_\ep(x_0)=\hc{x\in M\mid d(x,x_0)\le\ep}$ в метрическом пространстве $M$ является замкнутым.
\кзадача


\задача[принцип вложенных шаров]
    Докажите, что метрическое пространство полно тогда и только тогда, когда любая последовательность вложенных шаров, радиусы которых стремятся к нулю, имеет общую точку.
\кзадача

\сзадача
    Докажите, что стремление радиусов к нулю существенно, то есть существует полное пространство и последовательность вложенных шаров, имеющих пустое пересечение.\\
    \help{Можно построить метрику на $\N$ так, чтобы $d(x,y)>1$ при $x\ne y$}
\кзадача


\опр
    Топологическое пространство называется \выд компактным, если из любого покрытия его открытыми множествами можно выбрать конечное подпокрытие.
\копр


\задача
    \пункт
    Докажите, что любое конечное топологическое пространство компактно.
    \невСтрочку
    \пункт
    Приведите пример некомпактного топологического пространства.
    \пункт
    Докажите, что отрезок компактен.
\кзадача

\сзадача
    Докажите, что подмножество в $(\R^n,d_2)$ является компактным тогда и только тогда, когда оно замкнуто и ограничено.
\кзадача


\задача
    Докажите, что подмножество компактного топологического пространство компактно тогда и только тогда, когда оно замкнуто.
\кзадача


\задача
    \пункт
    Дайте определение \выд непрерывной функции $f\colon (X,\Tc_X)\to (Y,\Tc_Y)$ так, чтобы оно совпадало с определением непрерывности в случае метрических пространств.\\
    \пункт
    Дайте определение непрерывности функции в точке.
    \спункт
    Сформулируйте следующие утверждения как утверждения о непрерывности отображений топологических пространств (в некоторой точке):
    $\limn x_n = a$; $\lim_{x\ra x_0} f(x) = A$; $\lim_{x\ra +\bes} f(x) = A$;  $\lim_{x\ra x_0} f(x) = \bes$.
\кзадача


\задача
    Докажите, что непрерывная функция $f\colon(X,\Tc_X)\to\R$ на компактном $X$ достигает своего наименьшего и наибольшего значения.
\кзадача

\задача
    Докажите, что непрерывный образ компакта --- компакт.
\кзадача

\vfill
\ЛичныйКондуит{0mm}{6mm}
%\СделатьКондуитИз{6.2mm}{6.2mm}{sp_Top.tex}
% \GenXMLW

\end{document} 