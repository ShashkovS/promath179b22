% !TEX encoding = Windows Cyrillic
\documentclass[a4paper, 12pt]{article}
\usepackage[tikz]{newlistok}

\УвеличитьШирину{1.5truecm}
\УвеличитьВысоту{2.5truecm}
\ВключитьКолонтитул

\sloppy


\Заголовок{Кватернионы — 2}
\НомерЛистка{Q-2}
\ДатаЛистка{2022.01}
\begin{document}


\СоздатьЗаголовок


\задача (теорема Фробениуса)
	Пусть $A$ -- конечномерная ассоциативная алгебра с делением размерности $n$. Назовём её элемент $I$ \emph{чисто мнимым}, если $I^2$ -- неположительное вещественное число, и \emph{мнимой единицей}, если $I^2 = -1$. \пункт Докажите, что в $A$ нет \emph{делителей нуля}, то есть если $ab =0$, то либо $a=0$, либо $b=0$. \пункт Докажите, что для любого невещественного $x \in A$ существуют такие $a,b \in \mathbb{R}$, что $a + bx$ -- мнимая единица. (\emph{указание}: так как алгебра конечномерна, то элементы $1,x,x^2,x^3,\ldots,x^n$ линейно зависимы). \пункт Докажите, что множество чисто мнимых элементов $A$ образует векторное подпространство размерности $n-1$. \пункт Докажите, что не существует трёхмерной ассоциативной алгебры с делением. \пункт Докажите, что если $A$ либо одномерна, либо двумерна, и в $A$ есть мнимая единица, либо $A$ четырёхмерна, и в $A$ есть три линейно независимых мнимых единицы $I,J,K$, которые умножаются как кватернионы. Говоря проще, $\mathbb{R}, \mathbb{C}, \mathbb{H}$ -- единственные ассоциативные алгебры с делением.
\кзадача

{\small Имеет место обобщение теоремы Фробениуса, согласно которому алгебры с делением существуют только в размерностях $1,2,4$ и $8$. Единственное известное доказательство этого факта существенно опирается на алгебраическую топологию.}

\опр
	Пусть $s \in \mathbb{H}, s \neq 0$. Операция \emph{сопряжения с помощью} $s$ определяется по формуле $q \mapsto {}^sq = sqs^{-1}$.
\копр

\задача
	Проверьте, что ${}^s(q_1 + q_2) = {}^sq_1+{}^sq_2$, ${}^s(q_1q_2) = {}^sq_1 {}^sq_2$, ${}^sq^{-1} = ({}^sq)^{-1}$, $\overline{{}^sq} = {}^{\overline{s}^{-1}}\overline{q}$.
\кзадача

\задача
	Пусть $s = \alpha + t$, где $\alpha \in \mathbb{R}$, а $t$ -- чисто мнимый. Докажите, что операция сопряжения с помощью $s$ оставляет на месте пространство чисто мнимых кватернионов, и что если $q$ -- чисто мнимый, то ${}^sq$ -- это результат поворота вектора $q$ вокруг оси вектора $t$ на угол $2 \ \mathrm{arctg} \frac{|t|}{|\alpha|}$.
\кзадача

\задача
	Отождествим трёхмерное пространство с пространством чисто мнимых кватернионов. Докажите, что любое вращение трёхмерного пространства, сохраняющее начало координат, и сохраняющее ориентацию, имеет вид $q \mapsto {}^sq$ для некоторого $s$, по модулю равного единице, причём $s$ определён однозначно с точностью до знака.
\кзадача

\задача
	Проверьте, что в условиях предыдущей задачи сопряжения с помощью элементов групп \пункт $T^*$, \пункт $O^*$, \пункт $I^*$ задают в точности группу симметрий тетраэдра, октаэдра и икосаэдра, соответственно. \пункт Почему в нашем списке нет бинарной группы куба и бинарной группы додекаэдра?
\кзадача

\задача
	Пусть $\pi$ -- плоскость в пространстве чисто мнимых кватернионов, и пусть $s$ -- единичный вектор нормали к $\pi$. Запишите формулу для отражения относительно плоскости $\pi$.
\кзадача

\задача
	Пусть $q_1,q_2,q_3$ -- чисто мнимые кватернионы. Запишите формулу для площади прямоугольника, натянутого на $q_1$ и $q_2$ и объёма параллелепипеда, натянутого на $q_1,q_2$ и $q_3$.
\кзадача

\задача[расслоение Хопфа]
	Будем рассматривать множество кватернионов $x+yi+zj+wk$ как единичную сферу в четырёхмерном пространстве, заданную уравнением $x^2+y^2+z^2+w^2=1$. Обозначим его через $S^3$. \пункт Докажите, что ${s \in S^3 : {}^sk = k}$ -- это окружность, то есть пересечение $S^3$ с двумерной плоскостью в $\mathbb{R}^4$. \пункт Пусть $q$ -- чисто мнимый кватернион, по модулю равный единице. Докажите, что множества $S_q = \{s \in S^3 : {}^sk=q \}$ -- попарно непересекающиеся окружности, на которые разбивается (или, как ещё говорят, расслаивается) трёхмерная сфера. Стереографическая проекция $S^3 \to \mathbb{R}^3$ определяется так: вложим $\mathbb{R}^3$ в $\mathbb{R}^4$ как гиперплоскость, заданную уравнением $w=0$, соединим точку $s \in S^3$ с северным полюсом $(0,0,0,1)$, и продлим полученную прямую до пересечения с $\mathbb{R}^3$. \пункт Напишите формулу для стереографической проекции. \пункт Докажите, что окружности, не проходящие через северный полюс, при стереографической проекции переходят в окружности, а проходящие через северный полюс -- в прямые. Нарисуйте несколько окружностей из расслоения Хопфа после стереографической проекции и проверьте, что они \emph{зацеплены}, то есть круг, ограничивающийся одной из окружностей, пересекает все другие, причём ровно в одной точке.
\кзадача

{\small С помощью восьмимерной алгебры $\mathbb{O}$ можно построить аналог расслоения Хопфа, разбив семимерную сферу на зацепленные трёхмерные сферы. Несуществование расслоений такого типа для $n \neq 1, 3, 7$ и позволяет доказать теорему о размерностях алгебр с делением.}


\ЛичныйКондуит{0mm}{6mm}
% \GenXMLW
%\СделатьКондуит{5.4mm}{7mm}

\end{document}
