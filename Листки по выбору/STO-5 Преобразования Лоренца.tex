% !TeX encoding = windows-1251
\documentclass[a4paper,12pt]{article}
\usepackage[mag=1000]{newlistok}

\УвеличитьВысоту{2.5cm}
\УвеличитьШирину{1.9cm}

\newcommand{\wa}{\overrightarrow}
\newcommand{\smat}[1]{\hr{\begin{smallmatrix}#1\end{smallmatrix}}}
\newcommand{\мв}{\,м$_в$\xspace}
\DeclareMathOperator{\Arsh}{Arsh}
\DeclareMathOperator{\Arch}{Arch}
\DeclareMathOperator{\Arth}{Arth}
\DeclareMathOperator{\Arcth}{Arcth}
\DeclareMathOperator{\gd}{gd}


\global\abovedisplayskip=-11pt
\global\belowdisplayskip=-11pt
\global\abovedisplayshortskip=-11pt
\global\belowdisplayshortskip=-11pt


\Заголовок{Преобразования Лоренца}
\НомерЛистка{STO-5}
\ДатаЛистка{2022.01}

\sloppy

\begin{document}
\СоздатьЗаголовок

{\footnotesize
\noindent\textbf{Секунда и метр}.
Время и расстояния можно точно определить независимо от системы счисления даже в СТО (см. предыдущий листок).
% Секунда есть время, равное 9\,192\,631\,770 периодам излучения, соответствующего переходу между двумя сверхтонкими уровнями основного состояния атома цезия-133, а метр равен расстоянию, которое проходит свет в вакууме за $1/299\,792\,458$ секунды.
Конечно же, мы требуем, чтобы метр и секунда во всех системах отсчёта были <<одинаковыми>>.
Также мы будем требовать, чтобы сохранялось направление течения времени (например, следя за тем, чтобы распад частицы оставался распадом).

Опыты Майкельсона--Морли и Кеннеди--Торндайка показали, что скорость света \лк почти\пк не зависит от системы отсчёта.
Если быть точнее, то было установлено, что скорости света во всех направлениях в двух системах отсчёта, двигающихся относительно друг друга со скоростью 60\,км/с, отличаются не более, чем на 2\,м/с.
Позднее, постоянство скорости света было проверено множеством различных способов и с куда большей точностью.

Таким образом возникает основной постулат СТО: скорость света постоянна во всех системах отсчёта.
Скорость света обозначается через $c$. ($c\approx299\,792,458$\,м/с)
Преобразование пространства-времени $\R^4$, сохраняющие 1 метр, 1 секунду и скорость света называются преобразованиями \выд Лоренца.
Предполагается, что ничто не способно двигаться в пространстве-времени быстрее, чем со скоростью света.

Для удобства будем измерять время в метрах$_в$ (и писать \мв).
\мв --- время, за которое свет пролетает один метр.
Это удобно потому, что скорость света становится равной 1\,м/\мв.
Преобразования координат между инерциальными системами отсчёта в СТО называются  \выд{преобразованиями Лоренца}.
\par
}

\раздел{Одномерный мир}

\задача[Парадокс поезда]
Пусть на поезде, движущемся со скоростью, близкой к скорости света (такой поезд, видимо, стоит ожидать раньше всего в Японии (если где-нибудь ещё не научатся значительно влиять на скорость света)), едут три человека: $A$ в голове, $O$ --- в середине и  $B$ --- в хвосте поезда. На земле около пути стоит четвёртый человек $O'$. В тот самый момент, когда $O$ проезжает мимо $O'$, сигналы ламп от $A$ и $B$ достигают $O$ и $O'$.
Покажите, что на вопрос\лк Кто раньше включил фонарь?\пк наблюдатели $O$ и $O'$ дадут различные ответы.
\кзадача




\задача
Найдите все возможные мировые линии света в одномерном мире $\R^2$.
\кзадача

\задача
Изобразите в $\R^2$ и $\R^3$ множество точек:
\пункт в которые можно попасть из данной (это множество называется \выд конусом \выд будущего);
\пункт в которые можно посветить из данной;
\пункт из которых можно попасть в данную (\выд конус \выд прошлого). Какой физический смысл конуса будущего и прошлого?
\кзадача


\задача
Преобразования Лоренца будучи аффинным имеют вид $f\rbmat{x\\t} = \rbmat{a&b\\c&d}\cdot\rbmat{x\\t}+\rbmat{x_0\\t_0}$.
\\\пункт
Покажите, что $|a+b|=|c+d|$ и $|b-a|=|d-c|$;
\\\пункт
Покажите, что матрица $A=\smat{a&b\\c&d}$ имеет вид либо $\smat{\al&\be\\\be&\al}$, либо $\smat{-\al&-\be\\\phantom{-}\be&\phantom{-}\al}$;
\\\пункт
Покажите, что определитель матрицы $A$ должен быть равен $\pm1$;
\\\пункт
Покажите, что $\al=\ch\phi$, а $b=\sh\phi$ для некоторого $\phi$.
Число $\phi$ называется при этом \выд{параметром скорости} или \выд{быстротой}.
\кзадача


\задача
Фраза <<перейти в систему отсчёта ракеты>> означает перейти в такую систему отсчёта,
в которой мировая линия ракеты является прямой $(0,t), t\in\R$.
\\\пункт
Преобразование координат для перехода в систему отсчёта ракеты задаётся матрицей $\smat{\ch\ph&\sh\ph\\\sh\ph&\ch\ph}$.
С~какой скоростью летит ракета?
\пункт
Ракета движется в системе отсчёта лаборатории со скоростью $u$.
Найдите преобразование, позволяющее перейти в систему отсчёта ракеты,
а также преобразование, которое позволяет из системы отсчёта ракеты перейти в систему отсчёта лаборатории.
\кзадача



\задача[Эффект Допплера]
Ракета движется в системе отсчёта лаборатории со скоростью $u$.
Каждую секунду в точке с координатой 0 в лаборатории моргает лампочка.
С каким интервалом на ракете наблюдаются вспышки?
\кзадача


\задача
\пункт
Докажите, что для любой пары различных событий найдётся ракета, в системе которой события либо одновременны (говорят, что интервал между ними \выд пространственноподобный), либо происходят в одной и той же точке пространства (интервал временноподобный), либо принадлежат мировой линии света (интервал \выд светоподобный).
\\\пункт
Пускай одно из событий находится в начале координат.
Найдите множество точек пространства-времени, для которых интервал пространственно-, временно- и светоподобный.
\кзадача


\задача
В системе отсчёта лаборатории два события происходят одновременно
\пункт
в одной и той же точке;
\пункт
в разных точках.
В каких ещё системах отсчёта эти события также будут одновременными?
\пункт
Докажите, что для любых событий, соединяемых пространственноподобным интервалом, найдётся две ракеты такие, что в системе одной первое событие происходит раньше, а в системе другой --- наоборот.
\кзадача





\ЛичныйКондуит{0mm}{5mm}
% \GenXMLW

\end{document}























