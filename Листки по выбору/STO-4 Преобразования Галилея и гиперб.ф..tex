% !TeX encoding = windows-1251
\documentclass[a4paper,11pt]{article}
\usepackage[mag=1000]{newlistok}

\УвеличитьВысоту{2.5cm}
\УвеличитьШирину{1.9cm}

\newcommand{\wa}{\overrightarrow}
\newcommand{\smat}[1]{\hr{\begin{smallmatrix}#1\end{smallmatrix}}}
\newcommand{\мв}{\,м$_в$}
\DeclareMathOperator{\Arsh}{Arsh}
\DeclareMathOperator{\Arch}{Arch}
\DeclareMathOperator{\Arth}{Arth}
\DeclareMathOperator{\Arcth}{Arcth}
\DeclareMathOperator{\gd}{gd}


% \renewcommand{\spacer}{\vspace{2pt}}



\Заголовок{Преобразования Галилея и гиперболические функции}
\НомерЛистка{STO-4}
\ДатаЛистка{2022.01}


\sloppy

\begin{document}
\СоздатьЗаголовок

{\footnotesize
\noindent\textbf{Напоминание}.
Мы уже выяснили две основных вещи.
Преобразование координат между двумя инерциальными системами отсчёта должно быть аффинно.
Аффинное преобразование $f\colon\R^m\to\R^m$ может быть записано в координатах виде $f(\vec{x}) = A\cdot \vec{x} + \vec{w_0}$,
где $A$ --- матрица линейного преобразования, а $\vec{w_0}$ --- некоторый вектор.

\noindent\textbf{Секунда и метр}.
Оказывается, что время и расстояния можно точно определить независимо от системы счисления.
Так секунда есть время, равное 9\,192\,631\,770 периодам излучения, соответствующего переходу между двумя сверхтонкими уровнями основного состояния атома цезия-133, а метр равен расстоянию, которое проходит свет в вакууме за промежуток времени, равный $1/299\,792\,458$ секунды.
Конечно, мы требуем, чтобы метр и секунда во всех системах отсчёта совпадали.

\noindent\textbf{Преобразования Галилея}.
В классической теории мы не властны над временем.
Это означает, что если $f$ --- преобразование координат между инерциальными системами отсчёта в классической теории (преобразование \выд{Галилея}), то $f(x,y,z,t) = (*,*,*,t+t_0)$.
\par}


\задача[одномерный классический мир]
Будем рассматривать одномерный мир: одна координата в пространстве и одна во времени.
\пункт
Докажите, преобразование имеет вид $f\rbmat{x\\t} = \rbmat{a&b\\0&1}\cdot\rbmat{x\\t}+\rbmat{x_0\\t_0}$;
\пункт
Покажите, что число $a$ равно либо $1$, либо $-1$;
\пункт
За что <<отвечают>> каждое из чисел $a$, $b$, $x_0$ и $t_0$?
\кзадача


\задача
\пункт
На обычной плоскости заданы два обычных вектора $(x_1,y_1)$ и $(x_2,y_2)$. Докажите, что площадь параллелограмма, натянутого на эти вектора равна $x_1y_2-x_2y_1$. Это число называется \выд определителем матрицы $\smat{x_1&y_1\\x_2&y_2}$.
Что происходит с определителем, если
\пункт
переставить строки или столбцы?
\пункт
умножить строку или столбец на число?
\пункт
к одной строке прибавить другую, умноженную на число?
\кзадача

%
% {\small
% \соглашение
% Во всех случаям мы может делать <<параллельный перенос>> координат.
% За него отвечает вектор $\vec{w_0}$ из формулы в напоминании.
% На этот вектор не накладывается никаких ограничений.
% А вот на линейную часть, задаваемую матрицей $A$ ограничения наоборот возникают естественным образом.
% Именно эти ограничения мы изучаем в этом листке.
% Поэтому будем считать, что вектор $\vec{w_0}$ нулевой, то есть в момент времени 0 мы запускаем ракету из начала координат системы отсчёта лаборатории.
% Время на ракете в этот момент также установлено в 0.
% \ксоглашение
% }


\задача[двумерный классический мир]
\label{3dworld}
Будем рассматривать двумерный мир: две координаты в пространстве и одна во времени.
\пункт
Докажите, преобразование имеет вид $f\rbmat{x\\y\\t} = \rbmat{a&b&\al\\c&d&\be\\0&0&1}\cdot\rbmat{x\\y\\t}+\rbmat{x_0\\y_0\\t_0}$;
\\\пункт
За что <<отвечают>> числа $\al$ и $\be$?
\\
\пункт
Из физических соображений покажите, что
$a^2+c^2=1$,  $b^2+c^2=1$ и $ab+cd=1$;
\\\пункт
Докажите, что матрица $\smat{a&b\\c&d}$ имеет определитель, равный $1$ или $-1$;
\\\пункт
Известно, что существует  физический опыт, который позволяет вне зависимости от системы отсчёта определить вращение <<по часовой стрелке>>.
Покажите, что определитель из предыдущего пункта равен 1;
\\\пункт
Докажите, что матрица $\smat{a&b\\c&d}$ имеет вид $\smat{\cos\phi&-\sin\phi\\\sin\phi&\phantom{-}\cos\phi}$.
Какой физический смысл числа~$\phi$?
\кзадача

\раздел{Гиперболические функции}
\задача
Гиперболические функции — семейство элементарных функций, выражающихся через экспоненту и тесно связанных с тригонометрическими функциями.
По определению $\ch \ph = \dfrac{e^\ph + e^{-\ph}}{2}$ (\выд{гиперболический синус, чинус}), $\sh \ph = \dfrac{e^\ph - e^{-\ph}}{2}$ (\выд{шинус}), $\th\ph = \dfrac{\sh\ph}{\ch\ph}$, $\cth{\ph}=\dfrac{\ch\ph}{\sh\ph}$.
\невСтрочку
\пункт
Нарисуйте графики гиперболических функций.
\пункт[Основное соотношение]
Докажите, что $\ch^2\ph - \sh^2\ph = 1$;
\пункт[Геометрическое определение]
Как связаны гиперболические функции с гиперболой?
\пункт[Формулы сложения]
Выразите $\sh(x\pm y)$ и $\ch(x\pm y)$ через $\sh x$, $\ch x$, $\sh y$ и $\ch y$;
\пункт Выразите $\th(x\pm y)$ через $\th x$ и $\th y$;
\пункт[Производные]
Найдите производные гиперболических функций;
\пункт[Обратные гиперболические функции]
Обратные гиперболические функции обозначаются через $\Arsh$, $\Arch$, $\Arth$ и $\Arcth$,
и читаются как \выд{Ареа-синус} (от \выд{area}), \выд{Ареа-косинус} и т.д.
\\Выразите $\Arsh x$, $\Arch x$ и $\Arth x$ через $\ln$ и $x$.
\сспункт[Связь с тригонометрическими функциями]
Докажите, что $\sh x = -i\sin(ix)$, $\ch x = \cos(ix)$, $\th x = -i\tg(ix)$.
\vspace*{-3mm}
\сспункт[Функция Гудермана]
Функция \выд Гудермана определяется через интеграл: $\displaystyle\gd(x)=\int\limits_0^x \dfrac{dt}{\ch t}$.
\vspace*{-3mm}
\\Докажите, что $\gd(x) = \arctg(\sh(x))$, $\sh(x)=\tg(\gd(x))$, $\sin(\gd(x))=\th(x)$.
\кзадача

\задача
Пусть преобразование координат задаётся матрицей $A = \smat{\ch\ph&-\sh\ph\\-\sh\ph&\ch\ph}$.
\\\пункт
Куда это преобразование переводит прямые $y=0$, $y=x/2$, $y=x$, $y=2x$ и $x=0$?
\\\пункт
Докажите, что преобразование $A$ в области $y\ge x$ сохраняет \выд{интервал} --- величину $\sqrt{y^2-x^2}$.
\кзадача


\ЛичныйКондуит{0mm}{5mm}
% \GenXMLW

\end{document}





% Пусть преобразование координат задаётся матрицей $A = \smat{\ch\ph&-\sh\ph\\-\sh\ph&\ch\ph}$.
% С какой скоростью движется лаборатория в системе отсчёта ракеты?
% \пункт
% Куда переходит мировая линия центра лаборатории при замене координат?
% \пункт
% Вычислите $\Arsh u$, $\Arch u$ и $\Arth u$ через $u$.
% \кзадача




















\раздел{Специальная теория относительности}
{\footnotesize
Опыты Майкельсона--Морли и Кеннеди--Торндайка показали, что скорость света \лк почти\пк не зависит от системы отсчёта.
Если быть точнее, то было установлено, что скорости света во всех направлениях в двух системах отсчёта, двигающихся относительно друг друга со скоростью 60\,км/с, отличаются не более, чем на 2\,м/с.
Позднее, постоянство скорости света было проверено множеством различных способов и с куда большей точностью.

Постулат СТО: скорость света постоянна во всех системах отсчёта.
Ничто не может передвигаться быстрее скорости света. Скорость света обозначается через $c$. ($c\approx299\,792,458$\,м/с)
Преобразование пространства-времени $\R^4$, удовлетворяющие этому условию называются преобразованиями \выд Лоренца.

Для удобства будем измерять время в метрах$_в$ (и писать \мв). \мв --- время, за которое свет пролетает один метр.
Это удобно потому, что скорость света становится равной 1\,м/\мв.
Преобразования координат между инерциальными системами отсчёта в СТО называются  \выд{преобразованиями Лоренца}.

}

\задача
Найдите все возможные мировые линии света в одномерном мире $\R^2$.
\кзадача

\задача
Изобразите в $\R^2$ и $\R^3$ множество точек:
\пункт в которые можно попасть из данной (это множество называется \выд конусом \выд будущего);
\пункт в которые можно посветить из данной;
\пункт из которых можно попасть в данную (\выд конус \выд прошлого). Какой физический смысл конуса будущего и прошлого?
\кзадача


\задача
Преобразования Лоренца будучи аффинным имеют вид $f\rbmat{x\\t} = \rbmat{a&b\\c&d}\cdot\rbmat{x\\t}+\rbmat{x_0\\t_0}$.
\\\пункт
Покажите, что $|a+b|=|c+d|$ и $|b-a|=|d-c|$;
\\\пункт
Покажите, что матрица $A=\smat{a&b\\c&d}$ имеет вид либо $\smat{\al&\be\\\be&\al}$, либо $\smat{-\al&-\be\\\phantom{-}\be&\phantom{-}\al}$.
\кзадача


\задача
По определению $\ch \ph = \dfrac{e^\ph + e^{-\ph}}{2}$ (\выд{чинус}), $\sh \ph = \dfrac{e^\ph - e^{-\ph}}{2}$ (\выд{шинус}), $\th\ph = \dfrac{\sh\ph}{\ch\ph}$, $\cth{\ph}=\dfrac{\ch\ph}{\sh\ph}$.
\пункт
Докажите, что $\ch^2\ph - \sh^2\ph = 1$;
\пункт
Пусть преобразование координат задаётся матрицей $A = \smat{\ch\ph&-\sh\ph\\-\sh\ph&\ch\ph}$.
С какой скоростью движется лаборатория в системе отсчёта ракеты?
\пункт
Куда переходит мировая линия центра лаборатории при замене координат?
\пункт
Вычислите $\Arsh u$, $\Arch u$ и $\Arth u$ через $u$.
\кзадача
