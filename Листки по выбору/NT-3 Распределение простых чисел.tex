% !TeX encoding = windows-1251
\documentclass[a4paper,12pt]{article}
\usepackage{newlistok}
% \renewcommand{\spacer}{\vfil}

\УвеличитьВысоту{2.5cm}
\УвеличитьШирину{1.9cm}

\Заголовок{Распределение простых чисел}
\НомерЛистка{NT-3}
\ДатаЛистка{2022.01}

\begin{document}

\СоздатьЗаголовок

\noindent В этом листке мы будем часто использовать следующие
обозначения:\\
$p_n$~--- $n$-е простое число ($p_1 = 2$, $p_2 = 3$, $p_3 = 5$,
\dots);
$P$~--- множество всех простых чисел ($P = \{p_1, p_2, p_3,\dots\}$);\\
$\pi(x)$~--- количество простых чисел, не превосходящих $x\in\N$;
$\log x$~--- двоичный логарифм $x$ (т.~е.~$\log_2 x$).

\medskip
История определения асимптотики функции  $\pi(x)$ такова:

\begin{nums}{-4}
\item Евклид: $\pi(x) \to \infty$ при $x \to \infty$;
\item Эйлер: $\dfrac{\pi(x)}{x} \to 0$ при $x \to \infty$;
\item \vspace*{-3mm}
Чебышёв (1848 г.): Если предел $\dfrac{\pi(x)\ln(x)}{x}$ существует, то он равен 1;
\item Адамар и Валле-Пуссен (1896 г.): $\dfrac{\pi(x)\ln(x)}{x} \to 1$ при $x \to \infty$.
\end{nums}




\задача Докажите, что при $n\in\N$ \вСтрочку \пункт $p_{n + 1} \leq
p_1 \cdot p_2\dots\cdot p_n + 1$; \пункт $p_n \leq 2^{2^{n - 1}}$.
\кзадача

\задача Докажите, что $\pi(x) \geq \log\log x$ при $x \geq 2$.
\кзадача

\опр Пусть $n\in\N$. Определим функцию $F^n\colon \N \to \N$
следующим образом: $F^n$ есть количество натуральных чисел, не
превосходящих $x$, все простые делители которых принадлежат
множеству $\{p_1,p_2,\dots,p_n\}$. \копр

\задача \вСтрочку \пункт Найдите  $F^3(57)$; \пункт Найдите $F^n(x)$
при $x < p_{n + 1}$. \кзадача

\задача Докажите, что $F^n(x) \leq 2^n \cdot \sqrt x$. \кзадача

\задача Докажите следующие утверждения:\\
\вСтрочку \пункт простых чисел бесконечно много; \пункт
$\pi(x)\geq0{,}5\cdot\log x$; \спункт ряд $\displaystyle{\frac1{p_1} +
\frac1{p_2} + \dots}$ расходится. \кзадача

\vspace*{-2mm}
\раздел{$***$}
\vspace*{-2mm}

\задача Докажите следующие утверждения:\\
\вСтрочку \пункт $\prod\limits_{n < p \leq 2n,\atop p\in P} p <
C^n_{2n} < 2^{2n}$; \пункт $\prod\limits_{n + 1 < p \leq 2n +
1,\atop p\in P} p < C^n_{2n + 1} < 2^{2n}$; \пункт
$\prod\limits_{p\leq x,\atop p\in P} p < 2^{2x}$.
%$$
%{\bf\hbox{а)}}\prod\limits_{n < p \leq 2n,\atop p\in P}\!\!\!p \
%<C^n_{2n}\ <\ 2^{2n}; \qquad {\bf\hbox{б)}}
%\prod\limits_{n+1<p\leq2n+1,\atop p\in P}\!\!\!p\ <C^n_{2n+1}\ <\
%2^{2n}; \qquad {\bf\hbox{в)}}\prod\limits_{p\leq x,\atop p\in P}\!p\
%<\ 2^{2x}.
%$$
\кзадача

\задача Докажите следующие утверждения: \сНовойСтроки \пункт
$(\pi(x) - \pi([\sqrt x])) \cdot \log\sqrt x < 2x$; \пункт
существует такое $c_1\in\R$, что $\displaystyle{\pi(x) \leq
c_1\cdot\frac{x}{\log x}}$ при $x \geq 2$. \кзадача

\задача Пусть $n\in\N$, $p$~--- простое число. Докажите, что $p$
входит в каноническое разложение числа  $n!$ в степени
$\sum\limits_{i = 1}^{m}[n/p^i]$, где $m = [\log_p n]$. \кзадача

\задача Пусть $p$~--- простое число, $\alpha_p$~--- степень, в
которой $p$ входит в каноническое разложение числа  $C^n_{2n}$.
Докажите, что $\alpha_p \leq [\log_p 2n]$. \кзадача

\задача Докажите следующие утверждения: \вСтрочку \пункт
$\displaystyle{\frac{2^{2n}}{2n + 1} \leq C_{2n}^{n}}$; \пункт
$C_{2n}^{n} \leq \prod\limits_{p \leq 2n,\atop p\in P}\!p^{[\log_p
2n]}$. \кзадача

\задача Докажите следующие утверждения: \сНовойСтроки \пункт $2n -
\log(2n + 1) \leq \pi(2n) \cdot \log 2n$; \пункт существует такое
положительное $c_2\in\R$, что $\displaystyle{\pi(x) \geq c_2 \cdot
\frac{x}{\log x}}$ при $x \geq 2$. \кзадача


\сзадача Докажите, что для всякого достаточно большого $x\in\N$
справедливы неравенства
  $$
  0{,}9 \cdot \frac{x}{\log x} \leq \pi(x) \leq 4{,}1 \cdot
  \frac{x}{\log x}.
  $$
\vspace*{-5mm}
\кзадача

\сзадача Докажите, что при всяком достаточно большом $n\in\N$ между
$n$ и $5n$ обязательно найдется простое число. \кзадача

\задача В обозначениях задачи 9 докажите следующие утверждения:\\
\вСтрочку \пункт $\alpha_p \leq 1$ при $p > \sqrt{2n}$; \пункт
$\alpha_p = 0$ при $2n/3 < p \leq n$. \кзадача

\сзадача[Постулат Бертрана] Докажите, что при всяком достаточно
большом $n\in\N$ между $n$ и $2n$ обязательно найдется простое
число. \кзадача





% В этом листке мы докажем неравенства
% $$
% a\, \dfrac{x}{\ln x} \le \pi(x) \le b\, \dfrac{x}{\ln x}.
% $$
% Константы, которые получатся у нас, будут такими:
% $a = \frac{\ln2}{2} \approx 0.3465$, а $b = 5\ln2 \approx 3.4657$.
% У~Чебышёва константы были более точные: $a \approx 0.92129$, $b \approx 1.10555$.
%
% \задача[Нижняя оценка для НОК]
% Обозначим $\text{НОК}\hs{1,2,\ldots,2n+1}$ через $K$, а $\displaystyle\int\limits_0^1\br{x(1-x)}^n\,dx$ через~$I$.
% Докажите, что:
% \пункт
% $I < \dfrac{1}{4^n}$;
% \пункт
% число $K\cdot I$ целое;
% \пункт
% $K > 4^n$.
% \кзадача
%
% \задача[Нижняя оценка для $\pi(x)$]
% В обозначениях предыдущей задачи докажите, что
% \\
% \пункт
% $K<(2n+1)^{\pi(2n+1)}$;
% \пункт
% $\pi(2n+1) > \dfrac{2n}{\log_2(2n+1)}$;
% \пункт
% $\frac{\ln2}{2}\, \dfrac{x}{\ln x} \le \pi(x)$
% \кзадача
%
% \задача[Оценка произведения простых чисел]
% \невСтрочку
% \пункт
% Докажите, что число $C_{2m-1}^m$ больше произведения всех простых чисел, больших $m$, но меньших~$2m$;
% \medskip
% \пункт
% Докажите, что $\displaystyle \prod\limits_{p\le x} p < 4^x$, где $\displaystyle \prod\limits_{p\le x} p$ --- произведение всех простых чисел, не превосходящих $x$.
% \кзадача
%
% \задача[Верхняя оценка для $\pi(x)$]
% Докажите, что
%
% \medskip
% \пункт
% $\pi(x)^{\pi(x)/2} \le \pi(x)! \le 4^x$;
% \пункт
% $\pi(x) \le 5\ln2\, \dfrac{x}{\ln x}$.
% \кзадача
%
% \задача
% Пусть $p_1, p_2,\ldots,$ --- последовательность всех простых чисел.
% Докажите, что найдутся такие константы $\al$ и $\be$, что $\al n \ln n < p_n < \be n \ln n$ для всех $n$.
% \кзадача
%
% \задача
% Докажите, что ряд из обратных простых чисел расходится.
% \кзадача


%\vfill
\ЛичныйКондуит{0mm}{6mm}
% \GenXMLW

\end{document}
