% !TeX encoding = windows-1251
\documentclass[a4paper,12pt]{article}
\usepackage[mag=950]{newlistok}

\ВключитьКолонитул
\УвеличитьВысоту{25mm}
\УвеличитьШирину{19mm}

\expandafter\def\expandafter\normalsize\expandafter{%
    \normalsize
    \setlength\abovedisplayskip{2pt}
    \setlength\belowdisplayskip{2pt}
    \setlength\abovedisplayshortskip{2pt}
    \setlength\belowdisplayshortskip{1pt}
}
\sloppy

\newcommand{\0}[1]{\overline{#1}}
\renewcommand{\C}{\mathbb{C}}
\НомерЛистка{TOP-3}
\ДатаЛистка{2022.01}
\Заголовок{Гомотопии, степень отображения и основная теорема алгебры}
\begin{document}

\СоздатьЗаголовок

\vspace{-2mm}
\раздел{Гомотопии}
\vspace{-2mm}

\опр
Пусть $X$, $Y$ --- топологические пространства, $f,g$ --- два непрерывных
отображения $X$ в $Y$. Отображения $f$ и $g$ называются
\выд{гомотопными}, если существует такое непрерывное отображение
$F:[0,1]\times X\to Y$, что $F(0,x)=f(x)$, $F(1,x)=g(x)$ для любого
$x\in X$. Отображение $F$ называется \выд{гомотопией, связывающей $f$~с~$g$}.
\копр


\задача
Докажите, что гомотопность отображений --- отношение эквивалентности.
\кзадача

\задача
Докажите, что любые два непрерывных отображения из $X$ в $Y$ гомотопны,
если
\вСтрочку
\пункт $X$ --- любое, $Y=[0,1]$;
\пункт $X$ --- любое, $Y=\R$;
\пункт $X=[0,1]$, $Y$ --- линейно связное.
\кзадача

\опр
Пусть $X$ --- метрическое пространство, $a,b$ --- две его точки,
$f(t),g(t)$ --- два пути с началом в точке $a$ и концом в точке $b$.
Пути $f$ и $g$ называются \выд{гомотопными}, если существует такое
непрерывное отображение $F:[0,1]\times[0,1]\to X$, что $F(0,t)=f(t)$,
$F(1,t)=g(t)$, $F(\tau,0)=a$, $F(\tau,1)=b$ для любых $t,\tau\in[0,1]$.
Отображение $F$ называется \выд{гомотопией, связывающей $f$ с $g$}.
\копр



\noindent
{\bf Напоминание}.
Мы обозначаем через $S^1$ окружность единичного
радиуса в плоскости $\R^2$ с центром в начале координат, причем саму
плоскость мы отождествляем со множеством комплексных чисел $\C$.
Таким образом,
$$
S^1=\{z\in\C\mid |z|=1\}=\{e^{2\pi it}\mid t\in[0,1]\}.
$$



\задача
Пусть $f,g$ --- два пути в $X$. Определим отображение $h$ окружности
$S^1$ в $X$, положив
$$
h\left(e^{2\pi it}\right)=\left\{
\begin{array}{ll}
f(2t),&\mbox{при }0\le t\le 1/2,\\
g(2-2t),&\mbox{при }1/2\le t\le 1.
\end{array}
\right.
$$

Докажите, что пути $f$ и $g$ гомотопны тогда и только тогда, когда
отображение $h$ гомотопно постоянному отображению (то есть, переводящему
окружность $S^1$ в одну точку).
\кзадача

\раздел{Степень отображения}

Степенью непрерывного отображения окружности в себя называется, говоря
неформально, число раз, которое окружность на себя \лк наматывается\пк\
при этом отображении. Для того, чтобы дать точное определение, нам
понадобится некоторая подготовка.



\опр
Пусть $f:X\to S^1$ и $g:X\to \R$ --- непрерывные отображения
метрического пространства~$X$.
Отображение $g$ называется \выд{поднятием} отображения $f$ на $\R$,
если $f(x)=e^{2\pi ig(x)}$ для любого $x\in X$.
\копр

\задача
Пусть $X=[0,1]$, $f(x)=e^{2\pi ix}$. Опишите все
поднятия отображения $f$ на $\R$.
\кзадача

\задача
Пусть $X=S^1$, $f(x)=x$. Существует ли поднятие у этого отображения?
\кзадача




\задача
Пусть $f$ --- некоторый путь в окружности $S^1$, то есть непрерывное
отображение $f:[0,1]\to S^1$.
\сНовойСтроки
\пункт
Докажите, что у пути $f$ существует поднятие на прямую $\R$.
\пункт
Пусть $g_1,g_2$ --- поднятия $f$ на $\R$.
Докажите, что $g_1(t)-g_2(t)=k$, где $k$ --- некоторая целая константа.
\кзадача


\noindent{\bf Указание.} Рассмотрите сначала путь $f$, лежащий в
полуокружности; затем  воспользуйтесь равномерной непрерывностью
непрерывной функции на отрезке.


\опр
Пусть $h:S^1\to S^1$ --- непрерывное отображение. Рассмотрим путь
$f(t)=h(e^{2\pi it})$ и его поднятие $g(t)$. Число $g(1)-g(0)$
называется \выд{степенью отображения $h$}.
\копр

\задача
Докажите, что степень отображения $h:S^1\to S^1$
определена корректно (то есть не зависит от выбора поднятия).
\кзадача

\задача
Чему равна степень отображения $h(z)=z^n$, где $n$ --- целое число?
\кзадача





\задача
Пусть $h:S^1\to S^1$ --- непрерывное отображение, $w_0\in S^1$. Назовем
точку $z_0\in S^1$ \выд{положительным прообразом} точки $w_0$, если
$h(z_0)=w_0$ и при проходе $z$ через $z_0$ против часовой стрелки
значение $h(z)$ проходит через $w_0$ в том же направлении. Строго
последнее условие записывается так: для всех $z$ из некоторой
окрестности точки $z_0$ комплексное число
$h(z)/w_0$ имеет мнимую часть того же знака, что и мнимая часть $z/z_0$.
Аналогично определяем \выд{отрицательный прообраз} точки $w_0$ (при
проходе через него против часовой стрелки $h(z)$ проходит через $w_0$ по
часовой стрелке).

Предположим, что у точки $w_0$ конечное число прообразов, причем все они
либо положительные, либо отрицательные. Докажите, что степень
отображения $h$ равна разности числа положительных прообразов точки $w_0$
и числа ее отрицательных прообразов.
\кзадача

\vfill
\ЛичныйКондуит{0mm}{6mm}
\ОбнулитьКондуит
\newpage


\задача
\пункт
Докажите, что у любого непрерывного отображения квадрата $[0,1]\times
[0,1]$ в окружность есть поднятие на прямую.
\пункт
Пусть $f_1$ и $f_2$ --- гомотопные пути в окружности. Докажите,
что у них существуют гомотопные поднятия на прямую.
\кзадача


\задача
Докажите, что степень отображения $f:S^1\to S^1$ не меняется при гомотопии.
\кзадача

\задача
\пункт
Докажите, что если отображение $f:S^1\to S^2$ не является
сюрьекцией, то оно гомотопно отображению в точку.
\пункт
Докажите, что двумерная сфера $S^2$ не гомеоморфна двумерному тору
$T^2=S^1\times S^1$.
\кзадача

\задача
Отображение окружности в себя называется \выд{нечетным}, если
диаметрально противоположные точки переходят в диаметрально
противоположные, и \выд{четным}, если диаметрально противоположные точки
переходят в одну и ту же точку. Докажите, что степень нечетного
отображения --- нечетное число, а степень четного отображения --- четное
число.
\кзадача

\раздел{Порядок замкнутой кривой относительно точки}
\опр
\выд{Замкнутой кривой} в пространстве $X$ называется путь, у которого
конец совпадает с началом.
% Другими словами, замкнутая кривая --- это
%непрерывное отображение окружности в $X$.
\копр

\опр
Пусть $\gamma$ --- замкнутая кривая на плоскости $\R^2$, которую мы
отождествляем с $\C$, а $P$ --- точка
плоскости, не лежащая на кривой $\gamma$. Рассмотрим следующее
отображение окружности $S^1$ в себя:
$$
e^{2\pi it}\mapsto\frac{\gamma(t)-P}{|\gamma(t)-P|}.
$$
Степень этого отображения называется \выд{порядком кривой $\gamma$
относительно точки $P$} и обозначается $\ord_P(\gamma)$.
\копр

\задача
Нарисуйте кривые $\gamma_1$ и $\gamma_2$ и найдите их порядки
относительно точек $(0,0)$, $(1,0)$ и $(-1,0)$, где
$$\gamma_1(t)=(\cos(2\pi t)-1/2,\sin(4\pi t))\quad\mbox{и}\quad
\gamma_2(t)=(\cos(2\pi t)/2+3\cos(4\pi t)/4,\sin(4\pi t)).$$
\vspace{-\baselineskip}
\кзадача

\задача
Пусть $\gamma$ --- замкнутая кривая в плоскости $\R^2$, а $P$ --- точка
плоскости, не лежащая на кривой $\gamma$. Проведем произвольный луч $l$ из
точки $P$. Определите понятия положительной и отрицательной точки
пересечения луча $l$ с кривой $\gamma$ так, чтобы было верно
утверждение:
если существует лишь конечное число точек пересечения луча $l$ с кривой
$\gamma$, причем все они либо положительны, либо отрицательны, то
$\ord_P(\gamma)$ равен разности числа положительных и числа
отрицательных точек пересечения.
\кзадача

\задача
Пусть $f(x)$ --- многочлен степени $n$  с комплексными коэффициентами.
Докажите, что порядок кривой $\gamma_R(t)=f(Re^{2\pi it})$ относительно
точки $0\in\C$ при $R\gg0$ равен $n$.
\кзадача

\задача
Поставим в соответствие каждой замкнутой кривой $\gamma$ в $\R^2$, не
проходящей через точку $P$, отображение $\check\gamma:S^1\to
\R^2\setminus\{P\}$:
\vspace{-3mm}
$$
\check\gamma(e^{2\pi it})=\gamma(t).
$$
Докажите, что если отображения $\check\gamma_1$ и $\check\gamma_2$
гомотопны, то $\ord_P(\gamma_1)=\ord_P(\gamma_2)$.
%порядки кривых $\gamma_1$ и $\gamma_2$ относительно $P$ равны.
\кзадача

\задача
Пусть $f$ --- непрерывное отображение единичного круга в
плоскость,
$\gamma(t)=f(e^{2\pi it})$, $P$ --- точка, не лежащая
на кривой $\gamma$. Докажите, что если $\ord_P(\gamma)\ne0$, то
уравнение $f(x)=P$ имеет решение.  \кзадача

\задача[Основная теорема алгебры]
Докажите, что любой непостоянный многочлен с комплексными коэффициентами
имеет комплексный корень.
\кзадача

\задача
Пусть $\gamma$ --- замкнутая кривая на плоскости, $P$ --- точка, не
лежащая на ней. Докажите, что
\сНовойСтроки
\пункт
если точка $P$ соединена с точкой $P'$ путем, не пересекающим кривую
$\gamma$, то выполнено равенство:
$\ord_P(\gamma)=\ord_{P'}(\gamma)$.
%порядок $\gamma$ относительно $P'$ такой же, как и
%относительно $P$;
\пункт
найдется такая окрестность точки $P$, что относительно любой точки из
нее порядок $\gamma$ такой же, как и относительно $P$.
\кзадача

\задача[Н.Н.Константинов]
Из города $A$ в город $B$ ведут две непересекающиеся дороги. Известно,
что две машины, выезжающие по разным дорогам из $A$ и связанные
веревкой длины $2$, смогли проехать в $B$, не порвав веревки. Могут ли
разминуться, не коснувшись, два круглых воза радиуса $1$, центры которых
движутся по этим дорогам навстречу друг другу?
\кзадача

\задача
На плоскости нарисован граф $\Gamma$. Его образ при сдвиге на некоторый
вектор длины~$1$ не пересекается с $\Gamma$. По графу ползают два
круглых жука диаметром~$1$ (центр каждого жука все время принадлежит
$\Gamma$).  Могут ли они поменяться местами?
\кзадача

\vfill
\ЛичныйКондуит{0mm}{6mm}
% \GenXMLW

\end{document}

