\documentclass[a4paper,12pt]{article}
\usepackage{newlistok}

\УвеличитьШирину{1.0truecm}
\УвеличитьВысоту{2truecm}


\Заголовок{Топология прямой}
\НомерЛистка{51}
\ДатаЛистка{08.06.2020}
\Оценки{30/24/18}
\ВключитьКолонтитул

\begin{document}
\СоздатьЗаголовок


\опр
\выд Окрестностью точки называется произвольный
содержащий её интервал {\small (всюду в этом листке под интервалами
понимаются в том числе и бесконечные:~открытые лучи и вся прямая)}.
\\
Пусть $Y\subseteq \R$.
Точка $x\in\R$ называется
\\- \выд{внутренней точкой} множества $Y$, если у неё найдётся окрестность, целиком содержащаяся в $Y$,
\\- \выд{внешней точкой} множества $Y$, если у неё найдётся окрестность, не пересекающаяся с $Y$,
\\- \выд{граничной точкой} множества $Y$, если она не является ни внутренней, ни внешней,
\\- \выд{изолированной точкой} множества $Y$, если у неё найдётся окрестность, пересечение которой с~$Y$ состоит только из $x$.
% , если любая её окрестность содержит как точки из $Y$, так и точки не из $Y$.
% , принадлежащие $Y$, так и точки, не принадлежащие $Y$.
%она не является ни внутренней, ни внешней.
\копр


\задача
Докажите, что если $x$ — граничная точка множества $Y$, то
\\\пункт
либо $x\notin Y$, либо в любой её окрестности бесконечно много точек, не принадлежащих~$Y$;
\\\пункт
либо $x\in Y$, либо в любой её окрестности бесконечно много точек, принадлежащих~$Y$.
\\\пункт
Выкинем из множества все его граничные точки.
Может ли так оказаться, что мы ничего не выкинули? Выкинули всё?
\кзадача


\задача
Найдётся ли множество, у которого
\пункт
нет внутренних точек;
\пункт
ровно одна внутренняя точка?
\пункт
все точки внутренние?
\кзадача






\опр
Подмножество $U\subseteq \R$ называется \выд{открытым}, если все его точки --- внутренние.\\
Подмножество $V\subseteq \R$ называется \выд{замкнутым}, если оно содержит все свои граничные точки.
\копр

\задача
Найдите внутренние, внешние, граничные и изолированные точки для\\
\пункт
интервала $(0, 1)$;
\пункт
отрезка $[0, 1]$;
\пункт
множества $\hc{\frac{1}{n}\mid n\in\N}$;\\
\пункт
множества всех рациональных точек на прямой;
\кзадача

\задача
Докажите, что
\пункт интервал, дополнение к отрезку --- открытые подмножества прямой;
\\\пункт отрезок, дополнение к интервалу --- замкнутые подмножества прямой.
\кзадача


\задача
Верно ли, что открытое множество отрезка определяется рациональными числами, которые оно содержит?
\кзадача


\задача
Докажите, что множество граничных точек любого множества замкнуто.
\кзадача


\задача
Пусть $M\subset\R$ — непустое подмножество, и $x$ — произвольная точка.
Расстояние от $x$ до $M$ по определению равно $\inf \{|x-y|\colon y\in M\}$.
Докажите, что если $M$ — замкнуто, то в $M$ всегда найдётся точка $y$ такая, что $|x-y|$ равно расстоянию от $x$ до $M$.
\кзадача


\ввзадача
\пункт
Существуют ли множества, не являющиеся ни замкнутыми, ни открытыми?
\\\пункт
Всегда ли дополнение замкнутого множества открыто?
Всегда ли дополнение открытого множества замкнуто?
({\itshape Дополнением} множества $A$ называется разность $\R\setminus A$. Обозначения: $\overline{A}$.)
\кзадача



\задача
\пункт Докажите, что конечное пересечение
(то есть пересечение конечного числа) и произвольное объединение
(то есть объединение произвольного количества)
открытых множеств открыто.
\\\пункт
Докажите, что конечное объединение и любое пересечение замкнутых множеств замкнуто.
\кзадача

\задача
Найдите все множества, являющиеся одновременно открытыми и замкнутыми.
\кзадача


\задача
  Пусть $f$ --- непрерывная функция.
  Докажите, что множество $\{x\mid f(x)=0\}$ замкнуто.
\кзадача




\задача
Опишите все подмножества прямой, не имеющие граничных точек.
\кзадача

\ввзадача
Докажите, что любое открытое подмножество множества $\R$ либо совпадает с $\R$,
либо представляет собой объединение конечного или счетного множества непересекающихся интервалов и открытых лучей.
\кзадача


\ЛичныйКондуит{0mm}{5mm}
\ОбнулитьКондуит
\newpage


\опр
Непустое множество $M$ называется {\itshape компактом},
если из произвольного покрытия $M$ открытыми множествами
можно выделить конечное подпокрытие.
\копр


\ввзадача
Докажите, что
\\\пункт
компакты на прямой~--- это в точности непустые замкнутые ограниченные множества;
\\\пункт
у любой последовательности вложенных компактов
$K_1\supset K_2\supset K_3\supset\dots$ %имеет непустое
пересечение непусто.
\кзадача


%\newpage

\раздел{Канторово множество и канторова лестница}

Может показаться, что раз открытые множества — объединение интервалов, то замкнутые — объединения точек и отрезков.
Увы, это не так, замкнутые множества могут быть достаточно противными.

\задача[Канторово множество]
\label{Kantor}
Возьмём отрезок $K_0=[0,1]$. Разделим его на три равные
части и средний интервал $I_1^1=(\frac13, \frac23)$ выкинем.
Первый и третий отрезки образуют множество $K_1$.
Каждый из них разделим на три части
и выкинем средние интервалы $I_1^2=(\frac19, \frac29)$,
$I_2^2=(\frac79, \frac89)$.
Получится множество $K_2$. И так далее: на $n$-м шаге будем
делить каждый из $2^{n-1}$ отрезков, образующих $K_{n-1}$, на три
равные части и выкидывать все средние интервалы
$I_1^n, I_2^n, \dots, I_{2^{n-1}}^n$.
Так получается множество $K_n$, состоящее из $2^n$ отрезков.
Устремим $n$ к бесконечности.
Множество, получающееся в пределе, т.~е. $\bigcap\limits_{n=1}^{\infty} K_n$,
называется \emph{канторовым} (всюду дальше будем обозначать~его~$K$).
%\вСтрочку
\сНовойСтроки
\пункт
Конечно ли это множество? Счётно?
\пункт
Является ли оно открытым? Замкнутым?
\кзадача



\задача
Бесконечно ли множество рациональных чисел, принадлежащих канторову
множеству?
\кзадача


\задача
\пункт
Как выглядит запись точек из канторова множества в троичной системе счисления?
\пункт
Докажите, что канторово множество имеет мощность континуума.
\кзадача



\задача[Канторова лестница]
В точках 0 и 1 значение функции $K(x)$ принимается равным соответственно 0 и 1.
Далее интервал $(0, 1)$ разбивается на три равные части $\left[0,{\frac{1}{3}}\right]$, $\left[{\frac  {1}{3}},{\frac  {2}{3}}\right]$ и $\left[{\frac  {2}{3}},1\right]$.
На среднем отрезке полагаем $K(x)={\frac  {1}{2}}$.
Оставшиеся два отрезка снова разбиваются на три равные части каждый, и на средних отрезках $K(x)$ полагается равной ${\frac {1}{4}}$ (на левом) и ${\frac {3}{4}}$ (на правом).
Каждый из оставшихся отрезков снова делится на три части, и на внутренних отрезках $K(x)$ определяется как постоянная, равная среднему арифметическому между соседними, уже определенными значениями $K(x)$, и т.д.
На остальных точках единичного отрезка определяется <<по непрерывности>> (так чтобы в итоге получилась непрерывная функция).
Полученная функция называется \выд{канторовой лестницей}.

Докажите, что вот это «определяется по непрерывности» работает: $K(x)$ однозначно определена в каждой точке отрезка $[0, 1]$, нестрого монотонна и непрерывна.
\кзадача


\задача
Будем наугад брать точку $x$ из отрезка $[0, 1]$.
С какой вероятностью найдётся такая окрестность точки $x$, в которой канторова лестница $K(x)$ постоянна?
\кзадача


\ЛичныйКондуит{0mm}{5mm}
% \GenXMLW

\end{document}




% правильное название — точка конденсации
% \задача
% Пусть множество $M\subset\R$ несчётно (то есть как минимум бесконечно).
% Точка $a\in M$ называется \выд{точной конденсации}, если пересечение любой окрестности $a$ с $M$ несчётно.
% \\\пункт
% Докажите, что множество точек $b\in M$ не являющихся точками конденсации не более, чем счётно.
% \\\пункт
% Может ли множество точек конденсации иметь изолированные точки?
% \\\пункт
% Докажите, что множество всех точек конденсации $M$ замкнуто;
% \\\пункт
% Докажите, что несчётное замкнутое подмножество в $\R$ имеет мощность континуума.
% \кзадача

