\documentclass[a4paper, 11pt]{article}
\usepackage{newlistok}

\УвеличитьШирину{1.5truecm}
\УвеличитьВысоту{2.5truecm}




\begin{document}

\Заголовок{Плоские алгебраические кривые --- 2}
\НомерЛистка{73}
\ДатаЛистка{21.11 -- 03.12.2021}
\Оценки{23/16/9}

\СоздатьЗаголовок

\noindent

\задача
Докажите, что на $x-y$ делится \пункт $x^n-y^n$; \пункт $P(x,y)-P(x,x)$, где $P(x,y)$ --- любой многочлен.
\кзадача


\опр
{\em Степень одночлена} $ax^ny^m$, где $a\ne0$, --- число $m+n$. {\em Степень многочлена} $A(x,y)$ --- наибольшая из степеней его ненулевых одночленов (в записи, где <<приведены подобные>>). Обозначение: $\deg A (x,y)$.
\копр

\задача Пусть $A(x,y)$, $B(x,y)$~--- ненулевые многочлены.
Докажите, что $\deg AB = \deg A + \deg B$.
\кзадача

\задача
Кривая задана многочленом $A(x,y)$ степени $n$. Докажите, что если она пересекает некую прямую $y=kx+b$ более, чем в $n$ точках, то $A(x,y)$ делится на $y-kx-b$.
\кзадача

\задача
Дан многочлен $A(x,y)$ степени $n$.
\пункт Докажите, что либо у системы $A(x,y)=0$, $x=y^2$ не более $2n$ решений, либо
$A$ делится на $x-y^2$ (геометрический смысл: кривая $A$ либо пересекает параболу не более чем в $2n$ точках,
либо содержит в себе параболу).
\пункт Докажите то же самое для системы $A(x,y)=0$, $xy=1$.
\кзадача


\задача %\label{abcd}
Вершины четырёхугольника $ABCD$ лежат на параболе $y=x^2$.
Пусть прямые $AB$, $BC$, $CD$, $DA$ задаются многочленами
первой степени $l_1(x,y)$, \ $m_1(x,y)$, \ $l_2(x,y)$, \ $m_2(x,y)$ соответственно.
Рассмотрим кривую $H$ с уравнением $\lambda l_1l_2+\mu m_1m_2=0$, где $\lambda,$ $\mu\in\R$.
Докажите, что
\пункт $H$ проходит через $A$, $B$, $C$, $D$;
\пункт можно подобрать $\lambda,$ $\mu$ так, чтобы $H$ пересекалась с параболой ещё в одной точке и совпадала бы с параболой.
%\пункт кривая из пункта б) совпадает с параболой $y=x^2$.
% --- некоторые действительные числа.
\кзадача

\задача Можно ли на плоскости задать многочленом в точности одну из ветвей гиперболы $xy=1$?
\кзадача



\задача[Замена координат] На плоскости $Oxy$ рассмотрим
непараллельные  прямые $l_1(x,y) = 0$ и $l_2(x,y) = 0$ (где $l_1$ и $l_2$~--- многочлены первой степени).
Примем их за новые оси координат: $l_1$ --- за ось $z$, а $l_2$ --- за ось
$t$ (их пересечение $O_1$ --- за начало). Тогда $l_1$ и~$l_2$ имеют в новых координатах уравнения $t = 0$, $z = 0$ соответственно.
\\
\пункт Докажите, что можно так выбрать базисные вектора на осях
$O_1z$, $O_1t$, что новые координаты $(z,t)$ точки будут вычисляться
через её старые координаты $(x,y)$ по формулам $z = l_2(x,y)$, $t =
l_1(x,y)$. \пункт Докажите, что из уравнений $z = l_2(x,y)$, $t =
l_1(x,y)$ можно выразить старые координаты $x$ и $y$ через новые $z$
и $t$. \пункт Пусть $A(x,y)$~--- многочлен. Подставив в него вместо
$x$ и $y$ их выражения через $z$ и $t$, получим запись многочлена
$A$ в новых координатах $z$, $t$. Докажите, что при этом степень
многочлена не изменится: $\deg A(x,y) = \deg A(z,t)$.
\кзадача

\задача Докажите, что кривая, заданная уравнением $x^2 + 2xy + y^2 +
\sqrt2x - \sqrt2y + 1 = 0$, имеет ось симметрии.
\кзадача

\задача
\пункт
Две кривые пересекаются в конечном числе точек. Докажите, что в некоторой системе координат у них нет общих точек с одинаковыми ординатами.
\спункт
Решите задачу 3 для системы $A(x,y)=0$, $x^2+y^2-1$.
\кзадача



\задача Пусть шестиугольник
$ABCDEF$ вписан в окружность $x^2+y^2=1$,
прямые $AB$, $BC$, $CD$, $DE$, $EF$, $FA$ задаются многочленами
первой степени $l_1$, $m_1$, $l_2$, $m_2$, $l_3$, $m_3$ соответственно.
Рассмотрим кривую $H$ с уравнением $\lambda l_1l_2l_3+\mu m_1m_2m_3=0$, где $\lambda,$ $\mu\in\R$.
Докажите, что
\пункт $H$ содержит все вершины шестиугольника;\\
\пункт можно подобрать ненулевые $\lambda,$ $\mu$ так, чтобы $H$ имела с окружностью не менее\\ 7 общих точек;
\пункт полученная в пункте б) кривая $H$ делится на $x^2+y^2-1$. \\
{\em Замечание.} Задачу можно сдать для параболы (гиперболы), если не сдана 9б.
\кзадача
\ВосстановитьГраницы

%{\УстановитьГраницы{0cm}{6.5cm}
%}

\vspace*{-5mm}
%\раздел{Часть 2. Теорема Безу}

\УстановитьГраницы{0cm}{7cm}
\putpict{16cm}{-.3cm}{alg_curve_pask_papp_des-2}{}
\задача [Теорема Паскаля] Пусть $ABCDEF$ вписан \пункт в окружность;
\пункт в гиперболу или в параболу. Докажите, что точки пересечения прямых $AB$ и $DE$, $BC$ и $EF$, $CD$ и $FA$ лежат на одной прямой.
%\пункт То же верно, если $ABCDEF$ вписан
\кзадача

%\vspace*{10mm}
\задача  [Теорема Паппа] Пусть точки $A$, $B$, $C$ и $A'$, $B'$, $C'$
лежат на прямых $l$ и $l'$ соответственно.
Докажите, что тогда точки пересечения прямых
$AB'$ и $A'B$, $BC'$ и $B'C$, $CA'$ и $C'A$ лежат на одной прямой.
%(см.~рис.~слева).
\кзадача

%\vspace*{-2.1cm}
\putpict{16cm}{.1cm}{alg_curve_pask_papp_des-3}{}
%\vspace*{25mm}

\vspace*{-.4cm}
\задача
Три пунктирные прямые пересекают три сплошные прямые в
девяти точках (см.~рисунок). Докажите, что если 8
из этих точек лежат на некоторой \выд{кубике} (кривой, задающейся многочленом третьей степени), то и оставшаяся
девятая точка лежит на той же кубике. \кзадача
\ВосстановитьГраницы

\vspace*{-2.1cm}
\putpict{16cm}{-3cm}{list24d-5}{}
\vspace*{17mm}

% \задача \вСтрочку \пункт Пусть никакие три из точек $A$, $B$, $C$,
% $D$, $E$ на плоскости не лежат на одной прямой. Докажите, что через
% эти точки проходит ровно одна коника (кривая, задающаяся многочленом второй степени). \\ \пункт Докажите, что в
% пункте а) достаточно потребовать, чтобы никакие четыре из точек $A$,
% $B$, $C$, $D$, $E$ не лежали на одной прямой. \кзадача

\УстановитьГраницы{0cm}{6.1cm}
\задача
\пункт Для каждого $t$ найдите точки пересечения прямой $y=1-xt$ с окружностью $x^2+y^2=1$.
\пункт Докажите, что все точки этой окружности, кроме $(0,-1)$, задаются параметрически в виде $\left(\frac{2t}{1+t^2},\frac{1-t^2}{1+t^2}\right)$, $t\in\R$.
\пункт Докажите, что все рациональные точки этой окружности, кроме $(0,-1)$,  получаются по формулам предыдущего пункта при $t\in\Q$.
\кзадача

\ВосстановитьГраницы

\задача
Выведите из задачи 14 \пункт формулы пифагоровых троек: если $X^2+Y^2=Z^2$, где $X$ и  $Y$ взаимно просты и $X$ чётно, то $X=2mn$, $Y=m^2-n^2$, $Z=m^2+n^2$ для каких-то взаимно простых $m$ и $n$; \спункт задачу~9б.
\кзадача


\задача[М.Берже, С.Маркелов]
На плоскости даны парабола $p$ и окружность $\omega$, у них ровно 2
общие точки $A$ и $B$. Касательные к $p$ и $\omega$
в точке~$A$ совпадают. Обязательно ли касательные
к  $p$ и $\omega$  в точке $B$ совпадают?
\кзадача


\ЛичныйКондуит{0mm}{6.5mm}
% \GenXMLW


\end{document}

