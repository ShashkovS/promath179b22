\documentclass[a4paper,11pt]{article}
\usepackage[mag=1000]{newlistok}
\usepackage{tikz}
\usetikzlibrary{calc}

\УвеличитьШирину{1.5truecm}
\УвеличитьВысоту{2.5truecm}

\Заголовок{Геометрия на клетчатой бумаге}
\НомерЛистка{8д}
\renewcommand{\spacer}{\vfill}
\ДатаЛистка{13.10 -- 22.10/2018}
\Оценки{30/25/20}




\renewcommand{\spacer}{\vfil}
% !! Обрати внимание на эту команду. Она делает промежутки между задачами растяжимыми так,
% чтобы всё поместилось равномерно. Это позволяет немного сэкономить.


\begin{document}

\thispagestyle{empty}

\СоздатьЗаголовок

{\small Как связаны площадь многоугольника, лежащего на клетчатой бумаге (сетке) со стороной клетки 1, и число накрытых им узлов сетки?
Мы ответим на этот вопрос для многоугольников с вершинами в узлах сетки.}

%\опр Пусть дан многоугольник с вершинами в узлах сетки,
%$S$ --- его площадь, $i$ --- число узлов сетки,
%лежащих внутри него, $b$ --- число узлов
%сетки, лежащих на его сторонах (считая вершины).
%\выд{Формулой Пика} называют формулу:  $S=i+b/2-1$.
%\копр

%\задача
%Найдите площади многоугольников, изображенных на рисунках:\\
%\hbox{{\bf а)}\hspace*{5truecm} {\bf б)}\hspace*{5truecm}  {\bf в)}}
%\кзадача

%\putpict{15mm}{-5mm}{kletki-2}
%\putpict{70mm}{-5mm}{kletki-1}
%\putpict{150mm}{-5mm}{kletki-3}

%\vspace*{1.5cm}

%\задача
%Найдите \\площади \\многоугольников \\на рисунках \\справа:
%\кзадача

\УстановитьГраницы{0cm}{15.5cm}
\задача
Найдите площади многоуго\-льников, приведенных на рисунках справа:
\ВосстановитьГраницы
\vspace*{-1.8truecm}
\кзадача
\tikzset{gridst/.style={step=1cm,color=blue!50!black}}
\tikzset{polyst/.style={very thick,draw=green!30!black,fill=green!30,opacity=.5}}
\putthere{55mm}{-9.0mm}{%
  \begin{tikzpicture}[scale=.44]
      \draw (-1, 4) node {\пункт};
      \draw[gridst](.5,.5)grid(5.5,4.5);
      \filldraw[polyst] (1,4) -- (3,4) -- (2,3) -- cycle;
      \filldraw[polyst] (1,1) -- (3,3) -- (2,1) -- cycle;
      \filldraw[polyst] (3,2) -- (4,1) -- (5,4) -- cycle;
  \end{tikzpicture}%
}{0cm}{} %4-й параметр --- ширина подписи, 5-й --- сама подпись
\putthere{93mm}{-9.0mm}{%
  \begin{tikzpicture}[scale=.44]
      \draw (-1, 4) node {\пункт};
      \draw[gridst](.5,.5)grid(5.5,4.5);
      \draw[polyst] (1,1) -- (5,4) -- (4,2) -- cycle;
  \end{tikzpicture}%
}{0cm}{}
\putthere{155mm}{-9.0mm}{%
  \begin{tikzpicture}[scale=.36]
      \draw (-1, 5) node {\пункт};
      \draw[gridst](.5,.5)grid(13.5,5.5);
      \draw[polyst] (1,4) -- (3,3) -- (6,5) -- (4,3) -- (9,4) -- (8,3) -- (13,5) -- (10,3) -- (12,1) -- (5,3) -- (6,2) -- (5,1) -- (4,2) -- (2,1) -- (3,2) -- cycle;
  \end{tikzpicture}%
}{0cm}{}



\vspace*{1.3cm}

\задача
Найдите формулу,
выражающую площадь через $i$ (число узлов внутри) и $b$ (число узлов на границе)\\
%\вСтрочку
\пункт
для прямоугольника со сторонами, лежащими на линиях~сетки;\\
\пункт
для прямоугольного треугольника с вершинами в узлах и катетами, лежащими на
линиях сетки.
\кзадача

\задача
\пункт Проверьте, верна ли полученная вами в задаче 2 формула для примеров из задачи 1?\\
\пункт Два многоугольника (с вершинами в узлах) граничат по ломаной
и составляют
вместе новый многоугольник. Пусть Ваша формула верна для %некоторых
двух из этих многоугольников. Верна ли она для третьего? % многоугольника?
\кзадача

\задача
Придумайте формулу, выражающую площадь через число узлов внутри и на границе для любого\\
\пункт треугольника с вершинами в узлах сетки;
\пункт четырехугольника с вершинами в узлах сетки.
\кзадача


%\задача
%Пусть имеются три многоугольника (с вершинами в узлах), один из которых составлен
%из двух других.
%Пусть формула Пика верна для некоторых двух многоугольников
%из этих тр\"ех. Докажите, что тогда
%она верна и для третьего многоугольника.
%\кзадача

%\задача
%Докажите формулу Пика
%для любого
%\пункт треугольника; \пункт четырехугольника
%с вершинами в узлах сетки.
%\кзадача



%\задача
%Докажите формулу Пика для любого треугольника с вершинами в узлах сетки.
%\кзадача

\сзадача
%\пункт
%Пусть $M$  --- невыпуклый многоугольник, $A$ ---
%такая его вершина, что внутренний угол при этой вершине больше
%развёрнутого. Рассмотрим все лучи, исходящие из $A$ и направленные
%внутрь $M$; на каждом луче отметим первую точку пересечения
%с контуром $M$.  Могут ли все отмеченные точки принадлежать
%одной стороне $M$?
%\пункт
Докажите, что во всяком $n$--угольнике ($n>3$)
найдётся диагональ, принадлежащая ему целиком. \
({\sl Замечание:} результатом этой задачи можно пользоваться далее без
доказательства.)
\кзадача

\задача Докажите, что всякий многоугольник можно разбить диагоналями
на треугольники так, чтобы диагонали целиком принадлежали
многоугольнику и не пересекались внутри многоугольника.
\кзадача

\задача[Формула Пика]
Обобщите формулу из задачи 4 на любой многоугольник с вершинами в узлах~\hbox{сетки.}
\кзадача


%\vspace*{-1mm}

\раздел{***}

\vspace*{-2.5mm}

%\задача
%Вершины треугольника $\Delta$ лежат в узлах квадратной сетки,
%%причём
%других узлов на границе $\Delta$ %треугольника
%нет, а внутри %треугольника
%находится ровно один узел. Докажите, что он %этот узел
%совпадает с точкой пересечения медиан~$\Delta$. %треугольника.
%\кзадача


\задача
Король обошел шахматную доску, побывав на каждом поле по разу,
и последним ходом вернулся на исходное поле. Ломаная, соединяющая последовательно центры полей в пути короля, не имеет самопересечений.
\!\!\! \пункт \!\!\! Какую площадь может ограничивать эта ломаная?
\!\!\! \спункт \!\!\! Какую наибольшую длину она может~иметь?
\кзадача

\задача
Докажите, что у многоугольника с вершинами в узлах сетки площадь --- целое или полуцелое число, а квадраты длин сторон --- целые числа.
\кзадача

\задача Найдётся ли правильный
\вСтрочку
\пункт треугольник;
\пункт шестиугольник с вершинами в узлах сетки?
\кзадача

\задача Дан правильный $n$--угольник $M$
с вершинами в узлах сетки. Докажите, что существует
правильный $n$--угольник с вершинами и центром в узлах сетки
(используйте $M$ для его построения).
\кзадача

\сзадача
Для каких $n$ существует правильный $n$-угольник с вершинами в
узлах сетки?
\кзадача

\сзадача
На плоскости провели много параллельных прямых на равном расстоянии
друг от друга (\лк тетрадь в линейку\пк).
Для каких $n$ можно нарисовать правильный $n$-угольник %можно нарисовать %на плоскости
с вершинами на %проведённых
этих прямых?
\кзадача

\опр Назовём треугольник (или параллелограмм) с вершинами в узлах сетки
\выд{простым}, если внутри него и на его сторонах нет других узлов сетки кроме его вершин.
\копр

\задача
\вСтрочку
\!\!\!\!\! \пункт
\!\!\!\! Какова площадь простого треугольника?
\!\!\! \пункт
\!\!\!\! Может ли он иметь сколь угодно большой~\hbox{периметр?}\\
\пункт Какие простые треугольники прямоугольные?
\спункт У любого ли простого треугольника есть неострый угол?
%\пункт
%Докажите, что у простого треугольника один из углов тупой или прямой,
%причем последний случай возможен только для треугольника с вершинами
%в узлах одной клетки.
\кзадача



%\задача Пусть $A$ и $B$ --- узлы сетки, причём на отрезке $AB$
%нет других узлов. \\
%\пункт Докажите, что для некоторого узла сетки $C$
%треугольник $ABC$ будет простым.\\
%\пункт Найдутся ли среди не лежащих на прямой $AB$ узлов ближайшие к этой прямой и много ли
%их может быть? %Найдите расстояние от такого узла до прямой $l$.
%\кзадача

\сзадача
В трёх вершинах клетки сидит по кузнечику. Они начинают играть в чехарду:
каждый может прыгнуть через одного из двух других, после чего оказывается в симметричной
относительно него точке.
%прыгая друг через друга. При этом, если кузнечик А прыгает через кузнечика В, то после прыжка А оказывается от В на том же расстоянии (но по другую сторону и на той же прямой).
\сНовойСтроки
\пункт Докажите, что кузнечики всегда будут находиться в вершинах простого треугольника.
\пункт Может ли один из кузнечиков после нескольких прыжков попасть в четвертую вершину
исходной клетки?
%\пункт Докажите, что кузнечики, изначально расположенные в вершинах любого простого треугольника,
%могут через несколько прыжков попасть в вершины одной клетки.
\пункт Из каких простых треугольников можно прыжком получить треугольник
с меньшей наибольшей стороной?
\пункт В вершинах каких простых треугольников могут после нескольких прыжков оказаться наши кузнечики?
\кзадача


%\задача
%Прямая $l$ соединяет узлы сетки с координатами $(0;0)$ и
%$(p;q)$, причём между этими узлами на прямой $l$ нет других узлов.
%\кзадача

%\задача
%Пусть отношение площади многоугольника к квадрату длины какой-то его
%стороны иррационально. Докажите, что %никакой подобный ему многоугольник
%этот многоугольник нельзя нарисовать на клетчатой бумаге так, чтобы вершины лежали в узлах.
%\кзадача


\задача
Докажите, что параллелограмм $ABCD$ с вершинами в узлах сетки является
простым тогда и только тогда, когда все %возможные
параллелограммы, полученные из $ABCD$ параллельными переносами,
сдвигающими узел $A$
в разные узлы сетки, покрывают плоскость и не накладываются
друг на друга.
\кзадача

\сзадача
\!\!\!\!\пункт\!\!\!\!
На плоскости дана клякса площади больше 1. Докажите, что
у каких-то двух её точек
разности соответствующих координат целые.
\пункт \!\!\!\! ({\it Теорема Минковского})
На плоскости дана центрально-симметричная (относительно узла)
выпуклая фигура площади больше 4. Докажите, что она
содержит больше одного узла.
\кзадача

\сзадача
Каждый узел, кроме узла $O$, клетчатой бумаги со стороной клетки 1 см накрыли кругом
с центром в этом узле и радиусом $0{,}000001$ см.
% раз меньше ширины клеток.
Можно ли из узла $O$ %начала координат
выпустить луч, не пересекающий ни одного круга? % (кроме накрывающего начало) круга?
\кзадача

\УстановитьГраницы{0cm}{3cm}
\сзадача
\putthere{163mm}{-12mm}{%
\begin{tikzpicture}[scale=.3]
    \draw[gridst](-3.5,-3.5)grid(3.5,3.5);
    \draw[very thick,color=blue!50,->] (-3.7, 0) -- (4,0);
    \draw[very thick,color=blue!50,->] (0, -3.7) -- (0, 3.6);
    \draw (0,0) circle (3);
    \foreach \i/\j in {-3/0, 3/0, -2/-2, -2/-1, -2/0, -2/1, -2/2, 2/-2, 2/-1, 2/0, 2/1, 2/2, -1/-2, -1/-1, -1/0, -1/1, -1/2,  1/-2, 1/-1, 1/0, 1/1, 1/2, 0/-3, 0/-2, 0/-1, 0/3, 0/2, 0/1}
            {\filldraw[polyst] (\i,\j) circle (.2);}
    \draw[very thick,color=red!50!black] (0,0) --  (18:4);
    \draw[very thick,color=red!50!black] (0,0) --  (116:2);
    \draw[very thick,color=red!50!black,dashed] (116:2.5) --  (116:4);
\end{tikzpicture}%
}{0cm}{}
В парке, имеющем форму круга радиуса $s$ м ($s$ --- целое число),
деревья посажены во всех вершинах квадратной сетки со стороной квадратов
1 м, кроме центра (пример~для~$s=3$ см.~на рисунке). Докажите, что
вид из центра
\вСтрочку
\пункт
полностью заслонён,
если радиусы всех деревьев
больше $\displaystyle{\frac1s}$ м;
\пункт \!\!\! заслонён не полностью,
если радиусы всех деревьев  меньше $\displaystyle{1\over\sqrt{s^2+1}}$~м.
\ВосстановитьГраницы
\кзадача

\vspace{-2mm}
%\putpict{17.5cm}{13mm}{les}


\ЛичныйКондуит{0mm}{5mm}
%\СделатьКондуит{5.5mm}{7mm}

% \GenXMLW

\end{document}


\задача
Вершины некоторого треугольника лежат в узлах квадратной сетки,
прич\"ем других узлов на границе треугольника нет, а внутри
треугольника находится ровно один узел сетки. Докажите, что этот
узел совпадает с точкой пересечения медиан треугольника.
\кзадача 