\documentclass[a4paper,11pt]{article}
\usepackage[mag=1000]{newlistok}

\ВключитьКолонтитул

\УвеличитьШирину{1.4cm}
\УвеличитьВысоту{2.3cm}

\Заголовок{Комплексные числа в геометрии}
\НомерЛистка{40}
\renewcommand{\spacer}{\vspace{0.9pt}}
\ДатаЛистка{13.11 -- 27.11.2019}
% 45 задач
\Оценки{21/16/11}


\begin{document}
\newcommand{\0}[1]{\overline{#1}}


\СоздатьЗаголовок

%\medskip



}

\medskip

\задача[Эйлер] Докажите, что сумма квадратов длин сторон четырёхугольника
отличается от суммы квадратов диагоналей на учетверённый квадрат
длины отрезка, соединяющего середины диагоналей.
\кзадача

\задача Пусть $M$ --- точка на плоскости, $S$ --- окружность,
$AB$ --- её диаметр. Докажите, что величина $MA^2+MB^2$
не зависит от выбора диаметра $AB$ окружности $S$.
\кзадача

\задача[Теорема Лейбница] Пусть $F$ --- центр масс %(то есть, точка пересечения медиан)
треугольника $ABC$. Докажите, что для любой точки $M$ на
плоскости выполнено равенство: $MA^2+MB^2+MC^2=AF^2+BF^2+CF^2+3MF^2.$
\кзадача

\задача
Докажите теорему косинусов $BC^2=AB^2+AC^2-2\cdot AB\cdot AC\cdot\cos\alpha$, расположив вершины треугольника $ABC$ в точках $0$, $z$ и $w$ соответственно, где $w$ вещественно.
\кзадача





\задача На плоскости даны точки $A,B,C$. Пусть $A_1$ --- образ
точки
$C$ при повороте вокруг $A$ на $90^{\circ}$ против часовой
стрелки,
$B_1$ --- образ точки $C$ при повороте вокруг $B$ на
$90^{\circ}$
по часовой стрелке, $K$ --- середина $A_1B_1$, $M$ --- середина
$AB$.
Докажите, что отрезки $КM$ и $AB$ перпендикулярны. Как соотносятся их
длины?
\кзадача

\задача На сторонах треугольника $A_1A_2A_3$ во внешнюю сторону
построены
квадраты с центрами $B_1, B_2, B_3$. Докажите, что отрезки $B_1B_2$ и
$A_3B_3$ равны по длине и перпендикулярны.
\кзадача

\задача Пусть $A_1A_2A_3$ и $B_1B_2B_3$ --- правильные треугольники,
%причем
и их вершины занумерованы %в порядке обхода
против часовой стрелки.
Докажите,
что середины отрезков $A_1B_1, A_2B_2$ и $A_3B_3$ --- вершины
правильного
треугольника.
\кзадача

\задача Докажите, что три точки $z_1,z_2,z_3$ являются вершинами
правильного треугольника тогда и только тогда, когда
$z_1^2+z_2^2+z_3^2=z_1z_2+z_1z_3+z_2z_3$.
\кзадача


\опр
{\em Простое отношение} тройки различных точек $z_1$, $z_2$ и $z_3$ ---
это %называется
комплексное число
$\displaystyle\frac{z_1-z_3}{z_2-z_3}.$
\копр


\задача Докажите, что
\пункт  три различные точки $z_1,z_2,z_3$ лежат на одной
прямой тогда и только тогда, когда их простое отношение
вещественно;
\пункт прямая, проходящей через точки $z_1$ и $z_2$, задаётся уравнением $(z_1-z)(\0 z_2-\0 z)=(\0 z_1-\0 z)(z_2-z)$;
\пункт каково уравнение перпендикуляра к этой прямой, проходящего через $w$?
% прямая, проходящая через точку $w$ перпендикулярно прямой, проходящей через $z_1$ и $z_2$, имеет вид
% $(\0 z_1 -\0 z_2)z+(z_1-z_2)\0 z=(\0 z_1 -\0 z_2)w+(z_1-z_2)\0 w$.
\кзадача

\задача[Прямая Эйлера] В любом треугольнике центр тяжести треугольника,
его ортоцентр  и центр описанной окружности лежат на одной прямой.
\кзадача




\опр
{\em Двойное отношение} четвёрки различных точек $z_1$, $z_2$, $z_3$ и $z_4$ ---
это %называется комплексное
число
$\displaystyle\frac{z_1-z_3}{z_2-z_3}:\frac{z_1-z_4}{z_2-z_4}.$
\копр

\задача
\вСтрочку
\пункт Пусть четыре различные точки $z_1,z_2,z_3,z_4$ лежат на одной
окружности. Докажите, что тогда их двойное отношение
вещественно.
\пункт Пусть двойное отношение четырёх различных точек вещественно. Что
можно сказать об их взаимном расположении?
\кзадача

\задача
Докажите, что \пункт
$(z_1-z_2)(z_4-z_3)+(z_2-z_3)(z_4-z_1)=(z_2-z_4)(z_3-z_1);$\\
 \пункт  в любом четырёхугольнике произведение
 длин диагоналей не превосходит сумму произведений длин
противоположных сторон;
\пункт[теорема Птолемея]  для четырёхугольника, вписанного в окружность, достигается равенство.
\пункт Верно ли, что если равенство достигается, то четырёхугольник вписанный?
\кзадача

\задача
Докажите, что прямая, проходящая через точки $a$ и $b$ единичной окружности $z\0 z=1$, имеет уравнение $z+ab\0 z=a+b$, а касательная в точке $p$ этой окружности имеет уравнение $\0 p z+ p\0 z=2$.
\кзадача

\задача
\вСтрочку
\пункт Пусть $z_1$ и $z_2$ --- точки на единичной окружности
$z\0 z=1$. Докажите, что точка пересечения~касательных
к этой окружности, проходящих через $z_1$ и $z_2$, --- это точка $\frac{2z_1z_2}{z_1+z_2}$.
\пункт[Задача Ньютона] В описанном около окружности
четырёхугольнике середины диагоналей и центр окружности
лежат на одной прямой.
\кзадача




\сзадача Каждую сторону $n$-угольника продолжили на её длину (обходя по часовой стрелке). Пусть концы построенных
отрезков образуют правильный $n$-угольник. Докажите, что и исходный $n$-угольник правильный.
\кзадача



\сзадача
Пусть вписанная окружность треугольника $ABC$ задаётся уравнением $z\0 z=1$ и касается его сторон в точках $p$, $q$, $r$. Докажите, что
\пункт $\frac{2pqr(p+q+r)}{(p+q)(p+r)(q+r)}$ --- центр описанной окружности треугольника $ABC$;
\пункт $\frac{(pq+pr+qr)^2}{(p+q)(p+r)(q+r)}$ --- центр окружности Эйлера треугольника $ABC$;
\пункт точка $\frac{pq+pr+qr}{p+q+r}$ лежит и на вписанной окружности, и на окружности Эйлера (окружность 9 точек) треугольника $ABC$;
\пункт[теорема Фейербаха] вписанная окружность и окружность Эйлера треугольника $ABC$ касаются друг друга.
\кзадача

\задача
\пункт Пусть $a$, $b$, $c$, $d$ --- различные точки на единичной окружности
$z\0 z=1$. Докажите, что секущая, проходящая через $a$ и $b$, и секущая, проходящая через $c$ и $d$, пересекаются в точке, сопряжённой к
$\frac{(a+b)-(c+d)}{ab-cd}$.
\кзадача

\сзадача[Теорема Паскаля] Докажите, что точки пересечения прямых, содержащих противоположные стороны вписанного шестиугольника, лежат на одной прямой.
\кзадача

% \сзадача \выд{(Теорема Морли)}
% Трисектрисой угла называют луч, исходящий из вершины угла и
% отсекающий от угла втрое меньший угол (у каждого угла
% две трисектрисы). В треугольнике $ABC$ пусть $M$ --- точка
% пересечения трисектрис, примыкающих к стороне $BC$, $Q$ ---
% точка пересечения трисектрис, примыкающих к стороне $CA$ и
% $P$~---~точка пересечения трисектрис, примыкающих к $AB$.
% Докажите, что треугольник $MPQ$ правильный.
% \кзадача





\ЛичныйКондуит{0mm}{5mm}
% \GenXMLW

%\СделатьКондуит{3.7mm}{7.5mm}


\end{document}

\задача
Пусть карты из задачи 17 листка 22 лежат на комплексной плоскости.
Докажите, что найдутся такие $q,b\in\Cbb$, что
если $z\in\Cbb$ --- любая точка на первой карте, то этой же точкой местности
на второй карте будет точка $qz+b$.
Выразите с помощью  $q$ и $b$ точку,
изображающую на картах одну и ту же точку местности.
\кзадача



{\hsize 13.2cm

\задача
Запишите %в виде
как функцию комплексной переменной
\вСтрочку
\!\! \пункт \!\! симме\-трию относительно оси $y$;
\пункт ортогона\-ль\-ную проекцию на ось $x$;
\пункт центральную симметрию с центром $A$;
\пункт поворот на угол $\varphi$ относительно~точки~$A$;
\пункт гомотетию с коэффициентом $k$ и центром $A$;
\пункт %скользящую
симметрию относительно прямой $y=3$ со сдвигом на 1 влево;
\пункт поворот, %который
переводящий ось $x$ в прямую $y=2x+1$;
\пункт симметрию относительно прямой $y=2x+1$.
\кзадача


\задача
Куда отображение $z\longmapsto z^2$
переводит
\вСтрочку
\пункт
декартову координатную сетку;
\пункт
полярную координатную сетку;
\пункт
окружность $|z+i|=1$;
\пункт
кошку (см.~рис.~справа)?
\пункт
Те же вопросы для отображения
$z\longmapsto 1/z$.
\кзадача

}


\putpict{15.8cm}{2.5cm}{curve}{}

\vspace*{-5mm}


%\задача
%Куда отображение $z\longmapsto\sqrt z$ переводит
%$\{z\in\Cbb\ |\ {\rm Im}\,(z)>0\}$?
%верхнюю полуплоскость (без границы)?
%\кзадача

\задача
Куда отображение
\вСтрочку
\пункт
$z\longmapsto1/z$;
\спункт $z\longmapsto0,5(z+1/z)$
переводит %множество
$\{z\in\Cbb\ |\ {\rm Im}\,(z)>0,\ |z|\leq1\}$?
%полукруг радиуса 1 с центром в начале координат, лежащий в верхней
%полуплоскости?
\кзадача

\vspace*{-1mm}


\ЛичныйКондуит{0mm}{5mm}

%\СделатьКондуит{3.7mm}{7.5mm}


\end{document}





\задача
\вСтрочку
\пункт
Куда отображение $z\longmapsto1/z$ переводит
полукруг радиуса 1 с центром в начале координат, лежащий в верхней
полуплоскости?
\спункт Тот же вопрос для отображения
$z\longmapsto0,5(z+1/z)$.
\кзадача

\ЛичныйКондуит{0mm}{6mm}



\СделатьКондуит{5mm}{7.5mm}





\end{document}

%\задача
%Нарисуйте образы следующих множеств при отображениях, задаваемых
%функциями:
%\вСтрочку
%\пункт $f(z)=\bar z$;\hfil
%\пункт $f(z)=z^n,\ n\in\N$;\\ \\ \\ \\ \\
%\пункт $f(z)=\sqrt z$;\hfil
%\пункт $f(z)=1/z$;\\ \\ \\ \\ \\
%\пункт $f(z)=e^z$;\hfil
%\пункт $f(z)=\sin z$;\\ \\ \\ \\ \\
%\пункт $f(z)=\cos z$;\hfil
%\спункт $f(z)=\tg z$;\\ \\ \\ \\ \\
%\спункт $f(z)=0,5(z+1/z)$.
%\кзадача




На прямоугольную карту положили карту той же
местности, но меньшего масштаба
(меньшая карта целиком лежит внутри большей).
Пусть масштаб первой карты в $k$ раз больше масштаба второй карты,
вторая карта сдвинута на вектор $z$ и повернута на угол $\alpha$
относительно первой карты. Найдите координаты точки, которая
изображает на обеих картах одну и ту же точку местности.

\задача
Докажите, что вещественная и мнимая части любого корня квадратного
уравнения с комплексными коэффициентами выражаются через вещественные
и мнимые части коэффициентов уравнения с помощью арифметических операций
и извлечения действительного
квадратного корня (т.~е.~\лк выражаются в радикалах\пк).
\кзадача

\задача
%\вСтрочку
%\пункт
Выразите в радикалах
$\cos\frac{2\pi}{5}$ и
$\sin\frac{4\pi}{5}$ и
%\кзадача
%\задача
%\пункт
постройте %при помощи
циркулем и линейкой правильный пятиугольник.
\кзадача 