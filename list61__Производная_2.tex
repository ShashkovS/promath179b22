\documentclass[a4paper, 12pt]{article}
\usepackage{newlistok}
%\documentstyle[11pt, russcorr, listok]{article}

\УвеличитьШирину{1cm}
\pagestyle{empty}
\begin{document}

\Заголовок{Производная. Касательная}
\НомерЛистка{61}
\ДатаЛистка{15.02 -- 01.03.2021}
\Оценки{11/8/5}
\СоздатьЗаголовок

%\overfullrule=3pt

%\begin{document}



%\раздел{Касательная}

%Рассмотрим график функции $y=f(x)$. Фиксируем такую точку $(x_0,f(x_0))$,
\опр
Пусть функция $f$ определена в некоторой окрестности $U$ точки $a$.
Для каждой точки $b\in U$, $b\ne a$, рассмотрим {\em секущую}: прямую $l_b=k_b x +c_b$, проходящую
через точки $(a,f(a))$ и $(b,f(b))$ (напишите её уравнение).
Если существует предельная прямая $l=(\lim\limits_{b\to a} k_b)x + (\lim\limits_{b\to a} c_b)$ для семейства прямых $l_b$
при $b\to a$, то она называется {\em касательной} к графику $f$ в точке $a$. %Уточните это определение и
\копр

\пзадача
%\пункт
Напишите уравнение касательной к графику дифференцируемой функции $f(x)$ в точке~$x_0$.
%\пункт Совпадает ли оно для окружности %(график функции $y=\sqrt{R^2-x^2}$)
%с известным из~геометрии?
\кзадача

\пзадача
Найдите касательную к параболе $y=\frac{x^2-3x+3}3$, параллельную прямой
$y=x$.
\кзадача

\пзадача
Докажите, что отрезок любой касательной к графику функции $y=1/x$,
концы которого расположены на осях координат, делится точкой касания
пополам.
\кзадача


\задача
Под каким углом пересекаются кривые:
\вСтрочку
\пункт
$y=x^2$ и $x=y^2$;
\пункт
$y=\sin x$ и $y=\cos x$?\\
(Угол между кривыми в точке их пересечения --- это угол между касательными к кривым в этой точке.)
\кзадача

\задача
Докажите, что касательная к гиперболе $xy=a^2$ образует с осями координат
треугольник постоянной площади.
\кзадача

\пзадача \пункт Напишите уравнение касательной к окружности $x^2+y^2=1$ в её точке $(a,b)$.\\
\пункт Докажите, что эта касательная перпендикулярна радиусу, проведённому в точку $(a,b)$.
\кзадача

\задача
В каком наибольшем конечном числе точек прямая может касаться синусоиды?
\кзадача



\задача
Найдите геометрическое место точек, из которых парабола $y=x^2$
видна под прямым углом. (Угол, под которым видна парабола из данной точки --- это угол между касательными к параболе, проведёнными из этой точки).
\кзадача



\сзадача Пусть функция $f$ дифференцируема на $\R$, точка $A$ плоскости
не лежит на графике функции $f$, и $M$ --- такая точка графика
функции $f$, что расстояние $AM$ минимально. Докажите, что отрезок
$AM$ перпендикулярен касательной к графику $f$ в точке $M$.
\кзадача

\задача
%\пункт
Параллельный пучок лучей, падающий на параболу $y=x^2$ по
вертикали сверху, отражается от не\"е по закону
\лк угол падения равен углу отражения\пк.
Докажите, что все лучи этого пучка после первого отражения
пройдут через одну и ту же точку, и найдите эту точку (она называется <<фокусом>> параболы).
%\пункт
%Решите эту задачу для произвольной параболы $y=ax^2+bx+c$, где $a>0$.
\кзадача


\сзадача
%Гипербола --- это геометрическое место точек, разность расстояний
%от которых до двух данных точек $F_1$ и $F_2$ постоянна.
Дана гипербола $y=1/x$, пусть $F_1$ и $F_2$ --- точки $(\sqrt2;\sqrt2)$ и $(-\sqrt2;-\sqrt2)$ (её фокусы).
Докажите, что поток лучей из точечного источника света $F_1$, отразившись от гиперболы,
предстанет стороннему наблюдателю как поток лучей от точечного источника~в~$F_2$.
\кзадача

\сзадача
Докажите, что любая касательная к гиперболе $y=1/x$ образует равные
по величине углы с двумя прямыми, одна из которых проходит через
точку касания и точку $(\sqrt2;\sqrt2)$, а другая ---  через
точку касания и точку $(-\sqrt2;-\sqrt2)$.
\кзадача

% \сзадача
% \пункт Из точки $A$ проведены касательные $AB$ и $AC$ к эллипсу с фокусами $F_1$ и $F_2$. Докажите, что $\angle F_1AB=\angle F_2AC.$
% \пункт Докажите, что луч, выпущенный из внутренней точки эллипса, отражаясь от зеркальных стенок эллипса, будет всегда касаться некоторого другого эллипса или гиперболы, если он не проходит через фокусы эллипса и не летает по одной прямой.
% \кзадача


\сзадача
%\вСтрочку
%\пункт
%Найдите все $r$, при которых на %координатной
%плоскости $Oxy$
Существует ли окружность, % радиуса $r$,
пересекающая параболу $y=x^2$ ровно в двух точках так,
что в одной из этих точек у параболы и окружности есть общая
касательная, а в другой --- нет?
%\пункт
%Приведите пример такой окружности (хотя бы для одного~$r$).
\кзадача


\сзадача
\пункт Из точки $A$ проведены касательные $AB$ и $AC$ к эллипсу с фокусами $F_1$ и $F_2$. Докажите, что $\angle F_1AB=\angle F_2AC.$
\пункт Докажите, что луч, выпущенный из внутренней точки эллипса, отражаясь от зеркальных стенок эллипса, будет всегда касаться некоторого другого эллипса или гиперболы, если он не проходит через фокусы эллипса и не летает по одной прямой.
\кзадача

\ЛичныйКондуит{0mm}{8mm}

%\СделатьКондуит{4.5mm}{7.5mm}

% \GenXMLW


\end{document}

\раздел{Нули производной}


%\noindent {\Bf Соглашение.}
%Скажем, что функция $f$ удовлетворяет условию $(*)$, если $f$
%непрерывна на отрезке $[a,b]$  и дифференцируема на интервале $(a,b)$.

\задача
Пусть $f$
непрерывна на $[a,b]$, дифференцируема на $(a,b)$
и $f'(x_0)>0$ в некоторой точке~\hbox{$x_0\in (a,b)$.}
\пункт Найдется ли такая окрестность $U$ точки $x_0$, что
для всех \hbox{$x\in U$} если $x>x_0$, то
$f(x)>f(x_0)$, а если $x<x_0$, то $f(x)<f(x_0)$?
\спункт Верно ли, что $f$ монотонно
возрастает в некоторой окрестности $x_0$?
\кзадача


\опр Точка $x_0$ называется точкой \выд{локального максимума} функции $f$,
если $f(x_0)\ge f(x)$ для всех $x$ из некоторой окрестности $x_0$.
Если выполнено строгое неравенство $f(x_0)> f(x)$, говорят о строгом локальном максимуме.
Аналогично определяется точка \выд{(строгого) локального минимума}.
Такие точки называют точками \выд{(строгого) локального экстремума}
\копр

%\задача Докажите, что у любой функции $f$,
%непрерывной на отрезке $[a,b]$  и дифференцируемой на интервале $(a,b)$,
%существует точка локального максимума и точка локального минимума.
%\кзадача

\задача \пункт \выд{(Теорема Ферма)} Пусть $f$
непрерывна на $[a,b]$  и дифференцируема на $(a,b)$.
Докажите, что если $x\in (a,b)$ --- точка локального
максимума (минимума) $f$, то $f'(x)=0$. \пункт Верно ли обратное?
%\пункт Верно ли, что если $f'(x)=0$ для $x\in (a,b)$, то $x$ --- точка
%локального максимума или минимума?
\кзадача

\задача
Пусть функция определена на отрезке $[a,b]$. Тогда точками её локального экстремума
\кзадача


\задача Докажите для всех $x$:
\label{ex}
\вСтрочку
\пункт $x^4+x^3\ge -\frac{3^3}{4^4}$;
\пункт $x^6-6x+5\ge 0$;
\пункт $x^4-4x^3+10x^2-12x+5\ge 0$.
\кзадача


%\задача Найдите все точки локальных экстремумов:
%функций на интервале $(-\infty ,\infty )$:
%\label{ex}
%\вСтрочку
%\пункт $x+1$;
%\пункт $x^2-1$;
%\пункт $x^3+x$
%\пункт $\sin x$.
%\кзадача

\задача Найдите наибольшее и наименьшее значение при $x\in [0,1]$ функций
из задачи~\ref{ex}.
\кзадача



\задача Найдите наименьшее значение % выражения
при $x>0$:
\вСтрочку
\пункт $x+\frac{1}{x}$;
\пункт $x+\frac{1}{x^2}$;
\пункт $x^2+2x+\frac{4}{x}$.
\кзадача

\задача
\пункт Какую наибольшую площадь может иметь трапеция,
три стороны которой равны~1?\\
\пункт Какова наибольшая возможная площадь %может иметь
четыр\"ехугольника, 3 стороны которого равны~1?\\
\пункт У какого равностороннего шестиугольника со стороной 1 %имеет
площадь наибольшая?
\кзадача

\задача \выд{(Теорема Ролля)} %Пусть функция $f$ удовлетворяет условию $(*)$
Пусть $f$ непрерывна на $[a,b]$  и дифференцируема на $(a,b)$,
и, кроме того, $f(a)=f(b)$. Докажите, что найд\"ется такая точка $x\in
(a,b)$, что $f'(x)=0$.
\кзадача

\задача  \выд{(Теорема Лагранжа)}
%\вСтрочку
Пусть $f$ непрерывна на $[a,b]$  и дифференцируема на $(a,b)$.
%Пусть функция $f$ удовлетворяет условию $(*)$.
Докажите, что найд\"ется такое $x\in (a,b)$, что
$f'(x)=\frac{f(b)-f(a)}{b-a}$ и
объясните геометрический смысл этой теоремы. % Лагранжа.
\кзадача

\задача %Пусть функция $f$ удовлетворяет условию $(*)$
Пусть $f$ непрерывна на $[a,b]$  и дифференцируема на $(a,b)$.
Докажите, что если для всех $x\in (a,b)$ выполнено:
\вСтрочку
\пункт $f'(x)=0$, то $f$ постоянна
на $[a,b]$.
\пункт $f'(x)>0$,
то $f$ возрастает на $[a,b]$.
\кзадача

\задача Докажите, что для для всех $x>0$ выполнены неравенства:\\
\вСтрочку
%\пункт $e^x>1+x$;
%\пункт $e^x>1+x+\frac{x^2}{2}$;
\пункт $\sin x>x-\frac{x^3}{6}$;
\пункт $1-\frac{x^2}{2}<\cos x<1-\frac{x^2}{2!}+\frac{x^4}{4!}$;
\спункт $e^x>1+x+\frac{x^2}{2}+\ldots +\frac{x^n}{n!}$, где $n\in \N$.
\кзадача

\сзадача Найдите все дифференцируемые на $\R$ функции $f$, такие что
$f'(x)=f(x)$ для всех $x\in \R$.
\кзадача

%\задача
%Пусть функция $f$ дифференцируема на $\R$,
%и для каждой точки $a\in\R$ либо $f(a)=0$, либо $f'(a)=0$.
%Докажите, что $f$ --- константа.
%\кзадача


----------

\раздел{Касательная}

%Рассмотрим график функции $y=f(x)$. Фиксируем такую точку $(x_0,f(x_0))$,
\опр
Пусть функция $f$ определена в некоторой окрестности $U$ точки $x_0$.
Для каждой точки $x\in U$, $x\ne x_0$, рассмотрим прямую $l(x)$, проходящую
через точки $(x_0,f(x_0))$ и $(x,f(x))$. % (она называется секущей).
Если существует предельная прямая для семейства прямых $l(x)$
при $x\to x_0$, то она называется {\em касательной} к графику
$f$ в точке $x_0$. %Уточните это определение и
\копр

\взадача
%\пункт
Напишите уравнение касательной к графику функции $f(x)$ в точке $x_0$.
%\пункт Совпадает ли оно для окружности %(график функции $y=\sqrt{R^2-x^2}$)
%с известным из~геометрии?
\кзадача


\задача
Под каким углом пересекаются кривые:
\вСтрочку
\пункт
$y=x^2$ и $x=y^2$;
\пункт
$y=\sin x$ и $y=\cos x$?
\кзадача

\задача
Найдите геометрическое место точек, из которых парабола $y=x^2$
видна под прямым углом.
\кзадача

\задача
Докажите, что отрезок любой касательной к графику функции $y=1/x$,
концы которого расположены на осях координат, делится точкой касания
пополам.
\кзадача

\сзадача
%\пункт
Параллельный пучок лучей, падающий на параболу $y=x^2$ по
вертикали сверху, отражается от не\"е по закону
\лк угол падения равен углу отражения\пк.
Докажите, что все лучи этого пучка после первого отражения
пройдут через одну и ту же точку, и найдите эту точку.
%\пункт
%Решите эту задачу для произвольной параболы $y=ax^2+bx+c$, где $a>0$.
\кзадача

\сзадача
%Гипербола --- это геометрическое место точек, разность расстояний
%от которых до двух данных точек $F_1$ и $F_2$ постоянна.
Дана гипербола с фокусами $F_1$ и $F_2$.
Докажите, что поток лучей от точечного источника света в $F_1$, отразившись
от гиперболы, предстанет стороннему наблюдателю как поток лучей от точечного
источника~в~$F_2$.
\кзадача

%\сзадача
%\пункт Из точки $A$ проведены касательные $AB$ и $AC$ к эллипсу с фокусами $F_1$ и $F_2$. Докажите, что $\angle F_1AB=\angle F_2AC.$
%\пункт Докажите, что луч, выпущенный из внутренней точки эллипса, отражаясь от зеркальных стенок эллипса, будет всегда касаться некоторого другого эллипса или гиперболы, если он не проходит через фокусы эллипса и не летает по одной прямой.
%\кзадача




%\noindent {\Bf Соглашение.}
%Скажем, что функция $f$ удовлетворяет условию $(*)$, если $f$
%непрерывна на отрезке $[a,b]$  и дифференцируема на интервале $(a,b)$.

\сзадача
%\вСтрочку
%\пункт
%Найдите все $r$, при которых на %координатной
%плоскости $Oxy$
Существует ли окружность, % радиуса $r$,
пересекающая параболу $y=x^2$ ровно в двух точках,
прич\"ем в одной из этих точек у параболы и окружности есть общая
касательная, а в другой --- нет?
%\пункт
%Приведите пример такой окружности (хотя бы для одного~$r$).
\кзадача


====

\задача  Нарисуйте кривые:
\вСтрочку
%\пункт $x=y$;
\пункт $x^2=y^2$;
%\пункт $y=x^2$;
\пункт $х^2y-xy^2=x-y$;
%\пункт $ax^2+by^2=1$, где $a,b$ --- такие числа, что $a>b>0$;
%\пункт $ax^2-by^2=1$, где $a,b$ --- такие числа, что $a>b>0$;
\пункт $y^2=x^3$;\quad
\пункт $y-1=x^3$;\quad
\пункт $y^2-1=x^3$;\quad
\пункт $y^2-x=x^3$;\quad
\пункт $y^2-x^2=x^3$.
\кзадача

\сзадача Нарисуйте кривые:
\вСтрочку
\пункт $x^2=x^4+y^4$;
\пункт $xy=x^6+y^6$;
\пункт $x^3=y^2+x^4+y^4$;
\пункт $x^2y+xy^2=x^4+y^4$.
\кзадача 