\documentclass[a4paper,11pt]{article}
\usepackage[mag=1000]{newlistok}

\ВключитьКолонтитул

\УвеличитьШирину{1.4cm}
\УвеличитьВысоту{2.3cm}

\Заголовок{Комплексные числа}
\НомерЛистка{39}
% \renewcommand{\spacer}{\vspace{0.9pt}}
\ДатаЛистка{21.10 -- 11.11.2019}
% 45 задач
\Оценки{38/31/24}


\begin{document}

\СоздатьЗаголовок

%\medskip

\опр {\em Компл\'ексное число\/} $z$ --- это выражение вида
$z=a+bi$, где $a,\ b\in\R$, а $i$ --- {\em мнимая
единица\/}. По определению  $i^2=-1$.
Число $a$ называют \выд{вещественной частью} %комплексного числа
$z$ (пишут $a={\rm Re}\,(z)$), а число $b$ ---
\выд{мнимой частью} $z$ (пишут $b={\rm Im}\,(z)$).
Комплексные числа складывают и умножают,
\лк раскрывая скобки и приводя подобные\пк.
Множество комплексных чисел обозначают буквой $\Cbb$.
\копр




\пзадача
Найдите вещественную и мнимую части суммы и произведения чисел $a+bi$ и $c+di$.
\кзадача



\опр
Каждому комплексному числу $z=a+bi$ сопоставим точку $(a,b)$
и вектор %с координатами
$(a,b)$. Длина этого вектора называется {\em модулем\/} %комплексного
числа $z$ и обозначается $|z|$. Пусть $z\ne0$.
Угол (в радианах), отсчитанный против часовой стрелки
от вектора %с координатами
$(1,0)$ до вектора %с координатами
$(a,b)$,
называется {\em аргументом\/} %комплексного
числа $z$ и обозначается
${\rm Arg}\,(z)$. %(Углы измеряем в радианах.)
Аргумент %комплексного числа
определен с точностью до прибавления числа вида $2\pi n$, где $n\in\Z$.
\копр

\задача
\пункт Каков геометрический смысл суммы комплексных чисел?
\пункт Сравните $|z+w|$ и $|z|+|w|$ для $z,w\in\Cbb$.
\спункт Можно ли сравнивать комплексные числа так, чтобы сохранились основные свойства неравенств (домножение на число, большее 0, не меняет знак неравенства и т.п.)? Верно ли, что $i>0$, или что $i<0$?
\кзадача

\задача
Найдите модуль и аргумент чисел:  %следующих комплексных чисел:
\пункт $-4,\quad 1+i,\quad 1-i\sqrt{3},\quad \sin\alpha+i\cos\alpha$;
\пункт $1+\cos\alpha+i\sin\alpha.$
\кзадача


\пзадача [Тригонометрическая форма записи]
Докажите, что для любого ненулевого комплексного числа $z$ имеет место
равенство $z=r (\cos \varphi +i\sin \varphi )$, где $r=|z|$,
$\varphi={\rm Arg}\,(z)$.
\кзадача



\ввпзадача
Рассмотрим умножение точек комплексной плоскости на %комплексное число
$\cos\varphi+i\sin\varphi$ как преобразование $f$ этой плоскости, переводящее $z$ в $(\cos\varphi+i\sin\varphi)z$.
Куда при этом преобразовании перейдут\\ \пункт точки действительной оси; \quad \пункт точки мнимой оси?\\
\пункт Докажите, что %описанное
%то преобразование %--- это поворот плоскости
$f$ --- поворот против часовой стрелки на угол $\varphi$ вокруг начала координат.\\
\пункт
Пусть $z,w\in\Cbb$. % --- комплексные числа.
Выразите $|zw|$ и
${\rm Arg}\,(zw)$
через $|z|$, $|w|$, ${\rm Arg}\,(z)$, ${\rm Arg}\,(w)$.\\
\пункт Выведите из предыдущего пункта
формулы для косинуса суммы и синуса суммы.
\кзадача

\задача
\пункт Из любого ли комплексного числа можно извлечь
квадратный корень? \\
\пункт Решите уравнение $z^2=i$.
\пункт Найдите ошибку: $1=\sqrt 1=\sqrt{(-1)(-1)}=\sqrt{-1}\sqrt{-1}=i\cdot i=i^2=-1$.\кзадача
\задача
Докажите, что если и $m$ и $n$ --- суммы двух квадратов целых чисел, то и $mn$ --- тоже.
\кзадача

\ввпзадача
[Формула Муавра] Пусть $z=r(\cos\varphi+i\sin\varphi)$, $n\in\N$.
Докажите: %, что для любого натурального $n$ выполнено равенство
$z^n=r^n(\cos n\varphi+i\sin n\varphi)$.
\кзадача

\задача
Вычислите \пункт $(1+i)^{333}$; \пункт $(1+i\sqrt{3})^{150}$.
\пункт Выразите $\cos nx$ и $\sin nx$ через $\cos{x}$ и $\sin{x}$ ($n\in\N$).
\кзадача


\опр {(\em Формула Эйлера\/})
$e^{i\varphi}=\cos\varphi+i\sin\varphi$. Мы сможем доказать эту формулу в 11 классе, а пока  можно использовать выражение $e^{i\varphi}$ как короткое и удобное {\em обозначение} для $\cos\varphi+i\sin\varphi$.
%Чтобы убедиться в удобстве этого обозначения, решите следующую задачу.
\копр

\задача
\пункт Докажите, что $e^{i\varphi}e^{i\psi}=e^{i(\varphi+\psi)}$.\\
\пункт Можно ли найти $e^{i\varphi}+e^{i\cdot2\varphi}+\ldots+e^{i\cdot n\varphi}$ по формуле суммы геометрической прогрессии?
%\спункт Определите $e^z$ для $z\in\Cbb$ и докажите, что $e^ze^w=e^{z+w}$ для любых $z,w\in\Cbb$.
%\сспункт Подумайте, как возводить комплексное число в комплексную степень. %, чтобы выполнялись обычные свойства возведения в степень. Однозначна ли эта операция?
\кзадача

\пзадача
Найдите: % суммы
\пункт
$\sin \varphi +\sin 2\varphi +\ldots +\sin n\varphi$;
\пункт
$C_{n}^{1}-C_{n}^{3}+C_{n}^{5}-C_{n}^{7}+\dots;$
\пункт
$C_{n}^{0}+C_{n}^{4}+C_{n}^{8}+C_{n}^{12}+\dots$
\кзадача





\опр
Пусть $z=a+bi$, где $a,b\in\R$. Число $\overline{z}=a-bi$ называют
{\em комплексно-сопряжённым\/} к $z$.
\копр

\пзадача %$\!\!\!$
\пункт Выразите $|\overline{z}|$, ${\rm Arg}\,(\overline{z})$  через $|z|$, ${\rm Arg}\,(z)$.
% \пункт Что это за геометрическое преобразование:~$z\longmapsto\overline{z}$?
% \кзадача
%
% \задача
Докажите:
\вСтрочку
\пункт
$|z|^2=z\overline{z}$; % для любого $z\in\C$.
\пункт
$\overline{z+w}=\overline{z}+\overline{w}$,
$\overline{zw}=\overline{z}\,\overline{w}$;\\
\пункт
если $P(x)$ --- многочлен с вещественными коэффициентами и $P(z)=0$, то $P(\overline{z})=0.$
\кзадача


\задача
Докажите, что многочлен степени $n$ с %комплексными
коэффициентами из $\Cbb$ имеет не более $n$ %комплексных
корней из $\Cbb$.
\кзадача




\пзадача
\вСтрочку
\пункт Определите деление комплексных чисел;
\пункт вычислите $\frac{(5+i)(7-6i)}{3+i}$;
\пункт вычислите $\frac{(1+i)^5}{(1-i)^3}$.
\кзадача


%\задача
%Докажите, что многочлен степени $n$ с %комплексными
%коэффициентами из $\Cbb$ имеет не более $n$ %комплексных
%корней из $\Cbb$.
%\кзадача

\опр
{\em Корнем из 1 степени $n$} называется любое такое комплексное число $z$, что $z^n=1$.
\копр

\пзадача
\пункт Найдите и нарисуйте все корни из 1 степеней 2, 3, 4, 5 и 6.
\пункт Сколько всего корней из 1 степени $n$? Найдите их произведение и сумму их $s$-х степеней для каждого $s\in\N$.
% \пункт
% Пусть $\alpha_1$, \dots, $\alpha_n$ --- все корни степени $n$ из 1,
% $\alpha_1=1$.
% Найдите $\alpha_1^s+\ldots+\alpha_n^s$ (где $s\in\N$) и
% $(1-\alpha_2)\cdot\ldots\cdot(1-\alpha_n)$.
\кзадача

\задача Пусть $P$ --- многочлен степени $k$ с коэффициентами из $\Cbb$.
Докажите, что среднее арифметическое значений $P$ в вершинах
правильного $n$-угольника равно значению $P$ в центре многоугольника,
если $n>k$.
\кзадача

\сзадача
Вершины правильного $n$-угольника покрашены в несколько цветов так, что точки одного цвета --- вершины правильного многоугольника. Докажите: среди этих многоугольников есть равные.
\кзадача

\сзадача
\вСтрочку
\пункт
Пусть $z= (3+4i)/5$. Найд\"ется ли такое $n\in\N$, что $z^n=1$?
\пункт
Докажите, что $\frac1\pi\arctg\frac43\notin\Q$. % иррационально.
\кзадача



\пзадача
Нарисуйте: %множество комплексных чисел $z$, для которых
\вСтрочку
\!\!\!\! \пункт \!\!\!\! $\{ z\in\Cbb\,\,|\,\,z^n+1=0\}$;
\!\!\!\!  \пункт \!\!\!\! $\{z\in\Cbb\,\,|\,\,2\geq|z-i|\}$;
\!\!\!\!  \пункт \!\!\!\! $\{z\in\Cbb\,\,|\,\,{\rm Re}\,\frac{1}{z}=1\}$;
\!\!\!\! \пункт \!\!\!\!  $\{\frac{1+ti}{1-ti}\,\,|\,\,t\in\R\}$.
\кзадача



%\сзадача
%\вСтрочку
%\пункт
%Пусть $z= (3+4i)/5$. Найд\"ется ли такое $n\in\N$, что $z^n=1$?\\
%\пункт
%Докажите, что $\frac1\pi\arctg\frac43\notin\Q$. % иррационально.
%\кзадача





\ЛичныйКондуит{0mm}{5mm}
% \GenXMLW

%\СделатьКондуит{3.7mm}{7.5mm}


\end{document}

\задача
Пусть карты из задачи 17 листка 22 лежат на комплексной плоскости.
Докажите, что найдутся такие $q,b\in\Cbb$, что
если $z\in\Cbb$ --- любая точка на первой карте, то этой же точкой местности
на второй карте будет точка $qz+b$.
Выразите с помощью  $q$ и $b$ точку,
изображающую на картах одну и ту же точку местности.
\кзадача



{\hsize 13.2cm

\задача
Запишите %в виде
как функцию комплексной переменной
\вСтрочку
\!\! \пункт \!\! симме\-трию относительно оси $y$;
\пункт ортогона\-ль\-ную проекцию на ось $x$;
\пункт центральную симметрию с центром $A$;
\пункт поворот на угол $\varphi$ относительно~точки~$A$;
\пункт гомотетию с коэффициентом $k$ и центром $A$;
\пункт %скользящую
симметрию относительно прямой $y=3$ со сдвигом на 1 влево;
\пункт поворот, %который
переводящий ось $x$ в прямую $y=2x+1$;
\пункт симметрию относительно прямой $y=2x+1$.
\кзадача


\задача
Куда отображение $z\longmapsto z^2$
переводит
\вСтрочку
\пункт
декартову координатную сетку;
\пункт
полярную координатную сетку;
\пункт
окружность $|z+i|=1$;
\пункт
кошку (см.~рис.~справа)?
\пункт
Те же вопросы для отображения
$z\longmapsto 1/z$.
\кзадача

}


\putpict{15.8cm}{2.5cm}{curve}{}

\vspace*{-5mm}


%\задача
%Куда отображение $z\longmapsto\sqrt z$ переводит
%$\{z\in\Cbb\ |\ {\rm Im}\,(z)>0\}$?
%верхнюю полуплоскость (без границы)?
%\кзадача

\задача
Куда отображение
\вСтрочку
\пункт
$z\longmapsto1/z$;
\спункт $z\longmapsto0,5(z+1/z)$
переводит %множество
$\{z\in\Cbb\ |\ {\rm Im}\,(z)>0,\ |z|\leq1\}$?
%полукруг радиуса 1 с центром в начале координат, лежащий в верхней
%полуплоскости?
\кзадача

\vspace*{-1mm}


\ЛичныйКондуит{0mm}{5mm}

%\СделатьКондуит{3.7mm}{7.5mm}


\end{document}





\задача
\вСтрочку
\пункт
Куда отображение $z\longmapsto1/z$ переводит
полукруг радиуса 1 с центром в начале координат, лежащий в верхней
полуплоскости?
\спункт Тот же вопрос для отображения
$z\longmapsto0,5(z+1/z)$.
\кзадача

\ЛичныйКондуит{0mm}{6mm}



\СделатьКондуит{5mm}{7.5mm}





\end{document}

%\задача
%Нарисуйте образы следующих множеств при отображениях, задаваемых
%функциями:
%\вСтрочку
%\пункт $f(z)=\bar z$;\hfil
%\пункт $f(z)=z^n,\ n\in\N$;\\ \\ \\ \\ \\
%\пункт $f(z)=\sqrt z$;\hfil
%\пункт $f(z)=1/z$;\\ \\ \\ \\ \\
%\пункт $f(z)=e^z$;\hfil
%\пункт $f(z)=\sin z$;\\ \\ \\ \\ \\
%\пункт $f(z)=\cos z$;\hfil
%\спункт $f(z)=\tg z$;\\ \\ \\ \\ \\
%\спункт $f(z)=0,5(z+1/z)$.
%\кзадача




На прямоугольную карту положили карту той же
местности, но меньшего масштаба
(меньшая карта целиком лежит внутри большей).
Пусть масштаб первой карты в $k$ раз больше масштаба второй карты,
вторая карта сдвинута на вектор $z$ и повернута на угол $\alpha$
относительно первой карты. Найдите координаты точки, которая
изображает на обеих картах одну и ту же точку местности.

\задача
Докажите, что вещественная и мнимая части любого корня квадратного
уравнения с комплексными коэффициентами выражаются через вещественные
и мнимые части коэффициентов уравнения с помощью арифметических операций
и извлечения действительного
квадратного корня (т.~е.~\лк выражаются в радикалах\пк).
\кзадача

\задача
%\вСтрочку
%\пункт
Выразите в радикалах
$\cos\frac{2\pi}{5}$ и
$\sin\frac{4\pi}{5}$ и
%\кзадача
%\задача
%\пункт
постройте %при помощи
циркулем и линейкой правильный пятиугольник.
\кзадача 