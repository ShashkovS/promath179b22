\documentclass[a4paper, 12pt]{article}
\usepackage{newlistok}
%\documentstyle[11pt, russcorr, listok]{article}

\УвеличитьШирину{1cm}
\pagestyle{empty}
\begin{document}

\Заголовок{Производная. Определение и приёмы вычисления}
\НомерЛистка{60}
\ДатаЛистка{01.02 -- 15.02.2021}
\Оценки{31/25/19}
\СоздатьЗаголовок



\раздел{Определение производной.}

%\опр
%Пусть функция $f$ определена на некотором интервале, и пусть точка $x_0$
%принадлежит этому интервалу.
%Функция $f$ называется \выд{дифференцируемой\/} в
%точке $x_0$, если $f(x_0+t)=f(x_0)+\a\cdot t+o(t)$, где $\a$
%--- некоторая константа, а функция $o(t)$ такова, что
%$\lim\limits_{t\to0}o(t)/t=0$. Говоря неформально,
%дифференцируемость означает, что с точностью до \лк бесконечно
%малой\пк\ по сравнению с $t$ функции $o(t)$, разность
%$f(x_0+t)-f(x_0)$ ведет себя как {\it линейная однородная
%функция\/} $\a\cdot t$ от $t$. Коэффициент $\a$
%называется \выд{производной\/} функции $f$ в точке $x_0$ и
%обозначается $f'(x_0)$, а линейная однородная функция
%$\a\,t$ от $t$ называется {\it дифференциалом\/}
%функции $f$ в точке $x_0$ и обозначается $df(x_0)$. Например,
%функция $y=x$ дифференцируема в каждой точке $x_0\in\R$, е\"е
%производная $x'\equiv1$, а дифференциал $dx(x_0)=(x-x_0)$ в
%каждой точке $x_0$. Поэтому, допуская известную вольность,
%производную произвольной дифференцируемой функции $f(x)$ часто
%помимо $f'$ обозначают через $\frac{df}{dx}$.
%\копр


\опр
Пусть функция $f$ определена на некотором интервале, точка $x_0$
принадлежит этому интервалу.
\выд{Производной функции $f$ в точке $x_0$} называется число
$f'(x_0):=\lim\limits_{x\to x_0}\frac{f(x)-f(x_0)}{x-x_0}$,
если этот предел существует (тогда говорят, что функция $f$
\выд{дифференцируема} в точке~$x_0$).
\копр

\задача Докажите, что
$f'(x_0)=\lim\limits_{t\to 0}\frac{f(x_0+t)-f(x_0)}{t}$.
\кзадача

\задача
Для каждого $a\in\R$ найдите $f'(a)$, если\\
\вСтрочку
\пункт $f(x)=c$, где $c\in\R$;
\пункт $f(x)=x^n, n\in \N$;
\пункт $f(x)=x^{-n}, n\in \N$.
\кзадача

\ввпзадача Докажите, что $f'(x_0)=A$
тогда и только тогда, когда найд\"ется такая функция $\beta(t)$, что
для всех достаточно малых $t$ будет верно
$f(x_0+t)=f(x_0)+At+\beta(t)$,
прич\"ем $\lim\limits_{t\to0}\beta(t)/t=0$.
\кзадача


\задача
Докажите, что функция, дифференцируемая в точке,
непрерывна в этой точке. %А наоборот?
\кзадача


%\задача
%Для каждой точки $x=0,1,2$ найдите $f'(x)$, если
%\вСтрочку
%\пункт $f(x)=1$;
%\пункт $f(x)=x$;
%\пункт $f(x)=x^2$;
%\пункт $f(x)=x^n, n\in \N$.
%\кзадача

\опр Говорят, что функция $f$ \выд{дифференцируема} на интервале,
если она дифференцируема в каждой точке этого интервала. При этом
её \выд{производной} называется функция $f':x\mapsto f'(x)$.
\копр

%\задача Найдите производные функций из задачи 3.
%\кзадача

%\задача Найдите все точки, в которых дифференцируемы функции,
%и найдите производные:\\
%\вСтрочку
\пзадача Найдите производные функций (там, где они существуют):
\вСтрочку
\пункт $|x|$;
%\пункт $\frac{1}{x}$;
\пункт $\sqrt{x}$;
\пункт $x^{3/2}$.
\кзадача

%\задача
%Приведите пример функции, определенной в некоторой окрестности точки
%$x_0$, непрерывной в этой окрестности и не дифференцируемой
%в точке~$x_0$.
%\кзадача

%{\bf Факт.} Существует функция, определенная на интервале $(0,1)$,
%непрерывная на всем интервале и не имеющая производной ни в одной точке
%этого интервала.  \кзадача


%\раздел{Смысл производной.}

% \задача Автомобиль едет по прямой дороге так, что в
% каждый момент времени $t$ он находится в точке с координатой $s(t)$.
% Что будет показывать спидометр автомобиля в момент времени $t_0$?
% \кзадача
%

\раздел{Вычисление производных}

\взадача Пусть функции $f$ и $g$ дифференцируемы на некотором интервале.
Докажите, что
\сНовойСтроки
\пункт функция $f+g$ тоже дифференцируема на этом
интервале и $(f+g)'=f'+g'$;
\пункт для любой константы $C$ функция $Cf$ тоже дифференцируема на этом
интервале и $(Cf)'=Cf'$;
\пункт функция $fg$ тоже дифференцируема на этом
интервале и $(fg)'=f'g+fg'$;
\пункт функция $f/g$ дифференцируема во всех точках
интервала, где $g(x)\ne 0$, и $(f/g)'=(f'g-fg')/g^2$.
\кзадача

\пзадача Найдите производные функций (там, где они существуют):
\вСтрочку
%\пункт $f(x)=ax+b$;
%\пункт $f(x)=ax^2+bx+c$;
\пункт $a_nx^n+\ldots +a_1x+a_0$;
\пункт $\frac{5x+6}{7x+8}$;
\пункт $\frac{1}{x^3-5x-2}$.
\пункт $\sin x$;
\пункт $\cos x$;
\пункт $\tg x$;
\пункт $\ctg x$;
\пункт $x^{m/n}$, где $m\in\Z,\ n\in\N$;
\пункт $e^x$.
\кзадача

\взадача %[Производная сложной функции]
Пусть $F(x)=f(g(x))$. Докажите, что
если %функция
$g$ дифференцируема в точке $x_0$, а %функция
$f$
дифференцируема в точке $g(x_0)$, то $F(x)$ дифференцируема в точке
$x_0$, и $F'(x_0)=f'(g(x_0))g'(x_0)$.
\кзадача

%\задача Вычислите производные\\
%\вСтрочку
%\пункт $f(x)=e^{x^2}$;
%\пункт $f(x)=e^{\cos(x)}$;
%\пункт $f(x)=\cos(e^x)$;
%\пункт $f(x)=\sqrt{e^x+x^2}$.
%\кзадача

\взадача %[Производная обратной функции]
\вСтрочку
\спункт Пусть функция $f$ на некотором
интервале непрерывна и имеет обратную функцию~$g$. Докажите, что если $f$
дифференцируема в точке $x_0$ из этого интервала и $f'(x_0)\ne 0$, то %функция
$g$ дифференцируема в точке $f(x_0)$ и $g'(f(x_0))=\frac{1}{f'(x_0)}$.\\
\пункт Каков геометрический смысл формулы из пункта~а)?
%, привед\"енной в предыдущем пункте.
\кзадача

\задача
Найдите производную функции $\root 3 \of x$ через формулу производной обратной функции.
\кзадача


\пзадача Продифференцируйте:\\
%Найдите производные:
\вСтрочку
\пункт $\sin x^2$;
\пункт $\arcsin x$;
\пункт $\arccos x$;
\пункт $\arctg x$;
\пункт $\ln x$;
\пункт $2^x$;
\спункт $x^\alpha$.
%\пункт $f(x)=x^{1/3}$.
\кзадача

\опр
Говорят, что многочлен $f(x)$ имеет \выд{кратный корень $\alpha$}, если он делится на $(x-\alpha)^k$, где целое $k\geq2$.
Если при этом $f(x)$ не делится на $(x-\alpha)^{k+1}$, говорят, что $\alpha$ --- {\em корень кратности~$k$}.
\копр

\взадача
\сНовойСтроки
\пункт Докажите, что
при дифференцировании кратность корня многочлена
понижается на~1.
\пункт
Докажите, что многочлен имеет кратный
корень тогда и только
тогда, когда он имеет общий
корень со своей производной.
\пункт
Пусть многочлен из $\Q[x]$ не раскладывается на множители с рациональными коэффициентами (неприводим над $\Q$). Может ли он
иметь кратный комплексный корень?
\кзадача


\ЛичныйКондуит{0mm}{8mm}

%\СделатьКондуит{4.5mm}{7.5mm}

% \GenXMLW


\end{document}


\раздел{Касательная}

%Рассмотрим график функции $y=f(x)$. Фиксируем такую точку $(x_0,f(x_0))$,
\опр
Пусть функция $f$ определена в некоторой окрестности $U$ точки $x_0$.
Для каждой точки $x\in U$, $x\ne x_0$, рассмотрим прямую $l(x)$, проходящую
через точки $(x_0,f(x_0))$ и $(x,f(x))$. % (она называется секущей).
Если существует предельная прямая для семейства прямых $l(x)$
при $x\to x_0$, то она называется {\em касательной} к графику
$f$ в точке $x_0$. %Уточните это определение и
\копр

\взадача
%\пункт
Напишите уравнение касательной к графику функции $f(x)$ в точке $x_0$.
%\пункт Совпадает ли оно для окружности %(график функции $y=\sqrt{R^2-x^2}$)
%с известным из~геометрии?
\кзадача


\задача
Под каким углом пересекаются кривые:
\вСтрочку
\пункт
$y=x^2$ и $x=y^2$;
\пункт
$y=\sin x$ и $y=\cos x$?
\кзадача

\задача
Найдите геометрическое место точек, из которых парабола $y=x^2$
видна под прямым углом.
\кзадача

\задача
Докажите, что отрезок любой касательной к графику функции $y=1/x$,
концы которого расположены на осях координат, делится точкой касания
пополам.
\кзадача

\сзадача
%\пункт
Параллельный пучок лучей, падающий на параболу $y=x^2$ по
вертикали сверху, отражается от не\"е по закону
\лк угол падения равен углу отражения\пк.
Докажите, что все лучи этого пучка после первого отражения
пройдут через одну и ту же точку, и найдите эту точку.
%\пункт
%Решите эту задачу для произвольной параболы $y=ax^2+bx+c$, где $a>0$.
\кзадача

\сзадача
%Гипербола --- это геометрическое место точек, разность расстояний
%от которых до двух данных точек $F_1$ и $F_2$ постоянна.
Дана гипербола с фокусами $F_1$ и $F_2$.
Докажите, что поток лучей от точечного источника света в $F_1$, отразившись
от гиперболы, предстанет стороннему наблюдателю как поток лучей от точечного
источника~в~$F_2$.
\кзадача

%\сзадача
%\пункт Из точки $A$ проведены касательные $AB$ и $AC$ к эллипсу с фокусами $F_1$ и $F_2$. Докажите, что $\angle F_1AB=\angle F_2AC.$
%\пункт Докажите, что луч, выпущенный из внутренней точки эллипса, отражаясь от зеркальных стенок эллипса, будет всегда касаться некоторого другого эллипса или гиперболы, если он не проходит через фокусы эллипса и не летает по одной прямой.
%\кзадача




%\noindent {\Bf Соглашение.}
%Скажем, что функция $f$ удовлетворяет условию $(*)$, если $f$
%непрерывна на отрезке $[a,b]$  и дифференцируема на интервале $(a,b)$.

\сзадача
%\вСтрочку
%\пункт
%Найдите все $r$, при которых на %координатной
%плоскости $Oxy$
Существует ли окружность, % радиуса $r$,
пересекающая параболу $y=x^2$ ровно в двух точках,
прич\"ем в одной из этих точек у параболы и окружности есть общая
касательная, а в другой --- нет?
%\пункт
%Приведите пример такой окружности (хотя бы для одного~$r$).
\кзадача


====

\задача  Нарисуйте кривые:
\вСтрочку
%\пункт $x=y$;
\пункт $x^2=y^2$;
%\пункт $y=x^2$;
\пункт $х^2y-xy^2=x-y$;
%\пункт $ax^2+by^2=1$, где $a,b$ --- такие числа, что $a>b>0$;
%\пункт $ax^2-by^2=1$, где $a,b$ --- такие числа, что $a>b>0$;
\пункт $y^2=x^3$;\quad
\пункт $y-1=x^3$;\quad
\пункт $y^2-1=x^3$;\quad
\пункт $y^2-x=x^3$;\quad
\пункт $y^2-x^2=x^3$.
\кзадача

\сзадача Нарисуйте кривые:
\вСтрочку
\пункт $x^2=x^4+y^4$;
\пункт $xy=x^6+y^6$;
\пункт $x^3=y^2+x^4+y^4$;
\пункт $x^2y+xy^2=x^4+y^4$.
\кзадача 