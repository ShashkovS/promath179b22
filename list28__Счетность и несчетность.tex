\documentclass[a4paper,12pt]{article}
\usepackage[mag=1000]{newlistok}

\УвеличитьШирину{1truecm}
\УвеличитьВысоту{2.5truecm}
\renewcommand{\spacer}{\vspace{2.4pt}}

\Заголовок{Счётные и несчётные множества}
\НомерЛистка{28}
\ДатаЛистка{08.12 -- 26.12.2018}
\Оценки{22/18/14}

\begin{document}
\СоздатьЗаголовок


\ввпзадача
\пункт
Докажите, что
в любом бесконечном множестве найдется счётное подмножество.\\
{\footnotesize ({\em Указание:} выберите сначала первый элемент в подмножество, потом второй, \dots)}
\\
\пункт Докажите, что множество
$M$ бесконечно тогда и только тогда, когда
оно равномощно множеству, полученному из $M$ удалением
одного элемента.
\кзадача

\задача Равномощны ли следующие множества точек:\\
%\вСтрочку
\пункт интервал и отрезок;
\пункт полуокружность и прямая;
\пункт интервал и прямая;
\пункт два круга;\\
\пункт окружность и треугольник;
\пункт квадрат с внутренностью %\footnote{Квадрат в этом листке --- это квадрат с внутренностью, например множество точек $(x,y)$, где $0\leq x, y\leq1$.}
и плоскость;
\пункт квадрат с внутренностью и круг;
%\пункт  круг и %круговое
%кольцо;
\пункт отрезок и счётное объединение непересекающихся отрезков?
% {\small ({\em Замечание:} квадрат в этом листке --- это квадрат
% с внутренностью, например множество точек
% $(x,y)$, где $0\leq x, y\leq1$.)}
\кзадача

%\задача Разбейте отрезок
%на счётное число %объединение
%непересекающихся множеств, равномощных отрезку.
%\кзадача

\ввпзадача
Из бесконечного множества $M$ удалили некоторое счётное множество
и получили бесконечное множество $M'$. Докажите, что $M$ и $M'$ равномощны.
\кзадача


\задача
Равномощно ли множество иррациональных чисел множеству
всех действительных чисел?
\кзадача

\задача
Равномощно ли множество всех лучей множеству всех
окружностей (на плоскости)?
\кзадача


% Несчетные множества

\задача
Докажите, что %следующие множества равномощны:
множество $S$ бесконечных последовательностей из 0 и
1,~\hbox{множество~всех} подмножеств множества $\N$
и множество бесконечных вправо и вниз таблиц из 0 и 1 равномощны.
\кзадача


%\раздел{Дополнительные задачи}

\ввпзадача
\пункт Дана бесконечная вправо и вниз таблица из 0 и 1.
Покажите, как по этой таблице составить бесконечную строку из 0 и 1,
которая не совпадёт ни с одной из строк таблицы.\\
{\footnotesize ({\em Указание:} надо, чтобы новая строка отличалась
от каждой строки таблицы хотя бы в одном месте.)}\\
\пункт
Докажите, что
%множество $S$ из задачи 6
множество бесконечных последовательностей из 0 и 1
\выд{несчётно:} бесконечно, но не является счётным.
(Говорят, что множества из предыдущей задачи
имеют мощность \выд{континуум}).
\кзадача


\задача
Пусть $S$ --- множество из задачи 6. Докажите, что
множества $S$ и $S\times S$ равномощны.
\кзадача

\задача
Докажите, что множество всевозможных
прямых на %декартовой
плоскости
равномощно множеству точек этой плоскости.
\кзадача


\ввпзадача
%Докажите, что отрезок $[0;1]$ (множество точек $x$, где $0\leq x\leq1$)
%равномощен множеству бесконечных последовательностей из 0 и 1;
Докажите, что множество точек любого отрезка %$I$
равномощно множеству\\
%\сНовойСтроки
%\вСтрочку
\пункт
$S$~зада\-чи 6;
%множеству бесконечных последовательностей из 0 и 1;
%\кзадача
%
%\сзадача
%Докажите, что множество точек любого отрезка $I$ равномощно
%\вСтрочку
\пункт
точек квадрата с внутренностью; %$I\times I$ (множеству пар $(x,y)$, где $x, y$ --- любые точки~из~ $I$);\\
\пункт
точек куба с внутренностью.
%множеству точек плоскости.
\кзадача



\ввпзадача [Теорема Кантора--Бернштейна]
Если множество $A$ равномощно %некоторому
подмножеству множества $B$ и множество $B$ равномощно %некоторому
подмножеству множества $A$, то %множества
$A$ и $B$ равномощны. \\
{\footnotesize({\em Указание:} вам поможет задача 11 листка 27.)}
\кзадача


\задача
Число $x\in(0;1)$ назовём \выд{вычислимым}, если есть
конечный алгоритм (например, программа на Питоне), который позволяет для каждого $n\in\N$
определить $n$-ый знак после запятой в десятичной записи $x$.
\вСтрочку
\пункт Докажите, что множество вычислимых чисел из
интервала $(0;1)$ счётно.
\пункт
Выпишем десятичные записи всех вычислимых чисел
в таблицу, и диагональным методом (как в задаче 7) построим вычислимое число, не входящее в таблицу.
(Это можно сделать, написав программу на Питоне, которая последовательно будет перебирать программы, дающие вычислимые числа, и менять у $n$-го числа $n$-ю цифру.)
Объясните это противоречие.
\кзадача

\сзадача
%\вСтрочку
%\пункт
Отрезок представлен в виде объединения двух множеств.
%Объединение двух множеств есть квадрат.
Докажите, что одно из этих множеств
равномощно отрезку.
{\footnotesize ({\em Указание:} отрезок равномощен квадрату с внутренностью.)}
\кзадача



%\сзадача
%Докажите, что множества задачи 12 равномощны \
%\вСтрочку
%\пункт
%множеству взаимно однозначных отображений из $\N$ в $\N$; \
%\пункт
%множеству бесконечных последовательностей %из %целых
%\кзадача

\сзадача
Докажите, что множества задачи 6 равномощны \\
%\вСтрочку
\пункт
множеству взаимно однозначных соответствий между $\N$ и $\N$; \\
\пункт
множеству бесконечных последовательностей %из %целых
натуральных чисел.
\кзадача


\задача Найдётся ли бесконечное количество попарно неравномощных бесконечных множеств?
\кзадача

\сзадача Пусть $A$ --- счётное множество,
$M$ --- некоторое множество подмножеств $A$. Известно, что из
любых двух элементов $M$ один есть подмножество другого.
Обязательно ли $M$ счётно?
\кзадача

%\ссзадача
\smallskip
\noindent
{\bf Интересный трудный факт.}
{\em Из любых двух множеств одно равномощно подмножеству другого.}
%\кзадача

\smallskip

\ЛичныйКондуит{0mm}{6mm}
% \GenXMLW


% \СделатьКондуит{8mm}{6.2mm}


\end{document}

