% !TEX encoding = Windows Cyrillic

\documentclass[a4paper,12pt]{article}
\usepackage[mag=1000]{newlistok}
\usepackage{tikz}
\usetikzlibrary{calc}

\ВключитьКолонтитул

\УвеличитьШирину{.7truecm}
\УвеличитьВысоту{2truecm}

\Заголовок{Поля}
\НомерЛистка{52}
\renewcommand{\spacer}{\vspace{1.3pt}}
\ДатаЛистка{01.09.2020 -- 11.09.2020}
% 54 задач
\Оценки{31/25/19}
\sloppy

\begin{document}


\СоздатьЗаголовок

\smallskip

Говоря вольно, поле --- это набор элементов, на которых
есть четыре арифметических операции: сложение, вычитание, умножение и
деление, обладающие привычными свойствами.
Аксиоматизация этих свойств приводит к такому определению:

\smallskip

\опр
\выд{Полем} называется любое множество $\Bbbk$, на котором заданы операции
\emph{сложения} ($+$) и \emph{умножения} ($\cdot$),
удовлетворяющие следующим условиям (аксиомам поля):
\begin{itemize}
\item[(A1)]
Для любых $a,b\in\Bbbk$ выполнено равенство $a+b=b+a$
(\emph{коммутативность сложения}).
\item[(A2)]
Для любых $a,b,c\in\Bbbk$ выполнено равенство $(a+b)+c=a+(b+c)$
(\emph{ассоциативность сложения}).
\item[(A3)]
В $\Bbbk$ существует такой элемент $0$, что для любого $a\in\Bbbk$
выполнено равенство
$a+0=a$ (\emph{существование нуля}).
\item[(A4)]
Для любого $a\in\Bbbk$ существует такой $b\in\Bbbk$, что $a+b=0$
(\emph{существование противоположного элемента}: такой элемент $b$
называется \emph{противоположным} к $a$ и обозначается $-a$).
\item[(M1)]
Для любых $a,b\in\Bbbk$ выполнено равенство $a\cdot b=b\cdot a$
(\emph{коммутативность умножения}).
\item[(M2)]
Для любых $a,b,c\in\Bbbk$ выполнено равенство $(a\cdot b)\cdot
c=a\cdot (b\cdot c)$ (\emph{ассоциативность умножения}).
\item[(M3)]
В $\Bbbk$ существует такой элемент $1$, не равный нулю, что для
любого $a\in\Bbbk$ выполнено равенство $a\cdot 1=a$
(\emph{существование единицы}).
\item[(M4)]
Для любого $a\in\Bbbk$, не равного нулю, существует такой
$b\in\Bbbk$, что $a\cdot b=1$ (\emph{существование обратного
элемента}: такой элемент $b$ называется \emph{обратным} к $a$ и
обозначается $\dfrac1a$ или~$a^{-1}$).
\item[(AM)]
Для любых $a,b,c\in\Bbbk$ выполнено равенство $a\cdot(b+c)=a\cdot
b + a\cdot c$ (\emph{дистрибутивность умножения относительно
сложения}).
\end{itemize}
\копр

%Под аксиоматизацией имеется ввиду следующее. В полях выполнено много <<естественных>> свойств, которые мы привыкли использовать. Например, что <<минус на минус даёт плюс>>, то есть $-(-a)=a$ для любого элемента $a$. При естественно потребовать минимальное количество свойств, которые мы и будем требовать. То, что данный список аксиом --- минимальный (то есть никакая из аксиом не выводится из остальных) является хорошим упражнением.

\smallskip

Примерами известных вам полей являются $\mathbb Q$ — рациональные числа, $\mathbb R$ — действительные числа, $\mathbb C$ — комплексные числа.
%$\mathbb F_p=\Z/p\Z$ — вычеты по модулю простого числа $p$,
Более сложный пример: множество всех алгебраических чисел — корней многочленов с рациональными коэффициентами (основная трудность тут --- доказать,
что сумма и произведение алгебраических чисел тоже алгебраические числа).

Из множеств вроде целых чисел или многочленов, в которых выполнены все аксиомы, кроме М4 (существование обратного), и нет делителей нуля, можно изготовить \textit{поле частных}:
это множество дробей с ненулевым знаменателем (как обычно, дроби $a/b$ и $c/d$, у которых $ad = bc$, нужно считать равными), которые складываются и умножаются по обычным правилам.
Таким образом из целых чисел получаются рациональные числа, а из многочленов — поле рациональных дробей.

Если взять простое число $p$ и рассмотреть множество вычетов по модулю $p$, то у каждого элемента будет обратный (см.~листок 23).
Полученное множество $\mathbb F_p = \Z/p\Z$ будет полем вычетов по модулю простого числа $p$.

Аналогично если взять неприводимый многочлен $P$ и рассмотреть вычеты (остатки) по модулю многочлена~$P$
(точнее, это классы эквивалентности такого отношения: $A\sim B$, если $(A-B) \dv P$),
то тоже окажется, что у каждого вычета есть обратный.
Получится поле $\R[x]/P\R[x]$.
Если в качестве такого неприводимого многочлена взять $P=x^2+1\in\R[x]$, то вычеты образуют поле $\Cbb$.

Если взять корень $\alpha$ неприводимого многочлена $P\in\Q[x]$ и рассмотреть числа вида $a_0 \alpha^m + a_1 \alpha^{m-1}+\ldots+a_{m-1}\alpha + a_m$, где $m < \deg P$, $a_i \in \Q$, то они будут перемножаться и складываться в точности, как остатки от деления на многочлен $P$.
Это поле обозначается через $\Q[\alpha]$.
Например, $\Q[\sqrt{2}] =\{ a+b\sqrt{2}\mid a,b \in \Q\} = \Q[x] / (x^2-2)\Q[x]$.

Аналогичные конструкции можно делать для многочленов с коэффициентами из любого поля.


% Также поле можно получить при помощи следующей конструкции.
% Выберем неприводимый многочлен $f \in \Q[x]$.
% Рассмотрим множество вычетов $\mathbb Q [x] / f \mathbb Q[x]$ по модулю многочлена $f$,
% то есть множество всех остатков $a_0 x^m + a_1 x^{m-1}+\ldots+a_{m-1} x + a_m$, $m < \deg f$ от деления многочленов из $\Q[x]$ на $f$.
% Кстати, эту операцию можно обобщить на любое поле, и в частности,
% комплексные числа $\mathbb C = \{a +b i \mid a,b\in \R\}$ --- это в точности $\R[x]/(x^2+1)\R[x]$.



\newpage

\задача
Пусть $\Bbbk$ --- поле.  Докажите, что\\
%\вСтрочку
\пункт
в $\Bbbk$ есть только один ноль;
\пункт
у каждого элемента только один противоположный;\\
\пункт
для любого $a\in\Bbbk$ выполнено равенство
$-(-a)=a$;\\
\пункт
для любых $a,b\in\Bbbk$ уравнение $a+x=b$ имеет ровно одно решение
в $\Bbbk$ (оно обозначается $b-a$; таким образом, в поле
определена операция \emph{вычитания}).
\кзадача


\задача
Пусть $\Bbbk$ --- поле. Докажите, что\\
%\вСтрочку
\пункт
в $\Bbbk$ есть только одна единица;
\пункт
у каждого ненулевого элемента только один обратный;\\
\пункт
для любого ненулевого $a\in\Bbbk$ выполнено равенство
$(a^{-1})^{-1}=a$;\\
\пункт
для любого $b\in\Bbbk$ и любого ненулевого $a\in\Bbbk$ уравнение
$a\cdot x=b$ имеет ровно одно решение в $\Bbbk$ (оно обозначается
$\dfrac b a$; таким образом, в поле определена операция
\emph{деления} на ненулевые элементы).
\кзадача

\задача
Пусть $\Bbbk$ --- поле. Докажите, что\\
\пункт
для любого $a\in\Bbbk$ выполнено равенство $a\cdot 0=0$;
\пункт
если $a\cdot b=0$, то $a=0$ или $b=0$.\\
\пункт
Останется ли верным утверждение пункта б), если исключить из аксиом поля
аксиому М4?
\кзадача

\задача
Пусть $\Bbbk$ --- поле. Докажите, что для любого $a\in\Bbbk$
выполнены равенства\\
%\вСтрочку
\пункт
$a\cdot (-1)=-a$;
\пункт
$(-a)\cdot (-a)=a\cdot a$;
\пункт
$(-a)^{-1}=-(a^{-1})$, если $a\ne0$.
\кзадача

\задача
Пусть $\Bbbk$ --- поле. Докажите, что
%\сНовойСтроки
%\пункт
для любых $a,c\in\Bbbk$ и любых ненулевых $b,d\in\Bbbk$ выполнено
равенство
\вСтрочку
\пункт
$\dfrac a b\cdot\dfrac c d=\dfrac{a\cdot c}{b\cdot d}$;
\пункт
%для любых $a,c\in\Bbbk$ и любых ненулевых $b,d\in\Bbbk$ выполнено
%равенство
$\dfrac a b+\dfrac c d=\dfrac{a\cdot d+b\cdot c}{b\cdot d}$.
\кзадача

\задача\label{ex:field}
% \пункт
% Какие из следующих множеств являются полями:
% множество
% $\mathbb N$,
% натуральных чисел, множество
% $\mathbb Z$,
% целых чисел, множество
% $\mathbb Q$,
% рациональных чисел?
\пункт
Докажите, что $\Q[\sqrt{2}]$ — поле.
\пункт
При каких $m$ система вычетов $\Z/m\Z$ --- поле?
\кзадача

\задача
Образуют ли поле числа вида
\пункт $a+b\root 3 \of 2+c\root 3 \of 4$, где $a,b,c\in\Q$;
\пункт $a+b\sqrt p+c\sqrt q+d\sqrt{pq}$, где  $a,b,c,d\in\Q$, а $p,q$ --- фиксированные различные простые;
\пункт $\Z[i]/(3+4i)\Z[i]$; \пункт $\Z[i]/(2+i)\Z[i]$?
\кзадача

\задача
Избавьтесь от иррациональности в знаменателе: $\frac1{5+\root 3 \of 2+\root 3 \of 4}$.
\кзадача


\задача
%Является ли полем множество %рациональных
\пункт Пусть $\R(x)=\displaystyle{\left\{\frac{P(x)}{Q(x)}\
\Bigl|\ P(x),Q(x)\in\R[x],\ {\rm где}\ Q(x)\ {\rm
-\ не\ многочлен\ 0}\right\}}$.
Докажите, что $\R(x)$ --- поле (с обычным сложением и умножением).
% является полем.
\пункт Всегда ли для множества, удовлетворяющего всем аксиомам, кроме М4, множество классов эквивалентности его дробей будет полем?
\кзадача


\опр {\it Характеристика поля $\Bbbk$ } (обозначение: $\rm{char}\, \Bbbk$) --- такое наименьшее натуральное число~$m$, что $ \underbrace{1 + \ldots + 1}_m = 0$. Если такого числа нет, характеристика полагается равной $0$.
\копр

\задача
Докажите, что характеристика поля --- простое число или 0.
\кзадача

\задача
Найдите характеристики полей
 \пункт
% из задачи \ref{ex:field}.
$\Q$, $\Z/p\Z$, $\Q[\sqrt{2}]$, $\Cbb$;
\пункт $\Z[i]/(2+i)\Z[i]$.
\кзадача

\раздел{Поля характеристики $p$}

\задача Существует ли бесконечное поле характеристики $p$?
\кзадача

\задача Пусть $\Bbbk$ --- поле характеристики $p$. Докажите, что\\
\пункт элементы $1$, $1+1$, $1+1+1$, $\ldots$, $\underbrace{1 + \ldots + 1}_p$ образуют поле  $\F_p \subset \Bbbk$, <<точно такое же, как>>%
\footnote{Математический термин --- \выд{изоморфное}. Поле $F$ с операциями $+$, $\cdot$ и поле $G$ с операциями $\oplus$, $\odot$ называются {\it изоморфными}, если существует взаимно-однозначное соответствие $\varphi: F\rightarrow G$, сохраняющее обе операции, то есть для любых элементов $a,b\in F$ выполнены равенства $\varphi(a+b) = \varphi(a)\oplus\varphi(b); \varphi(a\cdot b) =\varphi(a)\odot \varphi(b)$.}
$\Z/p\Z$;\\
\пункт
для произвольных элементов $a,b \in \Bbbk$ выполнено равенство $(a+b)^p=a^p+b^p$.
\кзадача

\задача Пусть $\Bbbk$ --- конечное поле из $q$ элементов. \пункт Докажите, что для любого $x$ из $\Bbbk$ верно $x^q=x$. \пункт Для каждого $n$ найдите сумму всех $n$-х степеней элементов из $\Bbbk$.
\кзадача

\задача Пусть $\Bbbk$ --- конечное поле характеристики $p$.
Пусть $x \in \Bbbk \setminus \mathbb F_p$. Докажите, что\\
\пункт $\Bbbk$ содержит все элементы вида $\alpha x + \beta$, где $\alpha,\beta \in \mathbb F_p$;
\пункт в $\Bbbk$ как минимум $p^2$ элементов.\\
\пункт в $\Bbbk$ ровно $p^n$ элементов для некоторого $n$.
\кзадача


\ЛичныйКондуит{0mm}{6mm}

% \GenXMLW

\end{document}

