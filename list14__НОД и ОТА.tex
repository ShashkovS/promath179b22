\documentclass[12pt,a4paper]{article}
\usepackage[mag=1000, tikz]{newlistok}

\УвеличитьШирину{1.3cm}
\УвеличитьВысоту{2.3cm}
\renewcommand{\spacer}{\vspace{1.7pt}}

\ВключитьКолонтитул

\begin{document}

\Заголовок{Целые числа: наибольший общий делитель}
\Оценки{32/27/22}
\НадНомеромЛистка{179 школа, 7Б.}
\НомерЛистка{14}
\ДатаЛистка{10.02 -- 21.02/2018}


\СоздатьЗаголовок

\опр
Отрезки $a$ и $b$ \выд{соизмеримы}, если они имеют
\выд{общую меру}\ --- отрезок $d$, укладывающийся
и в $a$, и в $b$ целое число раз (т.е.~прямоугольник $a\times b$ можно разбить сеткой на равные квадраты).
\копр

\задача
От прямоугольника %размерами
$a\times b$ отрезают квадраты со стороной, равной меньшей стороне
прямоугольника, пока это возможно. %(будем называть это
%\лк операцией Евклида\пк).
С~оставшимся прямоугольником делают тоже самое,
%снова применяют операцию Евклида,
и т.~д.
\сНовойСтроки
\пункт
Сколько и каких квадратов получится, если
$a=324$, $b=141$?
\пункт
Докажите, что если $a$ и $b$ %стороны прямоугольника
соизмеримы, то
%в итоге %конце концов
%мы разрежем %его
прямоугольник разрежут на конечное число квадратов.
\пункт
Докажите, что если в итоге прямоугольник разрежут на
конечное число квадратов, то стороны прямоугольника соизмеримы, и
сторона $d$ %последнего
самого маленького
квадрата является их общей мерой.
\пункт Докажите, что любая общая мера сторон прямоугольника укладывается в $d$ целое число раз.
\пункт
%Пусть числа $a$ и $b$ целые.
%Докажите, что длина стороны последнего квадрата равна $(a,b)$.
Докажите, что $d$ --- \выд{наибольшая}
общая мера сторон прямоугольника.
\кзадача


\пзадача
Найдите наибольшую общую меру отрезков с длинами
$15/28$ и $6/35$.
%$\displaystyle{{15\over28}}$ и $\displaystyle{{6\over35}}$%,
%то есть найдите такое наибольшее число $\alpha$, что числа
%$\displaystyle{{15\over28\alpha}}$ и
%$\displaystyle{{6\over35\alpha}}$ --- целые.
\кзадача

\задача
От прямоугольника отрезали квадрат и получили прямоугольник,
подобный исходному.\\
\вСтрочку
\пункт
Соизмеримы ли его стороны?
\пункт
Найдите отношение сторон исходного прямоугольника.
\кзадача

\пзадача
Докажите, что отрезки $a$ и $b$ соизмеримы если и только если
\пункт есть отрезок $c$, в котором и $a$ и $b$ укладываются
целое число раз;
\пункт $a$ и $a+2b$ соизмеримы;
\пункт $\frac ab$ рационально.
%\пункт $ka$ и $kb$ соизмеримы ($k>0$).
\кзадача



\задача
%Имеются два шаблона: длины $a$ см и длины $b$ см, $(a,b)=d$.
Какие расстояния можно отложить на прямой от данной точки,
имея шаблоны $6$~см и $15$~см?
%\пункт длины $a$~см и $b$~см (где $(a,b)=d$)?
%\пункт $a$~см и $b$~см, где $(a,b)=d$?
\кзадача

\пзадача Синим на числовой оси отметили числа,
дающие при делении на 24 остаток 17, белым~--- дающие
при делении на 40 остаток 7.
Найдите наименьшее расстояние между белой и синей точками.
\кзадача

% \опр
% Если целое число~$d$ делит целые числа~$a$ и~$b$, то $d$~называется \выд{общим делителем} чисел $a$ и~$b$. Наибольший среди общих делителей чисел $a$ и~$b$ называется \выд{наибольшим общим делителем} $a$ и~$b$ (обозначение: $(a,b)$). В~том случае, когда $(a,b)=1$, говорят, что числа $a$ и~$b$ \выд{взаимно простые}.
% \копр

\опр {\it Наибольший общий делитель} $(a,b)$ целых чисел $a$ и $b$
--- это %называется
наибольшее целое число, делящее и $a$ и $b$.
% (предполагается, что $a$ и $b$ не равны одновременно нулю).
Обозначение: $(a,b)$. %или ${\rm НОД}(a,b)$.\\
Если $(a,b)=1$, то $a$ и $b$ называют \выд{взаимно простыми}.
% Число $(a,b)$ существует и единственно,
% если $a$ и $b$ не равны одновременно нулю (докажите!).
\копр

\задача
Докажите, что число $(a,b)$ существует и единственно,
если $a$ и $b$ не равны одновременно 0.
\кзадача

%\задача
%Докажите, что $(a,b)=|b|$ тогда и только тогда, когда $a$ делится на $b$.
%\кзадача

%\задача
%%Докажите: %, что
%Для каких пар целых чисел $a$, $b$ число
%$(a,b)$ существует (и единственно)?
%%для любых целых  $a$ и $b$,
%%кроме случая $a=b=0$. %(в этом случае считаем НОД неопредел\"енным).
%\кзадача


\ввпзадача
Докажите, что $(a,b)=(a-b,b)=(r,b)$, где $r$ --- остаток
от деления $a$ на $b$.
\кзадача

\задача
%Число $n$ целое.
Найдите возможные значения \вСтрочку
\пункт $(n,12)$;
\пункт $(n,n+1)$;
%\пункт $(n,n+6)$;
\пункт $(2n+3,7n+6)$;
\пункт $(n^2,n+1)$.
\кзадача

\ввпзадача
Пусть $a$ и~$b$ положительные целые, $I$ --- множество всех чисел, представимых в виде $ax+by$, где $x$ и~$y$ целые. Пусть $d$ ---
наименьшее положительное число в~$I$. Докажите, что каждое число~из~$I$
\вСтрочку
\пункт
делится на каждый общий делитель $a$ и~$b$;
\пункт
делится на~$d$.
\пункт
Докажите, что $d=(a,b)$.\\
\пункт Докажите, что $I$ --- множество всех целых чисел, делящихся на $(a,b)$.\\
\пункт Докажите, что $(a,b)$ --- наименьшее натуральное число, кратное каждому общему делителю~$a$~и~$b$.
\кзадача

\ввпзадача
Пусть $(a,b)=1$.
\пункт Докажите, что найдутся такие целые $x$ и $y$, что $ax+by=1$.\\
\пункт С помощью равенства из пункта а докажите, что если $c$ целое и $ac$ делится на $b$, то $c$ делится на $b$.
%\вСтрочку
%\пункт $(ac,b)=(c,b)$;
\кзадача

\smallskip
\noindent
{\bf Основная теорема арифметики.} {\em Для каждого целого числа $n > 1$ существует и единственно (с точностью до порядка сомножителей) его представление в виде $n=p_1^{\al_1}\cdot \ldots \cdot p_k^{\al_k}$, где $p_1,\dots,p_k$ --- различные простые. Такое представление называется каноническим разложением $n$ на простые множители.}

\smallskip

\задача
Пусть $a$ и $b$ --- целые числа, $p$ --- простое и $ab\del p$. Докажите, что если $a\!\not\!\del\, p$, то $b\del p$.
\кзадача


\ввпзадача
Докажите основную теорему арифметики.
\кзадача

\vspace*{-4pt}
\раздел{***}

\vspace*{-6pt}
%\задача Докажите, что $(a,b)$ --- общий делитель $a$ и $b$,
%делящийся на любой общий делитель~\hbox{$a$ и $b$.}
%\кзадача

%\задача Пусть $d>0$ делит $a$, $b$ и
%делится на любой общий делитель~\hbox{$a$ и $b$.}
%Докажите, что $d=(a,b)$.
%\кзадача

\задача На плоскости дана фигура, которая
при повороте вокруг точки $O$ на угол $48^\circ$
переходит в себя.
Обязательно ли эта фигура переходит в себя при повороте
вокруг $O$ на угол
\вСтрочку
\пункт $72^\circ$;
\пункт $90^\circ$?
\кзадача


\задача
Пусть $a$ и $b$ натуральные, $(a,b)=d$.
\пункт Какие из чисел $b,\ 2b,\ \ldots,\ ab$ делятся на~$a$?
\пункт По окружности длины $a$ катится колесо длины $b$, в него вбит гвоздь.
Сколько отметок оставит гвоздь?
%В~колесо вбит гвоздь, он оставляет отметки на окружности.
%\вСтрочку
%\сНовойСтроки
%\пункт
%\пункт
%Сколько раз прокатится колесо по окружности, %прежде чем
%пока гвоздь не попад\"ет в уже отмеченную %ранее
%точку?
\кзадача

\задача На клетчатой бумаге нарисован прямоугольник
размерами $a\times b$
клеток (стороны лежат на линиях сетки).
%, в н\"ем проведена диагональ.
На сколько частей делят его диагональ
\вСтрочку
\пункт узлы сетки;
\пункт линии сетки?
\кзадача

\сзадача
Даны $m$ целых чисел. За ход разрешается прибавить
по 1 к любым $n$ из них. При каких $m$ и $n$ всегда можно
за несколько ходов сделать числа одинаковыми, если
\пункт $(m,n)\ne1$;
\пункт $(m,n)=1$?
\кзадача




%\задача Числа $a$, $b$ и $c$ целые, $(a,b)=1$. Докажите, что
%если $ac\del b$,  то $c\del b$.
%%\вСтрочку
%%\пункт $(ac,b)=(c,b)$;
%\кзадача


%\задача
%\вСтрочку
%\пункт Докажите, что уравнение
%$ax+by=c$ имеет решение в целых числах~$x$,~$y$~тогда
%и только тогда, когда %$c$ делится на $(a,b)$.
%$c$ делится на $(a,b)$,
%и в этом случае
%найдется целое решение $x,\ y$, где $0\leq x<b$.
%\пункт Как найти одно из решений (укажите %какой-нибудь
%способ)?
%\пункт Как, зная одно решение,
%найти остальные? %решения?
%\кзадача


%\задача Пусть $a$ и $b$ --- натуральные числа, $(a,b)=d$.
%По окружности длины $a$ см катится колесо длины $b$ см.
%В колесо вбит гвоздь, который, ударяясь об окружность, оставляет
%на ней отметки.\\
%\вСтрочку
%%\сНовойСтроки
%\пункт
%Сколько всего таких отметок оставит гвоздь на окружности?
%\пункт
%Сколько раз прокатится колесо по окружности, %прежде чем
%пока гвоздь не попад\"ет в уже отмеченную %ранее
%точку?
%\кзадача





%\задача Пусть натуральные числа $a$ и $b$ взаимно просты
%и не равны одновременно 1. Докажите, что существуют
%такие целые числа $x$ и $y$, что $|x|<b$, $|y|<a$, и $ax+by=1$.
%\кзадача




\ЛичныйКондуит{0mm}{6mm}

% \GenXMLW

\end{document}

