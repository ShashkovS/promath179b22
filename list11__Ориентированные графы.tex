\documentclass[12pt,a4paper]{article}
\usepackage[mag=1000, tikz]{newlistok}

\УвеличитьШирину{1.5cm}
\УвеличитьВысоту{2cm}
\renewcommand{\spacer}{\vspace{1.7pt}}

\ВключитьКолонтитул

\begin{document}

\Заголовок{Ориентированные графы}
\НадНомеромЛистка{179 школа, 7Б.}
\Оценки{15/12/9}
\НомерЛистка{11}
\ДатаЛистка{13.12 -- 27.12/2017}


\СоздатьЗаголовок

\опр
Граф называется \выд{ориентированным} или \выд{орграфом}, если на каждом ребре указано
направление (тем самым у каждого ребра есть начало и конец). Определения для графа (путь, цикл, изоморфизм, компонента связности, \ldots) переносятся на ориентированные графы с учётом ориентации рёбер. К примеру,
\выд{путь} в орграфе определяется как и в графе, но начало каждого следующего ребра должно быть концом предыдущего
(двигаемся только по стрелкам).
\копр

%Пара вершин может при этом соединяться двумя рёбрами
%разных направлений.

\задача
Житель Врунии утверждает, что там есть несколько озёр,
соединённых между собой реками. Из каждого озера вытекают
3 реки, и в каждое озеро впадает 4 реки. Докажите, что он
ошибается.
\кзадача

\задача
Сколько всего на данных $n$ вершинах есть ориентированных графов, у которых нет петель и каждые две вершины соединены не более чем одним ребром?
\кзадача

\пзадача
Нарисуйте все неизоморфные орграфы из предыдущей задачи для $n=3$.
\кзадача


% \задача
% В некой стране каждый город соединен с каждым дорогой. Сможет ли сумасшедший король ввести на дорогах одностороннее движение так, что, выехав из любого города, в него нельзя будет вернуться?
% \кзадача

\задача
Докажите, что на рёбрах любого связного графа можно так расставить стрелки, что из некоторой вершины можно будет добраться по стрелкам до любой другой.
\кзадача



\пзадача
В связном графе степени всех вершин чётны. Докажите, что на рёбрах этого графа можно расставить стрелки так, что для каждой вершины число входящих рёбер будет равно числу выходящих, и из каждой вершины можно будет добраться до каждой, двигаясь по стрелкам.
\кзадача

\задача
На рёбрах связного графа стоят стрелки так, что у каждой вершины числа входящих и выходящих рёбер равны. Докажите, что от каждой вершины можно добраться (по стрелкам) до каждой.
\кзадача


\пзадача
\вСтрочку
\пункт Можно ли записать по кругу 100 цифр так, чтобы каждая двузначная комбинация от 00 до 99 при движении по часовой стрелке встречалась ровно по разу?
\пункт
Строка из 36 нулей и единиц начинается с 5 нулей.
Среди пятёрок подряд стоящих цифр~встречаются все 32 возможные комбинации.
Найдите 5 последних цифр строки. \пункт Почему такая строка есть?
\спункт Имеются флажки $k$ цветов. Надо сделать гирлянду, в которой каждая комбинация из $n$ подряд идущих цветов встречается ровно по разу. Докажите, что  следующий алгоритм всегда это делает: начинаем с $n$ красных флажков, и добавляем по флажку любого цвета, чтобы комбинации не повторялись, но, если~возможно,~берём~не~красный.
\кзадача

\сзадача
Схема проезда по городу --- связный граф (рёбра --- улицы,
вершины --- перекрёстки).
%%от $A$ до $B$ можно проехать единственным путём --- по ребру $AB$.
Докажите, что можно ввести на всех улицах, кроме мостов, одностороннее
движение (а на мостах --- двустороннее) так, чтобы от любого
перекрёстка можно было доехать по правилам до любого другого. %, не нарушая правил.
\кзадача

\пзадача
В турнире каждая команда сыграла с каждой по разу.
Ничьих не было. Всегда ли можно расположить команды в таком
порядке, чтобы 1-я команда выиграла у 2-й,
2-я --- у 3-й, и т.~д.?
\кзадача

\задача
В некоторой стране каждый город соединен с каждым дорогой с односторонним движением. Докажите, что
\пункт найдется город такой, что от него можно добраться до любого другого города не более, чем с одной пересадкой;
\пункт если городов больше двух, можно поменять направление движения не более чем на одной дороге так, чтобы от любого города можно было доехать до любого другого.
\кзадача

\пзадача
На сайте «Болтовня.ru» зарегистрировалось 2000 человек. Каждый из них пригласил к себе в друзья по 1000 человек. Два человека объявляются друзьями тогда и только тогда, когда каждый из них пригласил другого в друзья. Какое наименьшее количество пар друзей могло образоваться?
\кзадача



\задача
Каждый из 450 депутатов дал пощёчину ровно одному своему
коллеге. Докажите, что из них можно выбрать 150
человек,  среди которых никто никому не давал пощёчины.
% Выбежав после уроков на двор, каждый школьник кинул снежком ровно в одного другого школьника. Докажите, что всех учащихся можно разбить не более чем на три команды так, что члены одной команды друг в друга снежками не кидали.
\кзадача


\задача
\пункт
[Топологическая сортировка] В орграфе нет циклов. Докажите, что его вершины можно так упорядочить, что ни из какой вершины с большим номером не будет пути в вершину с меньшим номером.
\пункт
Процесс изготовления финтифлюшек состоит из множества взаимосвязанных шагов. Про каждый известно, какие шаги должны быть выполнены до него. Докажите, что финтифлюшку можно собрать, не нарушая процесс, если и только если в зависимостях нет циклов.
\кзадача

\сзадача
На рёбрах выпуклого многогранника расставлены стрелки так, что нет вершины, в которую только входят стрелки и нет вершины, из которой только выходят стрелки. Докажите, что найдется грань (а~на самом деле и две), контур которой можно обойти по стрелкам.
\кзадача

%На ребрах выпуклого многогранника расставлены стрелки так, что в каждую вершину хотя бы одна стрелка входит и из каждой вершины хотя бы одна %стрелка выходит. Докажите, что найдутся 2 грани, границу которых можно обойти по стрелкам.


% \задача
% В  поселке 20 жительниц. 1 марта одна из них узнала интересную новость и сообщила ее всем своим подругам. 2 марта те сообщили новость всем своим подругам, и так %далее. Может ли так случиться, что:
% \пункт 15 марта еще не все жительницы будут знать новость, а 18 марта уже все?
% \пункт 25 марта еще не все жительницы будут знать новость, а 28 марта уже все?
% \кзадача


%
% \задача
% Каждая из девочек до завтрака не более двух раз поболтала по телефону. Докажите, что их можно разбить на три группы так, чтобы в каждой группе не было болтавших %между собой девочек.
% \кзадача



% \задача
% Возможна ли компания, где у каждого ровно 5 друзей, а у любых двух --- ровно 2 общих друга?
% \кзадача

%\задача
%Докажите, что из каждого связного графа можно удалить одну вершину и все выходящие из нее ребра так, что останется связный граф.
%\кзадача

% \задача
% Из столицы выходит 101 авиалиния, из
% города Дальний --- одна, а из остальных городов по 100. Докажите,
% что из столицы можно долететь в Дальний (возможно, с пересадками).
% \кзадача
%

% \опр
% Пусть в каждой вершине ографа стоит символ (буква, цифра, \dots). Назовём комбинацию символов \выд{допустимой}, если в графе найдётся путь, который <<читает>> в точности эту комбинацию.
% \копр


\ЛичныйКондуит{0mm}{6mm}

%\GenXMLW

\end{document}



ЗАПАС


\задача
Какие-то две команды набрали в круговом волейбольном турнире одинаковое число очков.
Докажите, что найдутся такие команды А, В и С, что А выиграла у В, В выиграла у С, а С выиграла у А.
\кзадача


\задача
На соревнованиях было 100 судей. Каждый судья упорядочил всех участников (от лучшего по его мнению – к худшему). Оказалось, что ни для каких трёх участников A, B, C не нашлось трёх судей, один из которых считает, что A – лучший из трёх, а B – худший, другой – что B лучший, а C худший, а третий – что C лучший, а A худший. Докажите, что можно составить общий рейтинг участников так, чтобы для каждых двух участников A и B тот, кто выше в рейтинге, был бы лучше другого по мнению хотя бы половины судей.
\кзадача


\задача
В стране 2017 городов, каждый соединён
дорогами не менее, чем с 1008 другими.
Докажите, что из любого
города можно проехать в любой другой напрямую или
через один промежуточный город.
%\пункт найдутся 4 города, соединённые дорогами по циклу.
\кзадача

% \задача
% Группа островов соединена мостами.
%%так, что от каждого острова можно добраться до любого другого.
% Турист обошёл
% все острова, пройдя по каждому мосту %ровно
% один раз. На острове
% Светлом он побывал трижды. Сколько мостов ведёт со
% Светлого, если турист
% \вСтрочку
% \пункт
% не с него начал и не на нём закончил?
% \пункт
% с него начал, но не на нём закончил?
% \пункт
% с него начал и на нём закончил?
% \кзадача




% \задача
% \пункт Можно ли обойти всю шахматную доску конём по циклу?
% \пункт На шахматной доске стоит конь. Двое играют в игру, по очереди делая конём шахматный ход. Проиграет тот, кто поставит коня на клетку, где он уже был. Кто может обеспечить себе победу?
% \кзадача

\сзадача
% Известно, что в графе от любой вершины до любой другой можно добраться, пройдя суммарно не более 100 рёбер, а также можно добраться, пройдя суммарно чётное число рёбер (при подсчёте каждое ребро учитывается проходится несколько раз.
В~стране между некоторыми парами городов осуществляются двусторонние беспосадочные авиарейсы.
Известно, что из~любого города в~любой другой можно долететь, как сделав не~более 100 перелётов, так и сделав чётное число перелетов.
При каком наименьшем натуральном $d$ из~любого города можно гарантированно
долететь в~любой другой, сделав чётное число перелётов, не~превосходящее~$d$?
(Разрешается посещать один и~тот~же город или совершать один и~тот~же перелет
 более одного раза.)
\кзадача




% \задача
% \вСтрочку
% В некой стране $N$ городов, некоторые из которых соединены дорогами.
% Из любого города можно добраться в любой другой
% ровно одним способом (двигаясь по дорогам и нигде не разворачиваясь назад).
% \\
% \пункт Докажите, что в стране есть город, из которого ведёт ровно
% одна дорога.
% \пункт
% Сколько дорог в этой стране?
% \пункт
% Одну дорогу закрыли на ремонт. Можно ли теперь попасть из любого
% города в любой другой?
% \кзадача

ОРГРАФЫ

\опр
Граф называется \выд{ориентированным}, если на каждом ребре указано
направление.
%Пара вершин может при этом соединяться двумя рёбрами
%разных направлений.
\копр

\задача
В турнире каждая команда сыграла с каждой по разу.
Ничьих не было. Всегда ли можно расположить команды в таком
порядке, чтобы 1-я команда выиграла у 2-й,
2-я --- у 3-й, и т.~д.?
\кзадача

\задача
На новом сайте «Разговоры.ru» зарегистрировалось 2000 человек. Каждый из них пригласил к себе в друзья по 1000 человек. Два человека объявляются друзьями тогда и только тогда, когда каждый из них пригласил другого в друзья. Какое наименьшее количество пар друзей могло образоваться?
\кзадача


\задача
\вСтрочку
\пункт
Строка из 36 нулей и единиц начинается с 5 нулей.
Среди пятёрок подряд стоящих цифр~встречаются все 32 возможные комбинации.
Найти 5 последних цифр строки.
\пункт Почему такая строка есть?
\пункт ГИРЛЯНДЫ
\кзадача

\задача
Каждый из 450 депутатов дал пощёчину ровно одному своему
коллеге. Докажите, что из них можно выбрать 150
человек,  среди которых никто никому не давал пощёчины.
\кзадача

\сзадача
Схема проезда по городу представляет собой граф (рёбра --- улицы,
вершины --- перекрёстки). Назовём ребро $AB$ \выд{перешейком}, если
любой путь, соединяющий $A$ и $B$, содержит ребро $AB$.
%от $A$ до $B$ можно проехать единственным путём --- по ребру $AB$.
Докажите, что можно ввести на всех улицах, кроме перешейков, одностороннее
движение (а на перешейках --- двустороннее) так, чтобы от любого
перекрёстка можно было доехать по правилам до любого другого. %, не нарушая правил.
\кзадача

РАМСЕЙ

\задача
На плоскости отметили 17 точек и соединили каждые две из них
цветным отрезком: красным, желтым или зел\"еным.
Докажите, что
%\вСтрочку
%\пункт из каждой отмеченной точки
%выходит не меньше 6 одноцветных отрезков;
%\пункт
найдутся три точки в вершинах одноцветного треугольника.
\кзадача




\задача
У царя Гвидона было три сына (и больше детей не было).
Из его потомков сто имело по два сына, а остальные умерли бездетными.
Сколько потомков было у царя Гвидона?
\кзадача

%\vspace*{-5truemm}
\задача
%\вСтрочку
\пункт
На чаепитие пришли 27 школьников. Каждый принес по 2 пирожных.
Все пирожные раз\-ло\-жи\-ли на 27 тарелок (по 2 на тарелку).
Докажите, что, как бы ни были размещены пирожные,
можно так раздать тарелки школьникам, что каждому
достанется хотя бы одно пирожное, которое он сам принес.
\спункт  А если каждый принёс по 10 пирожных (и их разложили
по 10 штук на тарелку)?
\кзадача

\сзадача [Теорема Холла] В некоторой компании $n$ юношей.
При каждом $k$ от 1 до $n$ верно утверждение:
для любых $k$ юношей в компании число девушек,
знакомых хотя бы с одним из этих $k$ юношей, не меньше $k$.
Можно ли женить всех юношей на знакомых девушках?
\кзадача


\задача
На какое наименьшее число частей надо разделить проволоку длиной 12 см, чтобы
из них можно было сделать каркас куба со стороной 1 см? Части можно %только
изгибать и скреплять друг с другом.
%Можно ли из куска проволоки длиной 120 см
%сделать каркас куба со стороной 10 см, не ломая проволоки?
\кзадача
