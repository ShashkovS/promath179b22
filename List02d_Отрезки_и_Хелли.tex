% !TeX encoding = windows-1251
\documentclass[a4paper, 12pt]{article}
\usepackage{newlistok}
\usepackage[matrix,arrow]{xy}
%\newcommand{\ord}{\operatorname{ord}}
\renewcommand{\spacer}{\vspace*{1pt}}

\ВключитьКолонтитул
\makeatletter
\def\theFullListok{\hbox{\lefteqn{\raise2.3mm\hbox{\sf 179 школа, 7Б.}}\lower2.3mm\hbox{\sf \theListok\the\@listnum}}}
\def\theFullDate{\hbox{\lefteqn{\raise2.3mm\hbox{\sf\theDate\the\@listdate}}\lower2.3mm\hbox{21/16/11 з. на 5/4/3}}}
\makeatother

%\def\hang{\hangindent\parindent}

\УвеличитьШирину{1.3cm}
\УвеличитьВысоту{3cm}


\begin{document}

\Заголовок{Отрезки и теорема Хелли}
\Подзаголовок{}
\НомерЛистка{2д}
\ДатаЛистка{08.11 -- 29.11.2017}

\СоздатьЗаголовок

\vspace*{1mm}

\задача%[Взято из sol-01.09]
Поезд ехал в одном направлении 5,5~ч, и за любой
отрезок времени в 1~ч проехал ровно 100 км. Обязательно ли %тогда
\вСтрочку
\пункт
поезд ехал с постоянной скоростью;
\пункт
его средняя скорость была 100 км/ч?
\кзадача

\задача%[Взято из sol-03.09]
Коридор полностью покрыт несколькими прямоугольными
дорожками, ширина которых равна ширине коридора.
Некоторые  дорожки налегают друг на друга. Докажите, что
можно убрать несколько дорожек так, чтобы
\вСтрочку
\пункт  любой участок пола был покрыт, но не более чем двумя дорожками;
\пункт %можно убрать ещё несколько дорожек так, чтобы
дорожки не налегали друг на друга и покрывали не менее
половины~коридора.
\кзадача

\задача
Биологи 6 часов наблюдали за неравномерно ползущей улиткой так, что она всё это время была под
присмотром. Каждый биолог следил за улиткой ровно 1 час без перерывов
и зафиксировал, что она проползла за этот час ровно 1 м.
Могла ли улитка за время всего эксперимента проползти
\пункт $4$ м; \пункт $10$ м; \ппункт меньше $4$ м; \ппункт больше 10 м.
\кзадача

\раздел{***}

\vspace*{-1mm}
\пзадача
\пункт Каждый из учеников в течение дня один раз посидел в компьютерном классе. Известно, что там каждый встретился с каждым. Докажите: в некий момент все ученики были в компьютерном~классе.
\пункт [Теорема Хелли для прямой] На прямой дано конечное число отрезков. Известно, что любые два отрезка имеют общую точку. Докажите, что тогда и все отрезки имеют общую точку.
\кзадача

\задача На плоскости даны несколько прямоугольников со сторонами, параллельным осям координат. Любые два из них имеют общую точку. Докажите, что тогда и все они имеют общую точку.
\кзадача

\задача
В каждой клетке таблицы $10\times10$ записано целое число. Соседние по стороне числа отличаются не более чем на 1.
Докажите, что среди чисел таблицы найдутся \пункт 6; \спункт 10 одинаковых.
\кзадача

\задача
За день в библиотеке побывало 100 читателей, каждый по разу. Оказалось, что из любых трех по крайней мере двое там встретились. Докажите, что библиотекарь мог сделать важное объявление в такие два момента времени, чтобы в итоге все 100 читателей его услышали.
\кзадача

\задача
На прямой дано конечное число отрезков.
\пункт Пусть среди любых трёх отрезков какие-то два имеют общую точку. Докажите, что эти отрезки можно разбить не более чем на два подмножества так, что в каждом подмножестве все отрезки имеют общую точку.
\пункт Пусть среди любых трёх отрезков какие-то два не имеют общей точки. Докажите, что эти отрезки можно разбить не более чем на два подмножества так, что в каждом подмножестве никакие два отрезка не имеют общей точки.
\кзадача

\пзадача
Обобщите задачу 8 на случай, когда из любых $k$ отрезков какие-то два \пункт имеют общую точку;
\пункт не имеют общей точки.
\пункт На прямой даны $mn+1$ отрезков. Докажите, что есть или $m+1$ отрезков, имеющих общую точку, или $n+1$ отрезков,
никакие два из которых не имеют общей точки.
\кзадача

\пзадача
\пункт На прямой даны $2n+1$ отрезков, каждый пересекает не менее $n$ других. Докажите, что какой-то отрезок пересекает все остальные отрезки.\\
\пункт На прямой даны $2n-1$ синих и $2n-1$ красных отрезков. Каждый синий пересекает не менее $n$ красных и наоборот. Докажите, что какой-то синий отрезок пересекает все красные, и наоборот.
\кзадача

\пзадача
На окружности даны несколько дуг, каждые две имеют общую точку и каждая меньше трети окружности. Докажите, что все дуги имеют общую точку. Верно ли это для дуг большей длины?
\кзадача

\пзадача
На окружности даны несколько дуг, длина каждой меньше длины полуокружности. Докажите, что если каждые три дуги имеют общую точку, то и все дуги имеют общую точку.
\кзадача

\пзадача
На окружности даны несколько дуг, каждые две имеют общую точку. Докажите, что есть такие две диаметрально противоположные точки, что каждая дуга содержит одну из этих точек.
\кзадача

\раздел{***}

\vspace*{-1mm}
\опр
Множество называется <<выпуклым>>, если оно вместе с любыми двумя своими точками содержит и весь отрезок, соединяющий эти точки.
\копр

\пзадача [Теорема Хелли для плоскости] На плоскости дано конечное число выпуклых множеств, любые три из которых имеют общую точку. Докажите, что тогда и все множества имеют общую точку.
\кзадача

\пзадача
На плоскости дано конечное множество точек. Любые три из них можно накрыть кругом радиуса~1. Докажите, что и все точки можно накрыть кругом радиуса 1.
\кзадача

\задача
Дан выпуклый 7-угольник. Рассмотрим все выпуклые 5-угольники с вершинами в  вершинах 7-угольника. Докажите, что эти пятиугольники имеют общую точку.
\кзадача

% \сзадача
% На плоскости лежат несколько котлет (выпуклых многоугольников) и сидят несколько мух (точек). Пусть, какие бы 4 объекта мы не выбрали, можно накрыть скатертью в виде полуплоскости ровно те из них, которые котлеты. Докажите, что можно накрыть скатертью ровно все котлеты.
% \кзадача

% \задача
% Дана прямая $y-kx-b=0$. Докажите, что для точек $(x,y)$, лежащих по одну сторону от этой прямой, выполнено  $y-kx-b>0$, а для точек, лежащих по другую сторону, выполнено $y-kx-b<0$.
% \кзадача
%
% \задача
% Линейное неравенство от двух переменных --- это неравенство вида $ax+by+c\geq0$.
% Докажите, что если среди 100 таких неравенств любые три имеют общее решение, то и все 100 неравенств имеют общее решение.
% \кзадача
%
% \задача
% На плоскости дан вертикальный отрезок. Сопоставим каждой прямой $y=kx+b$, пересекающей этот отрезок, точку $(k,b)$ на другой плоскости. Докажите, что множеству всех прямых, пересекающих этот отрезок, будет сопоставлено выпуклое множество.
% \кзадача
%
% \сзадача
% На плоскости даны несколько параллельных отрезков. Известно, что любые три из них можно пересечь одной прямой. Докажите, что тогда и все 100 отрезков можно пересечь одной прямой.
%%Докажите,
% \кзадача
%
 
\ЛичныйКондуит{0mm}{6mm}

%\GenXMLW

\end{document}

Мухи и котлеты от Франка
