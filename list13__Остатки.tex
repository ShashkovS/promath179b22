\documentclass[12pt,a4paper]{article}
\usepackage[mag=1000]{newlistok}

\УвеличитьШирину{1.7cm}
\УвеличитьВысоту{2cm}
\renewcommand{\spacer}{\vspace{1.7pt}}

\ВключитьКолонтитул

\begin{document}

\Заголовок{Целые числа: остатки}
\НадНомеромЛистка{179 школа, 7Б.}
\Оценки{27/22/17}
\НомерЛистка{13}
\ДатаЛистка{24.01 -- 07.02/2018}


\СоздатьЗаголовок



\опр
Пусть $a$ и $m$ --- целые числа, $m\ne 0$.
\выд{Разделить} $a$ \выд{на} $m$ \выд{с остатком} значит найти
такие целые числа $k$ (\выд{частное}) и $r$ (\выд{остаток}),
что $a = km + r$ и $0\leq r < |m|$.
\копр

\задача
Числа $a$ и $b$ --- целые, $b>0$.
Отметим на числовой прямой все числа, кратные~$b$.
Они разобьют прямую на отрезки длины $b$.
Точка $a$ лежит на одном из них.
Пусть $kb$ --- левый конец этого отрезка.
Докажите, что $k$ --- частное, а
$r = a - kb$ --- остаток от деления $a$ на $b$.
%(и значит, частное и остаток определены однозначно).
\кзадача

\пзадача
Найдите частные и остатки от деления $2018$ на $23$, $-17$ на $4$ и
$n^2-n+1$~на~$n$ при каждом $n$.
\кзадача

\опр %Пусть $m\in\N$. % --- натуральное число.
Говорят, что \выд{$a$ сравнимо с $b$ по модулю $m$}, если
$a-b\del m$. Обозначение: $a\equiv b\!\pmod{m}$.
(Например, $29\equiv -1\!\pmod{6}$, $9N\equiv 2N\!\pmod{7}$ при натуральном $N$, и т.п.)
\копр

\задача
Докажите, что $a\equiv b\!\pmod{m}$ если и только если у $a$ и $b$
одинаковые остатки от деления~на~$m$.
\кзадача

\задача
Могут ли среди $m$ последовательных целых чисел какие-то два иметь равные остатки от деления на $m$?
\кзадача

% {{\small
% \noindent
% {\bf Замечание.}
% Иногда, для удобства, остатком от деления $a$ на $m$ называют целое число, не обязательно лежащее в пределах от $0$ до $m$. %сравнимое с $a$ по модулю $m$.
% Например, бывает удобно сказать, что 29 при делении на 6 дает остаток $-1$. Или что $9N$ дает остаток $2N$ при делении на 7 (число $2N$ может быть и больше 7, но главное, что оно сравнимо с $9N$ по модулю 7).
% }

%}

\пзадача
Пусть $a\equiv b\!\pmod{m}$,  $c\equiv d\ \!\pmod{m}$.
Докажите, что сравнения по одному и тому же~модулю\\ %\qquad
%\сНовойСтроки
\пункт
можно складывать и вычитать: $a+c\equiv b+d\!\pmod{m}$, $a-c\equiv b-d\!\pmod{m}$;\\
\пункт
можно перемножать: $ac\equiv bd\!\pmod{m}$;\\
%(сравнения по одному и тому же модулю можно умножать);\\
\пункт
можно возводить в натуральную степень $n$: $a^n\equiv b^n\!\pmod{m}$;\\
\пункт
можно домножать на любое целое число $k$: $ka\equiv kb\!\pmod{m}$.
\кзадача






\задача
Найдите остаток от деления
\пункт числа $1+31+331+\ldots+3333333331$ на $3$;
\пункт $6^{100}$ на 7.
\кзадача

\задача
Найдите остаток от деления числа  $1-11+111-1111+\ldots-1111111111$ на 9.
\кзадача

\пзадача
Найдите остатки от деления на 3 чисел $2N$, $100N$, $2^N$, $100^N$, $2007^N$ (ответ зависит от $N$).
\кзадача

\задача
Найдите остаток от деления  \пункт 10! на 11; \пункт 11! на 12.
\кзадача


\пзадача
\вСтрочку
\пункт
Какой цифрой оканчивается %число $14^{14}$?
%А число
$8^{18}$?
\пункт
При каких натуральных $k$ число $2^k-1$ кратно~$7$?
\кзадача

\задача
Найдите три последние цифры числа $1999^{2000}$.
\кзадача


\пзадача
Докажите, что
\пункт $30^{99}+61^{100}$ делится на $31$;
\пункт $43^{95}+57^{95}$ делится на $100$.
\кзадача

\задача
Докажите, что $1^n+2^n+\ldots+(n-1)^n$ делится на $n$ при нечётном $n$.
\кзадача


\пзадача
Числа $x$ и $y$ целые, причем $x^2+y^2$ делится на 3.
Докажите, что и $x$ и $y$ делятся на 3.
\кзадача

\сзадача
Докажите, что существует бесконечно много натуральных чисел, не представимых как сумма трёх или менее точных квадратов.
%Подсказка: рассмотреть остатки при делении на 8.
\кзадача

\пзадача
Даны 20 целых чисел, ни одно из которых не делится на 5. Докажите, что
сумма двадцатых степеней этих чисел делится на 5.
\кзадача

\пзадача
Какие целые числа дают при делении на 3 остаток 2,
а при делении на 5 --- остаток 3?
% (и докажите, что других нет).
\кзадача

\пзадача
Докажите, что остаток от деления простого  числа на 30 есть или простое
число или 1.
\кзадача

\сзадача
Сколько есть способов записать 2018 как сумму
натуральных слагаемых, любые два из которых равны или
различаются на 1?
%которые приблизительно равны?
(%Числа называются приблизительно равными,
%если они равны или отличаются на 1.
Способы лишь с разным
порядком слагаемых считаем равными.)
\кзадача

\задача
Докажите, что из любых 52 целых чисел всегда можно выбрать два
таких числа, что\\
\вСтрочку
\пункт
их разность делится на 51;
\пункт
их сумма или разность делится на 100.
\кзадача

\сзадача
Докажите, что из любых $n$ целых чисел всегда можно выбрать несколько,
сумма которых делится на $n$ (или одно число, делящееся на $n$).
\кзадача

\сзадача
\пункт Докажите, что для любого натурального $N$ существует делящееся на $N$ натуральное число,
все цифры которого только 0 и 1.
\пункт Найдётся ли такое число вида $1\ldots10\ldots0$?
%Найдётся ли натуральное число, все цифры которого только 0 и 1,
%делящееся на 2007?
\кзадача



%\сзадача Числа $a_1,\dots,a_n$ целые
%Для каждой пары целых чисел $i$ и $j$, где
%$1\leq j<j\leq n$, возьмем число $(a_i-a_j)/(i-j)$
%и перемножим все такие числа. Докажите, что получится
%целое число.
%%
%%Пусть $A$ --- произведение всевозможных разностей $a_i-a_j$, где
%%$1\leq j<j\leq n$, $B$ --- произведение всевозможных разностей $i-j$, где
%%$1\leq j<j\leq n$. докажите, что $A$ делится на $B$.
%\кзадача




%\опр
%Если разность чисел $a$ и $b$ делится на $m$, то говорят, что {\it $a$ %сравнимо с $b$ по модулю $m$} и пишут А?В (mod М).
%\копр





\сзадача
Шайка из $K$ разбойников отобрала у купца мешок с $N$ монетами. Каждая монета стоит целое число грошей. Оказалось, что какую монету ни отложи, оставшиеся монеты можно поделить между разбойниками так, что каждый получит одинаковую сумму. Докажите, что $N-1$ делится~на~$K$.
\кзадача



\ЛичныйКондуит{0mm}{6mm}

% \GenXMLW

\end{document}

