\documentclass[a4paper,12pt]{article}
\usepackage{newlistok}
\usepackage[matrix,arrow]{xy}
%\documentstyle[11pt, russcorr, listok]{article}
\renewcommand{\spacer}{\vspace{1.8pt}}

%\УвеличитьШирину{1.1truecm}
%\УвеличитьВысоту{2.5truecm}
%\hoffset=-2.5truecm
%\voffset=-27.3truemm
%\documentstyle[11pt, russcorr, ll]{article}

\pagestyle{empty}

\Заголовок{Перестановки: разложения}
\Подзаголовок{}
\НомерЛистка{48}
\ДатаЛистка{20.04.2020 -- 22.04.2020}
\Оценки{12/9/6}

\begin{document}


\СоздатьЗаголовок


\пзадача %\вСтрочку
%Представьте в виде произведения транспозиций (возможно, зависимых) цикл $(1\ 2\  \dots\ k)$.
Упростите (представьте в виде цикла или произведения независимых циклов):\\
\medskip
\пункт
$\begin{pmatrix}1&2&3&4&5\\4&5&2&1&3\end{pmatrix}^{100};$\\
\medskip
\пункт $(1\ k)(1\ k-1)\ldots (1\ 3)(1\ 2)$;\\
\medskip
\пункт $(i+1\ i+2)(i\ i+1)(i+1\ i+2)$;\\
\medskip
\пункт $(1\ 2\ \ldots\ n)^{n-1}$;\\
\medskip
\пункт $(1\ 2\ \ldots\ n)(1\ 2)(1\ 2\ \ldots\ n)^{n-1}$.
\кзадача


\пзадача
Докажите, что любая перестановка из $S_n$ есть произведение\\
\пункт
транспозиций;\\
\пункт
{\em элементарных} транспозиций (то есть транспозиций вида $(i\ i+1)$, где $1\leq i\leq n-1$);\\
\пункт
транспозиций вида $(1\ k)$, где $2\leq k\leq n$.
% \пункт
% Представьте в виде произведения элементарных транспозиций
% перестановки из задачи \ref{perm1}.
\кзадача

%\сзадача
%Докажите, что любую перестановку можно представить как
%произведение двух перестановок, порядок каждой из которых не
%больше двух.
%\кзадача


\пзадача
Пусть $T$ --- некоторое множество транспозиций из $S_n$.
Отметим на плоскости $n$ точек $A_1$, \dots, $A_n$ и соединим
некоторые из них р\"ебрами по правилу: точки $A_i$ и $A_j$
соединяются ребром, если во множестве $T$ есть транспозиция $(i,j)$.
Докажите, что получившийся граф будет связным %в том и только в том случае,
тогда и только тогда, когда любая перестановка из $S_n$
разлагается в произведение транспозиций, %каждая из которых
принадлежащих множеству $T$.
\кзадача

\сзадача
Пусть граф из предыдущей задачи --- дерево на $n$ вершинах. Для каждого ребра возьмём отвечающую ему транспозицию и перемножим их все в некотором порядке. Докажите, что получится цикл длины $n$.
\кзадача




\пзадача Пусть $n\geq2$. Какие перестановки из $S_n$ получаются композициями перестановок, каждая из которых --- % являются композицией \пункт транспозиций $(1\
транспозиция $(1\ 2)$ или цикл $(1\ 2\ \ldots\ n)$?
%; \пункт циклов длины $3$?
\кзадача

% \сзадача Для каких $k$ в $S_n$ существует перестановка, у которой
% ровно $k$ инверсий?
% \кзадача

% \задача
% В каждой клетке таблицы $2\times n$ стоит одно из целых чисел от 1 до $n$,
% прич\"ем в каждой строке стоят разные числа, и
% в каждом столбце стоят разные числа. Сколько таких таблиц?
% \кзадача


% \сзадача
% В таблице $n$ строк и $m$ столбцов.
% \выд{Горизонтальный ход} --- это любая
% перестановка элементов таблицы, при которой
% каждый элемент остается в той же строке, что и до перестановки.
% Аналогично определяется \выд{вертикальный ход}. За какое наименьшее
% число %таких
% горизонтальных и вертикальных
% ходов всегда удастся получить любую перестановку элементов
% таблицы?
% \кзадача
%

% \задача
% Найдите в $S_n$ перестановку с максимальным возможным числом инверсий.
% \кзадача
%
% \сзадача
% Сколько всего в $S_n$ имеется циклов длины $n$ с наименьшим числом инверсий?
% \кзадача
%
%



\пзадача
\пункт Постройте такое соответствие между элементами $S_3$ и движениями плоскости, переводящими равносторонний треугольник в себя, что композиции перестановок  соответствует композиция соответствующих движений.\\
\пункт Аналогично постройте соответствие между элементами $S_4$  и вращениями пространства, переводящими куб в себя.
\кзадача

\пзадача[Задача по геометрии]
Несколько человек хотят обменяться между собой квартирами.
У каждого есть по квартире, но каждый хочет переехать в другую
(разные люди --- в~разные квартиры).
По закону разрешены только парные
обмены: если двое обмениваются квартирами, то в тот же день не участвуют в других обменах.
Всегда ли можно устроить парные обмены так, что уже
через два дня каждый будет жить там, куда хотел переехать?
\кзадача



% \задача
% Рассмотрим всевозможные композиции перестановок $\sigma_1,\sigma_2,\ldots,\sigma_{n-1}$ (в зависимости от порядка). Сколько получится различных перестановок?
% \кзадача



\ЛичныйКондуит{0mm}{5mm}
% \GenXMLW

\end{document}


\опр
Две перестановки $\alpha$ и $\beta$ {\em сопряжены}, если существует такая перестановка $\gamma$, что $\alpha=\gamma\beta\gamma^{-1}$.
\копр

\задача Докажите, что \пункт порядки у любых двух сопряжённых перестановок совпадают;
\пункт две перестановки сопряжены, если и только если наборы длин циклов в их разложении на непересекающиеся циклы совпадают.
\кзадача

\задача
Верно ли, что перестановки с одинаковым числом инверсий сопряжены?
\кзадача

\задача
Найдите хотя бы одну перестановку $x$, такую что $x^2=\alpha$, где $\alpha$ -- перестановка.
\кзадача
