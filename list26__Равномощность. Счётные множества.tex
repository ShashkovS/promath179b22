\documentclass[a4paper,12pt]{article}
\usepackage{newlistok}

\УвеличитьШирину{1truecm}
\УвеличитьВысоту{2.5truecm}

\renewcommand{\spacer}{\vspace*{2pt}}


\begin{document}

\Заголовок{Равномощность. Счётные множества}
\Подзаголовок{}
\НомерЛистка{26}
\ДатаЛистка{07.11 -- 17.11/2018}
\Оценки{27/22/17}

\СоздатьЗаголовок


\опр
Говорят, что между множествами $A$ и $B$ задано
\выд{взаимно
однозначное соответствие}, если каждому элементу множества $A$
поставлен в соответствие какой-то определенный элемент множества $B$,
причём каждый элемент множества $B$ поставлен в соответствие ровно
одному элементу множества $A$. Говорят, что множества $A$ и $B$ %называются
\выд{равномощны (имеют одинаковые мощности)}, если между ними можно установить взаимно
однозначное соответствие.
%Множества $X$ и $Y$ называют \выд{равномощными},
%если их элементы можно разбить на пары $(x,y)$, где $x\in X$
%и $y\in Y$, причём каждый элемент $x\in X$ образует пару
%ровно с одним элементом $y\in Y$, и наоборот,
%каждый элемент $y\in Y$ образует пару
%ровно с одним элементом $x\in X$.
Обозначение: $|A| = |B|$.
%Говорят ещё, что между $X$ и $Y$ можно установить взаимно однозначное
%соответствие.
\копр

\задача Докажите следующие утверждения:\\
%Докажите:
\вСтрочку
\пункт $\!\!\!$ $|A| = |A|$;
\пункт $\!\!\!$ если $|A| = |B|$, то $|B| = |A|$;
\пункт $\!\!\!$ если $|A| = |B|$ и $|B| = |C|$,~то~\hbox{$|A| = |C|$.}
\кзадача

%\задача
%Обозначение $|X|$ уже встречалось для конечных множеств.
%Не будет ли недоразумений?
%%Когда равномощны два конечных множества?
%\кзадача

\задача
Сколько есть взаимно однозначных соответствий между двумя конечными множествами с одинаковым числом элементов?
\кзадача


\задача
Даны четыре множества: $\N$; множество чётных натуральных
чисел; $\Z$; множество натуральных чисел, кроме числа 3.
Докажите, что эти множества равномощны между собой.
%существует ли
%взаимно однозначное отображение из первого во второе.
%\сНовойСтроки
%\пункт множество натуральных чисел;
%\пункт множество чётных натуральных чисел;
%\пункт множество натуральных чисел без числа 3.
\кзадача

\задача Докажите, что равномощны следующие множества точек:\\
%\вСтрочку
\пункт любые два отрезка различной длины;
\пункт любые два интервала различной длины.\\
{\small ({\em Указание:} нарисуйте геометрическую картинку, устанавливающую соответствие.)}
\кзадача


\опр Множество называется \выд{счётным}, если
оно равномощно множеству $\N$.\\
Говорят, что множество $X$ \выд{не более чем счётно,} если $X$ конечно (например, пусто)
или счётно.
\копр

%\задача Докажите, что следующие множества
%счётны:
%\вСтрочку
%\пункт $\Z$;
%\пункт $\{x\in\N\ |\ x\ {\rm делится\ на}\ 9\}$.
%\кзадача

\пзадача
Любое ли счётное множество можно разбить на 3
непересекающихся счётных множества?
\кзадача

\задача
Ровно за минуту до Нового года Дед Мороз кладёт Васе под ёлку одну за другой 10 конфет, за полминуты до Нового года кладёт ещё 10 конфет (тоже по очереди), за четверть минуты --- так же кладёт ещё 10, и так далее до бесконечности. Баба Яга за полминуты до Нового Года съедает конфету, которую Дед Мороз положил первой, за четверть минуты до Нового года съедает конфету, которую Дед Мороз положил второй, и т.д. Сколько конфет будет под ёлкой в Новый год?
\кзадача

\ввзадача
%Даны три ящика $A$, $B$ и $C$.
В ящике $A$  счётное число орехов,
ящики $B$ и $C$  пусты. Берут 10 орехов из ящика~$A$
и перекладывают их в ящик $B$, после чего берут
один орех из ящика $B$ и перекладывают его в ящик~$C$.
Сколько орехов может оказаться в каждом из ящиков после
бесконечного числа таких действий?
\кзадача


\ввпзадача
Докажите, что
\вСтрочку
\пункт
подмножество счётного множества не более чем счётно;\\
\пункт %Докажите, что
если~множества $A$~и~$B$~счёт\-ны, % множества,
то $A \cup B$ тоже
счётно;
%объединение двух счетных множест счетно;
\пункт %Докажите, что
объединение конечного (не пустого) множества счётных множеств счётно.
\пункт %Докажите, что
объединение счётного множества счётных множеств счётно.
%\пункт %Докажите, что
%счётное объединение счётных множеств счётно.
\пункт Верно ли, что счётное объединение конечных множеств всегда
счётно?
\кзадача

\ввпзадача Докажите, что счётно
\вСтрочку
\пункт множество точек плоскости, координаты
которых --- целые числа;
\пункт множество $\Q$;
\пункт множество пар $A\times B=\{(a,b)\ |\ a\in A,\ b\in B\}$,
где $A$ и $B$ счётны.
\кзадача

\сзадача
Найдите алгебраическое выражение от двух переменных $x$ и $y$,
задающее взаимно однозначное соответствие между
%множеством неотрицательных целых чисел и
множеством точек плоскости с натуральными координатами и $\N$.
\кзадача

\задача Докажите, что счётно \quad
%\сНовойСтроки
\вСтрочку
\пункт множество конечных последовательностей из 0 и 1;\\
\пункт множество предложений русского языка;
\пункт множество конечных подмножеств множества $\N$.
\кзадача

\задача Счётно ли \quad
% следующие множества:
%\сНовойСтроки
\пункт множество точек плоскости, обе координаты которых рациональны;\\
\пункт множество всех треугольников на плоскости, координаты
вершин которых рациональны;\\
\пункт множество всех многоугольников на плоскости, координаты
вершин которых рациональны?
\кзадача

\задача
Счётно ли %следующие множества:
любое бесконечное множество непересекающихся\\
%\сНовойСтроки
\вСтрочку
%Счётно ли \ \ \
\пункт %любое бесконечное множество непересекающихся
интервалов длины более $1$ на~прямой;
\пункт %любое бесконечное множество непересекающихся
интервалов на прямой;
\пункт %любое бесконечное множество непересекающихся
кругов на плоскости;\\
\пункт %любое бесконечное множество непересекающихся
восьмёрок на плоскости
(восьмёрка --- это любые две касающиеся внешним образом
окружности);
%\пункт букв \лк{\sf Г}\пк\
%(одного размера) на плоскости;
\спункт %любое бесконечное множество непересекающихся
букв \лк{\sf Т}\пк\
(любых размеров) на плоскости?
%%%(любых размеров) на плоскости?
\кзадача

\задача
Счётно ли множество корней квадратных уравнений с рациональными
коэффициентами?
\кзадача


\сзадача
Может ли множество быть равномощно множеству всех своих подмножеств?
\кзадача


%\ссзадача


\ЛичныйКондуит{0mm}{6mm}
% \GenXMLW


%\СделатьКондуит{6mm}{6.2mm}


\end{document}

