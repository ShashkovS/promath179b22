\documentclass[a4paper, 12pt]{article}
\usepackage{newlistok}

\УвеличитьШирину{1.2truecm}
\УвеличитьВысоту{2truecm}




\begin{document}

\Заголовок{Условно сходящиеся ряды}
\НомерЛистка{70}
\ДатаЛистка{08.10 -- 15.10.2021}
\Оценки{11/8/6}

\СоздатьЗаголовок

\noindent


\задача
\пункт
({\em Теорема Лейбница.})
Пусть $a_n>0$ при $n\in\N$, $\lim\limits_{n\to{+\infty}}a_n=0$ и $a_1\geqslant a_2\geqslant
a_3\geqslant\dots$
Докажите, что знакочередующийся ряд $a_1-a_2+a_3-a_4+a_5-\dots$ сходится.\\
\пункт
Верно ли это, если $(a_n)$ не монотонна?
\кзадача

\опр
Ряд $\sum\limits_{n=1}^{+\infty} a_n$ называется
{\em абсолютно сходящимся}, если сходится ряд
$\sum\limits_{n=1}^{+\infty}|a_n|$.
\копр

\пзадача
Докажите, что \пункт абсолютно сходящийся ряд сходится;
\пункт при любой перестановке слагаемых абсолютно сходящегося ряда получается  ряд с той же суммой, и тоже абсолютно сходящийся.
\кзадача

% \задача
% Пусть ряд $\sum\limits_{n=1}^\infty a_n$ абсолютно
% сходится. Тогда абсолютно сходится произвольный ряд
% $\sum\limits_{n=1}^\infty b_n$, полученный из него
% перестановкой слагаемых, причём
% $\sum\limits_{n=1}^\infty b_n=\sum\limits_{n=1}^\infty a_n$.
% \кзадача

\опр
Ряд $\sum\limits_{n=1}^{+\infty} a_n$ называется
{\em условно сходящимся}, если он сходится,
но ряд $\sum\limits_{n=1}^{+\infty}|a_n|$ расходится.
\копр

\задача
Пусть ряд $\sum\limits_{n=1}^{+\infty} a_n$ сходится условно. Докажите, что
\сНовойСтроки
\пункт
ряд, составленный из его положительных
(или отрицательных) членов, расходится;
\пункт
({\em Теорема Римана})
ряд $\sum\limits_{n=1}^{+\infty} a_n$ можно превратить
перестановкой слагаемых как в расходящийся ряд, так и в сходящийся
с произвольной наперёд заданной суммой;
\пункт
можно так сгруппировать члены ряда
$\sum\limits_{n=1}^{+\infty} a_n$ (не переставляя их),
что ряд станет абсолютно сходящимся.
\спункт
Пусть $\sum\limits_{n=1}^{+\infty} a_n$~--- ряд из комплексных чисел,  $S$ --- множество всех перестановок $\sigma$
натурального ряда, для которых ряд $\sum\limits_{n=1}^{+\infty} a_{\sigma(n)}$
сходится. Каким может быть множество
$\big\{\sum\limits_{n=1}^{+\infty} a_{\sigma(n)}\ |\ \sigma\in S\big\}$?
\кзадача

\сзадача
Исследуйте на абсолютную и условную сходимость:\\
\вСтрочку
\пункт
$\sum\limits_{n=1}^{+\infty} \sin nx$, где $x\in\R$;
\пункт
$\sum\limits_{n=1}^{+\infty} \sin n^2$;
\пункт
$\sum\limits_{n=1}^{+\infty}\frac{\sin n}{n}$.
\кзадача

\пзадача
Пусть $s$ --- сумма ряда
%известно, что
$\sum\limits_{n=1}^\infty \frac{(-1)^{n+1}}{n}$. %=\ln 2$.
Найдите суммы
\\\пункт
$1+\frac13-\frac12+\frac15+\frac17-\frac14+\frac19+\frac1{11}-\frac16+\ldots$\,;
\пункт
$1-\frac12-\frac14+\frac13-\frac16-\frac18+\frac15-\frac1{10}-\frac1{12}+\ldots$\,.
% \пункт
% Переставьте члены ряда $\sum\limits_{n=1}^\infty \frac{(-1)^{n+1}}{n}$
% так, чтобы он стал расходящимся.
% \спункт
% Докажите, что
% $\sum\limits_{n=1}^\infty \frac{(-1)^{n+1}}{n}=\ln 2$.
\кзадача


\задача
Докажите:
\\\пункт
$1+\frac12+\frac13+\frac14+\dots+\frac1n=C+\ln n+\varepsilon_n$,
где $C$~--- константа ({\em постоянная Эйлера}) и $\lim\limits_{n\to\infty}\varepsilon_n=0$;
\\\пункт
({\em тождество Каталана})
$1-\frac12+\frac13-\frac14+\ldots-\frac1{2n}=
\frac1{n+1}+\frac1{n+2}+\ldots+\frac1{2n}$;
\\\пункт
$\displaystyle\sum\limits_{n=1}^\infty \dfrac{(-1)^{n+1}}{n}=\ln 2$.
\кзадача

% \задача
% \пункт
% Существует ли такая последовательность $(a_n)$, $a_n\ne0$ при $n\in\N$,
% что ряды $\sum\limits_{n=1}^{+\infty} a_n$ и
% $\sum\limits_{n=1}^{+\infty} \frac1{n^2a_n}$ сходятся?
% \пункт Можно ли выбрать такую последовательность из
% положительных чисел?
% \кзадача


\сзадача
Найдите сумму ряда: $\displaystyle\sum_{i=1}^{\infty} \frac{(-1)^i}{2i-1} = 1 - \frac{1}{3}+\frac{1}{5}-\frac{1}{7} + \cdots$.
\кзадача


\сзадача
Возможно ли, что
ряд $\sum\limits_{n=1}^{+\infty} a_n$ сходится, а ряд
$\sum\limits_{n=1}^{+\infty} a_n^3$ расходится?
\кзадача



% \сзадача
% Найдётся ли такая перестановка $\sigma$ натурального ряда, что
% для неё существует сходящийся ряд, который она переставляет
% в расходящийся, и существует расходящийся ряд, который она
% переставляет в сходящийся?
% \кзадача
%
\сзадача
Пусть функция $f\colon\R\to\R$ такова, что для любого сходящегося
ряда $\sum\limits_{n=1}^{+\infty} a_n$ ряд $\sum\limits_{n=1}^{+\infty} f(a_n)$ сходится. Докажите, что найдётся
такое $C\in\R$, что $f(x)=Cx$ в некоторой окрестности нуля.
\кзадача

\ЛичныйКондуит{0mm}{6.5mm}
% \GenXMLW

\end{document}


\задача
\пункт
({\em Признак сравнения Вейерштрасса.})
Пусть $\sum\limits_{n=1}^\infty a_n$,
$\sum\limits_{n=1}^\infty b_n$ --- ряды с неотрицательными членами.
Пусть найд\"ется такой номер $k$, что при всех $n>k$, $n\in\N$
будет выполнено неравенство
%$\forall n\ b_n\geqslant a_n\geqslant0$
$b_n\geqslant a_n$.
%(говорят, что
%{\em ряд $\sum\limits_{n=1}^\infty b_n$  мажорирует ряд}
%$\sum\limits_{n=1}^\infty a_n$).
Тогда если $\sum\limits_{n=1}^\infty b_n$ сходится, то
$\sum\limits_{n=1}^\infty a_n$ сходится;
если $\sum\limits_{n=1}^\infty a_n$ расходится, то
$\sum\limits_{n=1}^\infty b_n$ расходится.
%Верно ли это утверждение без предположения неотрицательности
%членов рядов?
\пункт
({\em Признак д'Аламбера.})
Пусть члены ряда $\sum\limits_{n=1}^\infty a_n$
положительны, и
существует %предел
$\lim\limits_{n\to\infty}\frac{a_{n+1}}{a_n}=q$.
Если $q<1$, то
ряд сходится, а если %$\lim\limits_{n\to\infty}\frac{a_{n+1}}{a_n}>1$,
$q>1$, то ряд расходится. Что можно сказать о сходимости
ряда, если $q=1$?
%в случае $\lim\limits_{n\to\infty}\frac{a_{n+1}}{a_n}=1$?
\пункт
({\em Признак Коши.})
Пусть члены ряда $\sum\limits_{n=1}^\infty a_n$ неотрицательны,
и существует %предел
%Если существует предел
$\lim\limits_{n\to\infty}\sqrt[n]{a_n}=q$.
Если $q<1$, то
ряд сходится, а если
%$\lim\limits_{n\to\infty}\sqrt[n]{a_n}>1$,
$q>1$, то ряд расходится. Что можно сказать о сходимости ряда,
%в случае $\lim\limits_{n\to\infty}\sqrt[n]{a_n}=1$?
если $q=1$?
\пункт
Приведите пример сходящегося ряда
с положительными членами, к которому применим признак Коши,
но не применим признак д'Аламбера. Бывает ли наоборот?
\кзадача

\задача
Исследуйте ряды на сходимость:\\
\вСтрочку
\пункт
$\sum\limits_{n=1}^\infty \frac1{n^p}$;
\пункт
$\sum\limits_{n=2}^\infty \frac1{n\ln n}$;
\пункт
$\sum\limits_{n=1}^\infty \frac{1\cdot3\cdot5\cdot\ldots\cdot(2n-1)}
{2\cdot4\cdot6\cdot\ldots\cdot2n}$;
\пункт
$\sum\limits_{n=1}^\infty \frac{n^k}{a^n}$;
\пункт
$\sum\limits_{n=1}^\infty \frac{a^n}{n!}$;
\пункт
$\sum\limits_{n=1}^\infty \frac1{\binom{2n}{n}}$;
%\пункт
%$\sum\limits_{n=1}^\infty \frac{n^{n+\frac1n}}{(n+\frac1n)^n}$;
\пункт
$\sum\limits_{n=1}^\infty (1-\cos\frac{x}n)$.
\кзадача

