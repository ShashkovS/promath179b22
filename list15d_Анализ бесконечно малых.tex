% !TEX encoding = Windows Cyrillic
\documentclass[a4paper,12pt]{article}
\usepackage[tikz,mag=1000]{newlistok}

\УвеличитьШирину{1.4truecm}
\УвеличитьВысоту{2.5truecm}

\Заголовок{Анализ бесконечно малых}
\НомерЛистка{15д}
%\renewcommand{\spacer}{\vfill}
\ДатаЛистка{18.12.2019}
\Оценки{19/16/13}


\begin{document}
\СоздатьЗаголовок

Пусть имеются две величины $x$ и $y$ (скажем, радиус круга и его площадь), связанные функциональной зависимостью: зная одну, можно вычислить другую. Изменим их немного $x\to x+\Delta x$, $y\to y+\Delta y$, сохраняя зависимость. Отношение $\frac{\Delta y}{\Delta x}$ (точнее, его предел, когда изменения бесконечно малы) называется \emph{производной $y$ по $x$} и обозначается $\frac{dy}{dx}$.  Если связь задаётся функцией $f$, то есть $y=f(x)$, то говорят о \emph{производной функции $f$  в точке $x$} и обозначают эту производную $f'(x)$:
\[
f'(x)=\lim_{\Delta x \to 0} \frac{f(x+\Delta x)-f(x)}{\Delta x}
\]

\задача
Пользуясь этим определением, вычислите производные функций
\\
\пункт $f(x)=x^2$;
\пункт $f(x)=(x+3)^2$;
\пункт $f(x)=x^2+3$;
\пункт $f(x)=1/x$;
\пункт $f(x)=\sqrt{x}$.
\кзадача

\задача
Заполните пробелы как в примере:\\
Скорость точки на прямой есть производная её $\langle$координаты$\rangle$ по $\langle$времени$\rangle$.
\невСтрочку
\пункт Ускорение точки на прямой есть производная $\langle\ldots\rangle$ по $\langle\ldots\rangle$.
\пункт Теплоёмкость тела есть производная его внутренней $\langle\ldots\rangle$ по $\langle\ldots\rangle$.
\пункт Ток через конденсатор есть производная его $\langle\ldots\rangle$ по $\langle\ldots\rangle$.
\пункт Длина окружности (границы круга) есть производная $\langle\ldots\rangle$  этого круга по $\langle\ldots\rangle$.
\пункт Площадь поверхности шара есть производная его $\langle\ldots\rangle$ по $\langle\ldots\rangle$.
\пункт Тангенс угла наклона касательной к графику есть производная $\langle\ldots\rangle$ по $\langle\ldots\rangle$.
\кзадача


% \begin{wrapfigure}{r}{0pt}
% \includegraphics{phys1.pdf}
% \end{wrapfigure}
\задача
Ширина прямоугольника равна $2$~м и растёт со скоростью $1$~мм/с. Высота прямоугольника равна $1$~м и растёт со скоростью $3$~мм/с. Чему равны и с какой скоростью растут периметр и площадь этого прямоугольника?
\кзадача


\УстановитьГраницы{0mm}{41mm}
\righttikz{0mm}{3mm}{\begin{tikzpicture}
  \foreach \y in {1.5, 1.4, ..., 0.0} \draw[gray](0,\y)-- (-0.1,\y-0.1);
  \foreach \x in {3.5, 3.4, ..., 0.0} \draw[gray](\x,0)-- (\x-0.1,-0.1);
  \draw [->,very thick] (2,0)--node[below right]{1м/с} (3,0);
  \filldraw[fill=black] (2,0) circle (2pt) node[above right] {$A$};
  \filldraw[fill=black] (0,1) circle (2pt) node[above right] {$B$};
  \draw[line width=2pt] (0,1)--(2,0);
  \draw (0,0)--(3.5,0);
  \draw (0,0)--(0,1.5);
  \draw[<->] (0, -0.3) -- node[below] {2м} (2, -0.3);
  \draw[<->] (-0.3, 0) -- node[left] {1м} (-0.3, 1);
\end{tikzpicture}}
\задача
Палка $AB$ соскальзывает вниз по стене. Чему равна скорость точки~$B$?
\кзадача


\сзадача
Cобаки находятся в вершинах квадрата со стороной $100$~м и начинают бежать со скоростью $1$~м/с, причём каждая всё время бежит в сторону следующей (по часовой стрелке). Когда и где они встретятся?
\кзадача

\ВосстановитьГраницы


\сзадача
Два корабля одновременно отходят по прямого берега. Оба движутся с постоянной по величине скоростью, но первый всё время движется перпендикулярно берегу, а второй~--- в направлении первого корабля. Какое расстояние будет между кораблями на бесконечности, если изначально было расстояние $d$?
\кзадача

\сзадача
Эллипс --- множество точек, для которых сумма расстояний до двух данных точек (фокусов) фиксирована. Докажите, что  любой луч, выходящий из одного фокуса эллипса и отражающийся от него по законам геометрической оптики (угол падения равен углу отражения), придёт в другой фокус. Объясните это же в терминах волновой оптики.
\кзадача

\сзадача
Две черепахи выползли из $A$ и приползли в $B$. Первая черепаха выползла раньше и ползла произвольным образом (с произвольной скоростью и по произвольному пути), а вторая выползла позже и всё время двигалась в направлении первой (с произвольной скоростью). Докажите, что путь ведомой черепахи не может быть длиннее пути ведущей.
\кзадача

\задача
Найдите $h'(x)$, если
\пункт $h(x)=f(x)+g(x)$;
\пункт $h(x)=f(x)\cdot g(x)$;
\пункт $h(x)=f(g(x))$.
\\
Как записать это в традиционных обозначениях? (Cкажем, в п.а) это $d(u+v)=du+dv$.)
\кзадача

\задача
Найдите $h'(x)$, если
\пункт $h(x)=1/g(x)$; \пункт $h(x)=f(x)/g(x)$.
\кзадача

\задача
Точка $X$ движется по единичной окружности с единичной скоростью: в момент $t$ её координаты равны $(\cos t, \sin t)$.
Куда направлена и чему равна (какие координаты имеет) её скорость?
% Чему равны производные функций $\cos(t)$ и $\sin(t)$?
\кзадача



\ЛичныйКондуит{0mm}{6mm}
% \GenXMLW
\end{document}
