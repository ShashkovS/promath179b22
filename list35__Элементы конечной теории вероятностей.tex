\documentclass[a4paper,12pt]{article}
\usepackage[mag=1000]{newlistok}
\usepackage{tikz}
\usetikzlibrary{calc}

\ВключитьКолонтитул

\УвеличитьШирину{1.2cm}
\УвеличитьВысоту{2.4cm}

\Заголовок{Элементы конечной теории вероятностей}
\НомерЛистка{35}
\renewcommand{\spacer}{\vspace{1.3pt}}
\ДатаЛистка{02.09 -- 11.09.2019}
% 54 задач
\Оценки{14/11/8}


\begin{document}


\СоздатьЗаголовок

%\centerline{\large\sf А Н О Н С.\hfil
%Конечная теория вероятностей, избранные задачи.  \hfil 02.2002       }


\medskip
%\раздел{Основные понятия}

{\small\rm
Пусть проводится серия экспериментов,
в результате которых наблюдаются различные
\выд{исходы (элементарные события)},
зависящие от случая.
Пример --- бросание игральной кости, здесь элементарное
событие --- выпадение одного из чисел $1, 2,\dots ,6$.
Или 5 бросаний игральной кости, элементарное событие --- любая последовательность 5 чисел, каждое из которых --- 1, 2, 3, 4, 5 или 6.
Совокупности исходов называют \выд{событиями}.
Пример события --- выпадение ч\"етного числа очков на игральной кости.
Каждому исходу $\omega$ сопоставляют число ${\sf P}(\omega)$ из отрезка
$[0;1]$, называемое \выд{вероятностью} этого исхода.
Сумма вероятностей всех %элементарных событий
исходов должна равняться 1.
Вероятность события $A$ --- сумма вероятностей исходов, составляющих %событие
$A$ (обозначается ${\sf P}(A)$).
%Тот факт, что некоторое событие при многократном повторении
%испытания происходит  примерно с частотой $p$, на языке математической модели
%означает, что вероятность данного события равна $p$.
%Дадим теперь более строгие определения.

В этом листке все исходы равновероятны: например, мы неявно считаем, что качественный игральный кубик не делает различий между числами, а у обычной монетки орёл и решка выпадают одинаково часто. В~реальной жизни всё может оказаться сложнее.

}

\smallskip

%\опр
%\выд{Конечное вероятностное пространство} --- это конечное множество
%$\Omega=\{\omega_1,\ldots,\omega_n\}$.
%Элементы $\Omega$ называются
%\выд{элементарными исходами}, а подмножества $\Omega$
%называются \выд{событиями}.
%При этом каждому исходу $\omega_i$  сопоставлено число $p_i=p(\omega_i)$
%из отрезка $[0;1]$, называемое вероятностью этого исхода.
%Сумма вероятностей всех элементарных исходов должна равняться 1.
%\выд{Вероятностью события} $A\in\Omega$ называется величина
%$P(A)=\sum\limits_{a\in A} p(a)$.
%\копр

%\опр
%\выд{Пространством элементарных событий} $\Omega$ может быть
%произвольное конечное множество.
%Всякое подмножество из $\Omega$ называется \выд{событием},
%множество всех событий обозначается буквой $\cal F$.\\
%\\
%\выд{Вероятностью}, или \выд{вероятностной мерой}
%называется числовая функция $P:{\cal F}\to {\Bbb R}$,
%такая что \\
%$1)$ $P(\emptyset)=0$, $P(\Omega)=1$, $P(A)\ge0$ для любого $A\in{\cal F}$;\\
%$2)$ \выд{(аддитивность вероятностной меры)} если $A\cap B=\emptyset$
%(т.~е.~события $A$ и $B$
%\выд{несовместны}), то $P(A\cup B)=P(A)+P(B)$.\\
%%События, вероятность которых равна 1, называются \выд{достоверными}.
%Множество $\Omega$, множество $\cal F$ и вероятностная мера $P$ вместе
%называются \выд{вероятностным пространством}. Обозначение:
%$(\Omega,\cal F, P)$.
%\копр


%\опр
%\выд{Суммой} событий $A$ и $B$ называется событие
%$A\cup B$, которое происходит тогда и только тогда, когда происходит
%хотя бы одно из событий $A$ или $B$.
%\выд{Произведением} событий $A$ и $B$ называется событие
%$A\cap B$, которое происходит тогда и только тогда, когда происходят
%оба события $A$ и $B$.
%Событием, \выд{противоположным} событию $A$,
%называется событие $\overline{A}=\Omega\setminus A$,
%которое происходит тогда и только тогда, когда не происходит
%событие $A$.
%\копр


\пзадача
Симметричную монету бросили 10 раз. %Считаем, что
%Монета равновероятно падает орлом или решкой.
Какова вероятность того, что
%\сНовойСтроки
\вСтрочку
\пункт
10 %десять
раз выпал ор\"ел;
\пункт
%Какова вероятность того, что
сначала выпало 5 орлов, а затем 5 решек;
\пункт
%Какова вероятность того, что
выпало 5 орлов и 5 решек (в любом порядке)?
\кзадача

\задача
В ящике лежат 4 шара, каждый белый или чёрный.
% Надо %Требуется
% угадать, сколько каких шаров~в~ящике.
За одну попытку вынимают наугад два шара,
смотрят на них и кладут обратно (после чего шары перемешиваются).
Сделали 100 попыток, и в 50-и из них вынули два чёрных шара.
Сколько каких шаров в ящике (скорее всего) и почему?
\кзадача

\задача
Тест состоит из 10-и вопросов, на каждый есть 4
варианта ответа. Двоечник Вася отвечает %на вопросы
\лк наобум\пк.
Какова вероятность того, что он ответит верно \вСтрочку
%\сНовойСтроки
\пункт
на все вопросы;
\пункт
ровно на 5 вопросов;
\пункт
не менее, чем на 5 вопросов?
\пункт В году проводят много таких тестов. Если Васе уда\"ется списать ответ %на вопрос %теста
у отличника Пети, он отвечает на вопрос верно, %в противном
иначе отвечает наугад. В конце года оказалось, что Вася
ответил верно на половину всех вопросов.
Какую долю вопросов (скорее всего) Вася списал?
\кзадача


%\задача
%Игральный кубик бросают 6 раз подряд. %Постройте соответствующее
%%вероятностное пространство (
%%Что такое в данном случае исход?
%Считая все элементарные события равновероятными,
%найдите вероятность выпадения %при четырех бросаниях игрального кубика
%\вСтрочку
%\пункт ровно одной шест\"ерки;
%\пункт хотя бы одной шестерки.
%\кзадача


%\задача
%Найдите вероятность выпадения при четырех бросаниях
%игрального кубика хотя бы одной шестерки.
%\кзадача


\пзадача
В кубке участвуют $2^n$ атлетов разной силы.
%Силы спортсменов постоянны и различны,
Более сильный всегда побеждает более слабого. %Найти вероятность того, что
С~какой вероятностью в финале
встретятся двое сильнейших? Все жеребьёвки равновероятны.
\кзадача



% \задача
% В Китае ввели закон с целью уменьшить
% прирост населения, минимально повлияв на традиции. Если
% в семье первый ребёнок --- мальчик (наследник), семье нельзя
% больше иметь детей, иначе можно
% завести ещё одного ребёнка. Повлияет ли выполнение закона на
% соотношение мужского и женского населения в Китае?
% (При каждом рождении вероятность рождения
% мальчика считаем равной~1/2.)
% \кзадача

\задача
В некой игре ведущий предложил игроку угадать,
за какой из тр\"ех закрытых дверей находится автомобиль.
Игрок наугад выбрал дверь. После этого ведущий
(зная, где автомобиль) открыл
одну из двух других дверей, за которой нет автомобиля, и предложил игроку либо
выбрать другую закрытую дверь,
либо настаивать на первоначальном выборе.
Как лучше поступить~игроку?
\кзадача

% \задача
% Пусть $B$ --- событие, обладающее ненулевой вероятностью. Дайте
% определение условной вероятности $P_B(A)$ события $A$ при условии,
% что событие $B$ произошло.
% (выразите е\"е через $P(A)$, $P(B)$ и $P(A\cap B)$).
% \кзадача


%\задача
%Пусть вероятность рождения мальчика равна $1/2$.
%Какова вероятность того, что в семье два мальчика, если один
%из детей --- мальчик?
%\кзадача

%\задача
%Вероятность попадания в цель при отдельном выстреле равна $0,2$.
%Какова вероятность поразить цель, если в $2\%$  случаев
%выстрел не происходит из-за осечки?
%\кзадача

%\опр
%События $A$ и $B$ называются \выд{независимыми}, если $P(AB)=P(A)\cdot P(B)$.
%\копр

% \задача
% Из колоды в 52 карты выбирается наудачу одна карта. Независимы ли события\\
% \вСтрочку
% \пункт
% \лк выбрать вальта\пк\ и \лк выбрать пику\пк;
% \пункт
% \лк выбрать вальта\пк\ и \лк не выбрать даму\пк?
% \кзадача
%
% \опр
% Событие $A$ называется \выд{независимым} от события $B$,
% если $P(B)=0$, или $P(B) > 0 $ и $P_B(A)=P(A)$.
% \копр
%
% \задача
% Верно ли, что $A$ и $B$ независимы тогда и только тогда, когда
% \пункт $A$ не зависит от $B$; ?
% \кзадача

% \задача
% Пусть вероятность попасть под машину, переходя улицу в неположенном
% месте, равна 0,01. Какова вероятность остаться целым, сто раз
% перейдя улицу в неположенном месте?
%%Как связана эта вероятность с числом $e$ (см.~листок 19)?
% Вычислите е\"е поточнее (см.~задачу 21 листка~15).
% \кзадача

\пзадача
Коля выучил 3 билета из 30. На экзамене все билеты
%(каждый~---~в одном экземпляре)
лежат на столе, студенты
по очереди тянут билеты, вытянутые билеты убирают со стола.
Каким по счёту выгоднее тянуть билет Коле? %, чтобы %иметь наибольшую вероятность
%вытянуть выученный билет?
\кзадача

%\задача
%\вСтрочку
%\пункт
%[Выборка без возвращения]
%В урне $M$ черных и $N$ белых шаров. Наугад выбрано $n$ шаров.
%Какова вероятность вытащить ровно $m$ белых шаров, если
%после взятия из урны шар
%\вСтрочку
%\пункт
%не возвращается назад;
%\пункт
%возвращается назад.
%\кзадача


\задача
Какое наименьшее число учеников должно быть в классе, чтобы вероятность
совпадения дней рождения у двух учеников была больше $1/2$?
(Разрешается посчитать на компьютере.)
\кзадача

%\задача
%\пункт Пусть $A_1,A_2$ --- события, вероятности которых больше 0.
%Докажите \выд{теорему умножения вероятностей}:
%$P(A_1 A_2)=P(A_1)\cdot P_{A_1}(A_2)$.
%\кзадача




%\задача
%Пусть $A$ и $B$ --- независимые события в вероятностном пространстве
%$\Omega$.  Верно ли, что независимы события
%\вСтрочку
%\пункт $A$ и $\Omega\setminus  B$;
%\пункт $\Omega\setminus A$ и $\Omega\setminus  B$?
%\пункт Какие события независимы сами с собой?
%\кзадача

%\задача
%Пару кубиков бросили два раза. Какова вероятность
%\кзадача

%\задача
%Сколько раз нужно бросить игральный кубик, чтобы вероятность
%получить хотя бы одну пят\"ерку стала больше половины?
%\кзадача


%\задача
%Есть 1000 симметричных монет, прич\"ем одна из них фальшивая
%(с двумя орлами). Выбрали случайно одну монету и подбросили е\"е 10
%раз, выпали все орлы. Какова вероятность того, что если эту монету
%подбрость ещ\"е 20 раз, то снова выпадут все орлы?
%\кзадача


% \задача
% Про некий вид бактерий известно, что каждая бактерия через секунду после появления
% на свет делится с вероятностью $p_k$ на $k$ потомков, где $k=1,\ 2,\ \dots,\ n$, и с вероятностью $p_0$ умирает.
% Докажите, что вероятность $x$ того, что весь род, начавшийся с данной
% бактерии,  когда-либо целиком вымрет, удовлетворяет уравнению
% $x=p_0+p_1x+p_2x^2+\dots+p_{n}x^{n}$.
% \кзадача

% \пзадача
% Определите {\em условную вероятность} ${\sf P}(A\,|\,B)$ события $A$ при условии,
% что событие $B$ произошло.
% \кзадача


% \задача
% Пусть вероятность рождения мальчика равна $1/2$.
% Какова вероятность того, что в семье два мальчика, если один
% из детей --- мальчик?
% \кзадача

% \задача
% Вероятность попадания в цель при отдельном выстреле равна $0,2$.
% Какова вероятность поразить цель, если в $2\%$  случаев выстрел не происходит из-за осечки?
% \кзадача

% \опр
% События $A$ и $B$ называются \выд{независимыми}, если
% ${\sf P}(A\,|\,B)={\sf P}(A)$.
% \копр

% \пзадача
% Из колоды в 52 карты выбирается наудачу одна карта. Независимы ли события\\
% \вСтрочку
% \пункт
% \лк выбрать вальта\пк\ и \лк выбрать пику\пк;
% \пункт
% \лк выбрать вальта\пк\ и \лк не выбрать даму\пк?
% \кзадача
%
% \пзадача
% Верно ли, что $A$ и $B$ независимы тогда и только тогда, когда
% \пункт ${\sf P}(B\,|\,A)={\sf P}(B)$;\\
% \пункт ${\sf P}(A\ {\text и} \ B)={\sf P}(A)\cdot{\sf P}(B)$;
% \пункт независимы события $A$ и \лк не $B$\пк;
% \пункт независимы события \лк не $A$\пк\ и~\лк не~$B$\пк.
% \кзадача


%\опр
%Событие $A$ называют \выд{независимым} от события $B$,
%если $P(B)=0$, или (иначе) %$P(B) > 0 $ и
%$P_B(A)=P(A)$.
%\копр
%
%\задача
%Верно ли, что $A$ и $B$ независимы тогда и только тогда, когда
%$A$ не зависит от $B$?
%\кзадача

% \задача
% Из 100 симметричных монет одна фальшивая
% (с двумя орлами). Выбрали случайно монету,~бросили 5
% раз: выпали все орлы. С какой вероятностью,  если е\"е
% бросить ещ\"е 10 раз, снова выпадут все орлы?
% \кзадача

% \задача [О вреде подхалимства]
% \пункт В жюри из трех человек вердикт %окончательное решение
% выносят большинством голосов. Председатель и эксперт принимают
% верное решение независимо с вероятностями $0,7$ и $0,9$,
% а третий для этого бросает монету. С~какой вероятностью жюри принимает верное решение?
% \пункт А если третий будет копировать решение председателя?
% \пункт А если третий будет копировать решение эксперта?
% \кзадача

% \задача
% Отец обещал сыну приз, если сын выиграет подряд хотя бы две теннисные
% партии против него и чемпиона по одной из схем:
% отец-чемпион-отец или чемпион-отец-чемпион.
% Чемпион играет лучше отца. Какую схему выбрать сыну?
% \кзадача

% \задача
% Два охотника одновременно выстрелили одинаковыми пулями в медведя.
% Медведь был убит одной пулей.
% Как %должны
% поделить охотникам шкуру, если вероятность попадания у первого --- 0,3,
% а у второго --- 0,6?
% \кзадача

%\задача
%Каждый из двух игроков пишет на бумажке по целому числу,
%потом они одновременно открывают эти числа.
%Если их сумма делится на 3, то второй
%платит первому рубль; если нет --- второй получает $a$
%рублей от первого. При каком значении $a$ эта игра будет честной?
%\кзадача

%\задача
%Каждый из двух игроков пишет на бумажке число 1 или 2, после чего
%они одновременно открывают бумажки.
%Если числа совпали, то первый платит второму столько рублей, каковы
%эти числа; если нет --- второй платит первому $a$ рублей.
%При каком значении $a$ эта игра будет честной?
%\кзадача

\задача
Даны 10 ч\"ерных и 10 белых шаров. Вы разложите их
по двум урнам, и вам предложат выбрать случайный шар из случайной урны.
Как максимально увеличить шансы вынуть белый~шар?
\кзадача

% \задача
% Юра ежедневно в случайное время между 16 ч и 18 ч едет ужинать к маме или невесте, которые живут по той же линии метро,
% но в разных концах. Юра садится в первый пришедший поезд (в любом направлении).
% Он считает, что его шансы ужинать у мамы или невесты равны,
% но за 20 дней был у мамы лишь дважды. Как это могло быть?
% \кзадача

\пзадача
В ящике 2019 носков --- синих и красных. Может ли синих носков быть столько, чтобы вероятность вытащить наудачу два носка одного цвета была равна 0,5?
\кзадача


\пзадача [Сумасшедшая старушка]
Каждый из $n$ пассажиров купил билет на $n$-местный самолет.
Первой зашла сумасшедшая старушка и села на случайное место.
Далее каждый вновь вошедший занимает своё~место, если оно свободно;
иначе занимает случайное. С какой вероятностью последний пассажир займет своё место?
\кзадача


\сзадача %[Задача о баллотировке] %Предположим, что
На выборах кандидат
$P$ набрал $p$ голосов, а кандидат $Q$ набрал $q$ голосов, %причем
$p>q$. Найдите вероятность того, что при последовательном подсчете голосов
$P$ все время был впереди $Q$.
\кзадача


\сзадача [Задача о разорении]
Игрок, имеющий $n$ монет, играет против казино, которое имеет
неограниченное число монет. За одну игру игрок либо проигрывает монету,
либо выигрывает с вероятностью 0,5. Он играет, пока не разорится. Какова
вероятность разориться ровно за $m$ игр?
\кзадача


\сзадача
Катя отправляла письма $N$ своим знакомым. Написав письма и подписав конверты, она устала и вложила письма в конверты наудачу. Какова вероятность того, что она устроила полную путаницу (то есть никто не получил письма, адресованного именно ему)?
\кзадача

\сзадача
Трое друзей хотят бросить жребий, кому идти в лес за дровами.
Как им это сделать, если у них есть только одна симметричная монета
(которую можно многократно бросать)?
\кзадача


\ЛичныйКондуит{0mm}{7mm}

%\СделатьКондуит{6.5mm}{7.7mm}
% \GenXMLW

\end{document}

\раздел {Дополнительные задачи}

\задача
Три завода выпускают одинаковые изделия. Первый производит 50\%
всей продукции, второй --- 20\%, третий --- 30\%.
Первый завод выпускает 1\% брака, второй --- 8\%, третий --- 3\%.
Выбранное наугад изделие --- бракованное. Какова вероятность
того, что оно %изготовлено на втором заводе?
со второго завода?
\кзадача

\задача
Трое друзей хотят бросить жребий, кому идти в лес за дровами.
Как им это сделать, если у них есть только одна монета
(которую можно многократно бросать)?
\кзадача

\задача
Про вид бактерий известно,
что каждая бактерия через минуту после появления
на свет делится с вероятностью $p_k$ на $k$ потомков, где $k=0,1,\dots,10$.
При этом $p_0$ --- вероятность смерти
бактерии через минуту после рождения.
Докажите, что вероятность $x$ того, что весь род, начавшийся с данной
бактерии,  когда-либо целиком вымрет, удовлетворяет уравнению
$x=p_0+p_1x+p_2x^2+\dots+p_{10}x^{10}$.
\кзадача



%\раздел {Дополнительные задачи}

%\задача
%\кзадача

%\задача
%Двое бросают монету --- один 10 раз, другой --- 11. Какова вероятность того,
%что у второго орлов выпало больше, чем у первого?
%\кзадача

\задача
Каждый из двух игроков пишет на бумажке число 1 или 2, после чего
они одновременно открывают бумажки.
Если числа совпали, то первый платит второму столько рублей, каковы
эти числа; если нет --- второй платит первому $a$ рублей.
При каком значении $a$ эта игра будет честной?
Разберите три случая:
\сНовойСтроки
\пункт
Каждый игрок равновероятно выбирает 1 или 2.
\пункт
Первый выбирает 1 с вероятностью $p$,
%и 2 --- с вероятностью $1-p$,
а второй выбирает 1 с вероятностью $q$ (где $0\leq p,q\leq1$).
%, и 2 --- с вероятностью $1-q$.\\
\спункт
То же, что и пункт б), но перед игрой первый случайным образом
выбирает $p$ (равновероятно из отрезка $[0;1]$),
а второй независимо выбирает $q$ (равновероятно из отрезка $[0;1]$).
\кзадача


\сзадача
Каждый из двух игроков пишет на бумажке по целому числу,
потом они одновременно открывают эти числа.
Если их сумма делится на 3, то второй
платит первому рубль; если нет --- второй получает $a$
рублей от первого. При каком значении $a$ эта игра будет честной
(разберите случаи, как и в задаче 24)?
\кзадача

\сзадача %[Сумасшедшая старушка]
Каждый из $n$ пассажиров купил по билету на $n$-местный самолет.
Первой зашла сумасшедшая старушка и села на случайное место.
Далее, каждый вновь вошедший занимает свое место, если оно свободно;
иначе занимает случайное. Какова вероятность того,
что последний пассажир займет свое место?
\кзадача



\сзадача %[Задача о баллотировке] %Предположим, что
На выборах кандидат
$P$ набрал $p$ голосов, а кандидат $Q$ набрал $q$ голосов, %причем
$p>q$. Найдите вероятность того, что при последовательном подсчете голосов
$P$ все время был впереди $Q$.
\кзадача


\сзадача [Задача о разорении]
Игрок, имеющий $n$ монет, играет против казино, которое имеет
неограниченное число монет. За одну игру игрок либо проигрывает монету,
либо выигрывает с вероятностью 0,5. Он играет, пока не разорится. Какова
вероятность разориться ровно за $m$ игр?
\кзадача


\сзадача
Датчик случайных чисел выдает конечное число чисел, каждое ---
со своей вероятностью. Датчик \выд{сильнее} другого, если
с вероятностью большей $1/2$ выданное им число больше числа, выданного другим
датчиком. Можно ли сделать датчики $A$, $B$ и $C$ так, чтобы $A$
был сильнее $B$, $B$ сильнее $C$, а $C$ сильнее $A$?
\кзадача

\сзадача %[Выбор невесты]
%Царь желает выбрать самую красивую невесту из $100$ претенденток.
%%Процедура выбора  невесты состоит в следующем:
%Претендентки в случайном
%порядке приходят к царю, и в момент прихода очередной претендентки
%царь может объявить
%ее своей невестой (царь заранее не знаком с претендентками, но легко
%упорядочивает их по красоте). Докажите, что царь может выбрать самую
%красивую с вероятностью, большей $1/3$.
Вам в случайном порядке предлагают 100 заранее неизвестных разных
денежных сумм, пока Вы не возьм\"ете предлагаемую сумму.
Как действовать, чтобы взять наибольшую с вероятностью, большей~$1/3$?
\кзадача

%\раздел{Геометрические вероятности}
%
%\noindent
%При решении требуется построить соответствующее бесконечное
%вероятностное пространство.

\сзадача
Палку %случайным образом
случайно ломают на 3 части. С какой вероятностью
из них можно сложить треугольник?
\кзадача

%\задача
%В течение часа к железнодорожной станции в случайные моменты времени
%подходят два поезда. Какова вероятность того, что удастся перебежать
%из поезда в поезд, не ожидая на платформе, если оба стоят по 5 минут?
%\кзадача

\сзадача [Задача Бюффона]
На плоскость, разлинованную параллельными прямыми (на расстоянии
$1$ друг от друга), брошена игла длины $\lambda<1$. Найдите
вероятность пересечения иглы с какой-нибудь прямой.
\кзадача


\сзадача [Парадокс Бертрана]
С какой вероятностью случайная хорда некой данной окружности будет больше
стороны правильного треугольника, вписанного в эту окружность?
\кзадача



\сзадача
Монету радиусом $r$ и толщиной $d$
бросают на горизонтальную поверхность (соударение неупругое).
Какова вероятность того, что монета упадет на ребро? %(Соударение считается неупругим.)
\кзадача

%\сзадача
%Человек, имеющий $n$ ключей, хочет отпереть свою дверь,
%независимо пробуя 1 ключ в минуту в случайном порядке.
%Сколько он в среднем провозится, если неподошедшие ключи
%из дальнейших испытаний
%\вСтрочку
%\пункт исключаются;
%\пункт не исключаются.
%\кзадача

\сзадача
Человек, имеющий $n$ ключей, хочет отпереть свою дверь, испытывая
ключи независимо один от другого в случайном порядке. Найдите
среднее число испытаний, если неподошедшие ключи\\
\вСтрочку
\пункт исключаются из дальнейших испытаний;
\пункт если они не исключаются.
\кзадача

\сзадача
Пачка жевачки содержит один из $n$ разных, но равновероятных вкладышей.
Сколько пачек нужно в среднем купить, чтобы собрать полную коллекцию
вкладышей?
\кзадача


\сзадача
%В расписании движения автобусов на остановке
%\лк Университет\пк\ написано, что
Средний интервал движения автобуса \No 57 равен 35 минут, а средний интервал
движения автобуса \No 661 равен 20 минут. Сколько в среднем нужно
ждать
\вСтрочку
\пункт автобус \No 57;
\пункт один из этих автобусов?
\кзадача

%\СделатьКондуит{4mm}{9mm}

\end{document}

\задача
Студент Иванов ездит из МГУ на 103-ем автобусе, а студент Петров ---
на 130-ом. Садятся они на одной остановке. Иванов утверждает, что 103-и
автобусы полные, а 130-е --- пустые. Петров же утверждает точно
противоположное. Предложите правдоподобное объяснение.
\кзадача 