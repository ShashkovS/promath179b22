\documentclass[a4paper,12pt]{article}
\usepackage[mag=930]{newlistok}

\УвеличитьШирину{1.2truecm}
\УвеличитьВысоту{2.5truecm}
\renewcommand{\spacer}{\vspace{2pt}}

\Заголовок{Фокусы с бесконечностью:\break последовательности, дроби, графы}
\Подзаголовок{}
\НомерЛистка{27}
\ДатаЛистка{17.11 -- 05.12.2018}
\Оценки{14/10/6}

\begin{document}


\СоздатьЗаголовок

\раздел{Последовательности и бесконечные десятичные дроби}


\задача
Число вкладчиков МегаБанка счётно, причем каждый вложил хотя бы 1 р.
Докажите, что деньги можно перераспределить между вкладчиками так,
чтобы у каждого стало не менее 1\,000\,000~р.
\кзадача

\задача
Докажите, что в любой бесконечной десятичной дроби можно так переставить
цифры, что полученная дробь станет периодической
(возможно, с предпериодом).
\кзадача

%5
\пзадача
Докажите, что
из любых одиннадцати бесконечных десятичных дробей можно
выбрать две, совпадающие в бесконечном числе позиций.
\кзадача

%0
%\задача Верно ли, что в натуральном ряду можно выделить\\
%\вСтрочку
%\пункт сколь угодно длинную; \пункт бесконечно длинную цепочку
%подряд идущих составных чисел? \кзадача

%1
%\задача
%Можно ли разместить внутри интервала \вСтрочку \пункт сколь угодно много,
%\пункт бесконечно
%много интервалов равной длины?
%\кзадача

\пзадача Можно ли в последовательности
%$\displaystyle{\frac{1}{1},\frac{1}{2},\frac13,\frac{1}{4},\ldots}$
$1, \ 1/2, \ 1/3, \ 1/4, \ \dots$
выделить\\
\вСтрочку
\пункт бесконечно длинную;
\пункт сколь угодно длинную арифметическую прогрессию?
\кзадача

\задача
Натуральные числа раскрасили в два цвета. Обязательно ли существует одноцветная бесконечная арифметическая прогрессия?
\кзадача

\пзадача
Существует ли такая бесконечная последовательность натуральных чисел,
что любая другая получается из неё вычёркиванием
\вСтрочку
\пункт некоторого конечного числа членов;
\пункт некоторых членов?
\кзадача



\сзадача
Докажите, что любое действительное число можно представить в виде
суммы девяти чисел, десятичные записи которых содержат только цифры 0 и 8.
\кзадача

%\задача
%Докажите, что
%\кзадача

%2

% \задача
% Счётно ли множество всевозможных бесконечных периодических дробей (с предпериодом и без)?
% \кзадача


%11.5
\сзадача Два джинна по очереди выписывают цифры бесконечной
десятичной дроби. Первый своим ходом приписывает в хвост любое
конечное число цифр, второй --- одну. Если в итоге получится
периодическая дробь,  выигрывает первый, иначе --- второй. Кто
выиграет при правильной игре? %наилучшей игре сторон?
\кзадача


\сзадача
Найдётся ли последовательность натуральных чисел, в которой каждое натуральное число встречается ровно по разу и для каждого $k=1,\ 2,\ 3, \dots$ сумма первых $k$ членов последовательности делится~на~$k$?
\кзадача

\сзадача
Найдутся ли такие два бесконечных
подмножества $A$ и $B$ целых неотрицательных чисел,
что каждое целое неотрицательное число %единственным образом
однозначно представляется в виде $a+b$, где $a\in A$,
$b\in B$?
\кзадача


\раздел{Бесконечные графы}

\пзадача
В некой Думе \пункт конечное; \пункт бесконечное число депутатов. Каждый депутат дал пощёчину ровно одному другому депутату, причём каждый депутат получил не более одной пощёчины. Докажите, что Думу можно разбить на три палаты, внутри каждой из которых никто никого не бил.
\кзадача

\задача
Рассмотрим связный граф, в котором бесконечно много вершин, но степень каждой вершины --- какое-то конечное число. Докажите, что в нём счётное число вершин.
\кзадача

\задача
Найдите ошибку в <<доказательстве>> леммы Кёнига (формулировку см.~в задаче \ref{kenig} а).\\ <<Возьмём любую вершину $A$ дерева. Так как дерево бесконечно и связно, можно найти простой путь любой сколь угодно большой конечной длины, идущий из $A$. (Ведь иначе все пути, идущие из $A$, имеют длину не больше какого-то числа $n$, но из $A$ выходит конечное число рёбер, из их концов --- тоже конечное число, и т.д. (так как степени вершин конечны), и не позже чем через $n$ шагов мы получим всё дерево, то есть оно окажется конечным.) Но если в дереве есть простой путь длины 1, длины 2, длины 3, и т.~д., то есть и простой путь бесконечной длины.>>
(Исправив ошибку, вы решите задачу \ref{kenig} а.)
\кзадача

\ввпзадача
\label{kenig}
\пункт [Лемма Кёнига]
Дано бесконечное дерево, в котором степень каждой вершины конечна. Докажите, что в нём есть бесконечный простой путь.
%Известно, что из вершины $A$ этого дерева можно найти простой путь из любого наперёд заданного числа рёбер.
\пункт А если в дереве есть вершина бесконечной степени?
%Обязательно ли верно утверждение предыдущего пункта, если в дереве есть вершина бесконечной степени?
\кзадача




%%8
%\задача Верна ли теорема Холла, если
%юношей счетное число, каждый знаком со счетным числом девушек?
%\кзадача



%7.4
\задача
%В неком языке (с конечным алфавитом)
Дан язык с конечным алфавитом (можете считать, что букв две: 0 и 1).
Любая последовательность букв (конечная или бесконечная) из алфавита этого языка называется {\em словом}.
Часть слов конечной длины в языке --- {\em неприличные}.
Слово называется {\em цензурным}, если в нём нет неприличных подслов.
Пусть существуют сколь угодно длин\-ные цензурные слова.\\
\пункт Докажите, что найдётся бесконечное цензурное слово.\\
\пункт Верно ли, что можно  любое приличное слово продолжить до бесконечного приличного слова? \\
\пункт  Верно ли, что если $w$ и $v$ приличные слова, то существует приличное слово, содержащее и $w$, и $v$?\\
\пункт Пусть неприличных слов конечное число. Докажите, что есть бесконечное периодическое цензурное слово.\\
\спункт Верно ли утверждение предыдущего пункта, если неприличных слов бесконечно много?
\кзадача


%10
\сзадача Каждое конечное слово в неком языке либо
хорошее, либо нехорошее. Докажите, что в любом бесконечном слове
можно откинуть несколько начальных букв так, что оставшееся
бесконечное слово можно будет нарезать либо только на хорошие
слова, либо только на нехорошие.
\кзадача

%6
%\задача
%\вСтрочку
%\пункт Есть три последовательности натуральных чисел:
%$(a_n)$, $(b_n)$ и  $(c_n)$.
%%$a_1,a_2,\ldots,a_n,\ldots$; \break
%%$b_1,b_2,\ldots,b_n,\ldots$; $c_1,c_2,\ldots,c_n,\ldots$.
%Докажите, что~найдутся такие номера $p$ и $q$,
%что $a_p\ge a_q$, $b_p\ge b_q$, $c_p\ge c_q$.
%\пункт
%%Верно ли аналогичное утверждение,
%%не три, а
%А если последовательностей счётное~число? %количество?
%\кзадача





%\задача
%Докажите, что существует такое подмножество $M\subset\N\cup\{0\}$, что
%каждое натуральное число единственным образом представляется в виде
%\пункт $a-b$, где $a,b\in M$;
%\пункт $a+2b$, где $a,b\in M$.
%\кзадача

%9
%\задача В некотором царстве с целью упрощения процесса торговли выпустили неограниченное число
%монет достоинством в $n_1, n_2, \dots, n_k, \dots$ тугриков, где $n_1<n_2<\cdots<n_k<\cdots$.
%Докажите, что в некоторый момент эту процедуру можно оборвать: найдётся такое число $N$, что
%\сНовойСтроки
%\пункт любую сумму, которую можно уплатить со сдачей выпущенными монетами, на самом
%деле можно уплатить со сдачей монетами достоинством в $n_1, n_2,\dots, n_N$ тугриков; \пункт то
%же самое, но суммы уплачиваются без сдачи. \кзадача






\ЛичныйКондуит{0mm}{6mm}



%\СделатьКондуит{6mm}{8mm}
% \GenXMLW

\end{document}

\задача
Игра происходит на плоскости. Играют двое: первый передвигает
одну фишку-волка, второй --- $k$ фишек-овец. После хода
волка ходит одна из овец, затем после следующего хода волка
опять какая-нибудь из овец и т.~д. И волк, и овцы
передвигаются за один ход в любую сторону не
более, чем на метр. Верно ли, что при любой
первоначальной позиции волк поймает хотя бы одну
овцу (окажется с ней в одной точке)?
\кзадача


\задача
Город представляет собой бесконечную клетчатую плоскость (линии --- улицы,
клеточки --- кварталы). На одной из улиц через каждые 100 кварталов на
перекрёстках стоит по милиционеру. Где-то в городе есть бандит
(его местонахождение неизвестно, но перемещается он только по улицам).
Цель милиции --- увидеть бандита.
Есть ли у милиции алгоритм наверняка достигнуть своей цели?
Максимальные скорости милиции и бандита
--- какие-то конечные, но неизвестные нам величины (у бандита
скорость может быть больше, чем у милиции). Милиция видит вдоль
улиц во все стороны на бесконечное расстояние.
\кзадача

\задача
В городе из  задачи 19 трое полицейских
ловят вора (местонахождение вора
неизвестно, но перемещается он только по улицам).
Максимальные скорости у полицейских и вора одинаковы.
Вор считается пойманным, если он оказался на одной улице с полицейским.
Смогут ли полицейские поймать вора?
\кзадача

