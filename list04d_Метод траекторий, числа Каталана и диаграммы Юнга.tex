% !TeX encoding = windows-1251
\documentclass[11pt,a4paper]{article}
\usepackage[mag=1000, tikz]{newlistok}

\УвеличитьШирину{1.5cm}
\УвеличитьВысоту{2.5cm}
\renewcommand{\spacer}{\vspace{1.5pt}}

\ВключитьКолонтитул



\begin{document}


\Заголовок{Метод траекторий, числа Каталана и диаграммы Юнга}
\НомерЛистка{4д}
\ДатаЛистка{03.2018}
\СоздатьЗаголовок




%\раздел{Разбиения чисел}

% \сзадача
% Докажите, что число разбиений\footnote[2]{
% Разбиения, отличающиеся только порядком слагаемых,
% считаются одинаковыми.}
% натурального $n$
% на $k$ натуральных слагаемых равно числу разбиений $n$
% в сумму натуральных слагаемых, наибольшее из которых равно $k$.
% \кзадача
%
%
%
% \ссзадача Какие $n$ столькими же способами
% представимы$^2$ в виде суммы~ч\"ет\-ного числа различных
% натуральных слагаемых,
% сколькими способами они представимы в виде суммы неч\"етного числа
% различных натуральных слагаемых? Что можно сказать об остальных $n$?
% \кзадача
%
% \сзадача
% Докажите, что число разбиений$^{2}$ натурального $n$
% на неч\"етные натуральные слагаемые равно числу разбиений $n$ на попарно
% различные натуральные слагаемые.
% \кзадача
%

%\раздел{Метод траекторий и числа Каталана}


\задача [Метод траекторий]
Будем рассматривать на клетчатой плоскости пути
с началом и концом в узлах клеток,
состоящие из диагоналей клеток, где каждая диагональ
ид\"ет либо вправо вверх, либо вправо вниз (если двигаться по пути
от начала к концу).
%Число диагоналей в пути называется его длиной.
\сНовойСтроки
\УстановитьГраницы{0cm}{5.5cm}
\пункт Сколько существует путей, выходящих из начала координат, в которых
$m$ диагоналей идут вправо вверх, а $n$ диагоналей идут вправо вниз?
\пункт Сколько путей соединяют узел $(0,0)$ с узлом $(x,y)$
(где $x,y\geq0$)?
\rightpicture{-5mm}{15mm}{50mm}{metod_otrazhenij}
\пункт [Принцип отражения]
Узлы $A$ и $B$ лежат не ниже оси абсцисс, $B$ лежит правее $A$.
Докажите, что число путей, идущих из $A$ в $B$, которые касаются
прямой $y=-1$ или пересекают е\"е, равно числу всех путей из $A$
в $B'$, где $B'$ --- узел, симметричный $B$ относительно прямой $y=-1$.
\ВосстановитьГраницы
\кзадача

\задача
\пункт
У кассы стоят $n+m$ человек; $n$ имеют по купюре 100 р, остальные $m$ ---  по купюре
50~р.~В~кассе нет денег, билет стоит 50 р. Сколько есть способов размещения людей
в очереди так, чтобы никто не ждал сдачи?
\пункт А если сначала в кассе было $k$ купюр по 50 р?
\пункт Какой ответ будет в п.~а), если надо, чтобы в любой момент времени, кроме, быть может, начального и конечного,
в кассе была хоть одна купюра в~50~р?
\кзадача

\задача Есть $a+b$ человек разного роста ($a\geq b$). Сколькими способами их можно построить в две шеренги ($a$ человек в первой и $b$ во второй), чтобы в каждой шеренге люди стояли по убыванию роста, шеренги стояли друг напротив друга (самый высокий в первой --- напротив самого высокого во второй, и т.д., пока вторая шеренга не кончится), и каждый человек во второй шеренге был ниже стоящего напротив в первой?
\кзадача

% \задача [Теорема о баллотировке] Кандидат $A$ собрал на выборах $a$ голосов,
% кандидат $B$ собрал $b$ голосов~\hbox{$(a>b)$.}
%%Избиратели голосовали последовательно.
% Сколько существует способов последовательного подсч\"ета голосов, при
% которых $A$ все время будет впереди $B$ по количеству голосов?
% \кзадача

%\txt{Числа Каталана можно определить многоми разными способами.
%}


\задача Игрок с $n$ монетами играет против казино с
бесконечным числом монет. За игру он либо проигрывает монету,
либо выигрывает с вероятностью $1/2$. % и играет, пока не разорится. %Какова доля способов
Какова вероятность его разорения ровно за $m$ игр?
\кзадача

\раздел{Числа Каталана}

\задача %[Числа Каталана]
Докажите, что следующие величины совпадают с числами Каталана (см. листок 5), и найдите их:\\
$\bullet$ Число путей из точки $(0,0)$ в точку $(n,n)$, идущих по линиям
клетчатой бумаги вверх и вправо, не поднимаясь
выше прямой $y=x$;\\ \\ \\ \\ \\
\rightpicture{-35mm}{20mm}{110mm}{katalan-2}
\vspace*{1cm}
% \пункт
$\bullet$ Число способов соединить данные $2n$ точек на окружности $n$ непересекающимися хордами.\\ \\ \\
\rightpicture{-35mm}{20mm}{110mm}{katalan-3}
$\bullet$
Число способов провести $2n$-звенную ломаную из
левого нижнего угла таблички $n\times2n$ в правый нижний угол.
(Ломаная не может выходить за границы таблички, каждое звено
ломаной --- диагональ клетки, идущая вправо вверх или вправо вниз,
если двигаться по ломаной слева направо.)
% Разберите два случая: когда ломаная может касаться нижней стороны
% таблички в точках, отличных от углов, и когда не может.
% \пункт
% На окружности отметили $2n$ точек.
% Сколькими способами их можно соединить $n$ непересекающимися хордами?
\кзадача

\задача
Найдите явную формулу для последовательности $C_n$, заданной
начальным условием $C_0=1$
и рекуррентной формулой $C_n=C_0C_{n-1}+C_1C_{n-2}+\ldots+C_{n-1}C_0$
(при $n\geq1$).
\кзадача

\задача
Ленту длиной $n+1$ см надо разрезать на куски в 1 см. На первом шагу режут в любом месте, на втором --- намечают разрез в каждой части, совмещают намеченные места и режут сразу обе части, на третьем режут <<одним махом>> четыре части и т.д. (получающиеся части в 1 см откладывают). Сколько есть способов так резать ленту? (Два способа разные, если хоть на каком-то шагу результаты разные).
\кзадача

\раздел{Диаграммы Юнга в задачах}

\задача %[Взято из sol-21.06]
Кассир считает деньги так: сначала считает, сколько всего купюр, потом прибавляет число купюр достоинством больше 1 р, затем --- достоинством больше 2 р, и т.~д. Почему у него получается верный ответ?
\кзадача

\задача
Печенья Алёши лежат в нескольких коробках.
Алёша записал, сколько печений в каждой.
Серёжа взял по печенью из каждой коробки и положил на 1-й поднос.
Затем снова взял по печенью из каждой непустой коробки
и положил на  2-й поднос --- и т.~д., пока все
печенья не попали на подносы. Тут Серёжа
записал, сколько печений на каждом подносе.
Докажите, что Алёша и Серёжа записали поровну различных~чисел.
% количество различных чисел среди записанных
% Алёшей равно количеству различных чисел среди записанных Серёжей.
\кзадача

\задача
В таблице $10\times10$ стоят 100 разных чисел. За ход выбирают любой клетчатый прямоугольник~и~переставляют числа в нём симметрично относительно его центра (<<повернули прямоугольник на $180^\circ$>>). Всегда ли за 99 ходов можно добиться, чтобы числа убывали в строках слева направо и в столбцах снизу вверх?
\кзадача


\vspace*{-2mm}
\ЛичныйКондуит{0mm}{6mm}
% \vspace*{-3mm}

% \GenXMLW

\end{document}

\vspace*{-0.2truecm}

\раздел{$***$}

\vspace*{-0.2truecm}

\задача
%\пункт
В НИИ работают 67 человек. Из них
47 знают английский язык, 35 --- немецкий, и 23 --- оба языка.
Сколько человек в НИИ
не знают ни английского, ни немецкого языков?
\пункт Пусть кроме этого  польский
знают 20 человек, английский и польский --- 12, немецкий и
польский --- 11, все три языка --- 5.
Сколько человек не знают ни одного из этих языков?
%\спункт [Формула включений и исключений]
%Решите задачу в общем случае: имеется $m$ языков,
%и для каждого набора языков известно, сколько человек знают все языки
%из этого набора.
\кзадача

\задача
В ряд записали 105 единиц, поставив перед каждой знак \лк$+$\пк.
Сначала изменили знак на противоположный перед каждой третьей единицей,
затем --- перед каждой пятой, а затем --- перед
каждой седьмой. Найдите значение полученного выражения.
\кзадача

\задача \вСтрочку
\пункт
На полке стоят 10 книг. Сколькими способами их можно переставить
так,~чтобы ни одна книга не осталась на месте?
\пункт А если на месте должны остаться ровно 3 книги?
\кзадача
