\documentclass[a4paper, 11pt]{article}
\usepackage{newlistok}
%\documentstyle[11pt, russcorr, listok]{article}
\newcommand{\0}[1]{\overline{#1}}
\def\C{\mbox{$\Bbb C$}}

\УвеличитьШирину{1.2truecm}
\УвеличитьВысоту{3.4truecm}


\newcommand{\RpR}{{\cal R}([a,b])}
\newcommand{\intab}{\int\limits_a^b}

\renewcommand{\spacer}{\vspace{1pt}}

\begin{document}

%\scalebox{1}{\vbox{%
%\ncopy{1}{
\Заголовок{Неопределенный интеграл}
\НомерЛистка{67}
\ДатаЛистка{03.09 -- 17.09.2021}
\Оценки{35/28/21}

\СоздатьЗаголовок

%\раздел{Первообразная}

\опр Пусть функция $f$ определена на интервале (или на любом открытом множестве, или на промежутке). {\em Первообразная} или {\em неопределённый интеграл} функции $f$ --- это такая дифференцируемая функция~$F$, что $F'=f$.
Обозначение: $\int f(x)\, dx$. Замечание: первообразная определена неоднозначно!
\копр


\задача
Пусть $F_1$ и $F_2$ --- первообразные $f$ на интервале $I$. Докажите, что %найдется
%константа $C$, такая, что
$F_1-F_2$ --- константа (на $I$).
\кзадача


\задача
\пункт Пусть функция $f$ непрерывна на некотором интервале. Зафиксируем точку
$a$ из этого интервала. Рассмотрим функцию $F(x)=\int\limits_a^xf(t)\, dt$.
Докажите, что $F$ дифференцируема на этом интервале. Чему равна е\"е производная?
\пункт Докажите, что у каждой функции, непрерывной на интервале, существует первообразная.
\спункт Приведите пример разрывной функции, у которой существует первообразная.
\кзадача

\задача
Пусть на некотором интервале существуют  $\int f(x)\, dx$
и $\int g(x)\, dx$. Докажите, что для любых $\alpha, \beta\in\R$
на этом интервале существует
 $\int (\alpha f(x)+\beta g(x))\, dx$, причём
$\int (\alpha f(x)+\beta g(x))\, dx=\alpha \int f(x)\, dx +\beta \int g(x)\, dx.$
\кзадача

\задача Найдите все первообразные функций (на их области определения):
\вСтрочку
\пункт $f=1$;
\пункт $f=x$;
\пункт $f=x^k,$ $k\in \N$;
\пункт $f=1/x$;
\пункт $f=x^k,$ $k\in \Z$;
\пункт $f=e^x$;
\пункт $f=\sin x $;
\пункт $f=\cos x $.
\кзадача

\взадача [Формула Ньютона-Лейбница]
Пусть $f$ --- непрерывная функция и $F$ --- е\"е первообразная. Докажите,
что $\int\limits_b^af(x)\, dx=F(a)-F(b)$.
\кзадача

\задача
Найдите площадь фигуры, ограниченной осью абцисс и одной дугой синусоиды.
\кзадача

\раздел{Формула замены переменных}

\задача
Пусть $\int f(x)\, dx=F(x)$.
Докажите, что $\int f(ax+b)\, dx =\frac{1}{a}F(ax+b)$.
\кзадача

\взадача
Пусть $\omega (x)$ --- дифференцируемая функция с непрерывной производной.
Пусть $f$ --- непрерывная функция, и $\int f(x)\, dx=F(x)$.
Докажите, что существует $\int f(\omega (x))\omega '(x)\, dx$ и
$\int f(\omega (x))\omega '(x)\, dx=F(\omega (x)).$
\кзадача

\задача
Найдите:
\вСтрочку
\пункт $\int e^{e^x+x}\, dx$;
\пункт $\int xe^{x^2}\, dx$;
\пункт $\int \frac{\ln x}{x}\, dx$;
\пункт $\int \sin x \cos x \, dx$;
%\пункт $\int \frac{\sin x }{\cos^3 x }\, dx$;
\пункт $\int\tg x $;
\пункт $\int\ctg x $.
\кзадача

\задача
\вСтрочку
Пусть $\omega$ отображает $[a,b]$ в $[c,d]$ так, что $\omega(a)=c$, $\omega(b)=d$, причём $\omega$ дифференцируема на
$[a,b]$, а $\omega'(x)$ непрерывна на $[a,b]$.
Докажите, что $\int\limits_c^df(t)\, dt=
\int\limits_a^bf(\omega (x))\omega'(x)\, dx$ для любой $f$, непрерывной
на $[c;d]$.
\кзадача

\задача
Вычислите интегралы
\вСтрочку
\пункт
$\int\limits_0^1\sqrt{1-x^2}\, dx$;
\пункт
$\int\limits_0^{\ln2}\sqrt{e^x-1}\, dx$.
\кзадача


\раздел{Интегрирование по частям.}
\взадача
\пункт
Пусть $u(x)$ и $v(x)$ --- дифференцируемые функции.
Пусть существует интеграл $\int u(x)v'(x)\, dx$.
Докажите, что существует интеграл $\int u'(x)v(x)\, dx$ и
$\int u'(x)v(x)\, dx=u(x)v(x)-\int u(x)v'(x)\, dx.$\\
\пункт
Пусть $u'(x)$ и $v'(x)$ непрерывны на $[a,b]$. Докажите, что
$
\int\limits_a^b u'(x)v(x)\, dx=u(x)v(x)\big|^b_a-
\int\limits_a^b u(x)v'(x)\, dx.
$
\кзадача

\задача
Найдите ($k\in\N$): %интегралы:
\вСтрочку
\пункт $\int \ln x \, dx$;
%\пункт $\int xe^x\, dx$;
\пункт $\int x^ke^x\, dx$; %, k\in \N$;
\пункт $\int e^x\sin x \, dx$;
%\пункт $\int e^x\cos x \, dx$;
\пункт $\int \ln^k x \, dx$; %, k\in \N$.
\пункт $\int\limits_0^\pi x\sin x\, dx$.
\кзадача

\взадача [Формула Тейлора] Пусть $f(x)$ --- функция с непрерывной $n+1$
производной. Докажите, что
$
f(x)=f(x_0)+\sum\limits_{k=1}^n\frac{f^{(k)}(x_0)}{k!}(x-x_0)^k+
\frac{1}{n!}\int\limits_{x_0}^x(x-t)^nf^{(n+1)}(t)\, dt.
$
\кзадача

\раздел{Разные задачи.}

\задача
Приведите пример функции, определённой на интервале и не имеющей на нём первообразной.
\кзадача

\задача
\пункт[Интегральный признак сходимости] Пусть $f:[1,+\infty]\to \R$ неотрицательна, монотонна и непрерывна. Докажите, что ряд $\sum\limits_{n=1}^{+\infty}f(n)$ сходится, если и только если существует $\lim\limits_{x\to+\infty}\int\limits_1^x f(t) dt$.\\ \пункт При каких $s>0$ сходится ряд $\zeta (s)=\sum\limits_{n=1}^{+\infty}\frac1{n^s}$? \пункт Найдите $\lim\limits_{\varepsilon\to0}\int\limits_{\varepsilon}^1 \ln t dt.$
\кзадача


\задача
Пусть $M$ --- максимум $|f'|$
на отрезке $[0;2\pi]$, $n\in\N$. Докажите, что
$
\left|\int\limits_0^{2\pi}f(x)\cos nx \ \! dx\right|\leq2\pi M/n.
$
\кзадача

\задача
\вСтрочку
\пункт
Найдите точную верхнюю грань чисел $\int\limits_0^1 xf(x)\,dx$ по всем
непрерывным неотрицательным на %отрезке
$[0;1]$ функциям
$f$, для которых $\int\limits_0^1 f(x)\,dx\leq2$.
\пункт
Найдите ответ, если не требовать неотрицательность~$f$.
\кзадача





\задача
Пусть $n\in\N$. Разделите отрезок $[-1;1]$ на черные и белые отрезки так, чтобы
суммы определ\"енных интегралов любого многочлена степени $n$ по белым
отрезкам и по чёрным были бы равны друг другу.
\кзадача

\ЛичныйКондуит{.1mm}{5mm}

% \GenXMLW

%}}}

%\СделатьКондуит{4.5mm}{9mm}

\end{document}

\задача
\пункт
\пункт
\кзадача
