% !TeX encoding = windows-1251
\documentclass[12pt,a4paper]{article}
\usepackage[mag=1000]{newlistok}

\УвеличитьШирину{1.7cm}
\УвеличитьВысоту{2.9cm}
\renewcommand{\spacer}{\vspace{2.2pt}}

\ВключитьКолонтитул

\parindent=0mm

\newthm{теория}{}{}{\it}{\it}{}


\begin{document}


\Заголовок{Математическая индукция}
\НадНомеромЛистка{179 школа, 7Б.}
\Оценки{16/12/8}
\НомерЛистка{8}
\ДатаЛистка{18.10 -- 11.11 /2017}
\СоздатьЗаголовок

\noindent
\выд{Математическая индукция} --- это способ доказать
бесконечную серию занумерованных натуральными числами
утверждений за два хода:\qquad
 1) \выд{база индукции:}
доказываем первое утверждение;\\
 2) \выд{шаг индукции:}
доказываем, что при любом натуральном $n$ из $n$-го
утверждения следует $(n+1)$-е.
%\копр

\smallskip


% \задача
% Докажите, что части, на которые $n$ прямых делят плоскость, можно
% раскрасить в два цвета, так чтобы соседние части (имеющие
% общий отрезок или луч) были окрашены в разные цвета.
% \кзадача

\задача
Докажите, что $1\cdot1!+2\cdot2!+\dots+n\cdot n!=(n+1)!-1$
при любом натуральном $n$.
\кзадача



% \задача
% Известно, что
% $a_1=1$ и  $a_{n+1}=2a_n+1$ при $n\geq1$.
% Найдите $a_n$.
% \кзадача


\задача
Докажите, что при любом натуральном $n$
\вСтрочку
\пункт $2^n>n$;
\пункт
$\displaystyle{\frac1{1^2}+\frac1{2^2}+\frac1{3^2}+\dots
+\frac1{n^2}\leq 2-\frac1{n}}$.
\кзадача


\ввзадача
Докажите неравенство Бернулли: $(1+a)^n\geq1+na$, если $a\geq-1$
и $n$ --- натуральное число.
\кзадача

\задача
Докажите: модуль суммы любого числа слагаемых не больше
суммы модулей~этих~слагаемых.
\кзадача


%\задача
%Докажите, что $2^{5n-2}+5^{n-1}\cdot3^{n+1}$ делится на 17
%при любом натуральном $n$.
%\кзадача

\задача
Найдите ошибку:
\лк Докажем, что в любом табуне все лошади одной масти, индукцией по числу лошадей.
Если в табуне одна лошадь, всё очевидно. %е лошади в этом табуне, очевидно, одной масти.
Пусть в любом табуне из $n$ лошадей все лошади одной
масти. Возьмём любой табун из $n+1$ лошади и построим в ряд.
По предположению, первые $n$ лошадей одной масти и последние $n$ тоже, то есть все лошади той же масти, что и <<средняя>> лошадь.\пк
%Поэтому все лошади в табуне одной масти.\пк
%что и требовалось доказать.\пк
\кзадача


\задача Верна ли теорема: \лк Если треугольник разбит отрезками
на треугольники, то хотя бы один из треугольников разбиения не
остроугольный\пк?
Вот е\"е доказательство (нет ли в н\"ем ошибки?):\\
{\лк
1. Если треугольник разбит отрезком на два треугольника,
то один из них не остроугольный (ясно).\\
2. Пусть имеется треугольник, как-то разбитый на $n$ треугольников.
Провед\"ем ещ\"е один отрезок, разбив один из маленьких треугольников
на два. Получим разбиение на $n+1$ треугольник, прич\"ем один из
двух новых треугольников будет не остроугольный.
По индукции теорема доказана.\пк}
\кзадача

\задача
На какое максимальное число частей могут разбить плоскость
\пункт $n$ прямых; \пункт $n$ окружностей?
%\кзадача
%
%\сзадача
\спункт На какое максимальное число частей могут разбить пространство
$n$ плоскостей?
\кзадача

\smallskip

{\footnotesize
\noindent
Есть разные варианты индукции. Иногда в качестве шага
приходится проверять, что $n$-е
утверждение верно~\hbox{если}~верны {\em все} предыдущие. Другой
вариант: предположим, что не все утверждения верны. Тогда
есть {\em наименьшее} на\-ту\-ра\-ль\-ное $n$, для которого $n$-е
утверждение неверно. Если из этого выводится противоречие,
то все утверждения верны.

}

\smallskip

\ввзадача
Докажите, что уравнение $n^2=2m^2$ не имеет решений в
натуральных числах.
%(т.~е.~$\sqrt 2$ иррационально).
\кзадача

\ввзадача
Докажите, что любое натуральное число можно представить как сумму нескольких
разных степеней двойки (возможно, включая и нулевую; сумма может состоять и из одного слагаемого).
\кзадача

\задача
Число $\displaystyle x+\frac1x$ --- целое.
Докажите, что
$\displaystyle x^n+\frac1{x^n}$ --- тоже целое при любом
натуральном $n$.
\кзадача


\задача[Ханойские башни]
Есть детская пирамида с $n$ кольцами и два пустых стержня
той~же~высоты.
Разрешается перекладывать верхнее кольцо с одного стержня на
другой, но нельзя класть~большее кольцо на меньшее.
Докажите, что
\вСтрочку
\пункт можно переложить все кольца на один из пустых стержней;
\пункт можно сделать это за $2^n-1$ перекладываний;
\пункт меньшим числом перекладываний не обойтись.
\кзадача

\сзадача
$N$ воров делят золотой песок. Каждый умеет делить на равные с его точки зрения части, но другие ему не верят.  Как действовать ворам,
чтобы каждый получил не менее $1\over N$ с его точки зрения?
% Разберите случаи:
% \вСтрочку
% \пункт
% $k=2$;
% \пункт
% $k=3$;
% \пункт
% $k$~--- любое.
\кзадача

\сзадача
При каких $n$ гири весом 1, 2, \dots, $n$ кг
можно разложить на три равные по весу кучи?
\кзадача

% \сзадача
% Докажите, что при $n>3$ можно представить~$n!$ в виде $ab$, где $\frac23 \leq\frac ab \leq \frac32$.
% \кзадача

\сзадача
На круговой трассе стоят машины.
Суммарно у них хватает бензина проехать один круг.
Докажите, что одна из машин сможет объехать трассу,
забирая по дороге бензин у других машин.
\кзадача

\сзадача На краю пустыни имеется неограниченный запас бензина и канистр, а также машина, которая при полной заправке может проехать 50 км. В канистры можно сливать бензин из бензобака машины и оставлять на хранение (в любой точке пустыни). Докажите, что машина может проехать любое расстояние. (Канистры с бензином возить нельзя, пустые можно возить в любом количестве.)
\кзадача

\сзадача Бизнесмен заключил с чёртом такую сделку: он может любую имеющуюся у него купюру обменять у чёрта на любой набор купюр любого меньшего достоинства (по своему выбору, без ограничения общей суммы). Он может также тратить деньги, но не может получать их в другом месте (кроме как у чёрта). При этом каждый день на еду ему нужен рубль. Сможет ли он так жить бесконечно долго?
\кзадача


\сзадача
Двое играют в игру, исход которой не зависит от случая. %Игроки
Ходят по
очереди, по правилам игра длится не более $n$ ходов. Ничьих
нет. Докажите, что у кого-то есть выигрышная стратегия.
\кзадача


\ЛичныйКондуит{0mm}{6mm}
%\GenXMLW
\end{document} 