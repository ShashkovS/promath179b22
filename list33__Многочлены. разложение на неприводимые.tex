\documentclass[a4paper,11pt]{article}
\usepackage[mag=1000]{newlistok}
\usepackage{tikz}
\usetikzlibrary{calc}

\УвеличитьШирину{1.5truecm}
\УвеличитьВысоту{2.4truecm}

\Заголовок{Многочлены: разложение на неприводимые}
\НомерЛистка{33}
\renewcommand{\spacer}{\vspace{1.2pt}}
\ДатаЛистка{03.04 -- 17.04.2019}
\Оценки{30/24/18}


\begin{document}


\СоздатьЗаголовок

\smallskip

\noindent
{\bfseries Обозначение.} Множества всех многочленов с коэффициентами из~$\R$,
$\Q$ %$\Z$
обозначают соответственно~$\R[x]$,~$\Q[x]$. %, $\Z[x]$.

\smallskip

%\раздел{Деление с остатком}

\опр\label{del}  Пусть $A$ и $B$ --- многочлены, причем $\deg B>0.$
Разделить $A$ на $B$ с остатком значит найти такие многочлены
$Q$ (частное) и $R$ (остаток), что $A=BQ+R,$ где либо $R=0$, либо $\deg R<\deg B$.
\копр

\пзадача  Разделите с остатком $2x^4-3x^3+4x^2-5x+6$ на $x^2-3x+1$.
\кзадача

\задача
%\пункт
Докажите, что  деление с остатком всегда возможно, и частное и остаток
% .\\
% \пункт  Докажите, что  при делении с остатком многочлены $Q$ и $R$
определены однозначно.
\кзадача

\ввпзадача[Теорема Безу] Докажите, что остаток от деления $A(x)$
на двучлен $x-s$ равен числу $A(s)$.
\кзадача


\пзадача
\вСтрочку
\пункт
Остаток от деления $A(x)$ на $x-1$ равен 5, а на $x-3$ равен 7.
Найдите остаток от деления $A(x)$ на $(x-1)(x-3).$
\пункт
Найдите остаток от деления $x^{1000}$ на $x^2+x-1$.
\кзадача

\опр Многочлен со старшим коэффициентом 1 называется \выд{приведённым}.
\копр

\опр \выд{Наибольшим общим делителем}
многочленов $A$ и $B$ из $\R(x)$ (из $\Q[x]$),
хотя бы один~из~которых ненулевой, назовём (и обозначим $\text{НОД}(A,B)$ или $(A,B)$) приведённый многочлен из $\R(x)$ (из $\Q[x]$), который
\\
1) делит и $A$, и $B$ (то есть, является общим делителем $A$ и $B$);
2) делится на любой общий делитель $A$ и $B$.
Многочлены называются взаимно простыми, если их НОД равен 1.
\копр


\noindent
{\small {\bf Замечание.} Из определения 3 не вполне ясно, почему НОД вообще существует. Задачи 6 и 7 проясняют этот вопрос.}

\задача Однозначно ли определяется {НОД} двух многочленов?
\кзадача

\задача
\пункт
Пусть $A$, $B$, $C$ --- многочлены. Докажите, что множество общих делителей у пары $A$ и $B$
такое же, как и у пары $A$ и $B-A\cdot C$, и если НОД есть у одной пары, то он есть и у другой, и эти НОДы равны.
%причём ${НОД}(A,B)={НОД}(A,B-A\cdot C)$.\\
%Верно ли, что ${\rm НОД}(A,B)={\rm НОД}(A,B-A\cdot C)$,где $C$ --- любой многочлен?\\
\пункт
Сформулируйте и докажите алгоритм Евклида вычисления {НОД} многочленов.
\кзадача

\ввзадача Докажите, что ${\rm НОД}(A,B)$ --- приведенный многочлен
наибольшей степени, делящий и $A$, и $B$.
\кзадача

\пзадача Найдите {НОД} многочленов: \вСтрочку
\пункт $x(x-1)^3(x+2)$ \ и \  $(x-1)^2(x+2)^2(x+5);$\\
\пункт $3x^3-2x^2+x+2$ \  и \  $x^2-x+1;$ \
\спункт $x^m-1$ \ и \ $x^n-1;$ \
\спункт $x^m+1$ \ и \ $x^n+1.$
\кзадача


\ввпзадача
Пусть $A$ и $B$ --- любые многочлены степени $m$ и $n$ соответственно. Докажите, что
\сНовойСтроки
\пункт
существуют такие многочлены $U$ и $V,$ что {НОД}$(A,B)=AU+BV$;
\пункт  если $m,\ n>0,$
то $U$ и $V$ можно выбрать так, чтобы  $\deg U<n$ (или $U=0$)
и $\deg V<m$ (или $V=0$).
\пункт Найдите такие $U$ и $V$, если $A$ и $B$ --- многочлены
из пункта б) предыдущей задачи.
\кзадача


\опр Многочлен положительной степени из $\R[x]$ (из $\Q[x]$) называется
\выд{неприводимым над $\R$ (над~$\Q$)},\/
если он не представляется в виде произведения двух многочленов
меньшей степени из~$\R[x]$ (из $\Q[x]$).
% Аналогично определяется понятие неприводимого над $\Q$ многочлена
% положительной степени из $\Q[x]$.
\копр

%\задача
%Докажите, что %над любым полем
%существует бесконечно много неприводимых над $\R$ многочленов.
%\кзадача

\задача Докажите, что многочлен степени 2 или 3, не имеющий корней в $\Q$, неприводим над $\Q$.
\кзадача

\задача
Может ли неприводимый над $\Q$ многочлен из $\Q[x]$
не быть неприводимым над $\R$?
\кзадача


%\задача Какие из указанных многочленов неприводимы над  $\R$:\\
%\вСтрочку
\пзадача Разложите на неприводимые множители над  $\R$
и на неприводимые множители над $\Q$:\\
\вСтрочку
\пункт $5x+7;$
\пункт $x^2-2;$
\пункт $x^3+x^2+x+1;$
\пункт $x^3-6x^2+11x-6;$
\пункт $x^3+3$;
\пункт $x^4+4.$
\кзадача

%\задача Разложите на неприводимые множители над  $\R$:\\
%\вСтрочку
%\пункт $x^3-6x^2+11x-6;$
%\пункт $x^4+4.$
%\кзадача

%\опр  Многочлены $A_1(x),\dots,A_k(x)$ из $\R[x]$
%называются \выд{взаимно простыми},\/ если
%не существует такого  многочлена $C(x)\in\R[x]$ ненулевой степени,
%на который
%делятся все многочлены $A_1(x),\dots,A_k(x)$.
%\копр
%


% \задача Пусть $A$ и $B$ из $\R[x]$ %, $\rm{НОД}(A,B)=1$.
% взаимно просты.
% Докажите, что найдутся такие %многочлены
% $U$  и  $V$ из $\R[x]$,  что $AU+BV=1$.
% \кзадача


\ввпзадача Докажите, что неприводимый над $\R$ многочлен из $\R[x]$, делящий произведение двух
многочленов  из $\R[x]$, делит один из этих многочленов. {\small ({\em Указание:}\/ используйте задачу 9а) для взаимно простых многочленов.)}
\кзадача

\ввзадача \пункт Докажите, что любой многочлен из $\R[x]$ ненулевой степени
однозначно (с точностью до множителей из~$\R$)
раскладывается в произведение неприводимых над  $\R$ многочленов
из $\R[x]$.\\ \пункт Верно ли аналогичное утверждение для многочленов из $\Q[x]$?
\кзадача

% \задача Верно ли, что любой многочлен из $\Q[x]$
% ненулевой степени  однозначно
% (с точностью до множителей из~$\Q$)
% раскладывается в произведение неприводимых над  $\Q$ многочленов
% из $\Q[x]$?
% \кзадача

\задача
\пункт %Докажите, что у двух взаимно простых многочленов нет общих корней. \пункт Верно ли обратное?
Два многочлена из $\Q[x]$ взаимно просты над $\Q$. Докажите, что у них нет общих действительных корней. 
\пункт Обязательно ли два многочлена из $\Q[x]$ без общих действительных корней взаимно просты над $\Q$?
\кзадача

\задача
Пусть $a\in\R$ --- общий корень многочленов $S$ и $T$ из $\Q[x]$,
%с рациональными коэффициентами,
причём $T$ неприводим над~$\Q$. Докажите,
что $S$ делится на $T$, и частное --- многочлен с рациональными
коэффициентами.
\кзадача


\задача
%\пункт
Делится ли
\вСтрочку
\пункт
многочлен $x^{100}-32x^{90}+x^4+5x^3-3x^2-10x+2$
на многочлен $x^2-2$?\\
\пункт
%Делится ли
многочлен $x^{11}+x^{9}-5x^{8}+x^7-6x^{6}-7x^4-98x^2-49$
на многочлен $x^3-7$?
\кзадача


\задача
Пусть $\alpha$ -- корень многочлена с рациональными коэффициентами. Докажите, что существует многочлен $P \in \mathbb{Q}[x]$, который делит все такие $Q\in\Q[x]$, что $Q(\alpha)=0$.
\кзадача

%\задача
%Степени двух неприводимых над $\Q$ многочленов из $\Q[x]$ различны.
%Могут ли эти многочлены иметь общий действительный корень?
%\кзадача

\сзадача
\вСтрочку
\пункт
Пусть  $\alpha\in\R$ --- корень некоторого ненулевого
многочлена из $\Q[x]$.
Пусть $G(x)$ --- произвольный многочлен из $\Q[x]$, такой что
$G(\alpha)\ne0.$
Докажите, что  существует такой многочлен $H(x)\in\Q[x]$, что
${1\over{G(\alpha)}}=H(\alpha)$.
\пункт
Найдите такой многочлен $H(x)$, если $\alpha=\root 3\of 2$ и $G(x)=x+1$.
\кзадача


\ЛичныйКондуит{0mm}{6mm}

%\СделатьКондуит{5mm}{7.7mm}


\end{document}

%\задача Обозначим многочлены из пункта б) предыдущей задачи как $f$ и $g$.
%Найдите такие многочлены $u$ и $v,$ что {НОД}$(f,g)=fu+gv$,
%причём $\deg u<\deg g$ и $\deg v< \deg f$.
%\кзадача


%\раздел{Дополнительные задачи}

\задача Пусть многочлен $A(x)$ таков, что $A(x)=A(-x)$ при любом $x$.
Докажите, что  существует такой многочлен $P(x),$ что
$A(x)=P(x^2)$ при любом $x$.
\кзадача

\задача
Пусть $p(x)$ --- непостоянный многочлен с целыми коэффициентами.
\сНовойСтроки
\пункт Докажите, что при любом целом числе $n$ либо
$p(n)$ делит $p(n+p(n))$, либо $p(n)=p(n+p(n))=0$.
\пункт Могут ли все числа $p(0)$, $p(1)$, $p(2)$, \dots\  быть простыми?
\кзадача

\задача Пусть $s_1$, \dots, $s_k$  --- корни многочлена
$a_nx^n+\dots+a_1x+a_0$.
Найдите корни многочленов
\вСтрочку \пункт $(-1)^na_nx^n+...+a_2x^2-a_1x+a_0;$
\пункт $a_0x^n+a_1x^{n-1}+...+a_{n-1}x+a_n$.
\кзадача

\задача Коэффициенты произведения двух многочленов с целыми
коэффициентами делятся на~5.  Докажите, что коэффициенты
одного из этих многочленов делятся на~5.
\кзадача

\задача Пусть $p(x)$ --- многочлен с целыми коэффициентами.
\сНовойСтроки
\пункт Докажите, что $a-b$ делит $p(a)-p(b)$  при любых различных
целых числах $a$ и $b$.
\спункт Пусть уравнения $p(x)=1$ и $p(x)=3$ имеют целое решение.
Может ли уравнение $p(x)=2$ иметь два различных целых решения?
\кзадача

\задача Используя равенство $(1+x)^p(1+x)^q=(1+x)^{p+q}$,
вычислите двумя способами коэффициент при $x^m$ в многочлене $(1+x)^{p+q}$
и решите задачу 34 листка 3.
\кзадача

\сзадача
У многочлена $P(x)$ есть  отрицательный коэффициент. Могут ли у всех
его степеней $P^n(x)$ (где $n>1$ --- целое) все коэффициенты быть
положительными?
\кзадача


\задача
\пункт Квадратный трёхчлен $ax^2 + bx + c$ принимает при каждом целом
$x$ целое значение. Верно ли, что среди его коэффициентов
хотя бы один --- целое число?
\пункт Верно ли, что все его коэффициенты --- целые числа?
\кзадача


\задача
Правильные треугольники со сторонами 1, 3, 5, \dots\
расположены в ряд  так, что их основания лежат
на одной прямой вплотную друг к другу.
Докажите, что вершины треугольников, противоположные основаниям,
лежат на некоторой параболе.
\кзадача



\задача \пункт Для каждого $x\in\{-3,-2,-1,-1/2,\ 0,\ 1/2,\ 1,\ 2,\ 3\}$
нарисуйте на плоскости $pOq$ графики прямых, задающихся уравнениями
$x^2+px+q=0.$
\пункт
Напишите уравнение, задающее множество таких точек $(p,q),$ что
квадратный \break
трёхчлен $x^2+px+q$ имеет кратный корень, и изобразите его на
плоскости.
\пункт Докажите, что все прямые из п.~а)
касаются\footnote{В этой задаче будем считать,
что прямая $l$ \выд{касается} некоторой кривой, если эта кривая лежит по одну
сторону от прямой $l$ и имеет с $l$ ровно одну общую точку.}
некоторой кривой. Что это за кривая?
\пункт Укажите на плоскости множества таких точек $(p,q),$ что квадратный
трёхчлен\break $x^2+px+q$ имеет два различных корня, не имеет корней.
\спункт Укажите на плоскости множества таких точек $(p,q),$ что
трёхчлен $x^2+px+q$ имеет на отрезке $[-1;1]$ два различных корня,
кратный корень, не имеет корней.
\кзадача


\end{document} 