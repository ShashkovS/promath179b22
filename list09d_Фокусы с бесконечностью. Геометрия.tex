\documentclass[a4paper,12pt]{article}
\usepackage[mag=930]{newlistok}

%\УвеличитьШирину{1.1truecm}
%\УвеличитьВысоту{2.2truecm}
\renewcommand{\spacer}{\vspace{3pt}}

\Заголовок{Фокусы с бесконечностью: геометрия}
\Подзаголовок{}
\НомерЛистка{9д}
\ДатаЛистка{12.2018}
\Оценки{17/13/9}

\begin{document}


\СоздатьЗаголовок

\задача
Замостите плоскость \\
\пункт
квадратами, среди которых ровно два одинаковых;\\
\пункт
квадратами, среди которых нет одинаковых;\\
\пункт треугольниками, среди которых нет одинаковых.
\кзадача

%3
\задача Можно ли покрыть плоскость\\
\вСтрочку
%\пункт  прямую конечным числом кругов;
\пункт   конечным числом полос;\\
\пункт внутренностями любого счётного набора углов;\\
\пункт конечным числом внутренностей углов, сумма которых
меньше 360$^\circ$;\\
\пункт счётным числом внутренностей парабол;\\
\пункт конечным числом внутренностей парабол?
\кзадача



%\раздел{Графы}

% \задача
% Из плоскости вырезали равносторонний треугольник.
% Можно ли оставшуюся часть плоскости замостить треугольниками, любые два из которых подобны, но не гомотетичны?
% \кзадача
%

\сзадача
Можно ли раскрасить все точки квадрата и круга в чёрный и белый цвета так, чтобы множества белых точек
этих фигур были подобны друг другу и множества чёрных точек также были подобны друг другу (возможно с
различными коэффициентами подобия).
\кзадача

\раздел{***}

%0
%\задача Верно ли, что в натуральном ряду можно выделить\\
%\вСтрочку
%\пункт сколь угодно длинную; \пункт бесконечно длинную цепочку
%подряд идущих составных чисел? \кзадача

\задача
Можно ли разбить бесконечную клетчатую доску на домино так, чтобы\\
\пункт каждая линия сетки разрезала пополам бесконечно много доминошек; \\
\пункт каждая линия сетки разрезала пополам лишь конечное число доминошек?
\кзадача


\задача
Клетки бесконечной клетчатой плоскости окрашены в 2 цвета.
%Обязательно ли
Найдётся ли бесконечное множество вертикалей и
бесконечное множество горизонталей, на пересечении которых
все клетки будут одного цвета?
\кзадача

\задача
\пункт На клетчатой стене закрашены некоторые клетки. Известно, что ладья может подняться по закрашенным клеткам сколь угодно высоко, не наступая ни на какую клетку дважды. Верно ли, что существует бесконечный путь по закрашенным клеткам, при прохождении которого ладья не наступает повторно ни на какую клетку и поднимается сколь угодно высоко?\\
\пункт На клетчатой плоскости закрашено некоторое множество клеток. Известно, что ладья может проделать по закрашенным клеткам путь сколь угодно большой длины, не наступая ни на какую клетку дважды. Верно ли, что ладья может двигаться по закрашенным клеткам бесконечно долго так, чтобы никогда не наступить повторно ни на какую клетку?\\
\спункт Из клетчатой плоскости вырезаны некоторые клетки. Для любого конечного набора оставшихся клеток часть плоскости можно замостить доминошками так, чтобы все клетки этого набора были покрыты доминошками. Верно ли, что тогда и всю плоскость можно замостить доминошками?
\кзадача


\раздел{***}



\задача В этой задаче разрешается поворачивать фигуры, накладывать их друг на друга.\\
\пункт Существуют ли такие 100 прямоугольников, что ни один из них нельзя покрыть остальными 99-ю?\\
\пункт Существует ли бесконечно много таких прямоугольников, что ни один из них нельзя покрыть никаким конечным набором остальных?\\
\пункт Существует ли бесконечно много таких прямоугольников, что ни один из них нельзя покрыть остальными?
\кзадача

\задача
В бесконечной последовательности бумажных прямоугольников площадь $n$-го прямоугольника равна $n^2$. Всегда ли ими можно покрыть ими плоскость? Наложения допускаются.
\кзадача

\сзадача
Дано бесконечно много квадратов. Всегда ли ими можно покрыть плоскость (наложения допускаются), если известно, что для любого числа $N$ найдется конечное число квадратов суммарной площади больше $N$?
\кзадача

\ЛичныйКондуит{0mm}{6mm}



%\СделатьКондуит{6mm}{8mm}
% \GenXMLW

\end{document}

\задача
Игра происходит на плоскости. Играют двое: первый передвигает
одну фишку-волка, второй --- $k$ фишек-овец. После хода
волка ходит одна из овец, затем после следующего хода волка
опять какая-нибудь из овец и т.~д. И волк, и овцы
передвигаются за один ход в любую сторону не
более, чем на метр. Верно ли, что при любой
первоначальной позиции волк поймает хотя бы одну
овцу (окажется с ней в одной точке)?
\кзадача


\задача
Город представляет собой бесконечную клетчатую плоскость (линии --- улицы,
клеточки --- кварталы). На одной из улиц через каждые 100 кварталов на
перекр\"естках стоит по милиционеру. Где-то в городе есть бандит
(его местонахождение неизвестно, но перемещается он только по улицам).
Цель милиции --- увидеть бандита.
Есть ли у милиции алгоритм наверняка достигнуть своей цели?
Максимальные скорости милиции и бандита
--- какие-то конечные, но неизвестные нам величины (у бандита
скорость может быть больше, чем у милиции). Милиция видит вдоль
улиц во все стороны на бесконечное расстояние.
\кзадача

\задача
В городе из  задачи 19 трое полицейских
ловят вора (местонахождение вора
неизвестно, но перемещается он только по улицам).
Максимальные скорости у полицейских и вора одинаковы.
Вор считается пойманным, если он оказался на одной улице с полицейским.
Смогут ли полицейские поймать вора?
\кзадача

