\documentclass[a4paper,11pt]{article}
\usepackage[mag=1000]{newlistok}
\usepackage{tikz}
\usetikzlibrary{calc}

\УвеличитьШирину{1.3truecm}
\УвеличитьВысоту{2.5truecm}

\Заголовок{Логика}
\НомерЛистка{24}
\renewcommand{\spacer}{\vfill}
\ДатаЛистка{10.10 -- 24.10/2018}
\Оценки{28/23/18}

\newcommand{\0}[1]{\overline{#1}}

%\documentstyle[11pt, russcorr, listok]{article}
%\newcommand{\del}{\mathrel{\raisebox{-.3 ex}{${\vdots}$}}}

\begin{document}

\СоздатьЗаголовок


% \задача Число $x$ натуральное. Среди утверждений 1) $2x>70$,
% 2) $x<100$, 3) $3x>25$, 4) $x \ge 10$, 5) $x>5$ три
% верных и два неверных. Чему равно $x$?
% \кзадача

% \задача Перед футбольным матчем команд  <<Север>>  и <<Юг>> было дано пять прогнозов:
% 1) ничьей не будет; 2) в ворота <<Юга>> забьют; 3) <<Север>> выиграет; 4) <<Север>> не проиграет; 5) в матче будет забито ровно $3$ гола. После матча выяснилось, что верными оказались ровно три прогноза. С каким счётом закончился матч?
% \кзадача

\пзадача Расставьте вместо многоточий слова «необходимо», «достаточно», и там, где это возможно, «необходимо и достаточно» так, чтобы получились верные утверждения.\\
\пункт Чтобы число $x$ делилось на $5$, $\ldots\ldots\ldots\ldots$, чтобы его десятичная запись кончалась цифрой $0$.\\
\пункт Чтобы число $x$ делилось на $9$, $\ldots\ldots\ldots\ldots$, чтобы сумма цифр его десятичной записи делилась на $3$.\\
\пункт Чтобы параллелограмм был ромбом, $\ldots\ldots\ldots\ldots$, чтобы его диагонали делили пополам внутренние~углы.\\
\пункт Чтобы параллелограмм был квадратом, $\ldots\ldots\ldots\ldots$, чтобы его стороны были равны.
\кзадача

\опр %Назовём {\it высказыванием} любое повествовательное
% предложение, %про
% которое %можно сказать,
% либо истинно, либо ложно.
Пусть  $A$ и $B$ --- некоторые утверждения. Определим
следующие утверждения:\\
%Высказывание
\лк не $A$\пк\ (обозначение $\overline A$) --- % и называют
{\it отрицание\/} $A$, истинно если и только если $A$ ложно;\\
%Высказывание
\лк $A$ и $B$\пк\ (обозначение $A\wedge B$) --- % и называют
{\it конъюнкция\/} %высказываний
$A$ и $B$, истинно если и только если и $A$, и $B$ истинны; \\
%Высказывание
\лк $A$ или $B$\пк\ ($A\vee B$) --- % и называют
{\it дизъюнкция\/} %высказываний
$A$ и $B$,
истинно если и только если хотя бы одно из $A$ и $B$ истинно; \\
%Высказывание
\лк если $A$, то $B$\пк\ (обозначение $A\Rightarrow B$), % --- {\it импликация\/},
истинно если и только если $A$ ложно или и $A$, и $B$ истинны.\\
Говорят, что $A$ и $B$ {\em равносильны} ($A\Leftrightarrow B$), если $A$ истинно тогда и только тогда, когда истинно $B$.
\копр

% \опр Будем говорить, что утверждение $\overline{P_1}$ является \выд{отрицанием} к утверждению $P_1$, если $\overline{P_1}$ верно тогда и только тогда, когда не верно $P_1$.
% \копр

\задача Докажите, что утверждения <<из $P_1$ следует $P_2$>> и <<из $\overline{P_2}$ следует $\overline{P_1}$>> равносильны.
\кзадача

% \задача Докажите, что если $m >1$ и $(m-1)! + 1$ делится на $m$, то число $m$ --- простое.
% \кзадача

\задача Равносильны ли утверждения «кто не с нами, тот против нас» и «кто не против нас, тот с нами»?
\кзадача

\пзадача %Рассмотрим утверждения вида «$P_1$ и $P_2$» (обозначается $P_1 \wedge P_2$ ) и «$P_1$ или $P_2$» (обозначается $P_1 \vee P_2$).
Докажите такие теоремы (правила де Моргана):
\пункт $\overline{P_1 \wedge P_2}\ \Leftrightarrow\ \overline{P_1} \vee\overline{P_2}$;
\пункт $\overline{P_1 \vee P_2}\  \Leftrightarrow\ \overline{P_1} \wedge \overline{P_2}$.
\кзадача

\пзадача Однажды принцесса сказала: «Хочу, чтобы мой муж был красивый, не был глупым или некрасивым, или чтобы был некрасивым, но не был глупым». Упростите это утверждение.
\кзадача

\пзадача Рассмотрим утверждения вида «для любого $h\in H$ верно $Q$» (обозначается $\forall\ h\in H : Q$) и «существует $h\in H$ такой, что верно $Q$» (обозначается $\exists\ h\in H : Q$). Постройте отрицания к этим утверждениям.
\кзадача



% \задача Истинно ли отрицание утверждения «для любого четырехугольника есть вписанная в него окружность»?
% \кзадача

% \задача В квадрате $3 \times 3$ закрашено $5$ клеток. Докажите, что найдется закрашенная клетка, в строке и в столбце которой найдется еще по одной закрашенной клетке.
% \кзадача

\задача Постройте отрицания к следующим утверждениям:\\
\пункт В каждом классе найдется ученик, который решил хотя бы одну задачу из контрольной.\\
\пункт Найдется класс, в котором каждый ученик решил хотя бы одну задачу из контрольной.\\
\пункт Существует такая задача, что в каждом классе хотя бы один ученик ее решил.\\
\пункт Для каждой задачи есть класс, в котором все ученики ее решили.\\
\пункт Есть город, в каждом районе которого есть улица, на которой в каждом доме есть однокомнатная квартира.\\
\пункт В каждом городе есть магазин, в котором нет хлеба, и никто из продавцов не знает, когда он будет.
\кзадача

\задача
Десять логиков пришли в кафе. Каждый заранее решил заказать себе кофе или чай, но ни один не знал
планов остальных. Официантка громко спросила: «ВСЕМ принести кофе?», --- а затем обошла логиков по одному, записывая ответы.
Каждый громко и правдиво ответил на её вопрос «Не знаю», «Да» или «Нет».
\пункт Пусть первые 9 логиков ответили «Не знаю»,
а 10-й сказал «Да». Сколько логиков решили заказать себе кофе?
\пункт Пусть 6-й и 7-й ответы были разными. Сколько каких ответов было? Найдите наименьшее
число логиков, наверняка заказавших себе кофе и наименьшее число логиков, наверняка заказавших себе чай.
\кзадача

\пзадача
Назов\"ем контрольную л\"егкой, если за каждой партой найд\"ется
ученик, решивший не менее 90\% задач. Дайте определение
трудной (т.~е.~не являющейся л\"егкой) контрольной, не используя
частицы \лк не\пк.
\кзадача


\задача Формализуйте фразу «ученики должны показывать свои тетради учителям», рассматривая множества учеников, тетрадок и учителей. Придумайте несколько вариантов, как это можно сделать.
\кзадача

\пзадача
Выразите
\вСтрочку
\пункт
$A\to B$;
\пункт
$A\wedge B$ через $A$ и $B$, используя только дизъюнкцию и отрицание.
\кзадача

\пзадача
Выразите
\вСтрочку
\пункт
$\overline{A\to B}$;
\пункт
$A\vee B$ через $A$ и $B$, используя только конъюнкцию и отрицание.
\кзадача

\сзадача
Докажите, что %любое
высказывание, истинность которого зависит
только от истинности высказываний $A_1,\dots,A_n$,
выражается через них с помощью %единой формулы, использующей
только дизъюнкции, конъюнкции и отрицания.
\кзадача


\сзадача
На острове рыцарей и лжецов некоторые жители знакомы между собой (знакомство взаимно). Турист встретил 10 аборигенов. Каждый абориген сказал про каждого из остальных 9 аборигенов одну из фраз: «Я его не знаю», «Это мой знакомый рыцарь», «Это мой знакомый лжец». Любые двое сказали друг про друга разные фразы. Про какое наибольшее число аборигенов турист может гарантированно узнать,~кто~они?
\кзадача

\задача
Солдату-цирюльнику пришел приказ: брить тех солдат его взвода,
которые не бреются сами\break (а~остальных не брить). Сможет ли он его выполнить?
\кзадача

\задача Являются ли следующие утверждения истинными или ложными (и вообще, утверждения ли это)?
\begin{center}
\vspace*{-1.7mm}
\framebox{Утверждение в рамке ложно}\hfil
\framebox{\framebox{Утверждение в двойной рамке истинно}}
\vspace*{-1mm}
\end{center}
\кзадача

\задача [Истинное происшествие]
Н.Н.Константинов сказал своим %математического
кружковцам: \лк В январе кружок %семинара
проходит 13, 17, 20, 24, 27 и 31 числа. В один из этих дней
вам будет дана %неожиданная
контрольная работа, но в какой
именно день, вы накануне знать ещ\"е не будете\пк.
Докажите, что эта контрольная не могла быть дана %\сНовойСтроки
\пункт
31 января;
\пункт
27 января;
\пункт
20 января.
\спункт
Но 20 января контрольная состоялась (единственную её задачу вы сейчас читаете), и накануне ни один кружковец об этом не знал. Как это совместить с решением пунктов а) -- в)?
\кзадача



\ЛичныйКондуит{0mm}{5mm}
% \GenXMLW


\end{document}

