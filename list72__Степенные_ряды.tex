% !TEX encoding = Windows Cyrillic
\documentclass[a4paper,12pt]{article}
\usepackage{newlistok}

\ВключитьКолонитул

\УвеличитьВысоту{2cm}
\УвеличитьШирину{1.5cm}
%\renewcommand{\spacer}{\vfil}

\Заголовок{Степенные ряды и производящие функции}
\НомерЛистка{72}
\ДатаЛистка{08.11 -- 19.11.2021}
\Оценки{32/24/16}
%\Подзаголовок{}

\renewcommand{\binom}[2]{C_{#1}^{#2}}

\begin{document}
\СоздатьЗаголовок


\раздел{Степенные ряды}

%\vfill
\опр
\выд{Формальным степенным рядом} от переменной $t$
называется бесконечное выражение вида
$
A(t)=a_0+a_1t+a_2t^2+\dots=\sum\limits_{k=0}^{+\infty} a_kt^k,
$
где $a_0,a_1,\dots$ --- числовая последовательность
(\выд{коэффициенты} ряда).
Два ряда считаются равными, если равны их соответствующие коэффициенты.
Слагаемые с нулевыми коэффициентами мы
будем, как правило, пропускать. Например, многочлен --- это ряд с конечным
числом ненулевых коэффициентов.

\noindent
Сопоставление рядам $F$, $G$ нового ряда $H$ называется
\выд{формальной алгебраической операцией}, если каждый коэффициент
ряда $H$ вычисляется конечным числом арифметических действий над
конечным числом коэффициентов рядов $F$, $G$. Например, сложение и
умножение рядов определяются так же, как для многочленов и
\выд{являются формальными операциями}, а
\лк вычисление значения ряда при данном числовом значении $t$\пк\
\выд{не является формальной операцией} (и потому здесь не определяется).
%
%Степенные ряды можно складывать и
%перемножать (раскрывая скобки и приводя подобные).
\копр








\задача
Проверьте, что сложение и умножение рядов являются формальными операциями.
% Как~вы\-ра\-зить коэффициенты суммы и коэффициенты произведения через
% коэффициенты сомножителей?
\кзадача









\пзадача
\пункт Пусть $F(t)=1+t+t^2+t^3+\ldots$,
$G(t)=1-t+t^2-t^3+\ldots$.
Найдите $F+G$ и $F\cdot G$.
\пункт Пусть $F=\sum\limits_{k=0}^{+\infty}\frac{1}{k!}t^k$,
$G=\sum\limits_{k=0}^{+\infty}\frac{(-1)^k}{k!}t^k$.
Найдите $F\cdot G$ и $F^2$.
\кзадача









\задача[Замена переменной]
Является ли формальной алгебраической операцией подстановка
в ряд вместо переменной $t$ произвольного ряда с нулевым свободным членом?
\кзадача








\пзадача
Найдите (если это возможно) такой ряд $F$, что
%\сНовойСтроки
\вСтрочку
\пункт
$(1-t)\cdot F=1$;
\пункт
$(2-t)\cdot F=1$;
\пункт
$(t^2+t^3+t^4+\dots)\cdot F=t^4-t^6+t^8-\dots$;
\пункт
$(t^3+t^4+t^5+\dots)\cdot F=t^2-t^4+t^6-\dots$.
\кзадача








\задача
Ряд $a(t)$ называется \выд{обратимым}, если существует такой ряд
$a^{-1}(t)$, что $a(t)a^{-1}(t)=1$. Докажите, что ряд
$a(t)$ обратим тогда и только тогда, когда $a_0\ne 0$,
причём $a^{-1}(t)$ единственен, и его отыскание есть
формальная операция.
\кзадача








\задача
%Числа $a$ и $b$ ненулевые.
%Выразите ряд $((a-t)(b-t))^{-1}$ через ряды $(a-t)^{-1}$ и $(b-t)^{-1}$.
При каких условиях на числа $a$ и $b$ ряд
$\displaystyle\frac1{(a-t)(b-t)}$ можно записать в виде
$\displaystyle\frac{c}{(a-t)}+\frac{d}{(b-t)}$
(подобрав подходящие числа $c$ и $d$)?
%Всегда ли существуют такие числа $c$ и $d$, что имеет место равенство
%степенных рядов $((a-t)(b-t))^{-1}=c(a-t)^{-1}+d(b-t)^{-1}$?
\кзадача








\пзадача
Вычислите все коэффициенты для ряда, обратного к
\\
\пункт $(1-t)(2-t)$;
\пункт $(1-t)^{2}$;
\пункт $(1-t)^{m}$;
% \\
\пункт $(t-1)(t+2)(t-3)$;
% \пункт $\bigl((t+1)^2(t-2)(t+3)^3\bigr)^{-1}$;
\пункт $t^{2}+t-1$.
% \спункт $t^{2}+t+1$.
\кзадача






\задача
Сформулируйте условия, при которых ненулевой степенной ряд $F$ можно
разделить на ненулевой степенной ряд~$G$ (иначе говоря, уравнение $G\cdot
X=F$ разрешимо относительно неизвестного степенного ряда~$X$).
Всегда ли результат деления определен однозначно?
\кзадача




\задача
При каких условиях на степенной ряд $F$ разрешимо уравнение $X^2 = F$ относительно неизвестного степенного ряда $X$?
\кзадача



\ЛичныйКондуит{0mm}{6.5mm}
\ОбнулитьКондуит

\newpage
\раздел{Производящие функции}

\опр
Пусть $(a_k)=(a_0,a_1,\dots)$ --- числовая последовательность, а
$t$ --- формальная переменная. Степенной ряд $\sum a_kt^k$
называется \выд{производящей функцией} последовательности
$(a_k)$.
\копр








\задача %\пункт
Пусть $F(t)$ --- производящая функция последовательности $(a_k)$. Для
какой последовательности производящей функцией будет степенной ряд
\вСтрочку
\пункт $tF(t)$;
\пункт $t^2F(t)$;
\пункт $(1+t)F(t)$?
\кзадача








\задача
\пункт
Напишите, пользуясь рекуррентным соотношением
$u_{k+2}=u_{k+1}+u_k$ и начальными условиями $u_0=0$, $u_1=1$,
уравнение для производящей функции чисел Фибоначчи и решите его.
\пункт
Найдите формулу для $n$-го числа Фибоначчи.
\кзадача








\пзадача Найдите явную формулу для последовательности $(g_n)$,
если $g_0=g_1=1$ и при $n\geq2$
\вСтрочку
\пункт
$g_n=5g_{n-1}-6g_{n-2}$;
\пункт
$g_n=6g_{n-1}-9g_{n-2}$;
\пункт
$g_n=g_{n-1}+2g_{n-2}+(-1)^n$;
\пункт
$g_n=g_{n-1}+2g_{n-2}+\ldots+n g_0$.
\кзадача

% \break








\задача Сколькими способами можно замостить
прямоугольник $3\times n$ плашками размера $2\times1$?
\кзадача








%\сзадача Сколько существует способов построить
%$2\times2\times n$-колонну из кирпичей размера $2\times1\times1$?
%\кзадача


\опр
Для произвольного числа $\alpha$ и натурального числа $k$
\выд{биномиальный коэффициент} $\displaystyle\binom{\alpha}k$
определяется формулой
$$
\displaystyle\binom{\alpha}k=\frac{\alpha(\alpha-1)(\alpha-2)\dots(\alpha-k+1)}{k!}.
$$
% (Для $\alpha=n\in\N$ определение совпадает со старым).
Для каждого $\alpha$ рассмотрим следующий степенной ряд:
$$
\displaystyle(1+t)^\alpha=\sum_{k=0}^{\infty}\binom{\alpha}kt^k.
$$
(Для натуральных $\alpha$ это уже знакомая вам формула бинома
Ньютона, а для остальных $\alpha$ правая часть равенства является
определением левой.)
\копр








\задача
\пункт
Ряд $(1+t)^{-1}$ определяется теперь двумя способами:
как обратный к ряду $1+t$ и по биномиальной формуле.
Согласуются ли эти определения?
\\\пункт
Докажите, что для любого натурального числа $n$ имеет место
равенство $(1+t)^{-n}(1+t)^n=1$.
\кзадача








\пзадача  Рассмотрим два многочлена от двух переменных:
$G(x,y) = \displaystyle\binom{x+y}n$ и  $\displaystyle F(x,y) = \sum\limits_{j=0}^n \displaystyle\binom{x}j \cdot \displaystyle\binom{y}{n-j}$.
\\\пункт Докажите, что $F(x,y)=G(x,y)$ для натуральных $x > n$, $y > n$.
\\\пункт Используя предыдущий пункт, выведите равенство
$(1+t)^{\alpha+\beta} = (1+t)^{\alpha}\cdot (1+t)^{\beta}$.
\\\пункт
Пусть $\al=\dfrac{m}{n}$ — рациональное число.
Докажите, что $\displaystyle(1+t)^{m}=\hr{\sum\limits_{j=0}^{\infty} \displaystyle\binom{\alpha}j t^j}^n$.
\кзадача



% (1+t)^m = sum^n




\задача
Пусть $c_0=1$, а при $n\geq1$ пусть
$c_n$ --- это число правильных расстановок $n$ открывающих и
$n$ закрывающих скобок ($n$-е число Каталана).

\пункт Докажите, что $c_n$ удовлетворяет рекуррентной формуле $c_n = c_0\cdot c_{n-1} + c_1 \cdot c_{n-2} + \ldots + c_{n-1}\cdot c_0$.

\пункт Докажите, что производящая функция $C(t)$ чисел Каталана удовлетворяет уравнению $$t \cdot C^2(t) - C(t)+1=0.$$

\пункт Решив квадратное уравнение, и использовав формулу для $(1+t)^{1/2}$ покажите, что
$$c_n = \dfrac{\dfrac{1}{2}\cdot \dfrac{1}{2}\cdot \dfrac{3}{2}\cdot \ldots \cdot \dfrac{2n-1}{2} \cdot 4^{n+1}}{2(n+1)!}.$$

\пункт Докажите, что $c_n = \dfrac{1}{n+1} C_{2n}^n$.

\кзадача





\ЛичныйКондуит{0mm}{6.5mm}
% \GenXMLW



\end{document}








\задача
Пусть $s_n$ --- число способов выплатить $n$ гривен купюрами
в 1 и 2 гривны. Является ли ряд
$(1+t+t^2+t^3+\ldots)(1+t^2+t^4+t^6+\ldots)$
производящей функцией для последовательности $(s_n)$?
\кзадача









\задача[Старинная задача]
%Примените производящие функции к решению такой старинной задачи:
Сколькими способами можно заплатить 1 рубль копейками, алтынами
(трёхкопеечными монетами) и пятаками (пятикопеечными монетами)?
\кзадача

\раздел{Разбиения чисел*}

\опр
\выд{Число разбиений} $p(n)$
--- это число всевозможных представлений $n$ в виде суммы
нескольких натуральных слагаемых\footnote[1]{
Разбиения, отличающиеся только порядком слагаемых,
считаются одинаковыми.}, то есть число всех
$n$-клеточных диаграмм Юнга; $p(0)\bydef1$.
\копр








\задача
Вычислите $p(n)$ для $n\le10$.
\кзадача








\задача
Пусть $P(t)=\sum\limits_{n=0}^{+\infty} p(n)t^n$ --- производящая
функция последовательности $(p(n))$.
Докажите, что
$$
%\sum\limits_{n=0}^{+\infty} p(n)t^n
P(t)=
(1+t+t^2+t^3+\ldots)\cdot(1+t^2+t^4+t^6+\ldots)
\cdot(1+t^3+t^6+t^9+\ldots)\cdot\ldots
$$
(проверив заодно, что это произведение
бесконечного числа рядов является формальной операцией).
\кзадача









%\задача
%Убедитесь, что корректно определено бесконечное
%произведение геометрических прогрессий
%Докажите, что
%$P(t)=(1-t)^{-1}\cdot(1-t^2)^{-1}\cdot(1-t^3)^{-1}\cdot\ldots$.
%\кзадача








\задача
Докажите, что
$1/P(t)=(1-t)\cdot(1-t^2)\cdot(1-t^3)\cdot\ldots$.
\кзадача








\задача Пусть $l(n)$ --- это число разбиений числа $n$ на нечётные
натуральные слагаемые,
и пусть $d(n)$ --- это число разбиений числа $n$ на различные
натуральные слагаемые$^1$. Докажите, что\\
\вСтрочку
\пункт
$\sum\limits_{n=0}^{+\infty}
l(n)t^n=\displaystyle\frac1{(1-t)\cdot(1-t^3)\cdot(1-t^5)\cdot\ldots}$;
\пункт
$\sum\limits_{n=0}^{+\infty}
d(n)t^n=(1+t)\cdot(1+t^2)\cdot(1+t^3)\cdot\ldots$.
\кзадача








\задача Докажите, что $l(n)=d(n)$ при всех натуральных $n$.
\кзадача








%\задача
%Докажите, что $1/P(t)=\prod\limits_{k\ge1}(1-t^{k})=
%1+\sum\limits_{n\ge1}\({\wht p}_{{\rm ч}}(n)-{\wht
%p}_{{\rm н}}(n)\) \cdot t^n$, где ${\wht p}_{{\rm ч}}(n)$ и
%${\wht p}_{{\rm н}}(n)$ --- это количества $n$-клеточных
%диаграмм Юнга, в которых длины всех строк различны,
%а общее число строк чётно и, соответственно, нечётно.
%\кзадача








\задача
Докажите, что
$\displaystyle\frac1{P(t)}=1+\sum\limits_{n\ge1}\(p_{{\rm ч}}(n)-
p_{{\rm н}}(n)\) \cdot t^n$, где $p_{{\rm ч}}(n)$ и $p_{{\rm н}}(n)$ ---
%это количества $n$-клеточных диаграмм Юнга, в которых длины строк различны,
%а общее число строк чётно и, соответственно, нечётно.
количества разбиений числа $n$ на чётное, и, соответственно, на нечётное
число различных натуральных слагаемых$^1$.
%(т.~е.~количества $n$-клеточных~ди\-а\-грамм Юнга, в которых длины строк
%различны, а общее число строк чётно и, соответственно, нечётно).
\кзадача








\задача
Рассмотрим множество $Y_n$ всех $n$-клеточных диаграмм
Юнга, строки которых имеют различные длины. Назовём
{\it торцом\/} такой диаграммы её правую верхнюю клетку и все
клетки, идущие от неё под углом $45^\circ$ по диагонали влево
вниз. Попробуем задать на $Y_n$ отображение, которое, в
зависимости от того, что у диаграммы длиннее --- торец или нижняя
строка, --- либо отрезает её торец и пририсовывет его новой
нижней строкой, либо, наоборот, отрезает нижнюю строку и
пририсовывет её новым торцом. Выясните, для каких диаграмм это
отображение не определено, и при каких $n$ оно устанавливает в
$Y_n$ взаимно однозначное соответствие между диаграммами из чётного
и из нечётного числа строк (а значит доказывает равенство
$p_{{\rm ч}}(n)=p_{{\rm н}}(n)$).
\кзадача








\задача [Пентагональная теорема Эйлера]
Докажите, что %$\displaystyle
$$\frac1{P(t)}=1+\sum\limits_{k\ge1}
(-1)^k\(t^{\(3k^2-k\)/2}+t^{\(3k^2+k\)/2}\)$$
и $p(n)=p(n-1)+p(n-2)-p(n-5)-p(n-7)+p(n-12)+p(n-15)-\,\cdots$.
\кзадача


\раздел{Дополнительные задачи.}








\задача
Для любого конечного множества $M$ положим
$C_{M}(t)\bydef\sum\limits_{k\ge0}c_{k}(M)t^{k}$, где $c_{k}(M)$ ---
это число всех $k$-эле\-мент\-ных подмножеств в $M$. Для двух
непересекающихся конечных множеств $A$ и $B$ выразите
$C_{{A\cup B}}(t)$ и $C_{{A\x B}}(t)$
%\footnote{{\it произведение\/} $A\x B$ множеств $A$ и
%$B$ представляет собой множество всех пар $(a,b)$ с $a\bl A$,
%$b\bl B$} $C_{{A\x B}}(t)$
через $C_{{A}}(t)$ и $C_{{B}}(t)$.
\кзадача









\ссзадача[Нерешённая проблема]
Докажите или опровергните: если все коэффициенты ряда $G(z)$
равны либо 0, либо 1 и при этом все коэффициенты $\bigl(G(z)\bigr)^2$
меньше некоторой константы $M$, то бесконечно много из
коэффициентов $\bigl(G(z)\bigr)^2$  равны нулю.
\кзадача


\end{document}








\задача
  Является ли $P_m(t)\bydef\sum\limits_{n\ge0}p_m(n)\cdot t^n$\,,
  где $p_m(n)$ --- это число $n$-клеточных диаграмм Юнга из
  $\le m$ строк, рациональной функцией от $t$? Выразите $p_m(n)$
  через $p_{m-1}(n)$ и $p_m(n-m)$.
\кзадача








\задача[Гауссовы биномиальные коэффициенты]\label{gbk}
  Пусть $F_n(t)=(1-t)(1-t^2)\,\cdots\,(1-t^n)$.
  Является ли $G_n^k(t)\bydef
  F_n(t)/\(F_{k}(t)\cdot F_{n-k}(t)\)$ многочленом? Вычислите
  $\lim\limits_{t\to0}G_n^k(t)$.
\кзадача








\задача
  Выразите через гауссовы биномиальные коэффициенты производящую
  функцию $P_m^k(t)=\sum\limits_{n\ge0}p_m^k(n)\cdot t^n$ чисел
  $p_m^k(n)$ $n$-кле\-то\-ч\-ных диаграмм Юнга высоты $\le m$  и
  ширины $\le k$\,.
\кзадача




\end{document}

Рассмотрим множество бесконечных последовательностей из действительных чисел\footnote{ Конечно, можно рассматривать здесь и числа из другие полей: рациональные или комплексные}. Как мы знаем (см. листок 54, задача 1з)) это множество является линейным пространством. Добавим к введённым операциям {\it умножение} двух последовательностей так, как мы делаем это для многочленов:

$$(a_0,\ldots,a_n,\ldots)\cdot (b_0,\ldots,b_n,\ldots) = (c_0,\ldots,c_n,\ldots),$$ $$ \mbox{ где } с_n = \sum_{j=0}^n {a_jb_{n-j}}$$








\задача Проверьте, что

$\bullet$ последовательности вида $(a_0,0,\ldots,0)$ относительно операций сложения и умножения изоморфны полю действительных чисел (в дальнейшем мы будем отождествлять такие последовательности с соответствующими числами);

$\bullet$ если обозначить через $t$ последовательность $(0,1,0,\ldots,0)$, то элементы $1,t,t^2,\ldots$ образуют базис в линейном пространстве всех последовательностей.

\кзадача

Другими словами,  любая последовательность $(a_0,a_1,\ldots,a_n,\ldots,)$ единственным образом представляется в виде $a_0 \cdot 1 + a_1 \cdot t + a_2 \cdot t^2 + \ldots + a_n \cdot t^n + \ldots$. Выражения $a_0 \cdot 1 + a_1 \cdot t + a_2 \cdot t^2 + \ldots + a_n \cdot t^n + \ldots$ (с операциями сложения и умножения) называют {\it формальными степенными рядами}. Ещё говорят, что $a_0+a_1\cdot t + a_2 \cdot t^2+\ldots + a_n \cdot t^n + \ldots$ --- это \выд{производящая функция} последовательности
$(a_k)$.


 $a_0$ называют {\it свободным членом} ряда. Формальные степенные ряды являются обобщением понятия многочлена. Все свойства многочленов
(коммутативность по сложению/умножению, ассоциативность, дистрибутивность) переносятся на формальные степенные ряды.


%\vfill
%\опр
%\выд{Формальным степенным рядом} от переменной $t$
%называется бесконечное выражение вида
%$
%A(t)=a_0+a_1t+a_2t^2+\dots=\sum\limits_{k=0}^{+\infty} a_kt^k,
%$

%где $a_0,a_1,\dots$ --- числовая последовательность
%(\выд{коэффициенты} ряда).
%Два ряда считаются равными, если равны их %соответствующие коэффициенты.
%Слагаемые с нулевыми коэффициентами мы
%будем, как правило, пропускать. Например, многочлен --- %это ряд с конечным
%числом ненулевых коэффициентов.

\noindent
Сопоставление рядам $F$, $G$ нового ряда $H$ называется
\выд{формальной алгебраической операцией}, если каждый коэффициент
ряда $H$ вычисляется конечным числом арифметических действий над
конечным числом коэффициентов рядов $F$, $G$. Например, сложение и
умножение рядов определяются так же, как для многочленов и
\выд{являются формальными операциями}, а
\лк вычисление значения ряда при данном числовом значении $t$\пк\
\выд{не является формальной операцией} (и потому здесь не определяется).
%
%Степенные ряды можно складывать и
%перемножать (раскрывая скобки и приводя подобные).









\задача
\пункт Пусть $F(t)=1+t+t^2+t^3+\ldots$,
$G(t)=1-t+t^2-t^3+\ldots$.
Найдите $F+G$ и $F\cdot G$.
\пункт Пусть $F=\sum\limits_{k=0}^{+\infty}\frac{1}{k!}t^k$,
$G=\sum\limits_{k=0}^{+\infty}\frac{(-1)^k}{k!}t^k$.
Найдите $F\cdot G$ и $F^2$.
\кзадача








\задача Существует ли два таких ненулевых степенных ряда $F,G$, что $F\cdot G = F+G$?
\кзадача








\задача[Замена переменной]
Является ли формальной алгебраической операцией подстановка
в ряд вместо переменной $t$ произвольного ряда с нулевым свободным членом?
\кзадача








\задача
Найдите (если это возможно) такой ряд $F$, что
%\сНовойСтроки
\вСтрочку
\пункт
$(1-t)\cdot F=1$;
\пункт
$(2-t)\cdot F=1$;
\пункт
$(t^2+t^3+t^4+\dots)\cdot F=t^4-t^6+t^8-\dots$;
\пункт
$(t^3+t^4+t^5+\dots)\cdot F=t^2-t^4+t^6-\dots$.
\кзадача








\задача
Ряд $a(t)$ называется \выд{обратимым}, если существует такой ряд
$a^{-1}(t)$, что $a(t)a^{-1}(t)=1$. Докажите, что ряд
$a(t)=\SER a,t$ обратим тогда и только тогда, когда $a_0\ne 0$,
причём $a^{-1}(t)$ единственен, и его отыскание есть
формальная операция.
\кзадача








\задача
%Числа $a$ и $b$ ненулевые.
%Выразите ряд $((a-t)(b-t))^{-1}$ через ряды $(a-t)^{-1}$ и $(b-t)^{-1}$.
При каких условиях на числа $a$ и $b$ ряд
$\displaystyle\frac1{(a-t)(b-t)}$ можно записать в виде
$\displaystyle\frac{c}{(a-t)}+\frac{d}{(b-t)}$
(подобрав подходящие числа $c$ и $d$)?
%Всегда ли существуют такие числа $c$ и $d$, что имеет место равенство
%степенных рядов $((a-t)(b-t))^{-1}=c(a-t)^{-1}+d(b-t)^{-1}$?
\кзадача








\задача
Найдите $n$-тый коэффициент ряда:
\вСтрочку
\пункт $\bigl((1-t)(2-t)\bigr)^{-1}$;
\пункт $\bigl((1-t)^{2}\bigr)^{-1}$;
\пункт $\bigl((1-t)^{m}\bigr)^{-1}$;\\
\пункт  $\bigl((t-1)(t+2)(t-3)\bigr)^{-1}$;
\пункт $\bigl((t+1)^2(t-2)(t+3)^3\bigr)^{-1}$;
\пункт $(t^{2}+t-1)^{-1}$;
\спункт $(t^{2}+t+1)^{-1}$.
\кзадача








\задача
Сформулируйте условия, при которых степенной ряд $F$ можно
разделить на степенной ряд~$G$ (иначе говоря, уравнение $G\cdot
X=F$ разрешимо относительно неизвестного степенного ряда~$X$).
Всегда ли результат деления определен однозначно?
\кзадача








\задача
\пункт
При каких условиях на степенной ряд $F$ разрешимо уравнение
$X^2=F$ относительно неизвестного степенного ряда $X$?
\пункт
Найдите коэффициенты такого степенного ряда $X$, что $X^2=1+t$.
\пункт
Существует ли степенной ряд $X$, удовлетворяющий уравнению
$tX^2-X+1=0$?
\кзадача









\задача %\пункт
Пусть $F(t)$ --- производящая функция последовательности $(a_k)$. Для
какой последовательности производящей функцией будет степенной ряд
\вСтрочку
\пункт $tF(t)$;
\пункт $t^2F(t)$;
\пункт $(1+t)F(t)$?
\кзадача








\задача
\пункт
Напишите, пользуясь рекуррентным соотношением
$u_{k+2}=u_{k+1}+u_k$ и начальными условиями $u_0=0$, $u_1=1$,
уравнение для производящей функции чисел Фибоначчи и решите его.
\пункт
Найдите формулу для $n$-го числа Фибоначчи.
\кзадача








\задача Найдите явную формулу для последовательности $(g_n)$,
если $g_0=g_1=1$ и при $n\geq2$
\вСтрочку
\пункт
$g_n=5g_{n-1}-6g_{n-2}$;
\пункт
$g_n=6g_{n-1}-9g_{n-2}$;
\пункт
$g_n=g_{n-1}+2g_{n-2}+(-1)^n$;
\пункт
$g_n=g_{n-1}+2g_{n-2}+\ldots+n g_0$.
\кзадача

\break








\задача Сколькими способами можно замостить
прямоугольник $3\times n$ плашками размера $2\times1$?
\кзадача








%\сзадача Сколько существует способов построить
%$2\times2\times n$-колонну из кирпичей размера $2\times1\times1$?
%\кзадача


\опр
Для произвольного числа $\alpha$ и натурального числа $k$
\выд{биномиальный коэффициент} $\displaystyle\binom{\alpha}k$
определяется формулой
$$
\displaystyle\binom{\alpha}k=\frac{\alpha(\alpha-1)(\alpha-2)\dots(\alpha-k+1)}{k!}.
$$
(Для $\alpha=n\in\N$ мы имеем $\displaystyle\binom{n}k=C_n^k$.)
Для каждого $\alpha$ рассмотрим следующий степенной ряд:
$$
\displaystyle(1+t)^\alpha=\sum_{k=0}^{\infty}\binom{\alpha}kt^k.
$$
(Для натуральных $\alpha$ это уже знакомая вам формула бинома
Ньютона, а для остальных $\alpha$ правая часть равенства является
определением левой.)
\копр








\задача
\пункт
Ряд $(1+t)^{-1}$ определяется теперь двумя способами:
как обратный к ряду $1+t$ и по биномиальной формуле.
Согласуются ли эти определения?
\пункт
Докажите, что для любого натурального числа $n$ имеет место
равенство $(1+t)^{-n}(1+t)^n=1$.
\кзадача

\ЛичныйКондуит{0mm}{6.5mm}


\end{document}









\задача
Пусть $c_0=1$, а при $n\geq1$ пусть
$c_n$ --- это число правильных расстановок $n$ открывающих и
$n$ закрывающих скобок ($n$-е число Каталана).
Напишите рекуррентную формулу, выражающую $c_n$
через $c_0,c_1,\ldots,c_{n-1}$, получите из неё уравнение на
производящую функцию $c(t)=c_0+c_1t+c_2 t^2+\ldots$
и, решив это уравнение, найдите явную формулу
для чисел Каталана.
\кзадача










\задача
Пусть $s_n$ --- число способов выплатить $n$ гривен купюрами
в 1 и 2 гривны. Является ли ряд
$(1+t+t^2+t^3+\ldots)(1+t^2+t^4+t^6+\ldots)$
производящей функцией для последовательности $(s_n)$?
\кзадача











\задача[Старинная задача]
%Примените производящие функции к решению такой старинной задачи:
Сколькими способами можно заплатить 1 рубль копейками, алтынами
(трёхкопеечными монетами) и пятаками (пятикопеечными монетами)?
\кзадача


\раздел{Разбиения чисел*}

\опр
\выд{Число разбиений} $p(n)$
--- это число всевозможных представлений $n$ в виде суммы
нескольких натуральных слагаемых\footnote[1]{
Разбиения, отличающиеся только порядком слагаемых,
считаются одинаковыми.}, то есть число всех
$n$-клеточных диаграмм Юнга; $p(0)\bydef1$.
\копр








\задача
Вычислите $p(n)$ для $n\le10$.
\кзадача








\задача
Пусть $P(t)=\sum\limits_{n=0}^{+\infty} p(n)t^n$ --- производящая
функция последовательности $(p(n))$.
Докажите, что
$$
%\sum\limits_{n=0}^{+\infty} p(n)t^n
P(t)=
(1+t+t^2+t^3+\ldots)\cdot(1+t^2+t^4+t^6+\ldots)
\cdot(1+t^3+t^6+t^9+\ldots)\cdot\ldots
$$
(проверив заодно, что это произведение
бесконечного числа рядов является формальной операцией).
\кзадача









%\задача
%Убедитесь, что корректно определено бесконечное
%произведение геометрических прогрессий
%Докажите, что
%$P(t)=(1-t)^{-1}\cdot(1-t^2)^{-1}\cdot(1-t^3)^{-1}\cdot\ldots$.
%\кзадача








\задача
Докажите, что
$1/P(t)=(1-t)\cdot(1-t^2)\cdot(1-t^3)\cdot\ldots$.
\кзадача








\задача Пусть $l(n)$ --- это число разбиений числа $n$ на нечётные
натуральные слагаемые,
и пусть $d(n)$ --- это число разбиений числа $n$ на различные
натуральные слагаемые$^1$. Докажите, что\\
\вСтрочку
\пункт
$\sum\limits_{n=0}^{+\infty}
l(n)t^n=\displaystyle\frac1{(1-t)\cdot(1-t^3)\cdot(1-t^5)\cdot\ldots}$;
\пункт
$\sum\limits_{n=0}^{+\infty}
d(n)t^n=(1+t)\cdot(1+t^2)\cdot(1+t^3)\cdot\ldots$.
\кзадача








\задача Докажите, что $l(n)=d(n)$ при всех натуральных $n$.
\кзадача








%\задача
%Докажите, что $1/P(t)=\prod\limits_{k\ge1}(1-t^{k})=
%1+\sum\limits_{n\ge1}\({\wht p}_{{\rm ч}}(n)-{\wht
%p}_{{\rm н}}(n)\) \cdot t^n$, где ${\wht p}_{{\rm ч}}(n)$ и
%${\wht p}_{{\rm н}}(n)$ --- это количества $n$-клеточных
%диаграмм Юнга, в которых длины всех строк различны,
%а общее число строк чётно и, соответственно, нечётно.
%\кзадача








\задача
Докажите, что
$\displaystyle\frac1{P(t)}=1+\sum\limits_{n\ge1}\(p_{{\rm ч}}(n)-
p_{{\rm н}}(n)\) \cdot t^n$, где $p_{{\rm ч}}(n)$ и $p_{{\rm н}}(n)$ ---
%это количества $n$-клеточных диаграмм Юнга, в которых длины строк различны,
%а общее число строк чётно и, соответственно, нечётно.
количества разбиений числа $n$ на чётное, и, соответственно, на нечётное
число различных натуральных слагаемых$^1$.
%(т.~е.~количества $n$-клеточных~ди\-а\-грамм Юнга, в которых длины строк
%различны, а общее число строк чётно и, соответственно, нечётно).
\кзадача








\задача
Рассмотрим множество $Y_n$ всех $n$-клеточных диаграмм
Юнга, строки которых имеют различные длины. Назовём
{\it торцом\/} такой диаграммы её правую верхнюю клетку и все
клетки, идущие от неё под углом $45^\circ$ по диагонали влево
вниз. Попробуем задать на $Y_n$ отображение, которое, в
зависимости от того, что у диаграммы длиннее --- торец или нижняя
строка, --- либо отрезает её торец и пририсовывет его новой
нижней строкой, либо, наоборот, отрезает нижнюю строку и
пририсовывет её новым торцом. Выясните, для каких диаграмм это
отображение не определено, и при каких $n$ оно устанавливает в
$Y_n$ взаимно однозначное соответствие между диаграммами из чётного
и из нечётного числа строк (а значит доказывает равенство
$p_{{\rm ч}}(n)=p_{{\rm н}}(n)$).
\кзадача








\задача [Пентагональная теорема Эйлера]
Докажите, что %$\displaystyle
$$\frac1{P(t)}=1+\sum\limits_{k\ge1}
(-1)^k\(t^{\(3k^2-k\)/2}+t^{\(3k^2+k\)/2}\)$$
и $p(n)=p(n-1)+p(n-2)-p(n-5)-p(n-7)+p(n-12)+p(n-15)-\,\cdots$.
\кзадача


\раздел{Дополнительные задачи.}








\задача
Для любого конечного множества $M$ положим
$C_{M}(t)\bydef\sum\limits_{k\ge0}c_{k}(M)t^{k}$, где $c_{k}(M)$ ---
это число всех $k$-эле\-мент\-ных подмножеств в $M$. Для двух
непересекающихся конечных множеств $A$ и $B$ выразите
$C_{{A\cup B}}(t)$ и $C_{{A\x B}}(t)$
%\footnote{{\it произведение\/} $A\x B$ множеств $A$ и
%$B$ представляет собой множество всех пар $(a,b)$ с $a\bl A$,
%$b\bl B$} $C_{{A\x B}}(t)$
через $C_{{A}}(t)$ и $C_{{B}}(t)$.
\кзадача

\опр
Для произвольного числа $\alpha$ и натурального числа $k$
\выд{биномиальный коэффициент} $\displaystyle\binom{\alpha}k$
определяется формулой
$$
\displaystyle\binom{\alpha}k=\frac{\alpha(\alpha-1)(\alpha-2)\dots(\alpha-k+1)}{k!}.
$$
(Для $\alpha=n\in\N$ мы имеем $\displaystyle\binom{n}k=C_n^k$.)
Для каждого $\alpha$ рассмотрим следующий степенной ряд:
$$
\displaystyle(1+t)^\alpha=\sum_{k=0}^{\infty}\binom{\alpha}kt^k.
$$
(Для натуральных $\alpha$ это уже знакомая вам формула бинома
Ньютона, а для остальных $\alpha$ правая часть равенства является
определением левой.)
\копр








\задача
\пункт
Ряд $(1+t)^{-1}$ определяется теперь двумя способами:
как обратный к ряду $1+t$ и по биномиальной формуле.
Согласуются ли эти определения?
\пункт
Докажите, что для любого натурального числа $n$ имеет место
равенство $(1+t)^{-n}(1+t)^n=1$.
\кзадача








\сзадача\пункт
Докажите, что для любого $\alpha$ и для любого натурального $n$
выполнено равенство $$(1+t)^\alpha(1+t)^n=(1+t)^{\alpha+n}.$$
\пункт Докажите,
что для любых $\alpha$ и $\beta$ выполнено равенство
$(1+t)^\alpha(1+t)^\beta=(1+t)^{\alpha+\beta}$.
\кзадача








\ссзадача[Нерешённая проблема]
Докажите или опровергните: если все коэффициенты ряда $G(z)$
равны либо 0, либо 1 и при этом все коэффициенты $\bigl(G(z)\bigr)^2$
меньше некоторой константы $M$, то бесконечно много из
коэффициентов $\bigl(G(z)\bigr)^2$  равны нулю.
\кзадача











\задача
  Является ли $P_m(t)\bydef\sum\limits_{n\ge0}p_m(n)\cdot t^n$\,,
  где $p_m(n)$ --- это число $n$-клеточных диаграмм Юнга из
  $\le m$ строк, рациональной функцией от $t$? Выразите $p_m(n)$
  через $p_{m-1}(n)$ и $p_m(n-m)$.
\кзадача








\задача[Гауссовы биномиальные коэффициенты]\label{gbk}
  Пусть $F_n(t)=(1-t)(1-t^2)\,\cdots\,(1-t^n)$.
  Является ли $G_n^k(t)\bydef
  F_n(t)/\(F_{k}(t)\cdot F_{n-k}(t)\)$ многочленом? Вычислите
  $\lim\limits_{t\to0}G_n^k(t)$.
\кзадача








\задача
  Выразите через гауссовы биномиальные коэффициенты производящую
  функцию $P_m^k(t)=\sum\limits_{n\ge0}p_m^k(n)\cdot t^n$ чисел
  $p_m^k(n)$ $n$-кле\-то\-ч\-ных диаграмм Юнга высоты $\le m$  и
  ширины $\le k$\,.
\кзадача 