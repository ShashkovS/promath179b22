% !TeX encoding = windows-1251
\documentclass[12pt,a4paper]{article}
\usepackage[mag=950, tikz]{newlistok}

\УвеличитьШирину{1.3cm}
\УвеличитьВысоту{2.3cm}
\renewcommand{\spacer}{\vspace{1.5pt}}

\ВключитьКолонтитул


\begin{document}


\Заголовок{Перечислительная комбинаторика -- 3}
\Оценки{32/27/22}
\НомерЛистка{16}
\ДатаЛистка{07.03 -- 21.03 /2018}
\СоздатьЗаголовок



%\раздел{Сочетания с повторениями и разбиения чисел}

%\раздел{Повторение}


\vspace*{-1mm}
\раздел{Шары и перегородки}

\vspace*{-1mm}
\задача
\пункт На полоске написано слово \лк снегопад\пк. Сколькими
способами её можно разрезать на 5 частей, если резать можно только между буквами?
\пункт Сколькими способами можно раздать 8 одинаковых орехов
5-ти разным детям так, чтобы каждый что-то получил?
\вСтрочку
\пункт А если можно давать орехи не всем?
\кзадача

\задача
Сколько букетов из пяти роз можно составить, если имеются розы трёх сортов?
\кзадача

%\задача
%Сколькими способами можно разложить десять одинаковых кусков
%сахара по пяти различным чашкам?
%\кзадача

\опр \выд{Числом сочетаний с повторениями из $n$ элементов по $k$}
называется число способов разложить $k$ одинаковых шаров по $n$
различным ящикам. Обозначение: $\overline C^k_n$.
\копр

\пзадача
\пункт Докажите, что
$\overline C^k_n=\overline C^k_{n-1}+\overline C^{k-1}_n$.
\пункт Найдите формулу для $\overline C^k_n$.
%=C_{n+k-1}^{n-1}$.
\кзадача

\пзадача Сколькими способами натуральное число $n$ можно
представить  как сумму
%\сНовойСтроки
\вСтрочку
\пункт $k$ натуральных;\\
\пункт $k$ неотрицательных целых;
\пункт нескольких натуральных слагаемых?
(Порядок слагаемых учитываем!)
\кзадача

\задача Автобусный билет называется \выд счастливым,  если сумма первых трёх
цифр его шестизначного номера равна сумме трёх последних цифр его номера.
%\сНовойСтроки
% \пункт Докажите, что счастливых билетов столько же, сколько номеров с суммой
% цифр~27.
\пункт Сколько имеется последовательностей из 6 неотрицательных целых чисел
с суммой~27?
\спункт Сколько существует счастливых билетов?
\кзадача

\vspace*{-1mm}
\раздел{Диаграммы Юнга}

\vspace*{-2mm}
\опр
Фигура типа $\sDY{6,5,5,3,1}$
(из выровненных по левому краю клетчатых горизонтальных полос, длина которых невозрастает сверху
вниз) называют {\it диаграммой Юнга\/}.
Число клеток в ней --- её {\it вес\/}.
\копр

\пзадача
Сколько существует диаграмм Юнга\quad
\вСтрочку
\пункт  веса 6;\quad
\пункт веса 7, имеющих не более 3 строк;\\
\пункт произвольного веса, но имеющих не более $p$ строк и не более $q$
 столбцов?
\кзадача

\задача
Имеются 4 различных чашки,
4 одинаковых стакана,
10 одинаковых кусков сахара
и 7 соломинок разного цвета.
Сколькими способами можно разложить:
\вСтрочку
\пункт соломинки по чашкам;
\пункт сахар по чашкам;
\пункт сахар по стаканам;
\спункт соломинки по стаканам.
\спункт  Как изменятся ответы в предыдущих пунктах, если потребовать,
чтобы после раскладывания пустых ёмкостей не оставалось?
\кзадача


\vspace*{-1mm}
\раздел{Повторение и разные задачи}

\vspace*{-1mm}
\пзадача
Коля и Гриша учатся в классе из 26 человек. Сколькими способами
можно выбрать из класса футбольную команду (11 человек) так, чтобы
Коля и Гриша не входили в команду одновременно?
\кзадача

\задача
После уроков 12 школьников хотят разделиться на две группы: одна пойдёт на футбол, а другая --- на дополнительное занятие по географии.
Сколькими способами они могут сделать такое разделение?
\кзадача

\пзадача
\пункт Сколько одночленов степени $d$ от $n$ переменных?
\пункт В $(a+b+c)^3$ раскрыли скобки. Сколько будет одночленов и с какими коэффициентами, если не приводить подобные?
А если привести?
%будут после приведения подобных?
\кзадача

\задача \пункт Сколькими способами из 15 разных цветков можно составить
3 букета: из 3, 5 и 7 цветков?
\пункт Сколько есть способов дать 11 разных цветков трём девушкам: какой-то --- 5, остальным ---~по~3?
\кзадача

\задача
На окружности отмечены 10 различных точек. Сколько можно провести незамкнутых
несамопересекающихся ломаных с вершинами во всех этих точках?
\кзадача

% \задача Сколькими способами можно расставить на шахматной доске
% две белые и две чёрные ладьи так, чтобы белые не били чёрных?
% (Ладьи одного цвета неразличимы. Вращать доску нельзя.)
% \кзадача

\пзадача
Сколькими способами можно переставить буквы в слове \texttt{НЕПОНИМАНИЕ} так, чтобы и гласные, и согласные (по отдельности) появлялись в алфавитном порядке?
\кзадача

%\задача
%В классе из 27 человек объявили о походе.
%Сколькими способами можно составить
%\кзадача



\задача  Между какими из множеств ниже имеется взаимно-однозначное соответствие? Постройте
их явно. Сколько элементов в каждом множестве?
\пункт горящие 12 лампочек на прямоугольном табло $5\times4$;
\пункт способы раскладывания 8 одинаковых кусков сахара по 13 разным чашкам;
\пункт слова, получающиеся при перестановке букв в слове
  {\tt аааааааабббббббббббб\/};
\пункт
пути по линиям сетки в клетчатом прямоугольнике $12\times8$ из левого нижнего угла в правый верхний;
%\ВосстановитьГраницы
\пункт
одночлены $a^8b^{12}$, получающиеся, если раскрыть скобки у
$(a+b)^{20}$ и не приводить подобные;
\пункт   диаграммы Юнга высоты $\le8$ и ширины $\le12$.
\кзадача


% \задача
%У королевы есть 12 одинаковых зеркал.
% Сколькими способами можно повесить в 8 разных залах 12 одинаковых зеркал, чтобы в каждом зале было хоть одно зеркало?
% \кзадача

\УстановитьГраницы{0cm}{6.5cm}
\задача На прямой отмечены $2n$ точек.
Разобьём их произвольно на пары, в каждой паре соединим точки дугой, как
показано на рисунке. Получившийся объект назовем \выд{дуговой
диаграммой}. Сколько существует различных дуговых диаграмм на этих точках?
\кзадача

\vspace*{-8truecm} \setbox5\vbox{ {\hsize 2.1truecm
\begin{picture}(250,200)
\linethickness{0.3mm}
 \put(0,0){\line(1,0){150}}
\put(0,0){\circle*{3.5}} \put(30,0){\circle*{3.5}}
\put(60,0){\circle*{3.5}} \put(90,0){\circle*{3.5}}
\put(120,0){\circle*{3.5}} \put(150,0){\circle*{3.5}}
\put(0,-20){{$A_{1}$}} \put(30,-20){{$A_{2}$}}
\put(60,-20){{$A_{3}$}} \put(90,-20){{$A_{4}$}}
\put(120,-20){{$A_{5}$}} \put(150,-20){{$A_{6}$}}
\qbezier(0,0)(45,25)(90,0) \qbezier(60,0)(90,25)(120,0)
\qbezier(30,0)(90,30)(150,0)
\end{picture}

}

}
\centerline{\hspace*{9.1cm}\copy5}


\vspace*{1.1truecm}
\ВосстановитьГраницы


\задача
Сколькими способами на трёхмерной доске $3\times3\times3$
можно расставить 9 одинаковых ладей так, чтобы они  не били друг друга?
(Ладья держит под боем свои строку, столбец и вертикаль.)
\кзадача

\сзадача
%Докажите, что любые два не крайних числа одной строки треугольника Паскаля не взаимно~просты.
Натуральные числа $k$ и $l$ оба меньше $n$. Докажите, что числа $C_n^k$ и $C_n^l$ не взаимно просты.
\кзадача

\vspace*{-1mm}
\ЛичныйКондуит{0mm}{5mm}

% \GenXML

\end{document}

% \задача
% В институте $N$ учёных, работающих в 500 отраслях науки. По каждой отрасли есть ровно 10 специалистов (один и тот же человек может быть специалистом в любом числе областей). Докажите, что можно разбить учёных на две группы, в каждой из которых будут специалисты по всем отраслям науки.
% \кзадача

%\задача
%\кзадача


%\раздел{Разные задачи}

%\задача
%Имеется сеть дорог (см.~рис. 1). Из вершины выходят $2^{100}$ человек.
%Половина идёт направо, половина --- налево. Дойдя до первого
%перекрёстка, каждая группа делится: половина идет направо,
%половина --- налево. Такое же разделение происходит на каждом
%перекрёстке. Сколько людей придёт в каждый из перекрёстков
%сотого ряда?
%\кзадача




%\раздел{Разбиения чисел}

% \сзадача
% Докажите, что число разбиений\footnote[2]{
% Разбиения, отличающиеся только порядком слагаемых,
% считаются одинаковыми.}
% натурального $n$
% на $k$ натуральных слагаемых равно числу разбиений $n$
% в сумму натуральных слагаемых, наибольшее из которых равно $k$.
% \кзадача
%
%
%
% \ссзадача Какие $n$ столькими же способами
% представимы$^2$ в виде суммы~чёт\-ного числа различных
% натуральных слагаемых,
% сколькими способами они представимы в виде суммы нечётного числа
% различных натуральных слагаемых? Что можно сказать об остальных $n$?
% \кзадача
%
% \сзадача
% Докажите, что число разбиений$^{2}$ натурального $n$
% на нечётные натуральные слагаемые равно числу разбиений $n$ на попарно
% различные натуральные слагаемые.
% \кзадача
%

\раздел{Метод траекторий и числа Каталана}

\задача Возле кассы собралось $n+m$ человек; $n$ из них
имеют по купюре 100 руб., а другие $m$ ---  по купюре
50 руб. Сначала в кассе нет денег, билет стоит 50 руб.
Сколько есть способов размещения всех покупателей
в очереди так, чтобы никто не ждал сдачи?
\кзадача

\задача [Метод траекторий]
Будем рассматривать на клетчатой плоскости пути
с началом и концом в узлах клеток,
состоящие из диагоналей клеток, где каждая диагональ
идёт либо вправо вверх, либо вправо вниз (если двигаться по пути
от начала к концу).
%Число диагоналей в пути называется его длиной.
\сНовойСтроки
\пункт Сколько существует путей, выходящих из начала координат, в которых
$m$ диагоналей идут вправо вверх, а $n$ диагоналей идут вправо вниз?
\пункт Сколько существует путей, соединяющих узел $(0,0)$ с узлом $(x,y)$
(где $x,y\geq0$)?
\пункт [Принцип отражения]
Узлы $A$ и $B$ лежат над осью абцисс, $B$ лежит правее $A$.
Докажите, что число путей, идущих из $A$ в $B$, которые касаются
оси абцисс или пересекают её, равно числу всех путей из $A'$
в $B$, где $A'$ --- узел, симметричный $A$ относительно оси абцисс.
\кзадача

\задача [Теорема о баллотировке] Кандидат $A$ собрал на выборах $a$ голосов,
кандидат $B$ собрал $b$ голосов~\hbox{$(a>b)$.}
%Избиратели голосовали последовательно.
Сколько существует способов последовательного подсчёта голосов, при
которых $A$ все время будет впереди $B$ по количеству голосов?
\кзадача

%\txt{Числа Каталана можно определить многоми разными способами.
%}

\сзадача %[Числа Каталана]
Докажите, что следующие величины совпадают с числами Каталана, и найдите их:\\
$\bullet$ Число путей из точки $(0,0)$ в точку $(n,n)$, идущих по линиям
клетчатой бумаги вверх и вправо, не поднимаясь
выше прямой $y=x$;\\ \\ \\ \\ \\
\rightpicture{-40mm}{20mm}{110mm}{cat-2}
\vspace*{1cm}
% \пункт
$\bullet$ Число способов соединить данные $2n$ точек на окружности $n$ непересекающимися хордами.\\ \\ \\
\rightpicture{-40mm}{20mm}{110mm}{cat-3}
$\bullet$
Число способов провести $2n$-звенную ломаную из
левого нижнего угла таблички $n\times2n$ в правый нижний угол.
(Ломаная не может выходить за границы таблички, каждое звено
ломаной --- диагональ клетки, идущая вправо вверх или вправо вниз,
если двигаться по ломаной слева направо.)
% Разберите два случая: когда ломаная может касаться нижней стороны
% таблички в точках, отличных от углов, и когда не может.
% \пункт
% На окружности отметили $2n$ точек.
% Сколькими способами их можно соединить $n$ непересекающимися хордами?
% \пункт Найдите явную формулу для последовательности $C_n$, заданной
% начальным условием\break
% $C_0=1$
% и рекуррентной формулой $C_n=C_0C_{n-1}+C_1C_{n-2}+\ldots+C_{n-1}C_0$
% (при $n\geq1$).
\кзадача



% \vspace*{-2mm}
%\ЛичныйКондуит{0mm}{6mm}
% \vspace*{-3mm}


\раздел{От Шашкова}


\задача
Докажите, что 1 — это единственное число треугольника Паскаля, которое встречается в нём бесконечно число раз.
\кзадача


\задача
Встречается ли в треугольнике Паскаля число $n=2^2\cdot2179\cdot6179$?
\кзадача


\задача
Сколькими способами можно переставить буквы в слове \texttt{НЕПОНИМАНИЕ} так, чтобы и гласные, и согласные (по отдельности) появлялись в алфавитном порядке?
\кзадача



\задача
Найдите коэффициент при $abcde$ после раскрытия скобок в $(a+b+c+d+e)^5$.
\кзадача


\задача
Пусть $n$, $k$, $l$ — натуральные числа.
Рассмотрим коэффициент перед $a^n\cdot b^k \cdot c^l$ после раскрытия скобок в $(a+b+c)^{n+k+l}$.
\пункт
Выразите его через несколько $C_{*}^{*}$;
\пункт
Вычислите его без $C_{*}^{*}$.
\кзадача


\задача
На клетчатой бумаге нарисован прямоугольник $200\times100$.
Сколько существует путей из левого нижнего угла в правый верхний таких, что путь
\\\пункт проходит через точки с координатами $(100,50)$ и $(190, 90)$;
\\\пункт не проходит ни через точку $(100,50)$, ни через точку $(190, 90)$?
\кзадача


\задача
В шкафу 5 красных и 7 синих носков. Каким числом способов 3 человека могут одеть носки?
(носки одного цвета считаются одинаковыми, однако левая нога отличается от правой)
\кзадача


\задача
Коля и Петя пришли в булочную. Там продаются булки 8 видов. Сколькими способами они
могут купить себе по две булочки?
\кзадача

\задача
\пункт
Докажите, что $n\cdot C_{n-1}^{k-1} = k\cdot  C_n^k$.
\пункт
Докажите это тождество комбинаторными методами, не используя явную формулу для $C_n^k$.
\кзадача

\задача
Мизеров, Распасной и Шестирной играют в преферанс.
Раздаются по 10 карт на руки и две в прикуп.
Какое количество раздач существует?
\кзадача

\задача
Докажите, что для любых натуральных $n$ и $k\le n$ выполнено неравенство: $C_n^k<C_{n+1}^k$.
\кзадача

\задача
Ведущий игры \лк Русское лото\пк Михаил Борисов достаёт 86 бочонков из 99, но он --- знатный жулик.
В тех случаях, когда число 49 --- год его рождения --- не выпало, он подменяет бочонок с минимальным номером на 49.
Во сколько раз меньше комбинаций становится от таких вот махинаций? (порядок бочонков не имеет значения)
\кзадача


\задача
Докажите, что для любых натуральных $n>1$ и $k<n$ выполнено неравенство: $C_{2n}^k<C_{2n}^n$.
\кзадача

\задача
В коробке имеется 100 различных бусин. Мы собираем из них бусы по 40 бусин (очевидно, что бусины по-прежнему различны).
Какое количество разных бус можно собрать?
\кзадача



\задача
Каким числом способов можно 20 школьников разбить на 5 групп по 4 человека?
\кзадача


\задача
На плоскости нарисовали $n$ различных прямых так, что никакие три не пересекаются в одной точке.
Сколько всего треугольников можно увидеть на картинке?
\кзадача


\задача
В дружном классе из $n=26$ человек $k=100$ пар подрались.
Докажите, что можно выбрать команду из 3 человек, которые друг с другом не дрались.
% (Троек всего $C_n^3=n(n-1)(n-2)/6$, каждая драка «убивает» не больше $k(n-2)$ троек.
% Значит, остались $n(n-1)(n-2)/6 - k(n-2)$ дружных троек.
% При этом $k<n(n-1)/6
\кзадача




\end{document}

\vspace*{-0.2truecm}

\раздел{$***$}

\vspace*{-0.2truecm}

\задача
%\пункт
В НИИ работают 67 человек. Из них
47 знают английский язык, 35 --- немецкий, и 23 --- оба языка.
Сколько человек в НИИ
не знают ни английского, ни немецкого языков?
\пункт Пусть кроме этого  польский
знают 20 человек, английский и польский --- 12, немецкий и
польский --- 11, все три языка --- 5.
Сколько человек не знают ни одного из этих языков?
%\спункт [Формула включений и исключений]
%Решите задачу в общем случае: имеется $m$ языков,
%и для каждого набора языков известно, сколько человек знают все языки
%из этого набора.
\кзадача

\задача
В ряд записали 105 единиц, поставив перед каждой знак \лк$+$\пк.
Сначала изменили знак на противоположный перед каждой третьей единицей,
затем --- перед каждой пятой, а затем --- перед
каждой седьмой. Найдите значение полученного выражения.
\кзадача

\задача \вСтрочку
\пункт
На полке стоят 10 книг. Сколькими способами их можно переставить
так,~чтобы ни одна книга не осталась на месте?
\пункт А если на месте должны остаться ровно 3 книги?
\кзадача
