\documentclass[a4paper,12pt]{article}
\usepackage[mag=1000]{newlistok}

\УвеличитьШирину{.3truecm}
%\УвеличитьВысоту{2.2truecm}
\renewcommand{\spacer}{\vspace{3pt}}

\Заголовок{Бескубные и бесквадратные последовательности}
\Подзаголовок{}
\НомерЛистка{10д}
\ДатаЛистка{01.2019}
\Оценки{9/7/5}

\begin{document}


\СоздатьЗаголовок


\опр
Начав с символа 0, будем заменять символы по правилу $0\rightarrow 01$,
$1\rightarrow10$. Получим последовательность слов:
$w_0=0,$ $w_1=01$, $w_2=0110$, $w_3=01101001$, $w_4=0110100110010110$, \ldots\\
Из задачи 1 следует, что можно построить бесконечное слово $w$, первые $2^n$ символов которого при каждом $n$ совпадают c $w_n$; оно называется {\em словом Туэ}.
\копр

\задача
По данному слову $w_n$ из задачи 1 построим слово $w'_n$, заменив все 0 на 1 и все 1 на 0.
Докажите, что при всех $n$ выполнено $w_{n+1}=w_nw'_{n}$.
% (поэтому можно построить бесконечное слово, первые $n$ символов которого прикаждом $n$ совпадают c $w_n$).
\кзадача

\задача
Для каждого целого $n\geq0$ определим $p_n$ так: $p_n=0$, если в двоичной записи числа $n$ чётное число единиц, и $p_n=1$, если нечётное. Докажите, что $w=p_0p_1p_2p_3\ldots$
\кзадача

\опр
Слово называется {\em бесквадратным}, если оно не содержит подслов вида $xx$,
и называется {\em бескубным} --- если не содержит подслов вида $xxx$ (где $x$ --- непустое слово).\\ Слово называется {\em сильно бескубным}, если в нём нет подслов вида $xx\alpha$, где $x$ --- непустое слово,\break $\alpha$ --- первая буква $x$. Иными словами, слово сильно бескубное, если в нём нет подслов вида $\alpha y\alpha y\alpha$, где $\alpha$ --- буква,
$y$ --- слово (возможно, пустое).
\копр

\задача
Докажите, что слово сильно бескубно тогда и только тогда, когда в нём нет подслов вида $\beta z z$, где $\beta$ --- последняя буква слова $z$.
\кзадача

\опр
Слово называется {\em словом без перекрытий}, если в нём нет <<перекрывающихся>> вхождений никакого слова, то есть нет подслова вида $xy=zx$, где слово $y$ короче слова $x$.
\копр

\задача
Слово свободно от перекрытий тогда и только тогда, когда оно сильно бескубно.
\кзадача

\задача [Теорема Туэ] Бесконечное слово Туэ (из определения 1) сильно бескубно.
\кзадача

\задача
Докажите, что слово Туэ непериодично.
\кзадача

\задача
Решите задачу 15д* листка 27, используя слово Туэ.
\кзадача


\опр
Определим новый алфавит из четырёх символов [00], [01], [10], [11]. Для любого слова $u=c_0c_1c_2\ldots$ из алфавита {0,1} определим слово $v=d_0d_1d_2\ldots$ по правилу $d_i=[c_ic_{i+1}]$.
\копр

\задача
Слово $u$ сильно бескубно тогда и только тогда, когда слово $v$ бесквадратно.
\кзадача

\задача
Пусть $v$ построено по слову Туэ $w$ из определения 1. Докажите, что тогда символу [00] всегда предшествует [10], а за [00] всегда идёт [01]; символу [11] всегда предшествует [01], а за [11] всегда идёт [10]. Докажите, что если заменить всюду в слове $v$ символы [00] и [11] на 1, символ [01] на 2 и символ [10] на 3, получим бесквадратное слово.
\кзадача


\ЛичныйКондуит{0mm}{6mm}



%\СделатьКондуит{6mm}{8mm}
%\GenXMLW

\end{document}
