\documentclass[a4paper,11pt]{article}
\usepackage[mag=1000]{newlistok}
\usepackage{tikz}
\usetikzlibrary{calc}

\УвеличитьШирину{1.2truecm}
\УвеличитьВысоту{2.5truecm}

\Заголовок{Последовательности: кормушки и ловушки}
\НомерЛистка{30}
\renewcommand{\spacer}{\vspace{1.5pt}}
\ДатаЛистка{26.01 -- 09.02.2019}
\Оценки{35/29/23}

\begin{document}

%\scalebox{.89}{\vbox{%
%\ncopy{1}{

\СоздатьЗаголовок

\опр Говорят, что задана \выд{последовательность} чисел
$x_1$; $x_2$; $x_3$; \dots ,  если каждому натуральному
числу $n$ поставлено в соответствие некоторое число $x_n$.
Другими словами, \выд{последовательность} ---
это произвольная числовая
функция, определ\"енная на множестве натуральных чисел.
Обозначение: $(x_n)$.  \копр

\задача
Есть ли последовательность, содержащая все \пункт рациональные; \спункт действительные числа?
\кзадача

\опр Последовательность $(x_n)$ называется \выд{ограниченной сверху},
если найд\"ется такое число $C$, что при всех натуральных $n$
будет выполнено неравенство $x_n<C$.
\копр

\пзадача
\пункт Дайте определение последовательности, ограниченной снизу.\\
\пункт Докажите, что %последовательность
$(x_n)$ \выд{ограничена}
(т.~е.~ограничена и сверху и снизу) тогда и только тогда, когда
найд\"ется такое число $C>0$, что при всех натуральных $n$
будет выполнено неравенство $|x_n|<C$.
\кзадача

%\опр Последовательность $\{x_n\}$ называется \выд{ограниченной},
%если найд\"ется такое число $C>0$, что при всех натуральных $n$
%будет выполнено неравенство $|x_n|<C$.
%\копр

\задача Найдите ограниченную последовательность, у которой\\
%\вСтрочку
\пункт  есть и наибольший, и наименьший члены;
\пункт  есть наибольший, но нет наименьшего;\\
\пункт  есть наименьший, но нет наибольшего;
\пункт  нет ни наименьшего, ни~\hbox{наибольшего.}
\кзадача


\задача Найти наибольший член последовательности:
%следующих последовательностей:
\вСтрочку
\пункт $\frac{n^2}{2^n}$;
\пункт $\frac{n}{100+n^2}$;
\пункт $\frac{1000^n}{n!}$;
\пункт $-n^2+5n-1$.
\кзадача

%\задача Найти наименьший член последовательности:
%следующих последовательностей:
%\вСтрочку
%\пункт $x_n=n+\dfrac{100}{n}$;
%\пункт $y_n=n+5\sin\dfrac{\pi n}2$.
%\кзадача

\пзадача
Перепишите, не используя отрицания:
\лк $(x_n)$ не  является ограниченной\пк.
\кзадача


\задача
Ограничена ли последовательность $(1+x)^n$, где $x>0$? %({\it Указание:} вспомните неравенство Бернулли.)
\кзадача



\пзадача При каких $q$ последовательность $x_n=1+q+q^2+\ldots+q^n$
ограничена?
\кзадача

\опр Сумма последовательностей $(x_n)$ и $(y_n)$ ---
%называется
последовательность $(z_n)$, где $z_n=x_n+y_n$
%при каждом натуральном
при всех $n\in\N$. Аналогично определяют %ся
разность,
произведение, отношение двух~\hbox{последовательностей.}
\копр

\пзадача Верно ли, что
\вСтрочку
\пункт сумма;
\пункт разность;
\пункт произведение;
\пункт отношение ограниченных
последовательностей --- тоже обязательно ограниченная последовательность?
\кзадача

\задача Любая ли последовательность есть отношение двух ограниченных последовательностей?
\кзадача


\опр
Интервал $(a;b)$ называется {\it ловушкой} для последовательности $(x_n)$, если $x_n\in (a;b)$ при $n\gg0$.\\
Интервал $(a;b)$ называется {\it кормушкой} для $(x_n)$, если $x_n\in (a;b)$ для бесконечного количества номеров $n$.\\
Определение кормушки и ловушки можно распространить и на отрезки, и на лучи (сделайте это!).
\копр

\задача
\пункт Докажите, что всякая ловушка для $(x_n)$ будет и кормушкой для $(x_n)$.\\
\пункт Приведите пример $(x_n)$ и кормушки для неё, которая не является ловушкой для $(x_n)$.
\кзадача


\задача
Пусть даны три отрезка, $U:\ [-0,5;0,5], \quad V: [-1;1], \quad W: [-2;2]$ \ \ и три последовательности\\
$$\alpha: \ 1;\ \frac12;\ \frac13;\ldots;\frac1n;\ldots,\quad
\beta:\  1;\ 2;\ \frac12;\ \frac32;\ \frac13;\ \frac43;\ldots;\frac1n;\ \frac{n+1}n;\ldots,\quad
\gamma:\ 1;\ \frac12;\ 3;\ \frac14;\ 5;\ \frac16;\ldots;2n-1;\ \frac1{2n};\ldots.
$$\\
Какие отрезки для каких последовательностей будут кормушками, какие для каких --- ловушками?
\кзадача

\пзадача
\пункт Докажите, что любые две ловушки для данной последовательности пересекаются\\
\пункт Верно ли, что если два интервала --- ловушки для $(x_n)$, то и их пересечение --- ловушка для $(x_n)$?
\кзадача

\сзадача
Найдётся ли последовательность, для которой любой интервал будет кормушкой?
\кзадача


\ввпзадача
Докажите, что у ограниченной последовательности есть кормушки сколь угодно малой длины.
\кзадача


\задача
Докажите, что последовательность не является ограниченной тогда и только тогда, когда \\
\пункт никакой интервал не будет для неё ловушкой;
\пункт можно выбрать хотя бы одно из двух направлений, что любой луч этого направления будет для неё кормушкой.
\кзадача

\задача
Докажите, что для последовательности $(x_n)$ из положительных чисел любой луч положительного направления будет ловушкой если и только если никакой интервал не будет для $(x_n)$ кормушкой.
\кзадача

% \задача
% Точка $a$ называется {\em предельной точкой последовательности $(x_n)$}, если любой интервал, содержащий точку $a$,
% будет кормушкой для $(x_n)$.
% \кзадача

\опр
Последовательность {\em фундаментальная}, если для неё есть сколь угодно короткие ловушки.
\копр

\задача
Верно ли, что
\пункт фундаментальная последовательность ограничена;
\пункт любая кормушка фундаментальной последовательности будет для неё и ловушкой;
\спункт ограниченная последовательность фундаментальна тогда и только тогда, когда любые две её кормушки, являющиеся отрезками, пересекаются.
\кзадача

\задача
\пункт Может ли быть фундаментальной сумма двух нефундаментальных последовательностей?\\
\пункт А произведение?
\пункт А сумма фундаментальной и нефундаментальной последовательностей?
\кзадача

\задача Верно ли, что фундаментальны
\пункт сумма двух фундаментальных;
\пункт произведение фундаментальной и ограниченной;
\пункт произведение двух фундаментальных последовательностей.
\кзадача



\ЛичныйКондуит{0mm}{6mm}
% \GenXMLW

\end{document}


\опр Последовательность $(x_n)$ называется \выд{бесконечно малой}, если
для каждого числа $\varepsilon>0$ найд\"ется такое число $N$,
что при любом натуральном $n\geq N$ будет выполнено неравенство $|x_n|<\varepsilon$.
%(другими словами, для любого числа $\varepsilon>0$ неравенство
%$|x_n|<\varepsilon$ будет выполнено при всех натуральных $n\gg0$).
\копр

\задача
Для %каждой из следующих
последовательности $(x_n)$ найдите
по данному $\varepsilon>0$ какое-нибудь $N$,
такое что при $n>N$ выполнено  неравенство
$|x_n|<\varepsilon$, если
\вСтрочку
\пункт $x_n = \frac1n$;
\пункт $x_n = \frac2{n^3}$;
%\пункт $x_n = \dfrac{\sin n}{n}$;
\пункт $x_n = \frac1{2n^2+n}$;
\пункт $x_n=(0,9)^n$;
\пункт $x_n=\frac1n+(0,9)^n$.
\кзадача

\задача %Известно, что %последовательности
Пусть $(x_n)$ и $(y_n)$
бесконечно малые. %Составим последовательность
Будет ли бесконечно малой последовательность $x_1,y_1,x_2,y_2,\dots$?
%Будет ли эта последовательность бесконечно малой?
\кзадача

\задача
Докажите, что сумма, разность и произведение бесконечно малых
последовательностей бесконечно малая.
\кзадача

\задача
Последовательность $(x_ny_n)$ бесконечно малая.
Верно ли, что одна из $(x_n)$, $(y_n)$
бесконечно малая?
\кзадача

\задача[Теорема о двух милиционерах] Последовательности $(x_n)$ и $(y_n)$
бесконечно малые, а последовательность
$(z_n)$ такова, что $x_n\leq z_n\leq y_n$, начиная с некоторого $n$. Докажите, что $(z_n)$ бесконечно малая.
\кзадача

\задача Является ли бесконечно малой последовательность
\вСтрочку
\пункт
$x_n=\frac{1-0,5^n}{n+7}$;
\пункт
$y_n=\frac{3^n+4^n}{2^n+5^n}$?
\кзадача

\задача
Можно ли в определении 4 заменить слова \лк каждого $\varepsilon>0$\пк\
на слова \лк каждого $\varepsilon$, где $1>\varepsilon>0$\пк?
\кзадача

\раздел{***}

\задача Дана последовательность $(x_n)$ с положительными членами.
Верно ли, что $(x_n)$ бесконечно малая
тогда и только тогда, когда последовательность $(\sqrt x_n)$
бесконечно малая?
\кзадача

\задача Даны две последовательности: $(x_n)$ --- бесконечно малая,
а $(y_n)$ --- ограниченная.
Докажите, что  $(x_n+y_n)$ --- ограниченная последовательность,
а $(x_ny_n)$ --- бесконечно малая последовательность.
\кзадача

\задача В бесконечно малой последовательности $(x_n)$
переставили члены (то есть взяли какое-то взаимно однозначное соответствие
$f:\N\rightarrow\N$ и получили новую последовательность $(y_n)$,
где $y_n=x_{f(n)}$ для всех $n\in\N$).
Обязательно ли полученная последовательность
будет бесконечно малой?
\кзадача



\задача Последовательность состоит из положительных
членов, причем сумма любого количества е\"е членов не превосходит 1.
Докажите, что эта последовательность бесконечно малая.
\кзадача




\задача
Дана бесконечная вправо и вниз таблица.
В каждой строчке записана бесконечно малая последовательность.
Пусть $x_n$ --- произведение верхних $n$ чисел $n$-го столбца.
Верно ли, что $(x_n)$ бесконечно малая?
\кзадача



\сзадача Любая ли последовательность есть отношение двух
\вСтрочку
\пункт ограниченных;
\пункт бесконечно малых? % последовательностей?
\кзадача


%\задача
%Есть ли последовательность, члены которой найдутся в любом
%интервале числовой оси?
%\кзадача


%\задача Какие из следующих последовательностей $(x_n)$
%являются бесконечно малыми:\\
%\вСтрочку
%\пункт
%$x_n=\dfrac{n^{100}}{7^n}$;
%\пункт
%$x_n=\dfrac{5^{100}}{n!}$;
%\пункт
%$x_n=\dfrac{(2n)!}{3^{n!}}$;
%$
%x_n=\frac{4\cdot7\cdot10\cdot\ldots\cdot(3n+1)}{2\cdot6\cdot10\cdot\ldots\cdot(4n+2)}?
%$
%\кзадача



%\ЛичныйКондуит{0mm}{6mm}

%\СделатьКондуит{4mm}{7.5mm}
%\GenXMLW

\end{document}

\опр Последовательность $(x_n)$ называется \выд{бесконечно большой}, если
для любого числа $C>0$ найд\"ется такое число $k$, что при всех
натуральных $n$, больших $k$,
будет верно неравенство
$|x_n|>C$.
\копр

\задача Последовательность не является ограниченной.
Обязательно ли она бесконечно большая?
\кзадача

\задача
Запишите без отрицания: %,
%что значит, что последовательность
$(x_n)$ не  является
\вСтрочку
\пункт бесконечно малой;
\пункт бесконечно большой.
\кзадача

\задача Какие из %следующих
последовательностей ограничены, какие
%являются
--- бесконечно малые, а какие --- бесконечно большие:
\вСтрочку %\\
\пункт $x_n = (1,1)^n$;
\пункт $y_n = (0,9)^n$;
\пункт $z_n = \sqrt{n^3+n} - \sqrt{n^3}$;
%\пункт $t_n=\sqrt[n]{n!}$;
\пункт $s_n=\dfrac{n^5+1}{n^4+n^2}$?
\кзадача


%\задача Последовательность $(x_n)$ неограничена. Верно ли, что
%последовательность $(1/x_n)$ бесконечно малая?
%\кзадача

\задача Верно ли, что последовательность $(x_n)$ с ненулевыми
членами %является
бесконечно малая тогда и только тогда, когда
последовательность $(1/x_n)$ %является
бесконечно большая?
\кзадача


\задача
Одна последовательность бесконечно большая,
а другая бесконечно малая.
Что можно сказать\\
\вСтрочку
\пункт о сумме;
\пункт об отношении;
\пункт о произведении этих последовательностей?
\кзадача

\задача
Для каждого натурального $n$ пусть $x_n$ --- сумма чисел вида $1/k$,
где $k$ --- натуральное,
$1\leq k\leq n$ и в десятичной записи числа $k$
нет цифры 9. Ограничена ли последовательность $(x_n)$?
%
%$x_n=a_1/1+a_2/2+\dots+a_n/n$,
%где $a_n=1$, если в десятичной записи числа $n$ нет цифры 9, и $a_n=0$
%иначе?
%Имеет ли эта последовательность предел?
\кзадача

\опр Число $a$ называют \выд{пределом последовательности} $(x_n)$,
если последовательность $(x_n-a)$ является бесконечно малой.
Обозначение: $\lim\limits_{n \to \infty} x_n = a$.
Говорят также, что $(x_n)$ стремится к $a$ при $n$,
стремящемся к бесконечности
(и пишут $x_n \to a$ при~\hbox{$n \to \infty$).}
\копр

\задача Докажите, что последовательность не может иметь более одного предела.
\кзадача


\задача Какие из следующих последовательностей
имеют пределы? Найдите эти пределы:
\\
\вСтрочку
\пункт $x_n=(-1)^n$;
\пункт $x_n=\dfrac{n^2+5n+7}{2n^2-n+1}$;
\пункт $x_n=\dfrac{2^n-1}{2^n+1}$;
\пункт $x_n=1+q+\dots+q^{n}$;
\пункт $x_n=\dfrac{C^{100}_n}{n^{100}}$.
\кзадача

\задача В два сосуда разлили 1 л воды. Из 1-го сосуда перелили
половину имеющейся в н\"ем воды
во 2-ой, затем из 2-го перелили половину оказавшейся
в н\"ем воды в 1-ый, затем из 1-го  пере\-ли\-ли половину
оказавшейся в н\"ем воды во 2-ой, и так далее. Докажите,
что независимо от того, сколько воды было сначала в каждом из сосудов,
после 100 переливаний в
них будет $2/3$~л и $1/3$~л воды с точностью до 1 миллилитра.
\кзадача

\vfill
\break

%\раздел{Часть 2.}

\опр
Последовательность $(x_n)$ называется \выд{монотонно
возрастающей}, если $x_{n+1}>x_n$ при любом натуральном $n$.
\копр

\задача
Дайте определение монотонно убывающей, монотонно невозрастающей
последовательностей.
\кзадача

\задача Исследуйте на монотонность последовательности из задачи 3 листка 11.
\кзадача

\опр Говорят, что $(y_n)$ --- \выд{подпоследовательность}
последовательности
$(x_n)$, если найд\"ется такая монотонно возрастающая последовательность
$(k_n)$ натуральных чисел, что $y_n=x_{k_n}$ при всех $n\in\N$.
\копр

\задача Является ли $(y_n)$ подпоследовательностью $(x_n)$, если
\сНовойСтроки
\пункт $x_n=n$, а $y_n=2n$;
\пункт $x_n=n$, а $y_1=2;\ y_2=1; y_n=n$ при $n\geq3$;
\пункт $x_n=(-1)^n$, а $(y_n)$ есть $1,\ -1,\ 1,\ 1, -1,\ 1,\ 1,\ 1, -1,
\ldots$?
\кзадача

\задача Последовательность обладает одним из свойств:
бесконечно мала, бесконечно велика, ограниченна, неограниченна, монотонна.
В каких случаях любая е\"е подпоследовательность обязательно
обладает тем же свойством?
\кзадача

\сзадача Верно ли, что любая последовательность имеет монотонную
подпоследовательность?
\кзадача

\опр Пусть $\varepsilon$ --- произвольное положительное число.
\выд{$\epsilon$-окрестностью} точки $a$ называется интервал
$(a-\epsilon, a+\epsilon) = \{ x: |x-a| < \epsilon \}$.
Обозначение: $U_\epsilon(a)$.
\копр

\задача хаусдорф
\кзадача


\опр Число $a$ называют \выд{пределом} последовательности
$(x_n)$,
если для~всякого $\epsilon>0$
в $\epsilon$-окрестности $a$ содержатся почти все члены последовательности
(т.~е.~все, кроме конечного числа).
\копр


\опр Число $a$ называют пределом последовательности $(x_n)$,
если для всякого $\epsilon >0$
найд\"ется такое натуральное число $N$, что при любом $k \geq N$ будет
выполнено неравенство $|x_k - a| < \epsilon$. \\
На языке кванторов:
$\forall \epsilon >0 \ \exists N\in \N \ : \ \forall n\in \N,
\ n \geq N \quad |x_n - a| < \epsilon.$
  \копр

\задача Докажите эквивалентность определений  5 и 6
(т.~е.~докажите, что
число $a$ является пределом последовательности $(x_n)$ в смысле
определения~5 тогда и только тогда, когда $a$ является пределом
последовательности $(x_n)$ в смысле определения~6).
\кзадача


\задача Докажите эквивалентность определения ... и определений
предела из листка 11.
\кзадача

\задача
Для каждой из следующих последовательностей $(x_n)$ найдите число $a$,
являющееся е\"е пределом, и для произвольного числа $\varepsilon>0$
укажите какой-нибудь номер $N$, начиная с которого выполняется
неравенство $|x_n-a|<\varepsilon$:
\вСтрочку\\
\пункт $x_n = \dfrac1n$;
\пункт $x_n = \dfrac2{n^3}$;
\пункт $x_n = \dfrac{\sin n}{n}$;
\пункт $x_n = \dfrac1{2n^2+n}$;
\пункт $\dfrac 12; 1; \dfrac 14; \dfrac 13; \dfrac 18; \dfrac 15;
\dfrac 1{16}; \dfrac17;\dots$.
\кзадача

\задача
\пункт Напишите, что значит, что число $a$ не  является
пределом последовательности $(x_n)$.
\пункт Напишите, что значит, что последовательность $(x_n)$ не имеет предела.
\кзадача

\задача Найдите пределы следующих последовательностей,
если они существуют (в пункте г) ответ будет зависеть от значений
$a$ и $k$, а в пункте д) --- от $q$):\\
\вСтрочку
\пункт $x_n =\sin\dfrac{\pi n}{2}$;\\
\пункт $x_n =\dfrac1n\sin\dfrac{\pi n}{2}$.
\пункт $\lim\limits_{n \to \infty} \dfrac {2^n} {n!}$;
\пункт $\lim\limits_{n \to \infty} \dfrac {a^n} {n^k}$, $a\in \R$, $k \in \N$;
\пункт $\lim\limits_{n \to \infty} q^n$, $q \in \R$.
\спункт $\lim\limits_{n \to \infty} \sqrt[n]{n}$.
\кзадача


\end{document}

\задача
Изобразите следующие последовательности:
\вСтрочку\\
\пункт $x_n = n$;
\пункт $x_n = 1$;
\пункт $x_n = (-1)^n$;
\пункт $x_n = \dfrac {(-1)^n}n$;
\пункт $1; \dfrac 12; 1; \dfrac 13; 1; \dfrac 14; \dots$;\\
\пункт $x_n=(n+1)/(2n+3)$;
\пункт $x_n=(n^2+1)/(2n+3)$;
\пункт $x_n=(n+1)/(2n^2+3)$.
\кзадача
