% !TEX encoding = Windows Cyrillic
\documentclass[a4paper,12pt]{article}
\usepackage[mag=990]{newlistok}
\usepackage{tikz}
\usetikzlibrary{calc}

\УвеличитьШирину{1.1cm}
\УвеличитьВысоту{2.5cm}



\Заголовок{Поля-2. Порождающие элементы}
\НомерЛистка{16д}
\renewcommand{\spacer}{\vspace{1.2pt}}
\ДатаЛистка{09.2020--10.2020}
\Оценки{21/17/13}

\begin{document}
	\СоздатьЗаголовок
	
	 Цель этого листка --- доказать следующую теорему.
	
\теорема\label{T:1} В любом конечном поле $F$ найдётся такой элемент $x$, что все ненулевые элементы $F$ имеют вид $x^n$, где $n \in \mathbb{N}$. \ктеорема

\опр Элементы $x$ из теоремы 1 называют {\it образующими} или {\it порождающими элементами}.\копр
{\footnotesize
Положим $p = \mathrm{char} F$, $q = |F|$. Множество ненулевых элементов поля обозначим через $F^*$. }

{\footnotesize Как и в полях $\mathbb{Q}$, $\mathbb{R}$, $\mathbb{C}$, в любом поле $F$ можно рассмотреть многочлены $F[x]$ над данным полем. Все соответствующие определения (сложение, умножение, деление, степень, корни, остатки, ....) дословно переносятся на $F[x]$. Многие свойства также сохраняются для многочленов над произвольным полем, в частности, свойства из задачи 1а)б).}

\задача \пункт Докажите, что элемент $a$ --- корень многочлена $f(x) \in F[x]$ если и только если $f(x)$ делится на $x-a$.

\пункт Докажите, что число различных корней многочлена $f(x) \in F[x]$ не больше $\deg f(x)$.

\пункт Докажите, что все элементы $F^*$ являются корнями многочлена $x^{q-1}-1$.

\пункт Пусть $d \mid (q-1)$. Докажите, что многочлен $x^d-1$ имеет ровно $d$ различных корней в $F^*$.

\кзадача

\опр Для элемента $a \in F^*$ обозначим через $d(a)$ {\it порядок} элемента $a$, то есть такое минимальное натуральное $k$, что $a^k=1$.

\задача\сНовойСтроки  Докажите, что \пункт $d(a)$ определён и не больше $q-1$. \пункт корни многочлена $x^k-1$ в $F^*$ --- это в точности элементы $F^*$, у которых порядок делит $k$.
\кзадача

\задача Найдите порядки элементов из $(\Z/p\Z)^*$ при $p=2,3,5,7$. Какие из этих элементов являются порождающими?
\кзадача

\задача Напомним, что функция Эйлера $\varphi(n)$ определяется, как количество обратимых остатков в $\Z/n\Z$. Докажите тождество: $$\sum_{d \mid n} \varphi(d) = n.$$

{\footnotesize \указание Приведите дроби $\frac{1}{n}$, $\ldots$, $\frac{n}{n}$ к несократимому виду. Сколько дробей будут иметь знаменатель $d$ (где $d \mid n$)?
\куказание}

\кзадача

\задача Обозначим через $\psi(d)$ количество элементов $F^*$ порядка $d$. Докажите, что \сНовойСтроки
\пункт При $k \mid (q-1)$ выполнено $\sum\limits_{d \mid k}\psi(d)=k$;
\пункт При $k \mid (q-1)$ выполнено $\psi(k) = \varphi(k)$.
\пункт Количество элементов порядка $q-1$ в $F^*$ равно $\varphi(q-1) \geq 1$. Выведите отсюда теорему \ref{T:1}.
\кзадача

\задача Проверьте, что указанное множество вычетов образует поле и найдите там порождающий элемент: \quad \пункт $\mathbb{F}_2 [x]/(x^2+x+1)\mathbb{F}_2 [x]$; \пункт $\mathbb{F}_2 [x]/(x^3+x+1)\mathbb{F}_2 [x]$; \пункт $\mathbb{F}_3 [x]/(x^2+x-1)\mathbb{F}_3 [x]$.
\кзадача


\задача Мы найдём сумму $\sum\limits_{x \in F^*} x^k$ разными способами. Пусть $a \in F^*$ --- порождающий элемент $F$.

\сНовойСтроки

\пункт Найдите сумму, используя тот факт, что умножение на $a$ является взаимно-однозначным соответствием $F^*$ на себя.
\пункт Найдите сумму, используя тот факт, что все элементы $F^*$  являются разными степенями элемента~$a$.
\пункт Как найти сумму, не используя Теорему 1?
\кзадача

Теорему \ref{T:1} можно попробовать обобщить с конечных полей на более общие структуры. К примеру, для каждого целого $m$ можно взять вычеты $\Z / m\Z$ по модулю $m$ и найти порождающий элемент в $(\Z/m\Z)^*$ (определение порождающего элемента дайте самостоятельно). Он также называется {\it первообразным корнем по модулю} $m$.

\задача \пункт Существует ли в $\Z / m\Z$ первообразный корень для $m=2,3,4,5,6,7$?\\
\пункт Существует ли такое $m$, что не существует первообразных корней в $\Z / m\Z$?
\кзадача

\задача Пусть $\varphi(m) = p_1^{\alpha_1} \cdot \ldots \cdot p_s^{\alpha_s}$ --- каноническое разложение числа $\varphi(m)$ на простые сомножители, $(g,m)=1$. В этом случае $g$ --- первообразный корень в $\Z/m\Z$ тогда и только тогда, когда $g$ не является решением ни одного из сравнений $g^{\frac{\varphi(m)}{p_k}} \equiv 1 \pmod{m}$ при $k=1,\ldots,s$.
\кзадача

\задача Найдите какой-нибудь первообразный корень по модулю \пункт $11$; \пункт $17$.
\кзадача

\задача Докажите критерий Вильсона (листок 23, задача 12), используя существование первообразных корней по простому модулю.
\кзадача

\ЛичныйКондуит{0mm}{5mm}
%\GenXMLW

\end{document}

\раздел{Альтернативное доказательство Теоремы 1}

\задача Пусть $m$ --- наименьшее общее кратное всех порядков элементов из $F^*$. Докажите, что
\пункт $m \leq q-1$;
\пункт $m = q-1$.
\кзадача

\задача \пункт Пусть $d(a) = n$. Докажите, что для любого $k \mid n$ найдётся элемент порядка $k$.
\пункт Пусть $d(b) = l$, $(n,l)=1$ Докажите, что $d(ab)=nl$.
\кзадача

\задача Рассмотрим элемент $a \in F^*$ максимального порядка.
\кзадача

\ЛичныйКондуит{0mm}{5mm}

\GenXMLW

\end{document}
	