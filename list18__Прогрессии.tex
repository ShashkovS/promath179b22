\documentclass[a4paper,11pt]{article}
\usepackage[mag=1000]{newlistok}

\УвеличитьШирину{1.3truecm}
\УвеличитьВысоту{2.5truecm}

\Заголовок{Прогрессии}
\НомерЛистка{18}
\renewcommand{\spacer}{\vfill}
\ДатаЛистка{21.04 -- 12.05/2018}
\Оценки{34/28/22}

\begin{document}

\СоздатьЗаголовок

%\vspace*{-1truemm}

%Последовательность --- это набор занумерованных чисел
%\опр Арифметической прогрессией называется после
%\копр

\опр {\it Арифметическая прогрессия\/}  --- это (конечная или бесконечная)
последовательность чисел $a_1,\,a_2,\,a_3,\,\ldots\,$,
в которой разность $d=a_k-a_{k-1}$ между соседними членами $a_k$ и $a_{k-1}$
одинакова для всех $k$; она называется {\it разностью\/} или {\it
приращением\/} прогрессии.
\копр

\пзадача
\пункт Выразите $n$-й член арифметической прогрессии через
первый член и разность.\\
%Найдите 50-е натуральное число, большее
%90, c остатком 3 от деления на~4.
\пункт Найдите 50-е натуральное число среди чисел, больших 90 и имеющих
остаток 3 при делении на 4.
\кзадача

\пзадача
\пункт Каждый член последовательности (кроме крайних, если
они есть) равен среднему арифметическому двух соседних членов.
%: $a_k=(a_{k-1}+a_{k+1})/2$.
Верно ли, что это %а последовательность ---
арифметическая прогрессия?
\пункт Верно ли обратное?
% утверждение?
\кзадача

\задача В некоторой арифметической прогрессии сумма первых
$n$ членов равна сумме первых $m$ членов (где $m<n$).
Докажите, что сумма первых $n+m$ членов этой прогрессии равна нулю.
\кзадача

%\задача Найдите 100-й член последовательности $a_n$, заданной условиями
%$a_1=1$, $a_2=4$, $a_{n}=2a_{n-1}-a_{n-2}$ при $n\geq3$.
%\кзадача


%\задача
%Будет ли арифметической прогрессией последовательность с $k$-м
%членом, равным:
%\вСтрочку
%\пункт
%$\underbrace{1\,1\,\ldots\,1}_{k}$\,,
%\пункт
%$k$-тому натуральному числу, оканчивающемуся на 13\,?
%\кзадача


%\задача
%Какие из перечисленных ниже свойств набора чисел $\OFAM a,n$
%необходимы, а какие --- достаточны для того, чтобы этот набор был
%арифметической прогрессией:
%\сНовойСтроки
%\пункт
%каждый элемент (кроме крайних) равен среднему арифметическому
%двух соседних:
%$a_k=(a_{k-1}+a_{k+1})/2$;
%\пункт
%$2\,a_i=a_{i-2}+a_{i+2}$ при всех $2\leq i\leq n-2$;\hfill
%\пункт
%сумма $a_i+a_{n-i}$ одна и та же для всех $0\leq i\leq n$?
%\кзадача

\ввпзадача
Выразите сумму всех членов конечной арифметической
прогрессии $a_1,\ \!a_2,\ \!\ldots,\ \!a_n$
через\\
%\сНовойСтроки
\вСтрочку
\пункт два крайних члена и число слагаемых;
\пункт начальный член, число слагаемых и приращение.
\кзадача

\задача
Найдите сумму всех трёхзначных чисел, оканчивающихся на 7.
\кзадача

%\задача
%Конечная арифметическая прогрессия состоит из целых
%чисел, и ее сумма --- степень двойки. Докажите, что
%количество членов прогрессии --- тоже степень двойки.
%\кзадача

\пзадача
По строкам и столбцам прямоугольной таблицы %размера
$m\times n$ стоят
арифметические прогрессии. Найдите сумму всех чисел в таблице,
если сумма четырёх угловых чисел равна~$S$.
\кзадача

%\задача
%%Как связаны арифметические прогрессии с $m$-угольными числами Диофанта?
%Что такое {\it $m$-угольные числа\/}?
%Чему равно $n$-тое $m$-угольное число?
%\кзадача

\задача Найдите арифметическую прогрессию, у которой при каждом натуральном $n$
сумма первых $n$ членов равна
\пункт $3n$; \пункт $n^2$; \пункт $n^2+n$; \пункт $2n^2-3n$.
\кзадача

\задача
Пусть $f(x)=ax^2+bx+c$. Докажите, что арифметическая прогрессия, сумма первых $n$ членов которой равна $f(n)$ при всех натуральных $n$ ,
\пункт существует при $c=0$;
%$(a_n)$, что $a_1+\ldots+a_n=f(n)$ при всех натуральных~$n$.
\пункт не существует при~$c\ne0$.
\кзадача

\задача
Фабрика выпускает наборы из $n>2$ белых слоников различной
величины и массы, стоящих по росту. По стандарту, разность масс
соседних слоников должна быть одной и той же. При каких $n$ контролер
гарантированно сможет это проверить с помощью чашечных весов без гирь?
\кзадача


\задача
Можно ли натуральный ряд покрыть $k$ арифметическими
прогрессиями с различными натуральными разностями, не равными 1,
если
\вСтрочку
\пункт $k=2$;
\пункт $k=3$;
\спункт $k=4$;
\спункт $k=5$?
\кзадача



%\задача
%Известно, что среди членов  некоторой арифметической прогрессии
%$a_1, a_2, a_3, a_4, \dots$ есть числа $a_1^2$, $a_2^2$ и $a_3^2$.
%Докажите, что эта прогрессия состоит из целых чисел.
%\кзадача

%\vspace*{-1pt}
\раздел{***}

\vspace*{-5pt}

\опр {\it Геометрическая прогрессия\/}  --- это (конечная или бесконечная)
последовательность ненулевых чисел $a_1,\,a_2,\,a_3,\,\ldots\,$,
в которой отношение $q=a_k/a_{k-1}$ соседних членов
одинаково для всех $k$; оно называется {\it знаменателем\/} прогрессии.
\копр


\пзадача
Будет ли геометрической прогрессией последовательность, $k$-й
член~\hbox{которой}~ра\-вен\\
\вСтрочку
\пункт
$0,\underbrace{0\ldots0}_{k}3$;
%\пункт
%  $3^{-k}$;
\пункт
 $\underbrace{1\ldots1}_{k}$;
\пункт
 $2^{3k+5}$;
%\пункт
% $\sqrt[k]3$;
\пункт
 $g_k\cdot h_k$, где $(g_k)$, $(h_k)$ --- геометрические~\hbox{прогрессии?}\\
\пункт
Выразите $n$-й член геометрической прогрессии через
первый член~и~\hbox{знаменатель.}
\кзадача

\пзадача
\пункт Квадрат каждого члена последовательности (кроме крайних,
если они есть) ненулевой и равен произведению двух соседних. %: $a_k^2=a_{k-1}\cdot a_{k+1}$,
%Верно ли,
%что это %последовательность ---
Геометрическая ли это прогрессия?
\пункт Верно ли обратное?
\кзадача

\задача Некто приезжает в город с новостью и
сообщает её двоим. Каждый из вновь узнавших новость через 5 минут
сообщает её ещё двоим (которые её не знают) и
т.~д.~(пока все в городе её не узнают).
Через сколько времени новость узнает весь город, если в нём
1\,000\,000 жителей?
\кзадача

\задача Торговец продавал одинаковые орехи.
Первый покупатель купил 1 орех,~\hbox{второй ---~2,}
третий --- 4, и т.\,д.: каждый следующий %покупатель
покупал вдвое больше орехов, чем предыдущий.
Орехи, купленные последним, весили 50 кг, после чего у торговца
остался 1 орех. Сколько орехов (по массе) было у него~вначале?
%(Все орехи одинаковые.)
\кзадача

\ввпзадача Найдите суммы:
\вСтрочку
%\пункт $1+3+3^2+\ldots+3^{10}$;
\пункт
$1+x+x^2+\ldots+x^n$;
\пункт
$1-\frac12+\frac14-\frac18+\ldots-\frac1{512}$.
%\кзадача
%
%\задача
\пункт Выразите сумму
всех членов конечной геометрической прогрессии через начальный член $a$,
число слагаемых $n$ и знаменатель~$q$.
\кзадача


\пзадача
По строкам и столбцам прямоугольной таблицы %размера
$m\times n$ стоят геометрические прогрессии.
Произведение четырёх угловых чисел равно $p$. Чему может равняться произведение всех чисел таблицы?
\кзадача

\сзадача
\пункт Будут ли все целые члены геометрической прогрессии образовывать геометрическую прогрессию?
\пункт Можно ли покрыть натуральный ряд конечным числом геометрических прогрессий?
\кзадача


%\vskip-5pt
%\раздел{***}
%
%\vskip-15pt

%\vspace*{-1pt}
\раздел{***}

\vspace*{-5pt}


\опр {\it Числа Фибоначчи} -- это члены последовательности
$f_0,f_1,f_2,\ldots,$
в которой $f_0=f_1=1$,\break а
%остальные вычисляются по формуле $u_n=u_{n-1}+u_{n-2}$ (при $n\geq3$).
каждый следующий член равен сумме двух предыдущих:
%:$u_1=u_2=1$,
$f_{n}=f_{n-1}+f_{n-2}$ при всех целых $n\geq2$.
\копр

% \задача Вычислите первые 15 чисел Фибоначчи.
% \кзадача


\пзадача Найдите все
\вСтрочку
\пункт
арифметические;
\пункт
геометрические прогрессии, у которых каждый член,
начиная с третьего, равен сумме двух предыдущих.
\кзадача


\задача
\пункт У двух последовательностей одинаковые первые члены и вторые члены, и каждый член, начиная с третьего, равен сумме двух предыдущих. Докажите,
что эти последовательности совпадают.\\
\пункт
%\пункт Является ли последовательность чисел Фибоначчи
%геометрической прогрессией?
Представьте последовательность Фибоначчи в виде
суммы двух геометрических прогрессий,
то есть найдите такие прогрессии $(g_n)$ и $(h_n)$, что $f_n=g_n+h_n$ при
всех целых $n\geq0$.  %Найдите формулу для чисел Фибоначчи.
\пункт Найдите $f_0+\ldots+f_n$.
\кзадача

%\vspace*{-1mm}
\ЛичныйКондуит{0mm}{5mm}
%\СделатьКондуит{6mm}{6.5mm}
% \GenXMLW


\end{document}

\сзадача
Даны две бесконечные вправо прогрессии: арифметическая
%$\alpha$
%$a_1,a_2,a_3\dots$
и геометрическая. %$\beta$.
%$b_1,b_2,b_3\dots$,
Известно, что все числа, которые встречаются среди членов геометрической
прогрессии, % $\beta$,
встречаются и среди членов арифметической прогрессии. % $\alpha$.
%причём вторая содержится в первой.
Докажите, что знаменатель геометрической прогрессии --- целое число.
\кзадача

\задача
\кзадача

\задача
\кзадача

\задача
\кзадача



\vspace*{-5pt}
\раздел{***}

\vspace*{-10pt}

%\noindent
%{\small {\Bf Суммирование разностей.}

\задача
Найдите суммы:\footnote{
Сумму $b_1+b_2+\ldots+b_n$ иногда удаётся вычислить,
представив каждое слагаемое в виде разности $b_i=c_{i+1}-c_i$ чисел
некого другого набора (тогда при подстановке в исходную сумму почти все $c_i$
сокращаются). }
\вСтрочку
\пункт
{\small
$\displaystyle{\frac1{1\cdot2}+\frac1{2\cdot3}+\,\cdots\,+\frac1{n\,(n+1)}}$;}
%\пункт
%$1\cdot2+2\cdot3+\ldots+n(n+1)$;
\пункт
%{\small
$\displaystyle{\frac1{a_1a_2}+\frac1{a_2a_3}+\,\cdots\,+\frac1{a_{n-1}a_n}}$
(где $a_1,\ldots,a_n$ --- арифметическая прогрессия с разностью $d$);\\
%\пункт
%$1\cdot1!+2\cdot2!+\ldots+n\cdot n!$ (где $k!=1\cdot2\cdot\ldots\cdot k$);
\пункт
{\small
$\displaystyle{\frac1{1\cdot2\cdot3}+\frac1{2\cdot3\cdot4}+\ldots
 +\frac1{n(n+1)(n+2)}}$;}
\пункт
{\small
$\displaystyle{\frac1{1+\sqrt2}+\frac1{\sqrt2+\sqrt3}+\ldots+\frac1{\sqrt{48}+\sqrt{49}}}$.}
%$1\cdot2\cdot3+2\cdot3\cdot4+\ldots+n(n+1)(n+2)$.
%{\small
%$\displaystyle{\frac1{1\cdot2\cdot3\cdot4}+\frac1{2\cdot3\cdot4\cdot5}+\,\cdots\,
% +\frac1{n\,(n+1)\,(n+2)\,(n+3)}}$.}
\кзадача


\задача
\пункт
Выразите $(n+1)^k$ и разность $(n+1)^k-n^k$ в виде многочленов от $n$
при $k=1,2,3,4,5$.
\пункт
Пользуясь п.~а), найдите сумму $k$-тых степеней первых
$n$ натуральных чисел для $k=0,1,2,3,4$.
\кзадача


\задача
Найдите суммы:
%\footnote{
%Сумму $b_1+b_2+\ldots+b_n$ иногда удаётся вычислить,
%представив каждое слагаемое в виде разности $b_i=c_{i+1}-c_i$ чисел
%некого другого набора (тогда при подстановке в исходную сумму почти все $c_i$
%сокращаются). }
\вСтрочку
\пункт
{\small
$\displaystyle{\frac1{1\cdot2}+\frac1{2\cdot3}+\,\cdots\,+\frac1{n\,(n+1)}}$;}
%%\пункт
%%$1\cdot2+2\cdot3+\ldots+n(n+1)$;
\пункт
{\small
$\displaystyle{\frac1{a_1a_2}+\frac1{a_2a_3}+\,\cdots\,+\frac1{a_{n-1}a_n}}$}
(где $a_1,\ldots,a_n$ --- арифметическая прогрессия с разностью $d$).
\кзадача
