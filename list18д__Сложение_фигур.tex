% !TEX encoding = Windows Cyrillic
\documentclass[a4paper,12pt]{article}
\usepackage[mag=1000]{newlistok}
\usepackage{tikz}
\usetikzlibrary{calc}

%\УвеличитьШирину{.6cm}
%\УвеличитьВысоту{1cm}



\Заголовок{Сложение фигур}
\НомерЛистка{18д}
\renewcommand{\spacer}{\vspace{1.2pt}}
\ДатаЛистка{март -- апрель, 2021}
\Оценки{30/23/16}

\begin{document}
	\СоздатьЗаголовок
	

\задача Даны два отрезка: $AB$ и $CD$. Найдите множество точек, в которые может попасть середина отрезка, один конец которого $P$ лежит на $AB$, а другой конец $Q$ -- на $CD$.
\кзадача

\опр Пусть заданы две фигуры $F$ и $G$ (два множества точек на плоскости или в пространстве). Назовём \выд{полусуммой} этих фигур множество всех середин отрезков, один конец которых принадлежит $F$, а другой -- $G$. Обозначим это множество так: $F*G$.
\копр

\задача Вычислите $F*G$ в случаях: \пункт $F$ и $G$ состоят из одной точки; \пункт $F$ -- отрезок,\break $G$~-- одна точка; \пункт $F$ и $G$ -- прямоугольники с параллельными соответственными сторонами.
\кзадача

\задача Докажите, что если $F$ и $G$ -- выпуклые фигуры, то и $F*G$ выпукла.
\кзадача

\задача Докажите, что количество сторон полусуммы $n$-угольника и $m$-угольника может быть любым числом от $\max(m,n)$ до $m+n$.
\кзадача

\задача Вычислите $F*G$ в случаях: \пункт двух непараллельно расположенных прямоугольников; \пункт правильного треугольника и его стороны; \пункт двух окружностей разного радиуса;\\ \пункт полуокружности с самой собой; \пункт полуокружностей, составляющих вместе окружность.
\кзадача

\задача Найдите полусумму фигур в пространстве: \пункт двух параллельно расположенных прямоугольных параллелепипедов; \пункт отрезка и многоугольника, не лежащих в параллельных плоскостях; \пункт окружности и шара; \пункт двух противоположных граней правильного октаэдра; \пункт двух половинок шара, разрезанного диаметральной плоскостью; \пункт двух скрещивающихся прямых.
\кзадача





\опр Зафиксируем некоторую точку $O$ (начало отсчёта). Множество всех концов $M$ векторов $OM=OP+OQ$, где $P$ и $Q$ -- произвольные точки фигур $F$ и $G$, называется \выд{суммой} (или \выд{суммой Минковского}) фигур $F$ и $G$. Сумма $F$ и $G$ обозначается $F+G$.
\копр

\опр Множество всех концов $M$ векторов $OM=\lambda OP$, где $P$ -- любая точка фигуры $F$, а $\lambda$ -- данное положительное число, называют \выд{произведением} $F$ на $\lambda$ и обозначают $\lambda F$.
\копр

\задача Как выразить $F*G$ через операции суммы и произведения?
\кзадача


\задача Докажите следующие свойства введённых операций:
 \begin{itemize}
  \item $F+G=G+F$;
  \item $(F+G)+H=F+(G+H)$;
  \item $\lambda (\mu F) = (\lambda \mu)F$;
  \item $\lambda(F+G)=\lambda F + \lambda G$;
  \item Если $F \subset G$, $F' \subset G'$, то $\lambda F + \mu G \subset \lambda F' + \mu G'$;
  \item $\lambda F + \mu F \supset (\lambda + \mu)F$, если $F$ выпукло, то имеет место равенство;
  \item $\lambda (F \cup G) = \lambda F \cup \lambda G$, $H+(F \cup G) = (H+F) \cup (H+G)$;
  \item $\lambda (F \cap G) = \lambda F \cap \lambda G$, $H+(F \cap G) \subset (H+F) \cap (H+G)$.
 \end{itemize}
 Значит ли это, что множество выпуклых фигур образует векторное пространство относительно операций суммы и произведения?
\кзадача

\задача Докажите, что если $F$ и $G$ -- выпуклые многоугольники, а $p()$ обозначает периметр, то $p(\lambda F + \mu G) = \lambda p(F)+\mu p(G)$.
\кзадача

\задача Докажите, что если $\lambda+\mu=1$, то множество $\lambda F+\mu G$ не зависит от выбора точки~$O$.
\кзадача

\задача Из точки $O$, лежащей на границе полуплоскости, внутрь полуплоскости направлено $n$ векторов длины $1$. Докажите, что если $n$ нечётно, то длина их суммы не меньше $1$.
\кзадача

\задача Пусть $F$ -- выпуклый многоугольник площади $S$ и периметра $p$, и пусть $K$ -- круг радиуса $1$ с центром в $O$. Докажите, что площадь фигуры $F+tK$ равна $S+tp+t^2\pi$.
\кзадача

\задача Докажите, что следующие свойства выпуклого многоугольника $F$ эквивалентны:
 \begin{itemize}
  \item $F$ имеет центр симметрии;
  \item $F$ можно разрезать на параллелограммы;
  \item $F$ есть сумма нескольких отрезков.
 \end{itemize}
\кзадача

\задача \пункт Докажите, что следующие свойства выпуклого многогранника $F$ эквивалентны:
 \begin{itemize}
  \item Все грани $F$ -- параллелограммы;
  \item $F$ есть сумма конечного набора отрезков, никакие три из которых не параллельны одной плоскости.
 \end{itemize}

\пункт Сколько граней может иметь такой многогранник, если число отрезков (во втором свойстве) равно $k$?
\кзадача

\задача От незагашенного окурка в одной точке загорелся лес. Ветер дул в течение времени $t_1$ со скоростью $v_1$, затем $t_2$ -- со скоростью $v_2$, $\ldots$, $t_n$ -- со скоростью $v_n$. Пожар распространяется от загоревшихся участков со скоростью ветра (причём эти участки продолжают гореть).\\
\пункт Какой участок выгорел за это время?\\ \пункт А если пожар, кроме того, распространяется равномерно по всем направлениями со скоростью $u$?
\кзадача

\задача[Теорема Брунна-Минковского] Для любых выпуклых фигур $F$ и $G$ и для любых положительных чисел $\lambda, \mu$ выполнено неравенство $$S(\lambda F + \mu G) \geqslant (\lambda \sqrt{S(F)}+\mu \sqrt{S(G)})^2.$$ Докажите эту теорему:\\ \пункт для прямоугольников с попарно параллельными сторонами;\\ \пункт для любых многоугольников площади $1$ при условии $\lambda+\mu=1$;\\ \пункт Для многоугольников любой площади и любых $\lambda,\mu$.\\ \пункт Для случая, когда одна из фигур является многоугольником, а другая -- кругом.
\кзадача

\задача Докажите \emph{изопериметрическое неравенство}: для любой выпуклой фигуры площади $S$ и периметра $p$ выполнено неравенство $S \leqslant \frac{p^2}{4\pi}$.
\кзадача

\задача Пусть $F$ и $G$ -- многоугольники. \\ \пункт Докажите формулу $$S(\lambda F + \mu G) = \lambda^2 S(F)+2\lambda \mu S(F,G)+\mu^2 S(G),$$ где число $S(F,G)$ не зависит от $\lambda$ и $\mu$. Его называют \выд{смешанной площадью} многоугольников $F$ и $G$.
\пункт Докажите неравенство $S(F,G)^2 \geqslant S(F)S(G)$. В каком случае имеет место равенство?
\кзадача



\ЛичныйКондуит{0mm}{5mm}

% \GenXMLW

\end{document}

