\documentclass[a4paper, 12pt]{article}
\usepackage{newlistok}

\УвеличитьШирину{1.3truecm}
\УвеличитьВысоту{2truecm}

\renewcommand{\spacer}{\vfill}


\begin{document}

\НомерЛистка{57}
\Заголовок{Число $e$}
\ДатаЛистка{30.11.2020 -- 11.12.2020}
\Оценки{14/10/6}
\СоздатьЗаголовок


\bigskip
\задача
Пусть $(x_n)$ --- последовательность положительных чисел, стремящаяся к $a$.
Докажите\footnote{\help{Домножение на сопряжённое будет крайне уместно и в этой задаче.}}, что тогда для любого $k\in\N$ существует предел последовательности $(\sqrt[k]{x_n})$,
равный $\sqrt[k]{a}$.
\кзадача

% \задача
% Докажите, что существует предел последовательности $\limn\hr{1+\dfrac1n}^n$.
% Число $e$ по определению равно пределу этой последовательности.
% \кзадача

\задача
Пусть $a,b>0$ и~$n\in\N$.
Докажите, что выполнено неравенство $\dfrac{a^{n+1}}{b^n}\geqslant(n+1)a-nb$.
\кзадача

\задача[о числе $e$]
\label{eee}
Докажите, что
\smallskip
\невСтрочку
\пункт
последовательность $e_n = \hr{1+\dfrac{1}n}^n$ монотонно возрастает;
\smallskip
\пункт
последовательность $E_n = \hr{1+\dfrac{1}{n}}^{n+1}$ монотонно убывает;
\smallskip
\пункт
$\limn e_n = \limn E_n$
(число $e$ по определению равно пределу этих последовательностей);
\smallskip
\пункт
выполнено неравенство $2{,}25<e<3{,}375$ (компьютер использовать нельзя);
\medskip
\пункт
Найдите такое $n$, что $|e - e_n| < 10^{-6}$ (компьютер использовать нельзя).
\кзадача

\задача[о числе $e^r$]
Докажите, что
\smallskip
\пункт
$\limn\hr{1-\dfrac1n}^n = \dfrac{1}{e}$;
\невСтрочку
\smallskip
\пункт
$\limn\hr{1+\dfrac{k}{n}}^n = e^k$, если число $k$ --- целое;
%\smallskip
\вСтрочку
\пункт
$\limn\hr{1+\dfrac{r}{n}}^n = e^r$, если число $r$ --- рационально;
\невСтрочку
\smallskip
\спункт
одна из последовательностей $\hr{1+\dfrac{r}{n}}^n$ и $\hr{1+\dfrac{r}{n}}^{n+r}$ монотонно возрастает,
другая --- монотонно убывает (начиная с какого-то момента), и пределы обеих последовательностей равны $e^r$.
\кзадача

\задача
Обозначим сумму $\bbr{ 1 + \dfrac{1}{1!} + \dfrac{1}{2!} + \ldots +  \dfrac{1}{n!}}$ через $s_n$,
а число $\hr{1+\dfrac{1}n}^n$ через $e_n$.
\невСтрочку
\smallskip
\пункт
Докажите, что для любого натурального $n$ выполнено неравенство $e_n \le s_n$;
\smallskip
\пункт
Зафиксируем натуральное число $N$ и рассмотрим любое натуральное $n>N$.
Раскроем скобки в выражении $\hr{1+\frac{1}n}^n$ по биному Ньютона
и оставим лишь первые $N+1$ слагаемых.
Докажите, что предел полученной таким образом последовательности равен $s_N$;
\smallskip
\пункт
Докажите, что $s_N \le e$;
\smallskip
\пункт
Докажите, что $\limn s_n = e$;
\smallskip
\спункт
Докажите, что
$\displaystyle\sum\limits_{i=m}^n \dfrac{1}{i!} \le \frac{1}{m!}\cdot\dfrac{1}{1-\frac{1}{m+1}}$;
\smallskip
\спункт
Найдётся ли $n < 100$ такое, что $|e - s_n| < 10^{-6}$? (Компьютер использовать нельзя.)
\smallskip
\спункт
Докажите, что для любого $r\in\Q$ существует предел
%\vspace{-2mm}
$\displaystyle
\limn \sumizn \dfrac{r^i}{i!} =
\limn \bbr{ 1 + \dfrac{r^1}{1!} + \ldots +  \dfrac{r^n}{n!}} = e^r
$.
\кзадача

\задача
Докажите, что $\limn n(e^{\frac{1}{n}} - 1) = 1$.
\help{Задача номер \ref{eee} здесь удивительным образом поможет}
\кзадача

\ЛичныйКондуит{0mm}{6mm}



\end{document}

\newpage
\УвеличитьВысоту{-1truecm}
\renewcommand{\spacer}{\vspace*{3mm}}
\ОбнулитьДанные
\НомерЛистка{20п}
\СоздатьЗаголовок

% \соглашение
% Арифметическим корнем $k$-ой степени ($k\in \N$) из положительного числа $a$ называется такое число $b$, что $b^k = a$.
% Обозначение: $a^{1/k}$ или $\sqrt[k]{a}$.
% Пусть $r = p/q$ --- рациональное число.
% Тогда $a^r$ по определению равно $(\sqrt[q]{a})^p$.
% В этом листочке мы считаем, что из любого положительного числа можно извлечь арифметический корень любой степени.
% Тем самым для любого положительного числа $a$ и рационального $r$ существует $a^r$.
% (Этот факт является следствием аксиом действительных чисел.)
% \ксоглашение

\bigskip
\задача
Пусть $(x_n)$ --- последовательность положительных чисел, стремящаяся к $a$.
Докажите, что тогда для любого $k\in\N$ существует предел последовательности $(\sqrt[k]{x_n})$,
равный $\sqrt[k]{a}$.
\кзадача



% \задача
% Докажите, что существует предел последовательности $\limn\hr{1+\dfrac1n}^n$.
% Число $e$ по определению равно пределу этой последовательности.
% \кзадача


\задача
Пусть $a,b>0$ и~$n\in\N$.
Докажите, что выполнено неравенство $\dfrac{a^{n+1}}{b^n}\geqslant(n+1)a-nb$.
\кзадача

\задача
\невСтрочку
\пунктн{3}
Докажите, что существует предел последовательности $\limn\hr{1+\dfrac1n}^n$.
Число $e$ по определению равно пределу этой последовательности.
\smallskip
\пунктн{4}
Докажите, что выполнено неравенство $2{,}4<e<3$;
\smallskip
\пункт
Найдите такое $n$, что $\hm{e - \hr{1+\dfrac1n}^n} < 10^{-9}$.
\кзадача

\задача
\label{e2e}
Докажите, что
\smallskip
\невСтрочку
\пунктн{3}
$\limn\hr{1+\dfrac{r}{n}}^n = e^r$, если число $r$ --- рационально;
\smallskip
\пункт
одна из последовательностей $e_n=\hr{1+\dfrac{r}{n}}^n$ и $E_n = \hr{1+\dfrac{r}{n}}^{n+r}$ монотонно возрастает,
другая --- монотонно убывает, и пределы обеих последовательностей равны $e^r$.
\кзадача

\задача
Обозначим сумму $\bbr{ 1 + \dfrac{1}{1!} + \dfrac{1}{2!} + \ldots +  \dfrac{1}{n!}}$ через $s_n$.
\невСтрочку
\smallskip
\пунктн{4}
Докажите, что $\limn s_n = e$;
\smallskip
\пунктн{6}
Найдётся ли $n < 100$ такое, что $|e - s_n| < 10^{-9}$?
\smallskip
\пункт
Докажите, что для любого $r\in\Q$ существует предел
\vspace{-4mm}
$\displaystyle
\limn \sumizn \dfrac{r^i}{i!} =
\limn \bbr{ 1 + \dfrac{r^1}{1!} + \ldots +  \dfrac{r^n}{n!}} = e^r
$.
\кзадача

\задача
Докажите, что $\limn n(e^{\frac{1}{n}} - 1) = 1$.
\кзадача

\ЛичныйКондуит{0mm}{6mm}



\end{document}

