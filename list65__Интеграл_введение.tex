% !TEX encoding = Windows Cyrillic
\documentclass[a4paper,12pt]{article}
\usepackage{newlistok}

\ВключитьКолонитул

\УвеличитьВысоту{2cm}
\УвеличитьШирину{1.5cm}
\renewcommand{\spacer}{\vfil}
\newcommand{\RpR}{{\cal R}([a,b])}
\newcommand{\intab}{\int\limits_a^b}

\Заголовок{Интеграл: введение}
\НомерЛистка{65}
\ДатаЛистка{30.04 -- 14.05.2021}
\Оценки{15/12/8}


\begin{document}

\СоздатьЗаголовок

\опр
Интегралом функции $f$ на отрезке $[a,b]$
называется площадь под графиком функции~$f$ на этом отрезке
(при этом на участках, где функция отрицательна, площадь
считается со знаком \лк минус\пк).
Обозначение: $\int\limits_{a}^{b} f(x)\,dx$.
\копр

Дайте своё строгое определение интеграла.
Оно должно работать для функций, непрерывных на отрезке: 
совпадать с «обычной» площадью, если её можно посчитать из геометрических соображений, и должны быть верны задачи \ref{firstMustHave}--\ref{lastMustHave}.

Будет здорово, если класс функций, интегрируемых по вашему определению функций, будет шире (кусочно-непрерывные, со счётным множеством точек разрыва, \dots).

\задача
Найдите:
\вСтрочку
\пункт
$\displaystyle\int\limits_{3}^{7} 5\,dx$;
\пункт
$\displaystyle\int\limits_{-1}^{2} x\,dx$;
\пункт
$\displaystyle\int\limits_{-2}^{1} |x|\,dx$;
\пункт
$\displaystyle\int\limits_{0}^{1} x^2\,dx$;
\пункт
$\displaystyle\int\limits_{0}^{1} (x^2-3x+1)\,dx$.
\кзадача


\задача
\label{firstMustHave}
Докажите, что
$$
\int\limits_a^b(f(x)+g(x))\,dx=
\int\limits_a^bf(x)\,dx+\int\limits_a^bg(x)\,dx.
$$
\кзадача

\задача
Пусть $c$ --- некоторое число.
Докажите, что
$$
\int\limits_a^bcf(x)\,dx  = c\int\limits_a^bf(x)\,dx.
$$
\кзадача

\задача Пусть  $a<b<c$. Докажите, что
$$
\int\limits_a^c f(x)\,dx=\int\limits_a^b f(x)\,dx+\int\limits_b^c f(x)\,dx.
$$
\кзадача


\задача Пусть при любом $x\in[a,b]$ выполнено $f(x)\le g(x)$.
Докажите, что
$$
\intab f(x)\,dx \le \intab g(x)\,dx.
$$
\кзадача

\задача
\пункт
У любой ли функции, определенной на $[a;b]$, есть интеграл на этом
отрезке?\\
\пункт
А если функция непрерывна?
\кзадача

\задача
\label{lastMustHave}
Пусть $F$ --- многочлен, $f$ --- его производная. Докажите,
что тогда\\
\вСтрочку
\пункт
$F(t)=\int\limits_0^tf(x)\, dx+F(0)$;
\пункт
$\int\limits_a^bf(x)\, dx=F(b)-F(a)$;
\кзадача

\задача
Пусть функция $f$ непрерывна (на $\R$). Зафиксируем точку $a$
и рассмотрим функцию $$F(t)=\int\limits_a^tf(x)\, dx$$ от переменной $t$.
\сНовойСтроки
\пункт
Докажите, что функция $F$ непрерывна.
\пункт
Верно ли, что функция $F$ дифференцируема?
Если верно, то найдите производную.
\кзадача

\задача
У каких функций, определенных на отрезке $[a;b]$, существует
интеграл на этом отрезке (по Вашему определению)?
Попробуйте найти как можно более широкий класс функций,
у которых интеграл существует (например, интегрируемы ли многочлены,
непрерывные функции, ... ?).
\кзадача

\ЛичныйКондуит{0mm}{8mm}

% \GenXMLW

\end{document} 