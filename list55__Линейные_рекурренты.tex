% !TeX encoding = windows-1251

\documentclass[a4paper,12pt]{article}
\usepackage{newlistok}

%\usepackage{mathtools}


\УвеличитьШирину{1truecm}
\УвеличитьВысоту{2truecm}

\Заголовок{Линейные рекурренты}
\НомерЛистка{55}
\ДатаЛистка{26.10.2020 -- 02.11.2020}
\Оценки{17/14/11}
\ВключитьКолонтитул

\begin{document}


	\СоздатьЗаголовок
	

\пзадача[Числа Фибоначчи] Пусть последовательности
$(a_n)$ и $(b_n)$ удовлетворяют соотношению $x_{n + 2} = x_{n + 1} + x_n$.
Докажите, что \\
\пункт при любом $\lambda\in\R$ последовательность
$\lambda (a_n) = (\lambda a_1,\lambda a_2,\dots)$
удовлетворяет этому соотношению; \\
\пункт при любых $\lambda,\mu \in\R$ последовательность $\lambda(a_n)
+ \mu(b_n) = (\lambda a_1 + \mu b_1,\lambda a_2 + \mu b_2,\dots)$
также удовлетворяет этому соотношению.\\
\пункт Пусть вектора $(a_1,a_2)$ и $(b_1,b_2)$ неколлинеарны.
Докажите, что любая последовательность, удовлетворяющая соотношению
пункта~а), представляется в виде $(\lambda a_n) +
(\mu b_n)$ при некоторых $\lambda,\mu \in \R$.\\
\пункт Найдите
геометрические прогрессии,  удовлетворяющие соотношению пункта~а).\\
\пункт Найдите явную формулу последовательности Фибоначчи с
начальными условиями $a_0 = a_1 = 1$. \кзадача

\пзадача Найдите явные формулы для следующих (вещественных)
последовательностей: \сНовойСтроки \пункт $a_{n + 3} = 2a_{n + 2} +
a_{n + 1} - 2a_n$, если $a_0 = 1$, $a_1 = 2$, $a_2 = 3$; \пункт
$a_{n + 2} = -2a_{n + 1} - a_n$, в случаях $a_0 = 1$, $a_1 = -1$;
$a_0 = 0$, $a_1 = -1$; $a_0 = 1$, $a_1 = 2$. \пункт $a_{n + 2} =
7a_{n + 1} - 12{,}5a_n$, если $a_0 = a_1 = 1$. \пункт $a_{n + 2} =
a_{n + 1} + a_n + 1$, если $a_0 = a_1 = 1$. \кзадача


 {\it Линейной рекуррентой} называется последовательность $a_n$, если для некоторого $k \in \N$
и для любого $n \geqslant 0$ выполнено {\it линейное рекуррентное уравнение}
$$a_{n+k} = a_{n+k-1}с_1+ \ldots + a_{n}с_k \mbox{ для всех } n \geqslant 0.$$

Другими словами, линейные рекурренты --- решения данного уравнения. Если $c_k \neq 0$, то $k$ --- {\it порядок} рекурренты.

Многочлен $x^k - с_1  x^{k-1} - \ldots - с_k=0$ называется \textit{характеристическим многочленом} данного рекуррентного уравнения.

\пзадача Докажите, что решения линейного рекуррентного уравнения образуют линейное пространство размерности $k$ над полем $\mathbb{C}$.
\кзадача

\пзадача Зафиксируем линейное рекуррентное уравнение 2-го порядка $a_{n+2} = \lambda a_{n+1} + \mu a_n$.

\пункт[Случай разных корней] Пусть характеристический многочлен имеет два различных (комплексных) корня $\alpha$, $\beta$. Докажите, что последовательности $a_n = \alpha^n$ и $b_n = \beta^n$ являются решениями данного уравнения, и более того, образуют базис пространства решений.

\пункт[Случай кратного корня] Пусть у характеристического  многочлена есть кратный (комплексный) корень $\alpha$. Докажите, что последовательности $a_n = \alpha^n$ и $b_n = n\alpha^n$ являются решениями данного уравнения, и более того, образуют базис пространства решений.
\кзадача

\пзадача Сколько существует последовательностей длины $n$ из символов \пункт $0$ и $1$; \пункт $0$, $1$ и $2$, в которых никакие два нуля не находятся рядом?
\кзадача

\пзадача Садовник, привив черенок редкого растения, оставляет его расти два года, а затем ежегодно берет от него по 6 черенков. С каждым новым черенком он поступает аналогично. Сколько будет растений и черенков на $n$-ом году роста первоначального растения?
\кзадача

\пзадача В вершине $A$ шестиугольника $ABCDEF$ сидит лягушка. Каждую секунду лягушка перепрыгивает в одну из соседних вершин, выбирая направление случайным образом равновероятно. С какой вероятностью она окажется в $A$ через $n$ прыжков?
 \кзадача

\пзадача Найдите количество $10$-значных чисел, удовлетворяющих следующим условиям:

\пункт все цифры числа принадлежат множеству 1, 2, 3, 4, 5, а любые две соседние цифры отличаются на 1;

\пункт никакие две чётные цифры не находятся рядом.

\кзадача



\сзадача Опишите все решения рекуррентного уравнения $k$-ого порядка для $k \geq 3$. \кзадача

\ЛичныйКондуит{0mm}{5mm}
% \GenXMLW

\end{document}

\bigskip

Выводим линейные рекуррентные соотношения. Линейным рекуррентным соотношением называется формула вида
$$a_n = p_1a_{n-1} + p_2a_{n-2}\ldots + p_ka_{n-k}$$
(и, дополнительно, первые $k$ членов $a_1$, $a_2$,\ldots, $a_k$).

Выведите рекуррентную формулу для следующих последовательностей:

1. $a_n = 2^n - 1$ (не сдаётся).

2. $a_n = 2^n + (-1)^n$.

3. $a_n = 2^n + 2n - 1$.

4. $a_n = 3n^2$.

5. $3n^2 + 2^n + 2n - 1$.

6. $a_{n+1} - a_n = n$.

7. $a_{n+1} - a_n = n^2$.

P.S. У Андрея есть два рублевых счета в банке, счет А и счет Б, на
каждом лежит по 128 рублей. Чтобы перевести деньги со счета на счет,
нужно написать заявление в банк, указать в нем сумму перевода (натуральное число рублей) и направление перевода (со счета A на счет Б, или
наоборот). После того, как все эти заявления были собраны и занесены
в базу данных, к их выполнению приступает некая программа. В начале
каждой секунды она находит все заявления, сумма перевода в которых
не превосходит количества денег на счете, с которого эта сумма должна
быть списана. Если такое заявление ровно одно, то в текущую секунду
программа совершает перевод согласно этому заявлению, само заявление из базы данных удаляет и затем переходит к следующей секунде.
В противном случае программа завершает свою работу. Андрей написал
заявление о переводе 64 рублей со счета А на счет Б и ещё несколько
заявлений. Какое наибольшее количество переводов можно совершить
при помощи этой программы?

Многочлен
$$x_k - (p_1x^{k-1} + p_2x^{k-2} + \ldots + p_k)$$

	
называется характеристическим многочленом рекуррентного соотношения $$a_n = p_1a_{n-1} + p_2a_{n-2}\ldots+ p_ka_{n-k}.$$

7. Пусть характеристический многочлен $x^2 -p_1x-p_2$ ( имеет два ненулевых различных корня, $\alpha$ и $\beta$.

а) [не сдаётся] проверьте, что $a_n = C_1\alpha^n + C_2\beta^n$, где $C_1$ и $C_2$ однозначно определяются по первым членам прогрессии $a_1$, $a_2$, является
явной формулой общего члена рекуррентного соотношения. б) выведите
формулу выше.

8. Пусть характеристический многочлен $x^2 - p_1x - p_2$ (имеет двойной
ненулевой корень.

а) [не сдаётся] проверьте, что $a_n = C_1\alpha_n +C_2n\alpha_n$, где
$C_1$ и $C_2$ однозначно определяются по первым членам прогрессии $a_1$, $a_2$,
является явной формулой общего члена рекуррентного соотношения.
б) выведите формулу выше.

9. Последовательность $a_n$, состоящая из натуральных чисел, такова,
что $a_1 > a_0$ и $a_{n+1} = 3a_n - 2a_{n-1}$. Докажите, что $a^{100} > 2^{100}$.

10. Решите систему рекуррентных соотношений: $a_1 = 2$, $b_1 = 1$, $a_{n+1} =
3a_n + b_n$, $b_{n+1} = -a_n + b_n$.

11. Пару кроликов поместили в загон. Сколько пар кроликов будет
через год, если считать, что каждый месяц пара дает в качестве приплода
новую пару кроликов, которые со второго месяца жизни также начинают
приносить приплод?

12. Сколько существует $n$-значных чисел, составленных из цифр 1,2,3
внутри которого не встречается числа 12 (т.е. цифры 2, идущей следом
за цифрой 1?


	
\end{document}


