\documentclass[a4paper,11pt]{article}
\usepackage{newlistok}


\newenvironment{напоминание}{\medskip\textbf{Напоминание.}}{\par}

\ВключитьКолонитул

\УвеличитьВысоту{2.4cm}
\УвеличитьШирину{1.5cm}
\renewcommand{\spacer}{\vfil}

\Заголовок{Группы преобразований}
\НомерЛистка{63}
\ДатаЛистка{29.03 -- 05.04.2021}
\Оценки{20/15/10}

\begin{document}
\СоздатьЗаголовок

{\small
\noindent\textbf{Напоминание}.
Отображение $\varphi\colon X\to Y$ из множества~$X$ в~множество~$Y$ называется \emph{взаимно однозначным} (или \emph{биекцией}), если для каждого элемента $y\in Y$ существует ровно один элемент~$x$ такой, что $\varphi(x)=y$.
\\
\emph{Преобразованием} множества~$X$ называется любая биекция множества $X$ в себя.
Множество всех преобразований — $S(X)$.
\\
Преобразование~$\psi\colon X\to X$ называется \emph{тождественным}, если оно переводит каждый элемент $x$ в себя.
Обозначение:~$\psi=\id_X$.
\\
Отображение $\varphi\colon X\to Y$ называется \emph{обратным} для отображения $\psi\colon Y\to X$, если справедливы равенства $\varphi\circ\psi=\id_Y$ и~$\psi\circ\varphi=\id_X$.
Обозначение:~$\varphi=\psi^{-1}$
\\
Количество элементов во множестве~$X$ обозначается через~$|X|$ или~$\#X$.
\par
}

\опр
\emph{Группой преобразований} множества~$X$ называется всякая непустая совокупность его преобразований~$G$, удовлетворяющая следующим свойствам:
\begin{items}{-5}
\item[(i)]
$G$~замкнута относительно композиции, то есть для всех $g,h\in G$ верно: $g\circ h\in G$;
\item[(ii)]
$G$~замкнута относительно взятия обратного преобразования, то есть для всех $g\in G$  преобразование~$g^{-1}$ лежит в~$G$.
\end{items}
\копр
\vspace*{-3mm}

\задача
Докажите, что группа преобразований любого множества содержит тождественное преобразование.
\кзадача



\задача
\label{triangle}
Пусть множество~$X$\т это правильный треугольник~$ABC$ (с внутренностью, точки $A$, $B$, $C$ идут по часовой стрелке).
Обозначим через~$s_{a}$, $s_{b}$ и $s_с$ симметрии относительно прямых,
содержащих соответствующие высоты исходного треугольника.
Далее, обозначим через~$r_0$, $r_1$ и $r_2$ повороты вокруг центра треугольника на $0^\circ$, $120^\circ$ и $240^\circ$ против часовой стрелки соответственно.
\невСтрочку
\пункт
Докажите, что $G=\{s_{a},s_{b},s_{c},r_0,r_1,r_2\}$ образует группу преобразований треугольника;
\ппункт
Выпишите таблицу <<умножения>> в этой группе (например, $s_b\circ s_a = r_1$);
\ппункт
Придумайте группу преобразований правильного треугольника, состоящую из трёх преобразований.
\кзадача


\задача
\label{sym}
\пункт
Докажите, что для любого множества~$X$ множество~$S(X)$ является группой;
\\\ппункт
Пусть $X$\т конечно, причём~$|X|=n$.
Найдите $|S(X)|$.
\кзадача
\замечание
Если множество $X$ конечно и состоит из $n$ элементов, то группа $S(X)$~называется \emph{симметрической группой} и~обозначается~$S_n$ (см. 47-й листок).
\кзамечание


\пзадача
Пусть множество~$X$ является подмножеством прямой, плоскости или пространства.
Рассмотрим множество преобразований $\Isom(X)=\{\varphi\in S(X)\mid \varphi\text{ сохраняет расстояния}\}$.
Докажите, что вне зависимости от~$X$ множество преобразований $\Isom(X)$ является группой.
Эта группа называется \emph{группой движений}~$X$.
\кзадача


\задача
\пункт
Докажите, что самый длинный отрезок, внутри треугольника — это одна из его сторон.
\\\пункт
Какой может быть группа движений треугольника (с внутренностью)?
\кзадача


\задача
Пусть в множестве $X\subset\R^2$ есть 3 точки, не лежащие на одной прямой.
Докажите, что для любого преобразования $\phi\in\Isom(X)$ можно найти ровно одно такое движение плоскости $\Phi\in\Isom(\R^2)$,
 что $\phi(x) = \Phi(x)$ для любого $x\in X$.
Говорят при этом, что движение $\phi$ \выд продолжается до движения всей плоскости.
\кзадача


\задача
\label{square}
\пункт
Опишите группу движений квадрата (то есть найдите и опишите все её элементы).
\\\ппункт
Придумайте две различных группы преобразований квадрата, состоящих из четырёх движений.
\\\спункт
Придумайте три таких группы и докажите, что других нет.
\кзадача




\опр
\emph{Порядком элемента}~$g$ группы преобразований~$G$ называется наименьшее натуральное~$k$ такое, что $g^k=\underbrace{g\circ\dots\circ g}_k=\id$. Обозначение: $\ord(g)$.
\копр

\опр
\emph{Порядком группы}~$G$ называется количество элементов в~$G$. Обозначение: $|G|$ или~$\#G$.
\копр


\задача
Найдите порядок каждого элемента групп из задач \ref{square} и~\ref{triangle}.
\кзадача

\пзадача
Докажите, что в конечной группе каждый элемент имеет конечный порядок.
\кзадача


\задача
\label{ord}
Перечислите все элементы и~их порядки в~группах движений следующих множеств:
\\\пункт
прямоугольник;
\\\ппункт
правильный $m$-угольник (эта группа называется \emph{группой диэдра} и~обозначается $D_m$);
\\\пункт
правильный тетраэдр;
\кзадача

\задача
Найдите порядок группы движений следующих множеств:\\
\пункт
куб;
\пункт
октаэдр;
\пункт
правильная $m$-угольная призма;
\пункт
икосаэдр;
\пункт
додекаэдр.
\\\пункт
Найдите все движения куба, которые не являются ни поворотом вокруг прямой, ни симметрией относительно плоскости;
\кзадача
\noindent\help{Как связаны между собой куб и~октаэдр? Тот же вопрос для икосаэдра и~додекаэдра.}


% \vfill
\ЛичныйКондуит{0mm}{6mm}
% \GenXMLW


\end{document}




