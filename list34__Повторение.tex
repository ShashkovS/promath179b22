\documentclass[a4paper,12pt]{article}
\usepackage[mag=1000]{newlistok}
\usepackage{tikz}
\usetikzlibrary{calc}

\ВключитьКолонтитул

\УвеличитьШирину{.5truecm}
%\УвеличитьВысоту{1truecm}

\Заголовок{Повторение}
\НомерЛистка{34}
% \renewcommand{\spacer}{\vspace{1.3pt}}
\ДатаЛистка{15.05 -- 29.05.2019}
% 54 задач
\Оценки{39/29/19}


\begin{document}


\СоздатьЗаголовок

\smallskip

\задача
У каждого целого числа от $n+1$ до $2n$ включительно
(где $n$ --- натуральное) возьмем наибольший нечетный
делитель и сложим все эти делители. Докажите, что получится $n^2$.
\кзадача

\задача
 Сколькими способами можно расставить на полке двухтомник Марка Твена,
трёхтомник Николая Гоголя и десятитомник Александра Пушкина, чтобы
\пункт <<свои стояли со своими>>, но не обязательно по порядку?
\пункт тома каждого автора встречались на полке в порядке возрастания, но не обязательно подряд?
\кзадача

% \задача
% \пункт Сколько разных слов (не обязательно осмысленных) можно получить, переставляя буквы в слове
% {\em аппарат}?
% \пункт Сколькими способами 10 детей можно разбить на пары?
% \кзадача


\задача
На карусели 7 двухместных верблюдов. Сколькими способами можно рассадить на неё 14 детей?
\кзадача

\задача
Дан квадрат $100\times100$ клеток. Сколько есть прямоугольников, чьи вершины лежат в центрах этих клеток?
\кзадача


\задача
Докажите, что
$C_a^0\cdot C_b^m+C_a^1\cdot C_b^{m-1}+\ldots+C_a^{m-1}\cdot C_b^1
+C_a^m\cdot C_b^0=C_{a+b}^m$.
\кзадача

\задача
Пусть для любого разбиения множества вершин графа на две
непересекающихся группы существует ребро, соединяющее вершины их
разных групп. Докажите, что тогда граф связен.
\кзадача


\задача
Докажите, что в связном графе любые два длиннейших несамопересекающихся по
ребру пути имеют общую вершину.
\кзадача


\сзадача
В стране между некоторыми парами городов осуществляются двусторонние беспосадочные авиарейсы. Известно, что из
любого города в любой другой можно долететь, совершив не более 100 перелетов. Кроме того, из любого города в любой другой можно долететь, совершив четное число перелетов. При каком
наименьшем натуральном $d$ из любого города можно гарантированно долететь в любой другой, совершив четное число перелетов,
не превосходящее $d$?
(Разрешается посещать один и тот же город или совершать
один и тот же перелет более одного раза.)
\кзадача

\задача Количество членов геометрической прогрессии чётно.
Сумма всех её членов в 3 раза больше суммы членов, стоящих
на нечётных местах. Найдите знаменатель прогрессии.
\кзадача

\задача В некоторой арифметической прогрессии $a_m=-a_n$
для каких-то натуральных $m$ и $n$, где $m<n$. При каких $m$ и $n$
эта прогрессия обязательно содержит нуль и под каким номером?
\кзадача

% \задача
% В таблицу $m\times n$ написаны числа так, что в каждой строке и в каждом столбце получилась геометрическая прогрессия. Произведение четырёх угловых чисел равно $p$. Можно ли однозначно восстановить \пункт произведение всех чисел таблицы; \пункт модуль этого произведения?
% \кзадача


\задача
Для натуральных $m$ и $n$ пусть $S_m(n)=1^m+2^m+\ldots+n^m$. Докажите, что\\
\пункт для любых натуральных $i$ и $k$ выполнено $(i+1)^{k+1}-i^{k+1}=
i^k\cdot C^1_{k+1}+i^{k-1}\cdot C^2_{k+1}+\ldots+i\cdot
C^k_{k+1}+1$;\\
\пункт
$(n+1)^{k+1}-1=
S_k(n)\cdot C^1_{k+1}+S_{k-1}(n)\cdot C^2_{k+1}+
\ldots+S_1(n)\cdot C^k_{k+1}+n$ (используйте пункт а!).\\
\пункт
Как по формулам для $S_1(n),\ldots, S_{k-1}(n)$ найти
формулу для $S_k(n)$?\\
\пункт Найдите $S_2(n)$, $S_3(n)$ и $S_4(n)$.\\
\пункт
Докажите, что $S_k(n)$ --- многочлен %степени $k+1$
от $n$. Найдите его степень, старший и свободный члены.
\кзадача

% \задача
% На какую цифру оканчивается число $7^{1000}$?
% \кзадача

\задача
При каких целых $n$ число $(n^2 - n + 1)/(n - 2)$ целое?
\кзадача

\задача
Решите уравнение $x^2+y^2=3z^2$ в целых числах.
\кзадача

% \задача
% Докажите, что для любых целых чисел $ab$ выполнено равенство  $ab=\text{НОK}(a,b)\cdot\text{НОД}(a,b)$.
% \кзадача

% \задача
% Докажите, что простых чисел вида $3k+1$ бесконечно много
% ($k$ --- натуральное).
% \кзадача

\задача
Пусть натуральные числа $a$ и $b$ взаимно просты. Докажите, что любое $k\in \N$, начиная с некоторого, представимо в виде линейной комбинации
$k=am+bn$, где $m,n\in \N$.
\кзадача


\задача
Равномощно ли отрезку множество точек $[0;1]\cup[2;3]\cup[4;5]\cup\ldots$?
\кзадача

\задача
Докажите, что множество точек любого треугольника
(с внутренностью) на плоскости равномощно множеству точек
любого прямоугольника (с внутренностью) на плоскости.
\кзадача




\задача
Равномощны ли множество всевозможных бесконечных последовательностей
целых чисел и множество всевозможных возрастающих
бесконечных последовательностей целых чисел?
\кзадача

\ЛичныйКондуит{0mm}{6mm}
\ОбнулитьКондуит
\newpage

\задача
Найдите коэффициент при $x^{179}$ у многочлена $(1-x+x^4)^{60}(1+x+x^4)^{60}$.
\кзадача

\задача
Найдите такой многочлен $P$ степени 3, что выполнены равенства $P(1)=1$, $P(2)=5$, $P(3)=0$, $P(4)+P(5)=8$.
\кзадача

\задача
Найдите все такие многочлены $P,$ что выполнено тождество $xP(x-1)=(x-10)P(x).$
\кзадача



\задача
Про последовательность $(a_n)$ известно, что
$\exists a\, \exists \varepsilon > 0 \, \exists N \in \N \, : \, \forall n > N \, |a_n - a| < \varepsilon$.
Обязательно ли эта последовательность ограничена? Обязательно ли она имеет предел?
%\пункт Те же вопросы о последовательности, для которой $\exists a\, \forall \varepsilon > 0 \, \forall N \in \N \, : \, \exists n > N \, |a_n - a| < \varepsilon.$
\кзадача

\задача
Рассмотрим условие
$\forall \varepsilon >0 \ \exists m\in\N \
\forall n\ge m\ : \ |x_n-a|>\varepsilon$. Эквивалентно ли оно
условию:
\пункт $a$ не является пределом $\{x_n\}$;
\пункт $\{x_n\}$ неограничена;
\пункт $\{x_n\}$ не имеет кормушек.
\кзадача


\задача Найдите такое $n \in \N$, что $\left(\dfrac{4}{5}\right)^n < 10^{-179}$.
\кзадача

\задача Найдите \пункт $\lim\limits_{n\to \infty} \frac{1+2+\ldots+n}{n+2} - \frac{n}{2}$;
\пункт $\limn(n - \sqrt{n^2 + 3n})$;
\пункт $\lim\limits_{n\to \infty}n(\sqrt{n^2-1} - n)$;\\
%\пункт $\limn \frac{1+3+\ldots+3^n}{5^n}$;
\пункт $\limn \frac{n^3 + 1}{2n^3 - n^2 - 7}$;
\пункт $\lim\limits_{n\rightarrow+\infty}\root n \of{1^n+2^n+\ldots+9^n}$;
\пункт $\limn\left(1+\frac{1}{n^2}\right)^n$.
\кзадача



\задача Последовательности $(a_n)$ и $(a_n\cdot b_n)$ имеют предел. Обязательно ли $(b_n)$ имеет предел?
\кзадача




\задача Верно ли, что для последовательности $(a_n)$ с ненулевыми членами условие <<в каждом отрезке находится конечное число элементов $(a_n)$>>
равносильно условию <<$(\frac{1}{a_n})$ бесконечно малая>>?
\кзадача



\задача
Найдите такое $N$, что при всех целых $k>N$
верно неравенство $100\cdot k^5+k^3+1000< k^6$.
\кзадача

\задача Что больше при $k \gg 0$: $100 \cdot k! + 200^k$ или $k^k + 2^k?$
\кзадача

% \сзадача
% Последовательность $(a_n)$ состоит из
% ненулевых членов и имеет предел 0. Известно, что
% последовательность $(\frac{a_{n+1}}{a_n})$ имеет предел
% $\alpha$. Найдите все возможные значения $\alpha$.
% \кзадача

% \задача
% Придумайте ограниченную последовательность без наименьшего и наибольшего членов.
% \кзадача




\задача
Про последовательность $(x_n)$ известно, что она имеет предел и содержит бесконечно много положительных и бесконечно много отрицательных членов. Докажите, что $(x_n)$ бесконечно малая.
\кзадача
%

% \задача
% Про последовательность $(x_n)$ известно, что она имеет предел и содержит бесконечно членов, квадрат которых больше 2, и бесконечно много членов, квадрат которых меньше двух. Докажите, что~$\br{\limn x_n}^2 = 2$.
% \кзадача



% \задача
% Пусть $\limn x_n = 1$. Найдите $\limn\sqrt{x_n}$.
% \кзадача


\задача
Найдите предел последовательности $\limn\dfrac{F_n}{3^n}$, где $F_n$ --- $n$-е число Фибоначчи.
\кзадача


\задача
Пусть $\limn x_n = 16$. Найдите $\limn\sqrt[4]{x_n}$.
\кзадача


\задача
Найдите предел $(x_n)$, если
\пункт $x_1=0$ и $x_{n+1}=\frac{x_n+3}4$;
\пункт $x_1=a$, $x_2=b$, $x_{n+2}=\frac{x_{n+1}+x_n}2$ при $n\in\N$.
\кзадача



\задача
Рассмотрим функцию $f \from \N \times \N \to \R$. Известно, что существуют пределы\break $\limn\left( \lim\limits_{m \to \infty} f(m, n)\right)$ и $\lim\limits_{m \to \infty}\left( \limn f(m, n)\right)$. Верно ли, что эти пределы обязательно совпадают?
\кзадача


\задача
Число $x\in(0;1)$ назов\"ем \выд{вычислимым}, если есть
конечное правило (данное в математических знаках или
словесное), которое позволяет для каждого $n\in\N$
определить $n$-ый знак после запятой в десятичной записи~$x$.
%\сНовойСтроки
\вСтрочку
\пункт Докажите, что множество вычислимых чисел из интервала $(0;1)$ сч\"етно.
\пункт %Занумеруем вычислимые числа,
Выпишем десятичные записи всех вычислимых чисел
в таблицу, и диагональным методом построим вычислимое число, не входящее в таблицу.
%Мы это число задали конечным словесным правилом, и в то же время его нельзя
%так задать.
%конечным словесным правилом.
Объясните это противоречие.
\кзадача


\задача
Решите в натуральных числах уравнения:\\
\вСтрочку
\пункт
$n(n+1)=m(m+2)$;
\пункт
$a!+b!+c!=d!$;
\пункт
$\frac1a+\frac1b=1$;
%\пункт $x+y+z=xyz$;
\пункт
$x^2+3x=y^2$.
%\пункт $xy+yz+xz=xyz+2$.
\кзадача

\задача
Можно ли уместить два точных куба между соседними точными квадратами?
\кзадача



\сзадача Найдите %предел последовательности $(a_n)$, если\\
%\вСтрочку
%\пункт
%$\displaystyle{\lim\limits_{n\to\infty}
%\left(\frac{1^k+2^k+\ldots+n^k}{n^k}-\frac{n}{k+1}\right)}$,
%где $k\in\N$;
%\пункт
$\displaystyle{\lim\limits_{n\to\infty}\frac{1^1+2^2+\ldots+n^n}{n^n}}$.
\кзадача

\ЛичныйКондуит{0mm}{6mm}

% \GenXMLW

\end{document}


\begin{center}
{\bf Задачи, которые требуют дополнительных знаний}
\end{center}


\задача
Последовательность чисел $(a_n)$ задана условиями: $a_1 = 1$ и $a_{n+1} = a_n + \dfrac1{a_n^{2013}}$.
Верно ли, что эта последовательность имеет предел?
\кзадача

\задача
Докажите, что существует предел $\limn\left(\dfrac{1}{1^3} - \dfrac{1}{2^3} + \cdots + (-1)^{n-1}\dfrac{1}{n^3}\right)$.
\кзадача

\задача
Докажите, что существует предел $\limn\left(\dfrac{1}{1^3} + \dfrac{1}{2^3} + \cdots + \dfrac{1}{n^3}\right)$.
\кзадача


\задача
Последовательность $(x_n)$ такова, что для всех $n \in \N$ выполнено неравенство $|x_{n+1} - x_n| < \dfrac{1}{n^2}$. Докажите, что существует предел $\limn x_n$.
\кзадача

\задача
Последовательность $(x_n)$ такова, что существует предел $\limn(|x_2-x_1|+\cdots+|x_n-x_{n-1}|)$. Докажите, что она имеет предел.
\кзадача


