\documentclass[a4paper,11pt]{article}
\usepackage[mag=1000]{newlistok}


\ВключитьКолонтитул

\УвеличитьШирину{1.4cm}
\УвеличитьВысоту{2.3cm}

\Заголовок{Условная вероятность}
\НомерЛистка{36}
\renewcommand{\spacer}{\vspace{0.9pt}}
\ДатаЛистка{09.09 -- 18.09.2019}
% 54 задач
\Оценки{25/20/15}


\begin{document}


\СоздатьЗаголовок

%\centerline{\large\sf А Н О Н С.\hfil
%Конечная теория вероятностей, избранные задачи.  \hfil 02.2002       }


%\medskip
%\раздел{Основные понятия}


%\опр
%\выд{Конечное вероятностное пространство} --- это конечное множество
%$\Omega=\{\omega_1,\ldots,\omega_n\}$.
%Элементы $\Omega$ называются
%\выд{элементарными исходами}, а подмножества $\Omega$
%называются \выд{событиями}.
%При этом каждому исходу $\omega_i$  сопоставлено число $p_i=p(\omega_i)$
%из отрезка $[0;1]$, называемое вероятностью этого исхода.
%Сумма вероятностей всех элементарных исходов должна равняться 1.
%\выд{Вероятностью события} $A\in\Omega$ называется величина
%$P(A)=\sum\limits_{a\in A} p(a)$.
%\копр

%\опр
%\выд{Пространством элементарных событий} $\Omega$ может быть
%произвольное конечное множество.
%Всякое подмножество из $\Omega$ называется \выд{событием},
%множество всех событий обозначается буквой $\cal F$.\\
%\\
%\выд{Вероятностью}, или \выд{вероятностной мерой}
%называется числовая функция $P:{\cal F}\to {\Bbb R}$,
%такая что \\
%$1)$ $P(\emptyset)=0$, $P(\Omega)=1$, $P(A)\ge0$ для любого $A\in{\cal F}$;\\
%$2)$ \выд{(аддитивность вероятностной меры)} если $A\cap B=\emptyset$
%(т.~е.~события $A$ и $B$
%\выд{несовместны}), то $P(A\cup B)=P(A)+P(B)$.\\
%%События, вероятность которых равна 1, называются \выд{достоверными}.
%Множество $\Omega$, множество $\cal F$ и вероятностная мера $P$ вместе
%называются \выд{вероятностным пространством}. Обозначение:
%$(\Omega,\cal F, P)$.
%\копр


%\опр
%\выд{Суммой} событий $A$ и $B$ называется событие
%$A\cup B$, которое происходит тогда и только тогда, когда происходит
%хотя бы одно из событий $A$ или $B$.
%\выд{Произведением} событий $A$ и $B$ называется событие
%$A\cap B$, которое происходит тогда и только тогда, когда происходят
%оба события $A$ и $B$.
%Событием, \выд{противоположным} событию $A$,
%называется событие $\overline{A}=\Omega\setminus A$,
%которое происходит тогда и только тогда, когда не происходит
%событие $A$.
%\копр



%\задача
%Игральный кубик бросают 6 раз подряд. %Постройте соответствующее
%%вероятностное пространство (
%%Что такое в данном случае исход?
%Считая все элементарные события равновероятными,
%найдите вероятность выпадения %при четырех бросаниях игрального кубика
%\вСтрочку
%\пункт ровно одной шест\"ерки;
%\пункт хотя бы одной шестерки.
%\кзадача


%\задача
%Найдите вероятность выпадения при четырех бросаниях
%игрального кубика хотя бы одной шестерки.
%\кзадача




% \задача
% В Китае ввели закон с целью уменьшить
% прирост населения, минимально повлияв на традиции. Если
% в семье первый ребёнок --- мальчик (наследник), семье нельзя
% больше иметь детей, иначе можно
% завести ещё одного ребёнка. Повлияет ли выполнение закона на
% соотношение мужского и женского населения в Китае?
% (При каждом рождении вероятность рождения
% мальчика считаем равной~1/2.)
% \кзадача

\пзадача
Пусть $B$ --- событие с ненулевой вероятностью. Определите {\em условную вероятность} $ P (A\,|\,B)$ события $A$ при условии,
что событие $B$ произошло (выразите е\"е через $P(A)$, $P(B)$ и $P(AB)$).
\кзадача


\задача
Пусть вероятность рождения мальчика равна $1/2$.
Какова вероятность того, что в семье два мальчика, если один
из детей --- мальчик?
\кзадача
%
\пзадача
Вероятность попадания в цель при отдельном выстреле равна $0{,}2$.
Какова вероятность поразить цель, если в $2\%$  случаев выстрел не происходит из-за осечки?
\кзадача
%
\опр
События $A$ и $B$ называются \выд{независимыми}, если
$ P (A\,|\,B)= P (A)$.
\копр
%

\пзадача
Верно ли, что $A$ и $B$ независимы тогда и только тогда, когда
\пункт $ P (B\,|\,A)= P (B)$;\\
\пункт $ P (AB)= P (A)\cdot P (B)$;
\пункт независимы $A$ и \лк не $B$\пк;
\пункт независимы \лк не $A$\пк\ и~\лк не~$B$\пк.
\кзадача
%

\пзадача
Из колоды в 52 карты выбирается наудачу одна карта. Независимы ли события\\
\вСтрочку
\пункт
\лк выбрать вальта\пк\ и \лк выбрать пику\пк;
\пункт
\лк выбрать вальта\пк\ и \лк не выбрать даму\пк?
\кзадача


\опр
События $A_1,\dots,A_n$ называются \выд{независимыми в совокупности}, если
для любых нескольких различных индексов $1\le i_1<i_2<\dots<i_k\le n$ выполнено:
$P(A_{i_1}A_{i_2}\dots A_{i_k})=P(A_{i_1})\cdot P(A_{i_2})\cdot\dots\cdot P(A_{i_k})$.
\копр

\задача
Следует ли из попарной независимости нескольких событий их независимость в совокупности?
\кзадача

\задача[Теорема умножения вероятностей]
Пусть $A_1,A_2,\ldots,A_n$ --- события, вероятности которых больше~0.
Докажите, что
$P(A_1 A_2\dots A_n) = P(A_1)\cdot P(A_2\mid A_1)\cdot P(A_3 \mid A_1 A_2)\cdot\ldots\cdot P(A_n \mid A_1\dots A_{n-1}).$
\кзадача

%
%

% \задача
% Верно ли, что $A$ и $B$ независимы тогда и только тогда, когда
% \пункт $A$ не зависит от $B$; ?
% \кзадача

% \задача
% Пусть вероятность попасть под машину, переходя улицу в неположенном
% месте, равна 0{,}01. Какова вероятность остаться целым, сто раз
% перейдя улицу в неположенном месте?
%%Как связана эта вероятность с числом $e$ (см.~листок 19)?
% Вычислите е\"е поточнее (см.~задачу 21 листка~15).
% \кзадача

%\задача
%\вСтрочку
%\пункт
%[Выборка без возвращения]
%В урне $M$ черных и $N$ белых шаров. Наугад выбрано $n$ шаров.
%Какова вероятность вытащить ровно $m$ белых шаров, если
%после взятия из урны шар
%\вСтрочку
%\пункт
%не возвращается назад;
%\пункт
%возвращается назад.
%\кзадача



%\задача
%\пункт Пусть $A_1,A_2$ --- события, вероятности которых больше 0.
%Докажите \выд{теорему умножения вероятностей}:
%$P(A_1 A_2)=P(A_1)\cdot P_{A_1}(A_2)$.
%\кзадача


%\пункт Какие события независимы сами с собой?

%\задача
%Пару кубиков бросили два раза. Какова вероятность
%\кзадача

%\задача
%Сколько раз нужно бросить игральный кубик, чтобы вероятность
%получить хотя бы одну пят\"ерку стала больше половины?
%\кзадача


% \задача
% Есть 1000 симметричных монет, прич\"ем одна из них фальшивая
% (с двумя орлами). Выбрали случайно одну монету и подбросили е\"е 10
% раз, выпали все орлы. Какова вероятность того, что если эту монету
% подбрость ещ\"е 20 раз, то снова выпадут все орлы?
% \кзадача



%
%

%\опр
%Событие $A$ называют \выд{независимым} от события $B$,
%если $P(B)=0$, или (иначе) %$P(B) > 0 $ и
%$P_B(A)=P(A)$.
%\копр
%
%\задача
%Верно ли, что $A$ и $B$ независимы тогда и только тогда, когда
%$A$ не зависит от $B$?
%\кзадача

%
\задача [О вреде подхалимства]
\пункт В жюри из трех человек вердикт %окончательное решение
выносят большинством голосов. Председатель и эксперт принимают
верное решение независимо с вероятностями $0{,}7$ и $0{,}9$,
а третий бросает монету. С~какой вероятностью жюри принимает верное решение?
\пункт А если третий будет копировать решение председателя?
\пункт А если третий будет копировать решение эксперта?
\кзадача
%
\пзадача
Отец обещал сыну приз, если сын выиграет подряд хотя бы две теннисные
партии против него и чемпиона по одной из схем:
отец-чемпион-отец или чемпион-отец-чемпион.
Чемпион играет лучше отца. Какую схему выбрать сыну?
\кзадача
%

\задача[Формула Байеса]
Выразите условную вероятность $P(A\mid B)$ через $P(B\mid A)$, $P(A)$ и $P(B)$.
\кзадача

\задача[Формула полной вероятности]
    Пусть $H_1,H_2,\ldots,H_n$ --- попарно непересекающиеся (несовместные) события с суммарной вероятностью 1.
    Докажите, что $P(B)=\sum\limits_{i=1}^{n}P(H_i)\cdot P(B\mid H_i)$ для любого события $B$.
\кзадача
\vspace{-1mm}
\пзадача
Два охотника одновременно выстрелили одинаковыми пулями в медведя и убили его одной пулей.
Как %должны
им поделить шкуру, если вероятность попадания у первого --- 0{,}3,
а у второго~---~0{,}6?
\кзадача

%\задача
%Каждый из двух игроков пишет на бумажке по целому числу,
%потом они одновременно открывают эти числа.
%Если их сумма делится на 3, то второй
%платит первому рубль; если нет --- второй получает $a$
%рублей от первого. При каком значении $a$ эта игра будет честной?
%\кзадача

%\задача
%Каждый из двух игроков пишет на бумажке число 1 или 2, после чего
%они одновременно открывают бумажки.
%Если числа совпали, то первый платит второму столько рублей, каковы
%эти числа; если нет --- второй платит первому $a$ рублей.
%При каком значении $a$ эта игра будет честной?
%\кзадача

% \задача
% Юра ежедневно в случайное время между 16 ч и 18 ч едет ужинать к маме или невесте, которые живут по той же линии метро,
% но в разных концах. Юра садится в первый пришедший поезд (в любом направлении).
% Он считает, что его шансы ужинать у мамы или невесты равны,
% но за 20 дней был у мамы лишь дважды. Как это могло быть?
% \кзадача

\задача
Три завода выпускают одинаковые изделия. Первый производит 50\%
всей продукции, второй --- 20\%, третий --- 30\%.
Первый завод выпускает 1\% брака, второй --- 8\%, третий --- 3\%.
Выбранное наугад изделие --- бракованное. Какова вероятность
того, что оно %изготовлено на втором заводе?
со второго завода?
\кзадача

% \задача
% Монета погнулась, и теперь <<орёл>> выпадает с вероятностью $\frac{1}{3}$ (а <<решка>> --- с вероятностью $\frac{2}{3}$).
%   Найдите вероятность того, что после 10 подбрасываниях этой монеты выпадет
%   \пункт ровно 5 <<орлов>>;
%   \пункт хотя бы 5 <<орлов>>;
%   \пункт выпадение какого количества <<орлов>> наиболее вероятно?
% \кзадача

\задача
Из 100 симметричных монет одна фальшивая
(с двумя орлами). Выбрали случайно монету,~бросили 5
раз: выпали все орлы. С какой вероятностью следующие 10 раз снова выпадут все орлы?
\кзадача

\задача
У некого вида бактерий каждая бактерия через секунду после появления
на свет делится с вероятностью $p_k$ на $k$ потомков, где $k=1,\ 2,\ \dots,\ n$, и с вероятностью $p_0$ умирает.
Пусть $x$ --- вероятность того, что весь род, начавшийся с данной бактерии, когда-либо целиком вымрет. Докажите, что
$x=p_0+p_1x+p_2x^2+\dots+p_{n}x^{n}$.
\кзадача

\задача
При обследовании вероятность обнаружить туберкулез у больного им равна 0{,}9, а вероятность принять
здорового за больного равна 0{,}01. Туберкулёзом болеет 1/1000 населения. С~какой вероятностью человек здоров,
если его признали больным
\пункт при одном обследовании;
\пункт при двух независимых обследованиях?
%Там ответ --- 91{,}7%  и  10.98%
\кзадача


\задача
Каждый житель города либо здоров, либо болен, а также либо богат, либо беден.
Богатство и здоровье независимы (доля богатых здоровых среди богатых равна доле здоровых среди всех).
Известно, что есть богатый горожанин и есть здоровый
горожанин. Обязательно ли найдётся богатый здоровый горожанин?
\кзадача



\сзадача
Два лекарства $A$ и $B$ испытывали на мужчинах и женщинах. Каждый
человек принимал только одно лекарство. Общий процент людей, почувствовавших улучшение, больше среди принимавших $А$. Процент мужчин, почувствовавших улучшение, больше среди мужчин, принимавших $В$. Процент женщин, почувствовавших улучшение, больше среди женщин, принимавших $В$.
\пункт Возможно ли это?
\пункт Какое лекарство нужно посоветовать принять пациенту, если его пол неизвестен?
\кзадача

% \задача
% В пятиместную машину садятся водитель, три пассажира и собака. Все люди садятся на отведенные им места, а собака может сесть куда угодно (случайным образом). Последовательность проникновения участников процесса в машину — случайная. Любой оказавшийся раньше собаки в машине человек гарантировано занимает свое место. А вот если собака села раньше кого–то из людей и заняла его место, обеспокоенный человек начинает вести себя как собака — садится на первое попавшееся (то есть, случайно выбранное) место (из оставшихся к этому моменту свободными, конечно).
% \кзадача
%

\ЛичныйКондуит{0mm}{6mm}
% \GenXMLW

%\СделатьКондуит{6.5mm}{7.7mm}

\end{document}

\раздел {Дополнительные задачи}


\задача
Трое друзей хотят бросить жребий, кому идти в лес за дровами.
Как им это сделать, если у них есть только одна монета
(которую можно многократно бросать)?
\кзадача

\задача
Про вид бактерий известно,
что каждая бактерия через минуту после появления
на свет делится с вероятностью $p_k$ на $k$ потомков, где $k=0{,}1,\dots,10$.
При этом $p_0$ --- вероятность смерти
бактерии через минуту после рождения.
Докажите, что вероятность $x$ того, что весь род, начавшийся с данной
бактерии,  когда-либо целиком вымрет, удовлетворяет уравнению
$x=p_0+p_1x+p_2x^2+\dots+p_{10}x^{10}$.
\кзадача



%\раздел {Дополнительные задачи}

%\задача
%\кзадача

%\задача
%Двое бросают монету --- один 10 раз, другой --- 11. Какова вероятность того,
%что у второго орлов выпало больше, чем у первого?
%\кзадача

\задача
Каждый из двух игроков пишет на бумажке число 1 или 2, после чего
они одновременно открывают бумажки.
Если числа совпали, то первый платит второму столько рублей, каковы
эти числа; если нет --- второй платит первому $a$ рублей.
При каком значении $a$ эта игра будет честной?
Разберите три случая:
\сНовойСтроки
\пункт
Каждый игрок равновероятно выбирает 1 или 2.
\пункт
Первый выбирает 1 с вероятностью $p$,
%и 2 --- с вероятностью $1-p$,
а второй выбирает 1 с вероятностью $q$ (где $0\leq p,q\leq1$).
%, и 2 --- с вероятностью $1-q$.\\
\спункт
То же, что и пункт б), но перед игрой первый случайным образом
выбирает $p$ (равновероятно из отрезка $[0;1]$),
а второй независимо выбирает $q$ (равновероятно из отрезка $[0;1]$).
\кзадача


\сзадача
Каждый из двух игроков пишет на бумажке по целому числу,
потом они одновременно открывают эти числа.
Если их сумма делится на 3, то второй
платит первому рубль; если нет --- второй получает $a$
рублей от первого. При каком значении $a$ эта игра будет честной
(разберите случаи, как и в задаче 24)?
\кзадача

\сзадача %[Сумасшедшая старушка]
Каждый из $n$ пассажиров купил по билету на $n$-местный самолет.
Первой зашла сумасшедшая старушка и села на случайное место.
Далее, каждый вновь вошедший занимает свое место, если оно свободно;
иначе занимает случайное. Какова вероятность того,
что последний пассажир займет свое место?
\кзадача



\сзадача %[Задача о баллотировке] %Предположим, что
На выборах кандидат
$P$ набрал $p$ голосов, а кандидат $Q$ набрал $q$ голосов, %причем
$p>q$. Найдите вероятность того, что при последовательном подсчете голосов
$P$ все время был впереди $Q$.
\кзадача


\сзадача [Задача о разорении]
Игрок, имеющий $n$ монет, играет против казино, которое имеет
неограниченное число монет. За одну игру игрок либо проигрывает монету,
либо выигрывает с вероятностью 0{,}5. Он играет, пока не разорится. Какова
вероятность разориться ровно за $m$ игр?
\кзадача


\сзадача
Датчик случайных чисел выдает конечное число чисел, каждое ---
со своей вероятностью. Датчик \выд{сильнее} другого, если
с вероятностью большей $1/2$ выданное им число больше числа, выданного другим
датчиком. Можно ли сделать датчики $A$, $B$ и $C$ так, чтобы $A$
был сильнее $B$, $B$ сильнее $C$, а $C$ сильнее $A$?
\кзадача

\сзадача %[Выбор невесты]
%Царь желает выбрать самую красивую невесту из $100$ претенденток.
%%Процедура выбора  невесты состоит в следующем:
%Претендентки в случайном
%порядке приходят к царю, и в момент прихода очередной претендентки
%царь может объявить
%ее своей невестой (царь заранее не знаком с претендентками, но легко
%упорядочивает их по красоте). Докажите, что царь может выбрать самую
%красивую с вероятностью, большей $1/3$.
Вам в случайном порядке предлагают 100 заранее неизвестных разных
денежных сумм, пока Вы не возьм\"ете предлагаемую сумму.
Как действовать, чтобы взять наибольшую с вероятностью, большей~$1/3$?
\кзадача

%\раздел{Геометрические вероятности}
%
%\noindent
%При решении требуется построить соответствующее бесконечное
%вероятностное пространство.

\сзадача
Палку %случайным образом
случайно ломают на 3 части. С какой вероятностью
из них можно сложить треугольник?
\кзадача

%\задача
%В течение часа к железнодорожной станции в случайные моменты времени
%подходят два поезда. Какова вероятность того, что удастся перебежать
%из поезда в поезд, не ожидая на платформе, если оба стоят по 5 минут?
%\кзадача

\сзадача [Задача Бюффона]
На плоскость, разлинованную параллельными прямыми (на расстоянии
$1$ друг от друга), брошена игла длины $\lambda<1$. Найдите
вероятность пересечения иглы с какой-нибудь прямой.
\кзадача


\сзадача [Парадокс Бертрана]
С какой вероятностью случайная хорда некой данной окружности будет больше
стороны правильного треугольника, вписанного в эту окружность?
\кзадача



\сзадача
Монету радиусом $r$ и толщиной $d$
бросают на горизонтальную поверхность (соударение неупругое).
Какова вероятность того, что монета упадет на ребро? %(Соударение считается неупругим.)
\кзадача

%\сзадача
%Человек, имеющий $n$ ключей, хочет отпереть свою дверь,
%независимо пробуя 1 ключ в минуту в случайном порядке.
%Сколько он в среднем провозится, если неподошедшие ключи
%из дальнейших испытаний
%\вСтрочку
%\пункт исключаются;
%\пункт не исключаются.
%\кзадача

\сзадача
Человек, имеющий $n$ ключей, хочет отпереть свою дверь, испытывая
ключи независимо один от другого в случайном порядке. Найдите
среднее число испытаний, если неподошедшие ключи\\
\вСтрочку
\пункт исключаются из дальнейших испытаний;
\пункт если они не исключаются.
\кзадача

\сзадача
Пачка жевачки содержит один из $n$ разных, но равновероятных вкладышей.
Сколько пачек нужно в среднем купить, чтобы собрать полную коллекцию
вкладышей?
\кзадача


\сзадача
%В расписании движения автобусов на остановке
%\лк Университет\пк\ написано, что
Средний интервал движения автобуса \No 57 равен 35 минут, а средний интервал
движения автобуса \No 661 равен 20 минут. Сколько в среднем нужно
ждать
\вСтрочку
\пункт автобус \No 57;
\пункт один из этих автобусов?
\кзадача

%\СделатьКондуит{4mm}{9mm}

\end{document}

\задача
Студент Иванов ездит из МГУ на 103-ем автобусе, а студент Петров ---
на 130-ом. Садятся они на одной остановке. Иванов утверждает, что 103-и
автобусы полные, а 130-е --- пустые. Петров же утверждает точно
противоположное. Предложите правдоподобное объяснение.
\кзадача
