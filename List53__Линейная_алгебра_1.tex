% !TeX encoding = windows-1251

\documentclass[a4paper,12pt]{article}
\usepackage{newlistok}
%\usepackage{mathtools}

\УвеличитьШирину{1.0truecm}
\УвеличитьВысоту{2truecm}


\Заголовок{Системы линейных уравнений}
\НомерЛистка{53}
\ДатаЛистка{11.09.2020 -- 28.09.2020}
\Оценки{26/20/14}
\ВключитьКолонтитул

\begin{document}
	\СоздатьЗаголовок
	
Пусть $K$ --- любое поле (можно считать, что это любое из полей $\Q,\ \R,\ \mathbb{C},\ \Z/p\Z$).

%\раздел{Системы линейных уравнений}

\опр
\выд {Линейное уравнение} с переменными $x_1, x_2, \ldots, x_n$ над полем $K$ --- это уравнение вида
$a_1x_1 + a_2x_2 + \cdots a_nx_n = b,$ где \выд коэффициенты $a_1, a_2,\ldots a_n$ и \выд{свободный член} $b$ лежат в~$K$.


\выд{Система $m$ линейных уравнений} с $n$ переменными $x_1, x_2, \ldots, x_n$ над полем $K$ в общем виде выглядит так:
\vspace*{-3mm}
	$$
	\begin{cases}
		a_{11}x_1 + a_{12}x_2 + \cdots + a_{1n}x_n = b_1,\\
		a_{21}x_1 + a_{22}x_2 + \cdots + a_{2n}x_n = b_2,\\
		\cdots \cdots \cdots \cdots 		\cdots \cdots \cdots \cdots \cdots \cdots \cdots \\
		a_{m1}x_1 + a_{m2}x_2 + \cdots + a_{mn}x_n = b_m.\\
	\end{cases}
	$$

\vspace*{2mm}
Говорят, что это система с прямоугольной матрицей $A = (a_{ij})$ размера $m \times n$ и правой частью $b = (b_i)$; коротко
система записывается так: $Ax = b$. %Решения такой системы обычно ищут в том же поле $K$.
Решение системы --- это такой упорядоченный набор $(x_1,\ldots,x_n)$ элементов поля $K$, что выполнены все уравнения системы.
\копр

% \задача \пункт Пусть $x^{(0)}$~--- некоторое решение системы $Ax = b$.
% Докажите, что любое её решение имеет вид $x^{(0)} + y$, где
% $y$~--- произвольное решение \выд{однородной} системы $Ax = 0$.\\
% \пункт Докажите, что для любых двух решений $x^{(1)}$ и $x^{(2)}$
% однородной системы их \выд{линейная комбинация} $ax^{(1)} + bx^{(2)}$ тоже является решением (при любых $a$ и $b$ из $K$).
% \кзадача


\опр	\выд {Элементарные преобразования} системы линейных уравнений --- это \\
$\bullet$ умножение строки на ненулевое число (ненулевой элемент поля);\\
$\bullet$ прибавление к строке другой строки, умноженной на число (меняется только одна строка!);\\
$\bullet$ перемена двух строк местами.
\копр


\задача Докажите, что элементарные преобразования не меняют множество решений системы (то есть, получается система, {\em эквивалентная} исходной).
\кзадача
	
\задача[Метод Гаусса] Докажите, что для любой системы уравнений
найдется эквивалентная ей система $Cx = d$, где матрица $C$ имеет
ступенчатый вид: в каждой ненулевой строке начальный отрезок нулей
длинней, чем в предшествующей строке, а все нулевые строки
располагаются в конце. \кзадача

\задача Как описать все решения системы, приведённой к ступенчатому виду?\footnote{\выд {Начало решения.} Посмотрим на последнюю ненулевую строку. Если она имеет вид $\alpha=0$, где $\alpha$ --- ненулевое число, система не имеет решений. Если она имеет вид $\alpha x_i=\beta$, где $\alpha\ne0$, мы можем однозначно выразить $x_i$ и подставить его в предыдущие уравнения, уменьшив число неизвестных. Если она имеет вид $\alpha x_i+\beta x_j=b$ с ненулевыми $\alpha,\beta$, мы можем придавать, например, переменной $x_j$ любое значение ($x_j$ при этом называют \выд{свободной переменной}), находить по нему значение $x_i$ и подставлять эти значения в предыдущие уравнения...}
\кзадача



	

% 	\задача
% Найдите все решения системы в зависимости от параметров $a,b$:
% 	$$
%     \begin{cases}
% 	x+2y+3z=0\\
% 	\hspace{0.75cm} 3y+5z=0\\
% 	\hspace{1.7cm} az=b
% 	\end{cases}
% $$
% 	\кзадача


	\пзадача Решите системы (над $\R$):
$$
	\пункт \begin{cases}
	x+3y+z=1\\
	2x+7y+3z=2\\
	-x-3y= 3\\
	3x+10y+5z=7
	\end{cases}
\hspace*{-2mm}
	\пункт 	\begin{cases}
	x+2y+3z=6\\
	4x+5y+6z=15\\
	7x+8y+9z=24
	\end{cases}
% 	\пункт 	\begin{cases}
% 	x_1+2x_2+3x_3+4x_4=5\\
% 	2x_1+4x_2+4x_3+5x_4=5\\
% 	3x_1+x_2+5x_3+x_4=0\\
% 	3x_1+x_2+7x_3+4x_4=5
% 	\end{cases}	
\hspace*{-2mm}
	\пункт	\begin{cases}
	x_1+x_2+x_3=0\\
	x_2 + x_3 + x_4 = 0\\
	\cdots \\
	x_{100} + x_1 + x_2 = 0
	\end{cases}	
\hspace*{-2mm}
	\пункт	\begin{cases}
	x+ay+a^2z = 0\\
	x + by + b^2z = 0\\
	x+cy+c^2z = 0
	\end{cases}	
$$
\кзадача

\задача Может ли система линейных уравнений с действительными коэффициентами иметь в точности два различных решения? \кзадача		
	
	
\пзадача Сколько решений в действительных числах может иметь система из $m$ линейных уравнений от $n$ переменных, если \пункт $m=n$; \пункт $m < n$; \пункт $m>n$? \кзадача
	
\опр Система из определения 1 называется \выд однородной, если $b_1 = b_2 = \cdots = b_m = 0$.\копр
	
\задача
Докажите, что если умножить любое решение однородной системы на число или сложить любые два её решения, снова получится её решение (кстати, а что такое <<сумма решений>>?).
\кзадача
	
\пзадача Пусть $X_O$ --- множество решений однородной системы $Ax = 0$. Докажите, что множество решений неоднородной системы $Ax = b$ (с той же левой частью) либо пусто, либо это множество $\{x_{частн} + x_{o}\ |\ x_{o}\in X_O\}$, где $x_{частн}$ --- произвольно выбранное решение неоднородной системы.
\кзадача

\задача
Докажите, что однородная система, в которой неизвестных больше, чем уравнений, имеет ненулевое решение.
\кзадача

\пзадача Пусть некоторая однородная система из $n$ линейных уравнений от $n$ переменных имеет только нулевое решение. Докажите, что у любой неоднородной системы линейных уравнений с такой же левой частью решений будет \пункт не больше одного; \пункт ровно одно.
\кзадача	


\задача Докажите, что если матрица $A$ такова, что для любой правой
части $b$ соответствующая система $Ax=b$ имеет единственное решение, то $n = m$
(то есть матрица квадратная).
\кзадача

\задача %[Задача Дирихле для уравнения Лапласа]
Старуха Шапокляк
расставила числа в граничных клетках прямоугольной таблицы\break (<<в~рамочке>>). Сможет ли
Чебурашка поставить числа в оставшиеся (<<внутренние>>) клетки таблицы так, что
каждое поставленное число будет средним арифметическим четырёх его
соседей? \кзадача

\задача 24 студента решали 25 задач. У преподавателя есть таблица размером 24$\times$25, в которой записано, кто какие задачи решил. Оказалось, что каждую задачу решил хотя бы один студент. Докажите, что можно отметить некоторые задачи <<галочкой>> так, что каждый из студентов решил чётное число (в частности, может быть, ноль) отмеченных задач. \кзадача

\раздел{***}

\задача Пусть все коэффициенты системы линейных уравнений рациональны (включая правую часть),
и система имеет действительное решение. Докажите, что она имеет и рациональное решение. \кзадача

\пзадача Известно, что некоторый многочлен принимает во всех рациональных
точках рациональные значения. Докажите, что его коэффициенты
рациональны. \кзадача

\пзадача Внутри отрезка $[0,1]$ выбрали $n$ различных точек. Отмеченной точкой назовём одну из $n$ выбранных или конец отрезка. Оказалось, что любая из внутренних $n$ точек является серединой какого-то отрезка с вершинами в отмеченных. Докажите, что все точки рациональные.
\кзадача

\задача Пусть коэффициенты матрицы $A$ рациональны и система $Ax = b$
разрешима. Докажите, что у нее есть решение вида $x^{(0)} = Cb$,
где $C$~--- матрица с рациональными коэффициентами (то есть, каждое из $x_i^{(0)}$ ---
\выд{линейная комбинация} чисел $b_1,\ldots,b_m$ с рациональными коэффициентами). \кзадача

% \задача Докажите, что у однородной системы с рациональными
% коэффициентами есть такие рациональные решения
% $x^{(1)},\dots,x^{(k)}$, что любое решение представляется в виде их
% линейной комбинации. \кзадача

\задача В стаде 101 корова. Любые 100 из них можно разбить на 2
стада по 50 коров так, что общие веса этих двух стад будут равны. Требуется доказать,
что все коровы одного веса. \\
\пункт Решите задачу, если веса коров --- целые числа; рациональные числа.\\
\пункт Докажите, что у системы с переменными --- весами коров, --- построенной по условию задачи, бесконечно много решений,
но при записи её решений получится ровно одна свободная переменная.\\
\пункт Решите задачу, если веса коров --- действительные числа.
\кзадача

\задача Пусть квадрат $1\times 1$ разбит произвольным образом на $n$ квадратов.\\ % Требуется доказать, что стороны всех квадратов рациональны.\\
\пункт
Пусть $x_1,\dots,x_n$ --- переменные, соответствующие сторонам квадратов. Проведём через стороны квадратов прямые, так что прямоугольник
разобьётся на прямоугольную сеточку. Введите дополнительные переменные --- длины сторон прямоугольничков этой сеточки, --- и составьте
такую систему линейных уравнений, коэффициенты которой --- только нули и плюс-минус единицы, а в правой части ---  только нули или единицы, что любое решение этой системы, состоящее из положительных чисел, даёт разбиение исходного квадрата $1\times1$ на $n$ квадратов.\\
\пункт\sloppy Докажите, что для любого решения нашей системы выполнено равенство
\mbox{$x_1^2+x_2^2+\ldots+x_n^2 = 1$}.\\
\пункт Докажите, что если при решении нашей системы появится хоть одна свободная переменная $t$, то некоторая квадратичная функция от $t$ будет постоянной на некотором интервале.\\
\пункт Докажите, что стороны всех квадратов рациональны.
\кзадача


\сзадача Пусть прямоугольник $a\times b$ разбит произвольным образом
на $n$ квадратов. \\
\пункт
Докажите, что система, составленная аналогично тому, как это было сделано в предыдущей задаче, имеет единственное решение, и оно состоит из чисел вида $\lambda a+\mu b$ с рациональными $\lambda$ и $\mu$.\\
\пункт Подставим это решение в каждое уравнение нашей системы, заодно преобразуя уравнение к виду $\lambda a+\mu b=0$. Докажите, что не может для каждого уравнения системы получиться $\lambda=\mu=0$.\\
\пункт Докажите, что число $a/b$ рационально, а стороны всех квадратов соизмеримы как с $a$, так и с $b$.
\кзадача


%\сзадача Прямоугольный лист бумаги размерами $a\times b$~см разрезан
%на прямоугольники, у каждого из которых одна сторона имеет длину
%1~см. Докажите, что хотя бы одно из чисел $a$ или $b$ целое.
%\кзадача



	

% \задача[Числа Фибоначчи] \пункт Известно, что последовательность
% $\{a_n\}$ удовлетворяет соотношению $a_{n + 2} = a_{n + 1} + a_n$.
% Докажите, что при любом $\lambda\in\R$ последовательность
% $\lambda\{a_n\} = (\lambda a_1,\lambda a_2,\dots)$ также
% удовлетворяет этому соотношению.\\ \пункт Пусть последовательности
% $\{a_n\}$, $\{b_n\}$ удовлетворяют соотношению пункта~а). Докажите,
% что при любых $\lambda,\mu \in\R$ последовательность $\lambda\{a_n\}
% + \mu\{b_n\} = (\lambda a_1 + \mu b_1,\lambda a_2 + \mu b_2,\dots)$
% также удовлетворяет этому соотношению.\\ \пункт Пусть
% последовательности $\{a_n\}$, $\{b_n\}$ удовлетворяют соотношению
% п.~а), причём вектора $(a_1,a_2)$ и $(b_1,b_2)$ неколлинеарны.
% Докажите, что любая последовательность, удовлетворяющая соотношению
% пункта~а), может быть представлена в виде $\lambda\{a_n\} +
% \mu\{b_n\}$ при некоторых $\lambda,\mu \in \R$.\\ \пункт Найдите
% геометрические прогрессии,  удовлетворяющие соотношению пункта~а).\\
% \пункт Найдите явную формулу для последовательности Фибоначчи,
% заданной рекуррентным уравнением $a_{n + 2} = a_{n + 1} + a_n$ и
% начальными условиями $a_0 = a_1 = 1$. \кзадача
%
% \задача Найдите явные формулы для следующих (вещественных)
% последовательностей: \сНовойСтроки \пункт $a_{n + 3} = 2a_{n + 2} +
% a_{n + 1} - 2a_n$, если $a_0 = 1$, $a_1 = 2$, $a_2 = 3$; \пункт
% $a_{n + 2} = -2a_{n + 1} - a_n$, в случаях $a_0 = 1$, $a_1 = -1$;
% $a_0 = 0$, $a_1 = -1$; $a_0 = 1$, $a_1 = 2$. \пункт $a_{n + 2} =
% 7a_{n + 1} - 12{,}5a_n$, если $a_0 = a_1 = 1$. \пункт $a_{n + 2} =
% a_{n + 1} + a_n + 1$, если $a_0 = a_1 = 1$. \кзадача
%
% \задача Опишите все решения рекуррентного уравнения $a_{n + k} =
% c_{k - 1}a_{n + k - 1} + \cdots + c_0a_n + c$, где $c_0,\dots,c_{k -
% 1} \in \R$, при любых начальных условиях, если \вСтрочку \пункт $c =
% 0$; \пункт $c \in \R$. \кзадача
%
% \задача \пункт Докажите, что через любые четыре точки плоскости
% проходит бесконечное количество кривых второго порядка (то есть
% линий, заданных уравнениями вида $ax^2 + bxy + cy^2 + dx + ey + f =
% 0$, где не все $a$, $b$, $c$ равны нулю). \пункт Докажите, что через
% любые пять точек плоскости проходит хотя бы одна кривая второго
% порядка. \пункт Существуют такие пять точек, через которые проходит
% ровно одна кривая второго порядка; бесконечное количество кривых
% второго порядка. \кзадача
%

		
	
	
	
% 	\задача Докажите, что если у СЛУ с рациональными коэффициентами существует вещественное решение, то у нее существует рациональное решение. \кзадача
% 	
% 	\задача Докажите, что если у однородной СЛУ существует рациональное решение, то существует и целочисленное решение. \кзадача
% 	
% 	\задача
% 	Имеется клетчатая таблица $(k+2)\times(l+2)$, в её граничных клетках(«в рамочке») расставлены произвольные вещественные числа.
% 	Докажите, что в клетках центрального прямоугольника $k\times l$ можно единственным образом расставить числа так, чтобы каждое из этих $kl$ чисел равнялось среднему арифметическому своих четырёх соседей по стороне.
% 	\кзадача
	
	
	
	\ЛичныйКондуит{0mm}{5mm}

% \GenXMLW
	
\end{document}




% правильное название ? точка конденсации
% \задача
% Пусть множество $M\subset\R$ несчётно (то есть как минимум бесконечно).
% Точка $a\in M$ называется \выд{точной конденсации}, если пересечение любой окрестности $a$ с $M$ несчётно.
% \\\пункт
% Докажите, что множество точек $b\in M$ не являющихся точками конденсации не более, чем счётно.
% \\\пункт
% Может ли множество точек конденсации иметь изолированные точки?
% \\\пункт
% Докажите, что множество всех точек конденсации $M$ замкнуто;
% \\\пункт
% Докажите, что несчётное замкнутое подмножество в $\R$ имеет мощность континуума.
% \кзадача

