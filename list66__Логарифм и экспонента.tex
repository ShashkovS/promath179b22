%\documentclass[a4paper,12pt]{article}
%\usepackage{russlh}
%\usepackage{russcorr,listok}
%\documentstyle[12pt,russcorr,listok]{article}
\documentclass[a4paper, 12pt]{article}
\usepackage{newlistok}

\newcommand{\RpR}{{\cal R}([a,b])}
\newcommand{\intab}{\int\limits_a^b}
\УвеличитьШирину{1truecm}
\УвеличитьВысоту{2.5truecm}
\hoffset=-2.5truecm
\voffset=-25.5truemm


\begin{document}

\Заголовок{Логарифм, экспонента и число $e$}
%\Заголовок{Интеграл}
%\Подзаголовок{Часть 1. Определение и свойства}
\НомерЛистка{66}
\ДатаЛистка{17.05 -- 28.05.2021}
\Оценки{20/15/10}

\СоздатьЗаголовок

\smallskip

\задача
Пусть $F_1(x)$ и $F_2(x)$ --- две такие функции, что на некотором промежутке
$(F_1(x))'=(F_2(x))'$. Докажите, что найдется такая
константа $C$, что $F_1(x)=F_2(x)+C$ на этом промежутке.
\кзадача


\опр
\выд{Натуральным логарифмом} положительного числа $t$ назовем
%площадь под графиком функции $y=1/x$ на отрезке $[1;t]$, то есть
интеграл $\int\limits_1^t\frac1x\,dx$.\break Обозначение: $\ln t$.
\копр

\пзадача
Докажите, что $(\ln t)'=1/t$.
\кзадача

\задача
Докажите, что натуральный логарифм --- монотонно возрастающая функция.
\кзадача

\пзадача
\пункт
Пусть $a$ --- положительное число. Найдите производную
функции $\ln at$.\\
\пункт
Докажите, что $\ln ab=\ln a+\ln b$ для любых положительных чисел $a$ и $b$.
\кзадача

\задача
Докажите, что
%\вСтрочку
%\пункт
%$\ln 1/t=-\ln t$;
%\пункт
$\ln t^r=r\ln t$ при любом рациональном $r$.
\кзадача

\пзадача
Докажите, что $\ln t$ неограниченно возрастает при $t\rightarrow+\infty$.
\кзадача

\задача
Докажите, что уравнение $\ln t=a$ имеет единственное %(положительное)
решение при любом  $a\in\R$.
\кзадача

\опр
Решение уравнения из задачи 7 при $a=1$ обозначается буквой $e$.
\копр

\пзадача
% \пункт Пусть $f$ --- непрерывная функция. Докажите, что
% $\lim\limits_{n\rightarrow+\infty}f(x_n)=
% f\left(\lim\limits_{n\rightarrow+\infty}x_n\right)$
% (если указанный справа предел существует и вместе с последовательностью
% $(x_n)$ принадлежит области определения функции $f$).
% \пункт
Докажите, что
$\displaystyle\lim\limits_{n\rightarrow+\infty}\frac{\ln
(1+1/n)}{1/n}=1$, и выведите отсюда, что
$e=\lim\limits_{n\rightarrow+\infty}\left(1+1/n\right)^{n},$\break
то есть новое определение числа $e$ совпадает со старым.
\кзадача

\задача
Докажите, что функция $y=\ln t$ имеет обратную функцию
(обозначим ее $E(y)$). Где определена эта функция?
Непрерывна ли она?
Как ведет себя эта функция при $y\rightarrow-\infty$ и
при $y\rightarrow+\infty$?
\кзадача

\пзадача
Докажите, что $E(a)\cdot E(b)=E(a+b)$ при любых $a$ и $b$ из $\R$.
\кзадача

\задача
Докажите, что $E(r)=e^r$ при любом рациональном $r$.
\кзадача

\опр
Пусть $x\in\R$. Определим $x$-тую степень числа
$e$ формулой $e^x=E(x)$. В~результате\\ %мы определили $e^x$ так, что\\
1) получилась непрерывная на $\R$ функция, \\
2) для рациональных $x$ определение $e^x$ эквивалентно
известному из алгебры,\\
3) $e^ae^b=e^{a+b}$ при любых $a$ и $b$ из $\R$.
\копр

\пзадача
Найдите производную функции $e^x$.
\кзадача

\пзадача
Докажите, что
$e^x=\lim\limits_{n\rightarrow+\infty}\left(1+x/n\right)^{n}.$
\кзадача

\задача
Пусть $a$ --- положительное число. Определите для каждого $x\in\R$
число $a^x$  так, чтобы $a^x$ была непрерывной функцией от $x$,
причем для рациональных чисел $x$ определение $a^x$
было эквивалентно уже известному из алгебры.
\кзадача

\задача
Докажите, что $a^xa^y=a^{x+y}$ при любых $x$ и $y$ из $\R$.
\кзадача

\пзадача
Найдите производную функции $a^x$.
\кзадача

\опр Пусть $\alpha\in\R$.
Функция $x\mapsto x^\alpha$, определ\"енная на множестве
$\R_+$, называется \выд{степенной функцией}, а число $\alpha$ называется
\выд{показателем степени}.
\копр

\пзадача Представьте степенную функцию в виде композиции
показательной и логарифмической функций.
\кзадача


\пзадача
Найдите производную функции $x^\alpha$, где $\alpha\in\R$.
\кзадача

\задача Найдите все непрерывные функции $f:\R\rightarrow\R$,
удовлетворяющие при всех $x,y\in\R$ условию
\вСтрочку
%\пункт
%$f(x+y)=f(x)+f(y)$;
\пункт
$f(x\cdot y)=f(x)\cdot f(y)$;
\пункт
$f(x+y)=f(x)\cdot f(y)$;
\пункт
$f(x\cdot y)=f(x)+f(y)$.
\кзадача

\сзадача Сколько решений имеет уравнение
$\log_{1/16}x=\left(\displaystyle\frac1{16}\right)^{\displaystyle x}$?
\кзадача

\ЛичныйКондуит{.1mm}{8mm}

%\СделатьКондуит{8mm}{9mm}

% \GenXMLW
\end{document}

\задача
\пункт
\пункт
\кзадача 