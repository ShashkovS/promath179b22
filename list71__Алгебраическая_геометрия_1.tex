\documentclass[a4paper, 12pt]{article}
\usepackage{newlistok}

\УвеличитьШирину{1.2truecm}
\УвеличитьВысоту{2.5truecm}




\begin{document}

\Заголовок{Плоские алгебраические кривые --- I}
\НомерЛистка{71}
\ДатаЛистка{18.10 -- 25.10.2021}
\Оценки{32/24/16}

\СоздатьЗаголовок

\noindent


\задача
Пусть две пересекающиеся окружности заданы обычными квадратичными уравнениями
$f(x,y)=0$ и $g(x,y)=0$. Докажите, что 
\пункт
уравнение прямой, проходящей через точки пересечения этих окружностей, можно записать в виде $\alpha f(x,y)-\beta g(x,y)=0$, подобрав числа $\alpha$ и $\beta$.
\пункт
$f(x_0,y_0)g(x,y)=g(x_0,y_0)f(x,y)$ --- уравнение окружности (или прямой), проходящей через точку $(x_0;y_0)$ и точки пересечения этих окружностей.
\кзадача

\задача
Найдите уравнение прямой, проходящей через точки пересечения окружностей \\
\пункт $x^2+y^2+6x+4y=12$ и $x^2+y^2-6x-2y=0$;
\пункт $x^2+y^2+12x-4y=9$ и $x^2+y^2-4x-16y=-60$.
\кзадача

% \задача
% Пусть на плоскости заданы три попарно пересекающиеся окружности, у которых центры не лежат на одной прямой. Для каждой пары
% из этих окружностей рассмотрим прямую, проходящую через точки их пересечения.
% Докажите, что %если среди этих прямых нет параллельных, то они
% эти прямые проходят через одну точку.
% \кзадача

\опр
{\it Степень точки} относительно данной окружности радиуса $r$ --- это число $d^2-r^2$, где $d$ ---
расстояние от этой точки до центра этой окружности.
{\it Радикальная ось} двух окружностей --- это множество точек, каждая из которых имеет равные степени
относительно этих окружностей.
\копр

\задача
\пункт Выразите уравнение радикальной оси двух окружностей через их обычные уравнения.
\пункт Всегда ли это прямая? \пункт Что можно сказать о попарных радикальных осях трёх окружностей?
%\пункт Докажите, что три попарные радикальные оси трех окружностей пересекаются в одной точке, или параллельны, или совпадают.
% \пункт Пусть на плоскости заданы три попарно пересекающиеся окружности, у которых центры не лежат на одной прямой. Для каждой пары
% из этих окружностей рассмотрим прямую, проходящую через точки их пересечения.
% Докажите, что %если среди этих прямых нет параллельных, то они
% эти прямые проходят через одну точку.
\кзадача

\задача
Докажите, что точки пересечения
кривых $x^2+4xy+3y^2=3$ и $4x^2-2xy+3y^2=11$
лежат на одной окружности. 
\кзадача


\задача
Даны две параболы на плоскости (не обязательно равные), оси симметрии которых взаимно перпендикулярны, пересекающиеся в четырех точках.
Докажите, что эти точки лежат на одной окружности (выразите уравнение этой окружности через уравнения парабол).
\кзадача

% \задача
% Найдите уравнение окружности, проходящей через точки пересечения окружностей из задачи 1а и точку $()$.
% \кзадача

\задача
Найдите уравнение какой-нибудь прямой, отделяющей друг от друга параболу
$y=x^2+0,5$ и параболу $y=-2x^2+12x-12$.
\кзадача

\опр
\выд{Одночленом от двух переменных}\/ $x$ и $y$ называется выражение
вида $ax^my^n$, где $a$ --- действительное число, $m,n$ --- целые неотрицательные.
Сумма нескольких одночленов такого~вида
(с приведенными подобными)
называется \выд{многочленом от двух переменных}\/ $x$ и $y$.
%Множество всех многочленов от $x$, $y$ с действительными коэффициентами обозначается $\R[x,y]$.
\копр

% \задача Дайте определение степени многочлена  $A\in\R[x,y]$
% (обозначается $\deg A$).
% \кзадача
%
% \задача Пусть $A(x,y)$, $B(x,y)$ --- ненулевые многочлены.
% \вСтрочку
% \пункт Докажите, что $\deg AB=\deg A+\deg B$.
% \пункт Что можно сказать о величине $\deg (A+B)$?
% \кзадача

%\noindent
\опр
{\it Плоская алгебраическая кривая}\/ --- это множество точек
плоскости, координаты которых удовлетворяют уравнению
$A(x,y)=0$,  где $A$ --- %некоторый
непостоянный многочлен от двух переменных (говорят, что он {\it задаёт}\/ эту кривую).
\копр

\задача
Могут ли два разных многочлена задавать одну и ту же кривую?
\кзадача

\задача %Даны кривые и их уравнения в другом порядке.
%Что чему соответствует?\\
Какому из уравнений соответствует каждая из
кривых, изображённых на рисунке:\\
%\сНовойСтроки
\пункт $x^2=x^4+y^4$;\\
\пункт $xy=x^6+y^6$;\\
\пункт $x^3=y^2+x^4+y^4$;\\
\пункт $x^2y+xy^2=x^4+y^4$.
\кзадача

%
\vspace*{-10mm}
\putpict{7.7cm}{0cm}{alg_curve_2d-1}{}
\putpict{10.6cm}{0cm}{alg_curve_2d-2}{}
\putpict{13.4cm}{0cm}{alg_curve_2d-3}{}
\putpict{16.7cm}{0cm}{alg_curve_2d-4}{}
\vspace*{5mm}

\задача  Нарисуйте плоские\\ кривые, задающиеся многочленами:
\вСтрочку
%\пункт $x-y$;
\пункт $x^2-y^2$;
%\пункт $y-x^2$;
%\пункт $x^2+y^2-1$;
%\пункт $xy-1$;
\пункт $х^2y-xy^2+y-x$;
\пункт $x^2+x+y^2$;
\пункт $4x^2+9y^2-36$;
\пункт $x^2-y^2-1$;
\пункт $y^2-x^3$;
\пункт $y-1-x^3$;
\пункт $y^2-1-x^3$;
\пункт $y^2-x-x^3$;
\пункт $y^2-x^2-x^3$.
\кзадача

\задача Произведение ли это двух многочленов (не констант):
\вСтрочку
\пункт $x^2+y^2-1$; \пункт $y^2-x$; \пункт $xy-1$.
\кзадача


\задача
Задайте на плоскости многочленом: \пункт одну точку; \пункт любое конечное множество точек.
\кзадача

\задача
Докажите, что система из конечного числа уравнений вида <<многочлен равен нулю>>
может быть задана одним уравнением вида <<многочлен равен нулю>>.
\кзадача

\задача
\пункт Докажите, что если многочлен от нескольких переменных $x$, $y$, $z$, ... равен 0 при $x=y$, то он делится на $x-y$;
\пункт а если он равен 0 и при $x=y$, и при $x=z$, то он делится на $(x-y)(x-z)$.
\кзадача

\задача
\пункт Пусть многочлен $P(x,y)$ равен нулю во всех точках с целыми координатами. Докажите, что это нулевой многочлен (если в его записи привести подобные, то всё сократится).\\
\пункт Решите аналогичную задачу для многочлена от нескольких переменных.
\кзадача


\сзадача
Существует ли многочлен $P(x,y)$, для которого множеством решений неравенства\break $P(x,y) > 0$ является квадрант $\{(x,y) : x>0, y>0\}$.
\кзадача

\сзадача
Существует ли такой многочлен $P(x, y)$, что $P(x, y) > 0$ для любых $x$, $y$, но $P$ принимает
значения, сколь угодно близкие к 0?
\кзадача

\ЛичныйКондуит{0mm}{6.5mm}
% \GenXMLW

\end{document}

