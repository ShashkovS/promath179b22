\documentclass[a4paper,12pt]{article}
\usepackage[mag=1000]{newlistok}
\usepackage{tikz}
\usetikzlibrary{calc}

\УвеличитьШирину{1.3truecm}
\УвеличитьВысоту{2.5truecm}

\Заголовок{Характеристические функции}
\НомерЛистка{20}
%\renewcommand{\spacer}{\vfill}
\ДатаЛистка{23.05 -- 30.05/2018}
\Оценки{17/13/9}

\newcommand{\0}[1]{\overline{#1}}

%\documentstyle[11pt, russcorr, listok]{article}
%\newcommand{\del}{\mathrel{\raisebox{-.3 ex}{${\vdots}$}}}

\begin{document}

\СоздатьЗаголовок

Пусть все изучаемые нами в данный момент множества являются подмножествами некоторого множества $\mathcal{U}$.
Для формального доказательства тождеств с этими множествами применяют \emph{метод
характеристических функций}. \emph{Характеристической функцией
$\chi_{{}_A}$ множества} $A$ называют функцию на множестве
$\mathcal{U}$, определённую так:
$$
\chi_{{}_A}(x)=
\begin{cases}
1,&\text{если }x\in A,\\
0,&\text{если }x\notin A.
\end{cases}
$$
Ясно, что само множество $A$ однозначно восстанавливается по своей
характеристической функции:
$A=\{x\in\mathcal{U}\mid\chi_{{}_A}(x)=1\}$.

Очевидно, что для характеристической функции произвольного множества $A$ верно
равенство ${\chi_{{}_A}}^2=\chi_{{}_A}$ (имеется в виду, что для
любого $x\in\mathcal{U}$ выполнено равенство
$\left({\chi_{{}_A}(x)}\right)^2=\chi_{{}_A}(x)$).

\задача
Докажите равенства:
\пункт $\chi_{{}_{A\cap B}}=\chi_{{}_A}\cdot\chi_{{}_B}$;
\пункт $\chi_{{}_{A_1\cap A_2\cap\ldots\cap A_n}}=\chi_{{}_{A_1}}\cdot\chi_{{}_{A_2}}\cdot\ldots\cdot\chi_{{}_{A_n}}$;
\пункт $\chi_{{}_{\0A}}=1-\chi_{{}_A}$;
\пункт $\chi_{{}_{A\cup B}}=\chi_{{}_A}+\chi_{{}_B}-\chi_{{}_A}\cdot\chi_{{}_B};$
\пункт $\chi_{{}_{A\setminus B}}=\chi_{{}_A}-\chi_{{}_A}\cdot\chi_{{}_B}$;
\пункт $B\subset A$ равносильно тому, что $\chi_{{}_A}\cdot\chi_{{}_B}=\chi_{{}_B}$.
\кзадача

\medskip

\noindent
\textit{%
Докажем с помощью характеристических функций тождество $\0{A\cup
B}=\0A\cap\0B$. 
\\ Для его левой части
$
\chi_{{}_{\0{A\cup
B}}}=1-(\chi_{{}_A}+\chi_{{}_B}-\chi_{{}_A}\cdot\chi_{{}_B})=
1-\chi_{{}_A}-\chi_{{}_B}+\chi_{{}_A}\cdot\chi_{{}_B}.
$
\\Для правой части
$
\chi_{{}_{\0A\cap\0B}}=(1-\chi_{{}_A})\cdot(1-\chi_{{}_B})=
1-\chi_{{}_A}-\chi_{{}_B}+\chi_{{}_A}\cdot\chi_{{}_B}.
$
\\Так как характеристические функции равны, то равны и множества.
Тождество доказано.
}
\medskip



\задача
Какие из следующих тождеств верны и почему:
%\пункт $(A\cap B)\cup C=(A\cup C)\cap (B\cup C)$;
\пункт $A\setminus(B\setminus C)=(A\setminus B)\cup (A\cap C)$;
\пункт $(A\setminus B)\setminus C=(A\setminus C)\setminus(B\setminus C)$;
\пункт $A\cup (B\setminus C)=(A\cup B)\setminus C$;
\пункт $(A\setminus B)\cup C=(A\cup C)\setminus(B\cup C)$;
\пункт $(A\setminus B)\cap C=(A\cap C)\setminus(B\cap C)=(A\cap C)\setminus B$?
\кзадача


\задача
Докажите, что $\chi_{{}_{\0{A_1\cup A_2\cup\ldots\cup A_n}}}=(1-\chi_{{}_{A_1}})(1-\chi_{{}_{A_2}})\cdot\ldots\cdot(1-\chi_{{}_{A_n}})$
% Выразите характеристическую функцию объединения $n$ множеств
% через характеристические функции самих множеств
% для
% \вСтрочку
% \пункт $n=3$;
% \спункт $n\in\N$
и выведите отсюда формулу включений-исключений.
\кзадача


\сзадача
По пустыне идёт караван из 9 верблюдов. Путешествие длится много дней, и наконец всем надоедает видеть впереди себя одного и того же верблюда. Сколькими способами можно переставить верблюдов так, чтобы впереди каждого верблюда шёл другой верблюд, чем раньше?
\кзадача

\сзадача
Сколько всего различных операций от трёх множеств можно выразить
через объединение, пересечение и дополнение? (Тождественно равные
операции считаются совпадающими.)
\кзадача

\раздел{***}

\опр
\emph{Симметрической разностью} множеств $A$ и $B$ называется
множество
$$
A\mathop\bigtriangleup B=(A\setminus B)\cup
(B\setminus A).
$$
\копр

\vspace*{-5mm}
\задача
Изобразите следующие множества на диаграммах
Эйлера--Венна и выразите через $\chi_{{}_A}$, $\chi_{{}_B}$ и
$\chi_{{}_C}$ их характеристические функции:
\пункт $A\mathop\bigtriangleup B$;
\пункт $(A\mathop\bigtriangleup B)\mathop\bigtriangleup C$.
\кзадача


Характеристическая функция принимает только значения 0 и 1. Поэтому, чтобы узнать значение характеристической функции, достаточно понять, чётно оно или нет: если чётно, то значение равно 0, а если нечётно --- то 1. Значит, можно вычислять значение характеристической функции <<по модулю 2>>. Обозначим операцию сложения по модулю 2 символом $\oplus$ (тогда $0\oplus0=0, 0\oplus1=1\oplus0=1, 1\oplus1=0$).

\задача %Обозначим операцию сложения по модулю 2 символом $\oplus$.
Докажите, что
\пункт $\chi_{{}_{A\mathop\bigtriangleup B}}=\chi_{{}_{A}}\oplus\chi_{{}_{B}}$;
\пункт $\chi_{{}_{A_1\mathop\bigtriangleup\ldots\mathop\bigtriangleup A_n}}=\chi_{{}_{A_1}}\oplus\ldots\oplus\chi_{{}_{A_n}}$.
\кзадача

\задача
Докажите, что $A\cap (B\mathop\bigtriangleup C) = (A\cap B)\mathop\bigtriangleup(A\cap C)$.
\кзадача

% символ "равно по определению"
%\newcommand{\eqdef}{\stackrel{\mathrm{def}}{=}}

% для переносов знаков бинарных операций с дублированием
%\newcommand*{\hm}[1]
%{#1\nobreak\discretionary{}{\hbox{$\mathsurround=0pt #1$}}{}}




% \задача
% В библиотеке $n$ книг и несколько читателей
% (каждый прочёл хотя бы одну книгу).~Про~лю\-бые $k$
% книг ($1\leq k\leq n$) известно, сколько читателей прочитало их все.
% Как найти общее число~читателей?
% \кзадача



%\сзадача
%Какое максимальное количество различных множеств можно получить
%из данных~$n$, используя операции $\setminus$, $\cup$ и $\cap\,$?
%\кзадача


%\vfill

\ЛичныйКондуит{0mm}{6mm}
% \GenXMLW


\end{document}

