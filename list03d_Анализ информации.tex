% !TeX encoding = windows-1251
\documentclass[12pt,a4paper]{article}
\usepackage[mag=980, tikz]{newlistok}

\ВключитьКолонтитул
\УвеличитьШирину{1.5cm}
\УвеличитьВысоту{1.5cm}
%\renewcommand{\spacer}{\vspace{3pt}}

\begin{document}


\Заголовок{Анализ информации}
\НадНомеромЛистка{179 школа, 7Б.}
\НомерЛистка{3д}
\ДатаЛистка{27.01.2018}
\СоздатьЗаголовок

{\small В этом листке есть задачи двух типов: в одних надо просто придумать алгоритм, а в других -- ещё и доказать,
что нет более быстрого алгоритма. Для примера разберём такую задачу:

\выд{Есть 9 монет, одинаковых на вид.
Из них одна фальшивая (легче настоящих, которые весят одинаково). За какое наименьшее число взвешиваний
на чашечных весах без гирь можно гарантированно найти эту монету?
}

Докажем, что двух взвешиваний хватит. Разделим монеты на три кучки
по три монеты. Сравним веса 1-й и 2-й кучек.
Если они равны, то фальшивая монета в 3-й кучке, иначе --- в более
лёгкой. Итак, за одно взвешивание мы нашли кучу из трёх монет,
среди которых и фальшивая. Теперь сравниваем
на весах две монеты из этой кучи: если весы в равновесии, то фальшивая монета 3-я,
а если не в равновесии, то фальшивая та, которая легче.

Докажем, что не удастся гарантированно найти монету за 1 взвешивание.
Заметим, что после первого взвешивания все монеты в любом случае разделятся на три группы: в 1-й
будут монеты, попавшие на 1-ю чашу весов, во 2-й --- монеты,
попавшие на 2-ю чашу весов, в 3-й --- монеты, которые не
взвешивались. Количество монет в группах может быть разным,
в какой-то группе может и совсем не быть монет, но главное: в одной из групп будет хотя бы
треть общего количества монет (то есть 3 монеты). И фальшивая монета
может оказаться как раз в этой группе. Но тогда узнать её уже невозможно:
мы истратили взвешивание, и перед нами три монеты (или больше),
которые вели себя при этом взвешивании одинаково, они для нас неразличимы.

Другое рассуждение: наклеим на каждую монету жёлтую бумажку, если она была на левой чаше при взвешивании,
красную --- если на правой, зелёную --- если не взвешивалась. Так как бумажек 3 вида, а монет 9, найдутся хотя бы 2 монеты с одинаковыми бумажками, и мы их не различим, а фальшивой может быть одна из них.

}


\задача
Дано $3^n$ одинаковых с виду монет, одна из них фальшивая (легче настоящих, которые весят одинаково).
За какое наименьшее число взвешиваний на чашечных весах без гирь можно гарантированно найти фальшивую?
\кзадача

\задача
Загадано целое число от 1 до 64. Можно задавать вопросы,
на которые дается ответ \лк да\пк\ или \лк нет\пк. За какое наименьшее число
вопросов всегда можно отгадать число, если
%\сНовойСтроки
\вСтрочку
\пункт
каждый следующий вопрос задаётся после того, как
получен ответ на предыдущий;
\пункт
надо заранее сказать все вопросы?
\кзадача

%2
\задача
В каждую клетку доски $8\times8$ записано целое число от 1 до 64
(каждое по разу).
За~один~вопрос, указав любой набор полей, можно
узнать числа, стоящие на этих полях
(без указания, какую клетку какое число занимает).
За какое наименьшее число
вопросов всегда можно узнать, какие числа где стоят?
\кзадача

\задача
Туристы взяли в поход 80 банок консервов, веса которых известны
и различны (есть список). Вскоре надписи на банках стёрлись
%стали нечитаемыми,
и только завхоз знает, где что. Он хочет
доказать всем, что в какой банке находится, не вскрывая
банок и пользуясь только списком и двухчашечными
весами со стрелкой, показывающей разницу весов на чашах.
Хватит ли ему для этого
\вСтрочку
\пункт четыр\"ех;
\пункт тр\"ех взвешиваний?
\кзадача

\задача
\пункт В жюри олимпиады 11 человек. Материалы
олимпиады хранятся в сейфе. Какое наименьшее число замков должен иметь сейф, чтобы можно было
изготовить сколько-то ключей и так их раздать членам жюри,
чтобы доступ в сейф был возможен если и только если соберётся не менее
6 членов жюри?\\
\пункт Круглая арена цирка освещается $n$ разными прожекторами. Каждый прожектор освещает некую выпуклую фигуру, причём если выключить любой один прожектор, то арена будет по-прежнему полностью освещена, а если выключить любые два --- будет освещена не полностью. При каких $n$ такое возможно?
\кзадача

\задача
\вСтрочку
\пункт
Двое показывают карточный фокус.  Первый  снимает
5 карт из колоды, содержащей 52 карты (заранее перетасованной
кем-то из зрителей), смотрит в них и после этого выкладывает
их в ряд слева направо, причем одну из карт кладет рубашкой
вверх, а остальные --- картинкой вверх. Второй участник фокуса
отгадывает закрытую карту. Докажите, что они могут  так  договориться,
что второй всегда будет угадывать карту.
\спункт
%Второй фокус отличается от первого тем,  что
Та же задача, но первый %участник
выкладывает слева направо четыре карты картинкой вверх,
а одну не выкладывает. Второй должен угадать невыложенную карту.
%Могут ли в этом случае  участники  фокуса
%так договориться,  чтобы  второй  всегда  угадывал  невыложенную
%карту?
\кзадача

\задача
\пункт В тюрьме 100 узников. Надзиратель сказал им:
«Я дам вам поговорить друг с другом, а потом
рассажу по отдельным камерам, и общаться вы уже не
сможете. Иногда я буду одного из вас отводить в комнату,
в которой есть лампа (вначале она выключена). Уходя из
комнаты, можно оставить лампу как включенной, так
и выключенной.
Если в какой-то момент кто-то из вас скажет мне, что
вы все уже побывали в комнате, и будет прав, я всех
выпущу на свободу. А если неправ --- скормлю всех
крокодилам. И не волнуйтесь, что кого-то забудут, ---
если будете молчать, то все побываете в комнате, и ни для
кого никакое посещение комнаты не станет последним.»
Придумайте стратегию, гарантирующую узникам освобождение.
\пункт А если неизвестно, была ли лампа в самом начале включена или нет?
\кзадача

\задача
Вам и мне надевают на голову шляпу. Каждая шляпа либо чёрная, либо белая.
Вы видите мою шляпу, я --- вашу, но никто не видит своей шляпы.
Каждый из нас (не подглядывая и не подавая друг другу никаких сигналов) должен попытаться угадать цвет своей шляпы.
Для этого по команде одновременно каждый называет цвет --- <<чёрный>> или <<белый>>.
Если хоть один угадал --- мы выиграли. Перед этим нам дали возможность посовещаться.
Как действовать, чтобы в любой ситуации выиграть?
\кзадача


\задача
\вСтрочку
\пункт
Мудрецам предстоит испытание: им завяжут глаза, наденут каждому
чёрный или белый колпак, построят в колонну и развяжут глаза.
Затем мудрецы по очереди, начиная с последнего (который видит всех), будут называть
цвет своего колпака. Кто ошибётся --- тому голову с плеч.
Сколько мудрецов гарантированно может спастись?
(Каждый видит всех впереди стоящих; у мудрецов до испытания есть время, чтобы договориться.)
\пункт А если колпаки могут быть $k$ данных цветов?
\спункт А если колпаки могут быть 100 данных цветов,
100 мудрецов стоят по кругу (видят друг друга) и
называют цвета своих колпаков одновременно, и нужно, чтобы цвет своего колпака угадал хотя бы один? % мудрец?
\кзадача

\задача Имеется 1000 бутылок с вином, в одной вино испорчено, и 10 белых мышей. Если мышь выпьет плохого вина, то через минуту станет фиолетовой. Разрешается один раз накапать каждой мыши вина из разных бутылок, дать им выпить одновременно и подождать минуту. Как найти испорченное вино?
\кзадача



\сзадача
Одиннадцати мудрецам завязывают глаза и надевают каждому колпак
одного из 1000 цветов. После этого глаза развязывают, и каждый видит все
колпаки, кроме своего. Затем одновременно каждый показывает
остальным одну из двух карточек --- белую или чёрную. После этого
все должны одновременно назвать цвет своих колпаков.
Как мудрецам заранее договориться, чтобы это удалось?
%Мудрецы могут заранее договориться о своих действиях; мудрецам известно, каких 1000 цветов
%могут быть колпаки.
\кзадача

%4

% \задача
% Обезьяна хочет определить, из окна какого самого низкого этажа
% 15-этажного дома нужно бросить кокосовый орех, чтобы он разбился.
% У нее есть
% \вСтрочку
% \пункт 1;
% \пункт 2;
% \пункт $N$ орехов.
% Какого наименьшего числа бросков ей заведомо хватит?
% (Неразбившийся орех можно бросать снова.)
% \кзадача
%
%3

% \задача
% Задача Маркелова
% \кзадача


%5
%\задача
%\пункт Есть 17 карт. Зритель загадывает одну из них. Фокусник
%раскладывает все карты на 4 стопки и узнает у зрителя, в какой
%стопке оказалась задуманная карта. Докажите, что он всегда может
%определить задуманную карту за 3 вопроса, а двух вопросов может и
%не хватить.
%\пункт При каком наибольшем количестве карт можно наверняка определить
%задуманную карту за 3 вопроса?
%\кзадача

% \задача
% Шляпы с ВМШ
% \кзадача

% \задача
% Вам и мне надевают на голову шляпу. Каждая шляпа либо чёрная, либо белая.
% Вы видите мою шляпу, я --- вашу, но никто не видит своей шляпы.
% Каждый из нас (не подглядывая и не подавая друг другу никаких сигналов) должен попытаться угадать цвет своей шляпы.
% Для этого по команде одновременно каждый называет цвет --- <<чёрный>> или <<белый>>.
% Если хоть один угадал --- мы выиграли. Перед этим нам дали возможность посовещаться.
% Как действовать, чтобы в любой ситуации выиграть?
% \кзадача


% \задача
% \вСтрочку
% \пункт
% Король объявил сотне мудрецов, что устроит им
% испытание. Мудрецам завяжут глаза, наденут каждому
% на голову чёрный или белый колпак, построят в колонну и развяжут глаза.
% Затем мудрецы по очереди, начиная с последнего, будут называть
% цвет своего колпака. Кто ошибётся --- тому голову с плеч.
% Сколько мудрецов гарантированно может спастись?
% (Каждый видит всех впереди
% стоящих; у мудрецов до испытания есть время, чтобы договориться.)\\
% \пункт А если колпаки могут быть $k$ данных цветов?
% \спункт А если колпаки могут быть 100 данных цветов,
% мудрецы стоят по кругу (видят друг друга) и
% называют цвета своих колпаков одновременно, причём
% король помилует всех, если
% цвет своего колпака угадает хотя бы один? % мудрец?
%\кзадача



% \задача
% Петя задумал целое число от 1 до 16. Вася может задавать Пете любые
% вопросы, на которые можно ответить \лк Да\пк\ или \лк Нет\пк.
% Отвечая на эти вопросы, Петя может один раз соврать
% (но неизвестно, когда).
% Как Васе узнать Петино число, задав не более 7 вопросов?
% \кзадача


\сзадача
Секретный код к любому из сейфов ФБР --- это целое число от 1
до 1700. Два шпиона узнали по одному коду каждый и решили ими обменяться.
Согласовав заранее свои действия, они встретились на берегу
реки возле кучи из 26 камней. Сначала 1-й шпион кинул в воду один
или несколько камней, потом --- 2-й, потом --- опять 1-й, и т.д.~до
тех пор, пока камни не кончились. Затем шпионы разошлись.
Каким образом могла быть передана информация?
(Шпионы не сказали друг другу ни слова.)
\кзадача

\задача
В тесте $30$ вопросов, на каждый есть 2 варианта ответа
(один верный, другой нет).
За одну попытку Витя отвечает на все вопросы, после чего ему сообщают,
на сколько вопросов он ответил верно.
Как Вите гарантированно узнать все верные ответы не позже, чем
\пункт после 29-й попытки
(и ответить верно на все вопросы при $30$-й попытке);
\спункт после 24-й попытки
(и ответить верно на все вопросы при $25$-й попытке)?
(Изначально Витя не знает ни одного
ответа, тест всегда один и тот же.)
\кзадача

\задача
Ботанический определитель использует 100 признаков.
Каждый признак либо есть у растения, либо нет.
Определитель считается \лк хорошим\пк, если любые два растения в нём отличаются
более чем по 50 признакам. Может ли хороший определитель
описывать более \вСтрочку
\пункт 50;
\спункт~34~растений.
\кзадача


\сзадача
Барон Мюнхгаузен убил на охоте 15 уток весом 50, 51,~\dots, 64 кг.
Ему известен вес каждой из уток. С помощью чашечных весов барон
собирается доказать зрителям, что первая
утка весит 50~кг, вторая~--- 51~кг, третья~--- 52~кг, и~т.\,д.\ (вначале зрители
не знают про веса уток абсолютно ничего). Какое наименьшее количество гирь
потребуется барону, если и~гири, и~уток можно размещать на обеих чашах весов,
а~количество взвешиваний не ограничено?
(Веса гирь известны как барону, так и зрителям.
В~наличии неограниченный запас гирь весами~1, 2,~\dots, 1000~кг.)
\кзадача

\задача
Фокуснику завязывают глаза, а зритель выкладывает в ряд $N$ одинаковых
монет, сам выбирая, какие --- орлом вверх, а какие --- решкой. Ассистент  фокусника
просит зрителя написать на листе бумаги любое число от 1 до $N$ и показать
всем присутствующим. Увидев число, ассистент
указывает зрителю на одну из монет ряда и просит
перевернуть её.
Затем фокуснику  развязывают
глаза,  он  смотрит на ряд монет и безошибочно определяет написанное
зрителем число.
\пункт  Докажите, что если у фокусника с ассистентом есть способ,
позволяющий фокуснику гарантированно отгадывать число
для $N=a$, то есть способ и для $N=2a$.
\спункт Найдите все $N$, для которых у фокусника с ассистентом
есть способ.
\кзадача

\сзадача
Есть $n$ разных ключей от $n$ разных замков (каждый
ключ подходит ровно к одной двери). За какое наименьшее число
попыток можно гарантированно узнать, какую дверь открывает какой ключ?
\кзадача

\ЛичныйКондуит{0mm}{5mm}

% \GenXMLW
%}

\end{document}
%6
%\задача
%Из 11 шаров два радиоактивны. Про любой набор шаров за одну проверку
%можно узнать, имеется ли в нём хотя бы один радиоактивный шар (но
%нельзя узнать, сколько их).
%%Доказать, что менее чем за 7 проверок
%За какое наименьшее число проверок
%можно гарантированно найти оба радиоактивных шара?
%%, а за 7 проверок их всегда можно обнаружить.
%\кзадача


%---сюжет 2------
%8

%9

%10
%\сзадача
%Имеются пять внешне одинаковых гирь с попарно различными массами.
%Разрешается выбрать любые три из них $A$, $B$ и $C$ и спросить,
%верно ли, что $m(A)<m(B)<m(C)$. (Через $m(x)$ обозначена масса гири
%$x$, при этом дается ответ \лк Да\пк\ или \лк Нет\пк). Можно ли за девять
%вопросов гарантированно узнать, в каком порядке идут веса гирь?
%\кзадача


%---просто задачи-----

%11
%\задача
%Оргкомитет по проведению олимпиады состоит из 9 человек. Материалы
%олимпиады хранятся в сейфе. Сколько замков должен иметь сейф, сколько
%ключей к ним нужно изготовить, и как их раздать членам комитета,
%чтобы доступ в сейф был возможен только тогда, когда соберется не менее
%6 членов комитета?
%\кзадача

% \задача
% Задача про 8 гирек и неточные весы
% \кзадача


%\задача
%Отряд девочек отправился в поход. После того, как они вернулись, их
%родителям стало известно, что хотя бы одна из них искупалась в походе
%без разрешения, и каждый решил высечь свою дочь, если узнает о том, что
%она купалась.
%Каждое утро девочки ходят в школу и обмениваются слухами
%о том, кого вчера высекли,
%а вечером сообщают слухи родителям. Кроме того, в первый же день каждая
%из девочек рассказала своим родителям, кто искупался в походе
%(исключая информацию о том, купалась ли она сама).
%Через 13 дней несколько отцов, получив очередную порцию информации,
%догадались о провинности их дочерей и высекли их. Сколько детей получило
%в этот вечер наказание?
%\кзадача

%13



\задача
Ботанический определитель использует 100 признаков.
Каждый признак либо есть у растения, либо нет.
Определитель \лк хороший\пк, если любые два растения в нем отличаются
более чем по 50 признакам. Может ли хороший определитель
описывать более
\вСтрочку
\пункт 50;
\спункт 34 растений.
\кзадача



\сзадача
Секретный код к любому из сейфов ФБР --- это целое число от 1
до 1700. Два шпиона узнали по одному коду каждый и решили обменяться
информацией. Согласовав заранее свои действия, они встретились на берегу
реки возле кучи из 26 камней. Сначала первый шпион кинул в воду один
или несколько камней, потом --- второй, потом --- опять первый, и т.д.~до
тех пор, пока камни не кончились. Затем шпионы разошлись.
Каким образом могла быть передана информация?
(Шпионы не сказали друг другу ни слова.)
\кзадача

%7
% \сзадача
% \вСтрочку
% \пункт Среди $M$ монет одна фальшивая. Настоящие весят
% одинаково; фальшивая отличается по~весу, но не известно,
% легче она или нет. %Есть еще эталонная (настоящая) монета.
%%Разрешено сделать $n$ взвешиваний на весах с двумя чашами.
% При каком наибольшем $M$ можно
% за $n$ взвешиваний на
% весах с двумя чашами найти фальшивую монету и узнать, легче ли она?
% \пункт А если надо лишь найти фальшивую монету?
% \кзадача
%
\ЛичныйКондуит{0mm}{5mm}

%
%\задача
%\пункт Фирма "Мелкософт" выпустила новый архиватор. Представитель фирмы
%утверждает, что этот архиватор уменьшает размер любого файла.
%Не ошибается ли он?
%\пункт Утверждается, что любой файл, который этот архиватор
%ужимает (а не увеличивает), уменьшается в размере по крайней мере на
%четверть. Какое наибольшее число файлов размера не более $2^6$ байт
%может ужиматься?
%\кзадача

%
%\задача
%из Кабатянского
%\кзадача



% \vspace*{-2mm}
\ЛичныйКондуит{0mm}{6mm}
% \vspace*{-3mm}

\end{document}

\vspace*{-0.2truecm}

\раздел{$***$}

\vspace*{-0.2truecm}

\задача
%\пункт
В НИИ работают 67 человек. Из них
47 знают английский язык, 35 --- немецкий, и 23 --- оба языка.
Сколько человек в НИИ
не знают ни английского, ни немецкого языков?
\пункт Пусть кроме этого  польский
знают 20 человек, английский и польский --- 12, немецкий и
польский --- 11, все три языка --- 5.
Сколько человек не знают ни одного из этих языков?
%\спункт [Формула включений и исключений]
%Решите задачу в общем случае: имеется $m$ языков,
%и для каждого набора языков известно, сколько человек знают все языки
%из этого набора.
\кзадача

\задача
В ряд записали 105 единиц, поставив перед каждой знак \лк$+$\пк.
Сначала изменили знак на противоположный перед каждой третьей единицей,
затем --- перед каждой пятой, а затем --- перед
каждой седьмой. Найдите значение полученного выражения.
\кзадача

\задача \вСтрочку
\пункт
На полке стоят 10 книг. Сколькими способами их можно переставить
так,~чтобы ни одна книга не осталась на месте?
\пункт А если на месте должны остаться ровно 3 книги?
\кзадача
