% !TEX encoding = Windows Cyrillic
\documentclass[a4paper,11pt]{article}
\usepackage[mag=970]{newlistok}
\usepackage{tikz}
\usetikzlibrary{calc}

\УвеличитьШирину{1.1cm}
\УвеличитьВысоту{2.5cm}



\Заголовок{Аксиома полноты}
\НомерЛистка{43}
\renewcommand{\spacer}{\vspace{1pt}}
\ДатаЛистка{13.01.2020 -- 29.01.2020}
\Оценки{33/25/17}

\begin{document}

\СоздатьЗаголовок



\опр Пусть дано подмножество $M$ множества действительных чисел $\R$. \\
Число  $c\in\R$ называют \выд{верхней гранью} множества $M$, если $c\geq m$ для всех $m\in M$.\\
Число  $c\in\R$ называют \выд{точной верхней гранью} множества $M$,
если $c$ является верхней гранью %множества
$M$, но никакое меньшее число
не является верхней гранью %множества
$M$.
Обозначение: $\sup M$ (читается \лк супр\'емум\пк{}  %множества
$M$).\\
Аналогично определяется точная нижняя грань множества $М$ ($\inf M$, \лк инф\'имум\пк{} $M$).
%, т.~н.~г.~$M$).
%Множество, ограниченное и сверху, и снизу, называют
%\выд{ограниченным}.
%для любого действительного числа $C$ $$  \exists C \in \R: \qquad \forall x\in M \quad |x| < C.$$
\копр


%\задача \пункт Докажите, что конечное объединение
%ограниченных множеств ограничено.  \пункт Верно ли то же
%самое для счетного объединения ограниченных множеств?
%\пункт Верно ли то же самое для счетного пересечения ограниченных множеств?
%\кзадача

%\опр
%%Число $C\in \R$ называется \выд{верхней гранью}
%%множества $M\subset \R$, если при любом $x\in M$
%%выполнено неравенство $x\leq C$.
%%Пусть $\Bbbk$ --- упорядоченное поле.
%Элемент $c\in\Bbbk$ % упорядоченного поля $\Bbbk$
%называют \выд{точной верхней гранью} множества
%$M\subset \Bbbk$, если $c$ является
%верхней гранью %множества
%$M$, но никакой меньший элемент
%не будет верхней гранью %множества
%$M$.
%Обозначение: $\sup M$ (читается \лк супремум\пк{}  %множества
%$M$).
%%т.~в.~г.~$M$.
%\копр



\задача
Докажите, что число $c$ есть $\sup M$ тогда и только тогда, когда
выполнены два условия:\\
1) для всех $x\in M$ верно, что $x\leq c$; \quad
2) для любого числа $c_1<c$ найд\"ется такое $x\in M$, что $x>c_1$.
\кзадача

%\задача Докажите, что если у множества $M$ существует точная верхняя (нижняя)
%грань, то она единственна.
%\кзадача

\задача Может ли у множества быть несколько точных верхних
(нижних) граней?
\кзадача

\пзадача
%Найдите т.~в.~г.~и т.~н.~г.~множеств: %а (если они существуют):
Найдите $\sup M$ и $\inf M$, если
\вСтрочку
\пункт
$M=\{a^2+2a\ |\  -5<a\leq 5\}$;
\пункт
$M=\{\pm \frac{n}{2n+1}\ |\ n\in\N\}$.
\кзадача

\задача
Последовательность $(a_n)$ имеет предел.
Докажите, что какой-то из её членов
совпадает либо с точной верхней гранью
множества $\{a_1,a_2,a_3,\dots\}$, либо с точной нижней гранью этого
множества.
\кзадача

%\задача \label{sqrt}
%\вСтрочку
%\пункт Докажите, что не существует такого $q \in \Q$, что $q^2 = 2$.
%Докажите, что %т.~в.~г.~множества
%$\sup\ \!\{q \in \Q\ |\ q>0 \mbox{ и } q^2 < 2\}$
%никакое
%не может быть рациональным числом.
% не является точной верхней гранью множества
%\кзадача

%\begin{quote}
%\small В действительных числах, как и в рациональных, можно выполнять
%сложение, вычитание, умножение и деление на число, отличное от нуля. Можно также
%сравнивать числа между собой. Важным отличием действительных чисел от рациональных
%является их \выд{полнота}.
%\end{quote}

\пзадача Пусть $A$ и $B$ --- некоторые подмножества $\R$,
и пусть известны $\sup A$ и $\sup B$.\\
\вСтрочку
\пункт Найдите $\sup (A \cup B)$. \quad
\пункт Найдите $\sup (A+B)$,
где $A+B = \{ a+b\ |\ a\in A, b\in B\}$.\\ %\stackrel{\mbox{\small def}}
\пункт Найдите $\inf(A\cdot B)$, где
$A\cdot B=\{ a\cdot b\ |\ a\in A, b\in B\}$,
если $A$ и $B$ состоят из отрицательных чисел.
\кзадача

\smallskip
\noindent {\bfseries Аксиома полноты.}
\выд{Всякое ограниченное сверху непустое
подмножество в $\R$ имеет точную верхнюю грань.}
%\end{itemize}

\smallskip

\задача
%Сформулируйте и докажите аксиому о точной нижней грани.
Каждое ли ограниченное снизу непустое подмножество в $\R$
имеет точную нижнюю грань?
\кзадача

\ввпзадача [Теорема Вейерштрасса]
Докажите, что любая
неубывающая ограниченная сверху последовательность
действительных чисел имеет предел.
\кзадача

\пзадача Найдите пределы %следующих
последовательностей:
\вСтрочку
\пункт $x_1=2$, $x_{n+1}=(x_n+1)/2$;
\пункт $y_1=\sqrt2$, $y_2=\sqrt{2\sqrt2}$,\\
$y_3=\sqrt{2\sqrt{2\sqrt2}}$, \dots; %\quad
\пункт $z_1=\sqrt2$, $z_2=\sqrt{2+\sqrt2}$,
$z_3=\sqrt{2+\sqrt{2+\sqrt2}}$, \dots; %\quad
\спункт $t_1=1$, $t_{n+1}=1/(1+t_n)$.
\кзадача

\пзадача
Пусть $a_1=1$. Ограничена ли последовательность
\пункт $a_{n+1}=a_n + \frac1{a_n}$;
\пункт $a_{n+1}=a_n + \frac1{a_1+\ldots+a_n}$?
\кзадача

\задача
Докажите, что последовательность
$\displaystyle{x_n=1-1/2+1/3-...+(-1)^{n+1}/n}$
имеет предел.
\кзадача




%\задача [Принцип вложенных отрезков] Пусть
%$
%[a_1,b_1]\supset [a_2,b_2] \supset \dots \supset [a_n, b_n] \supset \dots
%$
%--- последовательность вложенных отрезков. Докажите, что:
%%\сНовойСтроки
%\вСтрочку
%\пункт пересечение этих отрезков непусто;\\
%%($\bigcap_{i=1}^{\infty} [a_i, b_i] \ne \emptyset;$
%\пункт если $\lim\limits_{i\to\infty} (b_i-a_i) = 0$, то эти отрезки имеют
%единственную общую точку.
%\кзадача

\ввпзадача [Принцип вложенных отрезков] Пусть
$
[a_1,b_1]\supseteq [a_2,b_2] \supseteq \dots
$
--- последовательность вложенных отрезков. Докажите, что
%\сНовойСтроки
\вСтрочку
\пункт у отрезков есть общая точка;
%($\bigcap_{i=1}^{\infty} [a_i, b_i] \ne \emptyset;$
\пункт если $\lim\limits_{k\to\infty} (b_k-a_k) = 0$,
то общая точка %ровно
одна.
\кзадача


\задача
\вСтрочку
\пункт
Любая ли последовательность $(a_1,b_1)\supseteq(a_2,b_2)\supseteq
\dots$
% \supseteq (a_n,b_n)\supseteq \dots $
вложенных интервалов имеет общую~точ\-ку?
\пункт
А если %известно, что
и %последовательность
в $(a_n)$ и
%последовательность
в $(b_n)$
бесконечно много разных элементов?
\пункт Верна ли <<аксиома полноты>>~в~$\Q$?
\кзадача


\задача \пункт На прямой дано некоторое (бесконечное) множество отрезков.
Известно, что любые два из них имеют общую точку.
Докажите, что существует точка, принадлежащая всем отрезкам.\\
\пункт Верна ли та же задача для прямоугольников на плоскости, если их стороны параллельны осям координат?
\кзадача

\задача Верно ли, что последовательность вложенных кругов на плоскости всегда имеет общую точку?
\кзадача

\задача
На отрезке отметили бесконечное множество точек. Докажите, что\\
\пункт хотя бы одна из половин этого отрезка содержит бесконечно много отмеченных
точек;\\
\пункт на этом отрезке
найдётся точка $x$, %удовлетворяющая условию:
любая окрестность %точки $x$
которой содержит бесконечно много отмеченных
точек.
\кзадача

%\задача Каждое ли ограниченное снизу непустое множество $M\subset \R$
%имеет точную нижнюю грань?
%\кзадача

\ввпзадача[Лемма Больцано -- Вейерштрасса] Докажите, что любая ограниченная последовательность имеет сходящуюся подпоследовательность.
\кзадача





%\задача Назовем $\Q$-отрезком пересечение отрезка с множеством рациональных
%чисел. Верно ли, что любая последовательность вложенных $\Q$-отрезков имеет
%непустое пересечение?
%\кзадача

%\сзадача Докажите, что множество точек отрезка несчетно.
%\кзадача

\ввпзадача
\вСтрочку
\пункт [Компактность отрезка]
Отрезок покрыт интервалами. Докажите, что
можно выбрать конечное число этих интервалов так,
что они покроют отрезок.
%\кзадача
%\задача
\пункт
%Будет ли верным утверждение пункта а), если в н\"ем
А если заменить отрезок на интервал?
\кзадача


% \задача
% [Компактность отрезка]
% Отрезок покрыт системой интервалов. Докажите, что
%  можно выбрать из системы конечное число интервалов,
% покрывающих отрезок.
% \кзадача
%
% \задача
% Верно ли утверждение предыдущей задачи, если в ней
% заменить отрезок на интервал?
% \кзадача





%\задача
%Докажите, что последовательность
%$\displaystyle{x_n=1-\frac12+\frac13-...+\frac{(-1)^{n+1}}{n}}$
%имеет предел.
%\кзадача


%\задача[Число Эйлера]
%Докажите, что
%\вСтрочку
%\пункт
% Докажите, что последовательность
%$\displaystyle{x_n=(1+1/n)^n}$
%возрастает и ограничена (а значит, имеет предел
%$\lim\limits_{n \to \infty}x_n$; его обозначают буквой $e$).
%Докажите, что
%существует
%$\displaystyle{\lim\limits_{n \to \infty}(1+1/n)^n}$
%(его обозначают буквой $e$);\\
%\пункт
%$\displaystyle{\lim\limits_{n \to
%\infty}(1+k/n)^n=e^k}$ при любом %натуральном
%$k\in\N$;
%\пункт
%$\displaystyle{\lim\limits_{n \to
%\infty}(1-1/n)^n=1/e}$;
%\пункт
%Докажите, что
%$e=\lim\limits_{n \to \infty}(1+1/1!+1/2!+\dots+1/n!)$.
%\кзадача

%\сзадача
%Докажите, что
%$\displaystyle{e=1+1/1!+1/2!+1/3!+\dots}$.
%\кзадача


%\задача Для вычисления квадратного корня из положительного
%числа $a$ можно пользоваться следующим методом
%последовательных приближений. Возьмите любое положительное число
%$x_0$ и постройте последовательность по такому закону:
%$x_{n+1}=0,5\cdot(x_n+a/x_n).$\\
%%\сНовойСтроки
%\вСтрочку
%\пункт
%Докажите, что $\lim\limits_{n\to\infty}x_n=\sqrt a$.
%\спункт Сколько понадобится последовательных приближений,
%чтобы найти значение $\sqrt{10}$ с точностью до $0,0001$,
%если в качестве первого приближения взять $x_0=3$?
%\кзадача

\задача[Вычисление квадратного корня методом последовательных
приближений] Пусть $a>0$. Возьмем любое %положительное число
$x_0>0$ и построим последовательность $(x_n)$ по закону:
$x_{n+1}=0{,}5\cdot(x_n+a/x_n)$ при $n\in\N$.\\
%\сНовойСтроки
\вСтрочку
\пункт
Докажите, что $\lim\limits_{n\to\infty}x_n=\sqrt a$.
\спункт Для $a=10$ найдите такое $n$, что
%Сколько понадобится последовательных приближений,
%приближение
$|x_n-\sqrt{10}|<0{,}0001$, если %начальное приближение
$x_0=3$.
\кзадача


\пзадача[Существование корня]
\вСтрочку
\пункт
Докажите существование квадратного корня из положительного числа с помощью аксиомы полноты. ({\sl Указание:} докажите, что если
$c=\sup\{ q \in \Q\ |\ q>0 \mbox{ и } q^2 < 2\}$, то $c^2 = 2$.)\\
%\кзадача
%
%\задача
%\пункт
%Докажите, что для любого $a\geq0$ есть %существует
%ровно одно такое %неотрицательное число
%$x\geq0$, что $x^2=a$ (это число
%обозначают $\sqrt{a}$).
%\кзадача
%\задача
%\пункт
%Является ли $\Q$ полным? % поле~$\Q$? % рациональных чисел?
\спункт
Для любого $a\geq0$ и любого $r\in\Q$ определите $a^r$ и докажите его существование и единственность.
\кзадача



\ввпзадача \пункт [Критерий Коши]
Докажите, что последовательность $(x_n)$ \выд{сходится}
(то есть имеет предел) тогда и
только тогда, когда она фундаментальна (см. листок 30): \
$\forall \ \varepsilon>0 \quad \exists \ k\in\N\quad \forall \
m,n\ge k
\quad |x_m-x_n|<\varepsilon$.
Верен ли критерий Коши для последовательностей из \пункт рациональных чисел; \спункт комплексных чисел?
\кзадача


% \сзадача На прямоугольную карту положили карту той же
% местности, но меньшего масштаба
% (меньшая карта целиком лежит внутри большей).
% Докажите, что можно проткнуть иголкой сразу обе карты так,
% чтобы точка прокола изображала на обеих картах одну и ту же
% точку местности.
% \кзадача


\задача
\пункт
С помощью аксиомы полноты докажите аксиому Архимеда: для любого $c \in \R$ найд\"ется такое $n\in \N$, что $n>c$.
\спункт Выведите из принципа вложенных отрезков и аксиомы Архимеда аксиому полноты.
%точной верхней грани.
\кзадача


% \задача Докажите, что любая последовательность имеет монотонную подпоследовательность.
% \кзадача


\ЛичныйКондуит{0mm}{6mm}


%\СделатьКондуит{5mm}{7.5mm}




% \GenXMLW

\end{document}

\раздел{$***$}

\опр Число $x\in \R$ называется \выд{предельной точкой}
множества $M\subset \R$,
если в любой окрестности $x$ содержится бесконечно много чисел из $M$.
% для всякого $\varepsilon > 0$ множество $(x-\epsilon,x+\epsilon)\cap M$
%бесконечно.
\копр


\задача Найдите все предельные точки следующих числовых множеств: %\сНовойСтроки
\пункт конечное множество;\\
\пункт $\{ \frac 1n\ | \ n\in \N\}$;
\пункт $\Z$;
\пункт $(0,1)$;
\пункт $\Q$;
\пункт рациональные числа, знаменатели которых --- степени двойки.
\кзадача

%\задача Заменим в определении 3 слово \лк бесконечно\пк{}
%на слова \лк содержит не менее двух элементов\пк. Будет ли это определение
%эквивалентно старому?
%\кзадача


%\задача Верно ли, что точная верхняя грань бесконечного множества
%является его предельной точкой?
%\кзадача

%\задача Пусть $M\subset \R$ --- ограниченное множество, $A$ --- множество
%предельных точек M.
%Докажите, что для всякого $\varepsilon > 0$ множество
%$\{x\in M\ |\ x>\sup A +\varepsilon\}$ конечно.
%\кзадача

\задача \пункт Пусть $a$ --- предельная точка множества $M$. Докажите, что
существует такая последовательность $(x_n)$ элементов этого множества, что
$\lim\limits_{n \to \infty} x_n = a$.
\пункт Верно ли обратное утверждение?
\кзадача


\задача
Докажите следующие утверждения: \quad
\вСтрочку
%\сНовойСтроки
\пункт Для любого $c \in \R$
найд\"ется такое $n\in \N$, что $n>c$.\\
\пункт Для любого $\varepsilon > 0$ найд\"ется  такое $n\in \N$, что
$\frac 1n<\varepsilon$. \quad
\пункт[Аксиома Архимеда] Для любого положительного $h\in\R$ и для
любого $a\in\R$ существует и единственно такое целое число $n$, что
$nh\le a<(n+1)h$.
\кзадача

\задача Докажите, что любой отрезок из $\R$ содержит
%между любыми двумя различными действительными числами есть
бесконечно много чисел
\вСтрочку
\пункт
из $\Q$;
%рациональных;
%чисел;
\пункт
%ррациональных чисел.
из $\R\setminus\Q$.
\кзадача

\задача Выведите из принципа вложенных отрезков и аксиомы Архимеда
аксиому полноты.
%точной верхней грани.
\кзадача

\сзадача
Докажите единственность действительных чисел: для двух полных
упорядоченных полей существует биекция $f$ между ними такая,
что $f(a+b)=f(a)+f(b)$,
$f(a\cdot b)=f(a)\cdot f(b)$
и если $a\leqslant b$, то
$f(a)\leqslant f(b)$.
\кзадача

\сзадача
Докажите существование действительных чисел
(считая известными натуральные числа).
\кзадача


\опр Число $x\in \R$ называется \выд{предельной точкой}
множества $M\subset \R$,
если для всякого $\varepsilon > 0$ множество $(x-\epsilon,x+\epsilon)\cap M$
бесконечно.
\копр


\задача Найдите все предельные точки следующих множеств: \сНовойСтроки
\пункт конечное множество;
\пункт множество $\{ \frac 1n: n\in \N\}$;
\пункт множество целых чисел;
\пункт интервал $(0,1)$;
\пункт множество рациональных чисел;
\пункт множество рациональных чисел, знаменатель которых есть степень двойки.
\кзадача

\задача Заменим в определении 3 слово \лк бесконечно\пк{}
на слова \лк содержит не менее двух элементов\пк. Будет ли это определение
эквивалентно старому?
\кзадача


\задача Верно ли, что точная верхняя грань бесконечного множества
является его предельной точкой?
\кзадача

\задача Пусть $M\subset \R$ --- ограниченное множество, $A$ --- множество
предельных точек M.
Докажите, что для всякого $\varepsilon > 0$ множество
$\{x\in M\ |\ x>\sup A +\varepsilon\}$ конечно.
\кзадача

\задача \пункт Пусть $a$ --- предельная точка множества $M$. Докажите, что
существует такая последовательность $(x_n)$ элементов этого множества, что
$\lim\limits_{n \to \infty} x_n = a$.
\пункт Верно ли обратное утверждение?
\кзадача

\задача Докажите, что из любой ограниченной последовательности можно
выделить сходящуюся подпоследовательность.
\кзадача

\end{document} 