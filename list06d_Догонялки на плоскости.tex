\documentclass[a4paper, 12pt]{article}
\usepackage{newlistok}
\usepackage[matrix,arrow]{xy}
%\newcommand{\ord}{\operatorname{ord}}
\renewcommand{\spacer}{\vspace*{1pt}}

%\def\hang{\hangindent\parindent}

%\УвеличитьШирину{1.2cm}
\УвеличитьВысоту{2.5cm}

\begin{document}

\Заголовок{Догонялки на плоскости}
\Подзаголовок{}
\НомерЛистка{6д}
\ДатаЛистка{19.05.2018}
\Оценки{10/7/4}

%\ncopy{2}{
\СоздатьЗаголовок

\раздел{Конус Маха}

\задача
В поле проходит прямая дорога, по ней со скоростью 10 км/ч едет автобус. Укажите все точки поля, из которых можно догнать автобус, если бежать\\
\пункт с той же скоростью;
\пункт со скоростью 5 км/ч.
\кзадача

\задача
В поле проходит прямая дорога. Человек, стоящий на дороге в точке $A$, может  идти по полю со скоростью не более 3 км/ч и по дороге со скоростью не более 6 км/ч. Нарисуйте, куда он может попасть~за~1~ч.
\кзадача

\задача
Самолёт, летящий в два раза быстрее скорости звука, вылетел из точки $A$ и летит в точку $B$ по прямой. Самолёт непрерывно издаёт звук, который распространяется во все стороны. Нарисуйте все точки, до которых успеет дойти звук самолёта за время, пока самолёт летит из $A$ в $B$.
(Считайте, что всё происходит в плоскости.)
\кзадача


\задача
В поле проходят две перпендикулярные друг другу прямые дороги. Человек, стоящий на перекрестке, может идти по полю со скоростью не более 3 км/ч и по дорогам со скоростью не более \пункт 6 км/ч; \пункт $3\sqrt2$ км/ч. Нарисуйте все точки, в которые он может попасть за 1 час.
\кзадача

\сзадача
Пункт $A$ находится в лесу в 5 км от прямой дороги, пункт $B$ --- на дороге, расстояние от $A$ до $B$ --- 13 км (по полю). Скорость пешехода на дороге – 5 км/ч, в лесу – 3~км/ч. За какое наименьшее время пешеход сможет попасть из $A$~в~$B$?
\кзадача

\раздел{Найди стратегию}

\задача
Миша стоит в центре круглой лужайки радиуса 100 м.
Каждую минуту он шагает на 1 м, заранее объявляя, в
каком направлении хочет шагнуть.
Катя имеет право заставить его сменить направление
на противоположное. Может ли Миша действовать так, чтобы
когда-нибудь гарантированно выйти с лужайки?
%, или Катя всегда сможет ему помешать?
\кзадача

\задача
В центре квадрата сидит заяц, в каждом %из %четыр\"ех
углу --- %одном
волк. Может ли заяц выбежать из квадрата, если волки %могут
бегают лишь
по сторонам квадрата,
отношение максимальных скоростей волка и зайца равно 1,4.
%  с мак\-си\-маль\-ной скоростью, большей %которая больше
% максимальной скорости зайца
% в 1,4 раза? %;
%\пункт в 1,5 раза?
\кзадача

\задача
На плоскости играют %двое: один передвигает
%одну фишку-волка, другой --- несколько фишек-овец.
волк и несколько овец. Сначала ходит волк, потом
какая-нибудь овца, потом волк, потом опять
какая-нибудь овца, и т.~д. И волк и овцы передвигаются за ход
в любую сторону не более, чем на 1 м. Для любого ли числа
овец существует такая начальная позиция, что волк не поймает ни одной~овцы?
\кзадача

\сзадача
Город представляет собой бесконечную клетчатую плоскость (линии --- улицы,
клеточки --- кварталы). На одной из улиц через каждые 100 кварталов на
перекр\"естках стоит по милиционеру. Где-то в городе есть бандит
(его местонахождение неизвестно, но перемещается он только по улицам).
Цель милиции --- увидеть бандита.
Есть ли у милиции алгоритм наверняка достигнуть своей цели?
Максимальные скорости милиции и бандита
--- какие-то конечные, но неизвестные нам величины (у бандита
скорость может быть больше, чем у милиции). Милиция видит вдоль
улиц во все стороны на бесконечное расстояние.
\кзадача

\раздел{Ловим в несколько этапов}

\задача
На плоскости играют Левша и невидимая блоха.
За ход Левша проводит прямую, а блоха прыгает на 1 м, не пересекая ни одной прямой Левши (иначе проигрывает).
Может ли Левша гарантированно выиграть?
\кзадача

\задача
%Полицейский участок расположен на прямой дороге, бесконечной в обе стороны.
Некто угнал старую полицейскую машину, максимальная скорость которой
составляет 90\% от максимальной скорости новой, и едет по бесконечной в обе стороны дороге.
% В некий момент
%в участке спохватились и послали вдогонку %на поимку угонщика своего лучшего
%полицейского на новой машине.
Полицейский на новой машине не знает, ни когда машину угнали, ни в каком направлении
%вдоль %бесконечной в обе стороны прямой
%дороги
уехал угонщик.
Сможет ли полицейский поймать~угонщика?
% По бесконечной в обе стороны дороги едет угонщик. В некой точке дороги находится полицейский, максимальная скорость машины которого
% составляет 90\% от максимальной скорости
\кзадача

\сзадача
На бесконечной клетчатой сетке (линии --- улицы, клетки --- кварталы) трое полицейских
ловят вора. Местонахождение вора неизвестно, но перемещается он только по улицам.
Максимальные скорости у полицейских и вора одинаковы.
Вор считается пойманным, если он оказался на одной улице с полицейским.
Смогут ли полицейские гарантированно поймать вора? (Полицейские тоже движутся только по улицам.)
\кзадача

% \задача
% \кзадача
%
% \задача
% \кзадача

\vspace*{-1mm}
\ЛичныйКондуит{0mm}{6mm}
% \GenXMLW
%}

\end{document}

