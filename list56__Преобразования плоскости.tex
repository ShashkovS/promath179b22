% !TeX encoding = windows-1251

\documentclass[a4paper,12pt]{article}
\usepackage{newlistok}

%\usepackage{mathtools}


%\УвеличитьШирину{1truecm}
%\УвеличитьВысоту{2truecm}

\Заголовок{Линейные преобразования плоскости}
\НомерЛистка{56}
\ДатаЛистка{09.11.2020 -- 27.11.2020}
\Оценки{30/24/18}
\ВключитьКолонтитул

\begin{document}


	\СоздатьЗаголовок
	
	Рассмотрим двумерное линейное пространство над $\R$. Отождествляя векторы (рассматриваемые как радиус-вектор из начала координат) с точками на плоскости, можно считать, что векторы --- это точки из  $\R^2$.
	
	Точку с декартовыми координатами $x$ и $y$ будем обозначать через $\begin{pmatrix} x \\ y \end{pmatrix}$.

\опр {\it Линейным отображением} плоскости в себя называется отображение вида 

$$\begin{pmatrix} x \\ y \end{pmatrix} \mapsto \begin{pmatrix} ax+by \\ cx+dy \end{pmatrix}.$$

Таблица вида $\begin{pmatrix} a & b  \\ c & d \end{pmatrix}$ называется {\it матрицей} этого отображения. \копр

\пзадача Докажите, что тождественное отображение имеет матрицу $E :=  \begin{pmatrix} 1 & 0  \\ 0 & 1 \end{pmatrix}.$\кзадача

\пзадача Найдите образы точек $\begin{pmatrix} 1 \\ 0 \end{pmatrix}$, $\begin{pmatrix} 0 \\ 1 \end{pmatrix}$, $\begin{pmatrix} 1 \\ 1 \end{pmatrix}$, а также образ единичного квадрата при отображении с матрицей

 \пункт $\begin{pmatrix} 1 & 0  \\ 0 & 1 \end{pmatrix}$;  \пункт $\begin{pmatrix} 0 & 1  \\ -1 & 0 \end{pmatrix}$;  \пункт $\begin{pmatrix} 1 & 0  \\ 0 & 0 \end{pmatrix}$;  \пункт $\begin{pmatrix} 0 & 1  \\ 0 & 0 \end{pmatrix}$; \пункт  $\begin{pmatrix} 1 & 1  \\ -1 & 1 \end{pmatrix}$;  \пункт $\begin{pmatrix} 1 & 1  \\ 0 & 1 \end{pmatrix}$;  \пункт$\begin{pmatrix} 2 & 1  \\ 1 & 0 \end{pmatrix}$.

\кзадача
	
\пзадача Найдите все матрицы линейных преобразований, переводящих

\пункт  $\begin{pmatrix} 1 \\ 0 \end{pmatrix} \mapsto \begin{pmatrix} 1 \\ 1 \end{pmatrix}$,  $\begin{pmatrix} 2 \\ 1 \end{pmatrix} \mapsto \begin{pmatrix} 1 \\ 2 \end{pmatrix}$;

\пункт  $\begin{pmatrix} 1 \\ 1 \end{pmatrix} \mapsto \begin{pmatrix} 1 \\ 0 \end{pmatrix}$,  $\begin{pmatrix} 2 \\ 2 \end{pmatrix} \mapsto \begin{pmatrix} 1 \\ 2 \end{pmatrix}$;

\пункт  $\begin{pmatrix} 1 \\ 1 \end{pmatrix} \mapsto \begin{pmatrix} 1 \\ 0 \end{pmatrix}$,  $\begin{pmatrix} 2 \\ 2 \end{pmatrix} \mapsto \begin{pmatrix} 2 \\ 0 \end{pmatrix}$;

\кзадача

\пзадача Докажите, что линейное отображение
\сНовойСтроки
\пункт  однозначно задается образами векторов любого базиса;
\пункт оставляет начало координат на месте;
\пункт переводит прямые в прямые;
\пункт сохраняет параллельность прямых.

 \кзадача

%\сзадача Всякое ли \пункт биективное; \пункт произвольное преобразование плоскости, оставляющее на месте начало координат и переводящее прямые в прямые, является линейным?
%\кзадача

\пзадача  Докажите, что отображение $A : \R^2 \to \R^2$ является линейным тогда и только тогда, когда оно обладает следующими тремя свойствами

$\bullet$ $A(\vec{0})=\vec{0}$;

$\bullet$ для любых $u,v \in \R^2$ выполнено $A(u+v) = A(u) + A(v)$;

$\bullet$ для любого  $v \in \R^2$ и любого $\lambda \in \R$ выполнено: $A(\lambda v) = \lambda A(v)$.

\кзадача

\опр \выд Образом линейного отображения $\varphi: \R^2 \to \R^2$ называется множество\break $\Im \varphi = \{\varphi(a): a \in \R^2\}$, а \выд ядром\ --- множество $\Ker \varphi = \{a \in \R^2: \varphi(a) = \vec{0}\}$.
	\копр
	
\пзадача
\пункт
Докажите, что и ядро, и образ --- это либо $\vec{0}$, либо проходящая через начало координат прямая, либо вся плоскость.
\пункт Приведите соответствующие примеры.
\кзадача

	
\пзадача Докажите, что $\dim \Ker \varphi + \dim \Im \varphi = \dim V$.
\кзадача

\пзадача Докажите, что линейное отображение сохраняет отношение отрезков, лежащих на одной прямой (если не переводит всю эту прямую в $\vec{0}$).
\кзадача

\пзадача Пусть три чевианы делят три стороны треугольника в отношениях $\alpha_1$, $\alpha_2$ и $\alpha_3$. Докажите, что то, пересекаются ли они в одной точке, зависит только от чисел $\alpha_i$ (а от треугольника не зависит).
%Указание: любые два треугольника равны с точностью до линейного преобразования.
\кзадача

\сзадача Как при линейном отображении с матрицей $\begin{pmatrix} a & b  \\ c & d \end{pmatrix}$ изменяется площадь\\
\пункт единичного квадрата;\\ \пункт произвольного параллелограмма;\\ \пункт произвольного многоугольника?\\
\пункт Докажите, что отображение с матрицей $\begin{pmatrix} a & b  \\ c & d \end{pmatrix}$ биективно тогда и только тогда, когда $ad-bc \neq 0$.
\кзадача

\опр {\it Полярными координатами} точки плоскости называются ее расстояние до начала координат и азимут (отсчитываемый против часовой стрелки угол радиус-вектора с осью~$x$).\копр

\пзадача Докажите, что точка с полярными координатами $(r, \varphi)$ имеет декартовы координаты $\begin{pmatrix} r \cos \varphi \\ r \sin \varphi \end{pmatrix}$.
\кзадача

\пзадача Найдите \пункт матрицу поворота на $90^\circ$; \пункт матрицу $R(\varphi)$ поворота на угол $\varphi$.
\кзадача

\пзадача \пункт Докажите, что линейное преобразование сохраняет углы тогда и только тогда, когда
его матрица имеет вид либо $\begin{pmatrix} a & b \\ -b & a \end{pmatrix}$ , либо $\begin{pmatrix} a & b \\ b & -a \end{pmatrix}$ . (Что это за преобразования
геометрически?)
\пункт Какие линейные преобразования сохраняют расстояния?
\кзадача

\опр {\it Произведением} матриц, соответствующих линейным отображениям $A$ и $B$, называется матрица, соответствующая композиции 
$A \circ B$ этих отображений. Она обозначается~$AB$.\копр

\пзадача \пункт Вычислите произведение матриц $\begin{pmatrix} 0 & 0 \\ 1 & 0 \end{pmatrix} \begin{pmatrix} 0 & 1 \\ 1 & 0 \end{pmatrix}$.\\
\пункт Вычислите произведение $ \begin{pmatrix}
a_{11} & a_{12} \\ a_{21} & a_{22} \end{pmatrix} \begin{pmatrix} b_{11} & b_{12} \\ b_{21} & b_{22} \end{pmatrix}$.\\
\пункт Коммутативно ли умножение матриц?
\кзадача

\сзадача Решите уравнения \пункт $A^2 = E$; \пункт $A^2 = -E$; \пункт $A^2=A$.
\кзадача

\пзадача Вычислите явно произведение $R(\varphi)R(\psi)$. Какие тригонометрические тождества даёт равенство $R(\varphi)R(\psi)= R(\varphi+\psi)$?
\кзадача

\ЛичныйКондуит{0mm}{5mm}

%\GenXMLW

\end{document}


.