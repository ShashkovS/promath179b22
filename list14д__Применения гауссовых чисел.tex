% !TEX encoding = Windows Cyrillic
\documentclass[a4paper,12pt]{article}
\usepackage[mag=990]{newlistok}
\usepackage{tikz}
\usetikzlibrary{calc}

\УвеличитьШирину{1.4cm}
\УвеличитьВысоту{2.3cm}



\Заголовок{Применения гауссовых чисел}
\НомерЛистка{14д}
% \renewcommand{\spacer}{\vspace{1.2pt}}
\ДатаЛистка{18.12.2019}
\Оценки{16/13/10}

\begin{document}
	\СоздатьЗаголовок
	
        \задача Натуральное число представимо в виде суммы двух квадратов (целых чисел) тогда и только тогда, когда любое простое число вида $4k+3$ входит в его разложение на простые множители в чётной степени.
\кзадача



	\задача  \пункт Докажите, что если два числа представимы в виде $a^2+db^2$ (где $d$ --- фиксированное натуральное число), то и их произведение представимо в таком виде.
	
	\пункт Найдите все простые числа, которые представляются в виде $a^2+4b^2$.
	\кзадача
	
	\задача \label{r} \пункт Обозначим через $r(n)$ количество решений уравнения $a^2+b^2=n$ в натуральных числах. Докажите, что если $m,n$ взаимно просты, то $r(mn)=r(m)r(n)$. \спункт Обозначим через $r_d(n)$ количество решений уравнения $a^2+db^2=n$ в целых числах. Верно ли предыдущее утверждение при $d>1$?
	\кзадача
	
	\сзадача \пункт Как посчитать $r(n)$ из задачи \ref{r}? \пункт Для каких натуральных $k$ существует окружность, на которой лежит ровно $k$ точек с целыми координатами?
	\кзадача
	
	
	
	
	
	\задача(Пифагоровы тройки) \пункт Пусть $a,b,c$ --- такие взаимно простые целые числа, что $a^2+b^2=c^2$. Тогда $c=|z|^2$ для некоторого $z\in \Z[i]$.
\пункт Укажите все тройки целых чисел $a,b,c \in \Z$, таких что $a^2+b^2=c^2$. (То есть напишите формулу, которая дает в точности все такие тройки при подстановке в нее целых чисел)
\кзадача
	
	\задача Решите в целых числах уравнение $y^3=x^2+1$.
	\кзадача
	\задача \пункт Докажите, что ни при каком $n$ число $(3+4i)^n$ не является вещественным.\\ \пункт Докажите, что угол $\arctg{3/4}$ не выражается рациональным числом градусов.\\ \спункт Найдите все такие натуральные $k$, что $\cos{\pi/k}$ и $\sin{\pi/k}$ рациональны.
	\кзадача

	\задача \сНовойСтроки\пункт Дайте определение множеству чисел $\mathbb{Z}[\sqrt{-n}]=\mathbb{Z}[\sqrt{n}i]$. Определения из прошлого листка (делимость, ассоциированные элементы, обратимые, простые) дословное переносятся на произвольное множество вида $\mathbb{Z}[\sqrt{n}i]$. \пункт Найдите все обратимые числа в $\mathbb{Z}[\sqrt{-n}]$. \пункт При каких $n$ число 2 является простым в $\mathbb{Z}[\sqrt{n}i]$?
	\пункт Для каких $n$ работает алгоритм Евклида, описанный в листке 42? \кзадача
	
	\задача \сНовойСтроки Есть ли в $\mathbb{Z}[\sqrt{n}i]$ однозначное разложение на простые множители с точностью до ассоциированности для
	\пункт $n=2$;
	\пункт $n=3$ (Указание: используйте равенство $(1+\sqrt{3}i)\cdot(1-\sqrt{3}i)=2\cdot 2$);
	\пункт $n=4$ (Годится ли равенство $2 \cdot 2 = (-2) \cdot (-2)$? $2 \cdot 2 = 2i \cdot (-2i)?)$;
	\пункт $n=5$;
	\пункт $n>5$?
	\кзадача



\ЛичныйКондуит{0mm}{5mm}
% \GenXMLW

\end{document}

%	\опр Функция $f: \mathbb{N} \to \mathbb{C}$ называется \emph{мультипликативной}, если для любых взаимно простых чисел $m,n$ имеется равенство $f(mn)=f(m)f(n)$. Важный для дальнейшего пример мультипликативной функции доставляет функция $\chi$, заданная по правилу $\chi(1)=1, \chi(2)=0,\chi(3)=-1, \chi(4)=0$, $\chi(4k+r)=\chi(r)$.
%	\копр
%	\задача Рассмотрим функцию $\rho(n) = \sum_{d|n} \chi(d)$. \пункт Докажите, что $\rho$ мультипликативна. \пункт Докажите, что $\rho(p^k)=r(p^k)$ (функция $r$ определяется в задаче \ref{r}) для всех простых $p$ и натуральных $k$, и тем самым эти две функции равны.
%	\кзадача
%	\задача Введём функцию $R(n) = \sum_{i=1}^n r(i)$, равную числу решений неравенства $a^2+b^2 \leqslant n$ в натуральных числах. \пункт Докажите, что $R(n)=\sum_{i=1}^n [\frac{n}{i}] \chi(i)$. \пункт Докажите, что $\lim\limits_{n \to \infty} \frac{R(n)}{n^2}=\frac{\pi}{4}$. \пункт Докажите формулу $\frac{\pi}{4}=1-\frac{1}{3}+\frac{1}{5}-\frac{1}{7}+\frac{1}{9}-\ldots$.
%	\кзадача
%	\задача \emph{Дзета-функция Римана} определяется равенством $\zeta(s) = \sum_{i=1}^{\infty} \frac{1}{i^s}$. \emph{$L$-функция Дирихле} определяется равенством $L(\chi,s)=\sum_{i=1}^{\infty} \frac{\chi(i)}{i^s}$. \пункт Докажите, что дзета-функция Римана равна эйлерову произведению $\prod_p (\frac{1}{1-1/p^s})$ и докажите аналогичную формулу для $L$-функции.  Следовательно, $\zeta(s)L(\chi,s)= \\ \big( \prod_{p=4k+1} (\frac{1}{1-1/p^s}) \big) \cdot \prod_{p=4k+3} (\frac{1}{1-1/p^{2s}})$. \пункт Докажите, что второе произведение имеет конечный ненулевой предел при $s \to 1$. Так как $L(\chi,1) \neq 0$, первое произведение не может иметь конечного предела при $s \to 1$. Следовательно, простых чисел вида $4k+1$ бесконечно много. \пункт Используя функции $\chi_0(n)$, равную $1$ при нечётных $n$, и $0$ иначе, и $L(\chi_0,s)=\sum_{i=1}^{\infty} \frac{\chi_0(i)}{i^s}$, докажите, что и произведение $\prod_{p=4k+3} (\frac{1}{1-1/p^s})$ не имеет конечного предела при $s \to 1$.
%	\кзадача
%	Петер Густав Лежён Дирихле в 1837 году доказал, что если $n,r$ взаимно просты, то найдётся бесконечно много простых чисел вида $kn+r$. Мы разобрали частный случай этого доказательства при $n=4$. Доказательство Дирихле использует обобщение рассмотренных нами результатов на случай области чисел вида $a_0+a_1\zeta+\ldots+a_{n-1}\zeta^{n-1}$, где $a_i \in \mathbb{Z}$, а $\zeta$ -- примитивный корень степени $n$ из $1$.
	
	
