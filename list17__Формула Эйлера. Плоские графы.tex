% !TeX encoding = windows-1251
\documentclass[12pt,a4paper]{article}
\usepackage[mag=1000, tikz]{newlistok}

\УвеличитьШирину{1.5cm}
\УвеличитьВысоту{2.5cm}
\renewcommand{\spacer}{\vspace{1.5pt}}

\ВключитьКолонтитул



\begin{document}



\Заголовок{Формула Эйлера. Плоские графы}
%Графы 2}
%\Подзаголовок{Деревья. Формула Эйлера. Плоские графы}
\НадНомеромЛистка{179 школа, 7Б.}
\НомерЛистка{17}
\ДатаЛистка{04.04 -- 14.04/2018}
\Оценки{21/16/11}


\thispagestyle{empty}

\СоздатьЗаголовок

\vspace*{1mm}



\задача
Докажите, что любой граф можно нарисовать в пространстве так,
чтобы его рёбра не пересекались нигде, кроме вершин.
\кзадача

\опр \выд{Плоский граф}\ --- граф, который можно нарисовать
на плоскости так, что вершины будут точками, рёбра --- линиями, и рёбра не будут пересекаться (нигде, кроме вершин).~Нарисован\-ный граф делит плоскость на части (одна часть неограничена), их называют \выд{гранями} графа.
\копр

\задача
Какие графы задач 4, 5а листка 9 плоские? Нарисуйте, найдите число вершин, рёбер,~граней.
\кзадача

% \УстановитьГраницы{0mm}{120mm}
% \righttikz{-1mm}{0mm}{
% \begin{tikzpicture}[scale=.6]
% \begin{scope}[xshift=0mm]
% \coordinate (A) at (90:1);\coordinate (B) at (162:1);\coordinate (C) at (234:1);\coordinate (D) at (306:1);\coordinate (E) at (378:1);
% \foreach \pt in {(A), (B), (C), (D), (E)} \fill \pt circle (3pt);
% \draw (A)--(B)--(C)--(D)--(E)--(A);
% \end{scope}
% \begin{scope}[xshift=30mm]
% \coordinate (A) at (45:1);\coordinate (B) at (135:1);\coordinate (C) at (225:1);\coordinate (D) at (-45:1);
% \foreach \pt in {(A), (B), (C), (D)} \fill \pt circle (3pt);
% \draw (A)--(B)--(C)--(D)--(A) (A)--(C);
% \end{scope}
% \begin{scope}[xshift=60mm]
% \coordinate (A) at (90:1);\coordinate (B) at (162:1);\coordinate (C) at (234:1);\coordinate (D) at (306:1);\coordinate (E) at (378:1);
% \foreach \pt in {(A), (B), (C), (D), (E)} \fill \pt circle (3pt);
% \draw (A)--(B)--(D)--(A) (A)--(C)--(E)--(B);
% \end{scope}
% \begin{scope}[xshift=90mm]
% \coordinate (A) at (90:1);\coordinate (B) at (162:1);\coordinate (C) at (234:1);\coordinate (D) at (306:1);\coordinate (E) at (378:1);
% \foreach \pt in {(A), (B), (C), (D), (E)} \fill \pt circle (3pt);
% \draw (A)--(C)--(E)--(B)--(D)--(A);
% \end{scope}
% \begin{scope}[xshift=120mm]
% \coordinate (A) at (45:1);\coordinate (B) at (135:1);\coordinate (C) at (225:1);\coordinate (D) at (-45:1);
% \foreach \pt in {(A), (B), (C), (D)} \fill \pt circle (3pt);
% \draw (A)--(C)--(B)--(D)--(C)--(A)--(D);
% \end{scope}
% \begin{scope}[xshift=150mm]
% \coordinate (A) at (90:1);\coordinate (B) at (162:1);\coordinate (C) at (234:1);\coordinate (D) at (306:1);\coordinate (E) at (378:1);
% \foreach \pt in {(A), (B), (C), (D), (E)} \fill \pt circle (3pt);
% \draw (A)--(B)--(C)--(D)--(E)--(B);
% \end{scope}
% \end{tikzpicture}
% }
% \задача
% Для каждого из графов найдите число вершин, рёбер и граней.
% \кзадача
% \ВосстановитьГраницы
%
%
% \УстановитьГраницы{0mm}{13cm}
% \задача
% Нарисуйте все неизоморфные друг другу графы с не более
% чем четырьмя вершинами.
% \кзадача
% \ВосстановитьГраницы


% \righttikz{-7mm}{15mm}{
% \begin{tikzpicture}[scale=.5]
% \node[above] at (-1,0) {\пункт};
% \begin{scope}[xshift=0mm]
% \coordinate (A) at (0,0);\coordinate (B) at (5,0);\coordinate (C) at (5,3);\coordinate (D) at (0,3);
% \coordinate (AA) at (1.25,1);\coordinate (BB) at (3.75,1);\coordinate (CC) at (3.75,2);\coordinate (DD) at (1.25,2);
% \foreach \pt in {(A), (B), (C), (D), (AA), (BB), (CC), (DD)} \fill \pt circle (3pt);
% \draw (A)--(B)--(C)--(D)--(A) (AA)--(BB)--(CC)--(DD)--(AA) (A)--(AA) (D)--(DD);
% \end{scope}
% \begin{scope}[xshift=60mm]
% \coordinate (A) at (0,0);\coordinate (B) at (5,0);\coordinate (C) at (5,3);\coordinate (D) at (0,3);
% \coordinate (AA) at (1.25,1);\coordinate (BB) at (3.75,1);\coordinate (CC) at (3.75,2);\coordinate (DD) at (1.25,2);
% \foreach \pt in {(A), (B), (C), (D), (AA), (BB), (CC), (DD)} \fill \pt circle (3pt);
% \draw (A)--(B)--(C)--(D)--(A) (AA)--(BB)--(CC)--(DD)--(AA) (B)--(BB) (D)--(DD);
% \end{scope}
% \node[above] at (12,0) {\пункт};
% \begin{scope}[xshift=150mm, yshift=17mm]
% \coordinate (A) at (90:2); \coordinate (B) at (141:2); \coordinate (C) at (193:2); \coordinate (D) at (244:2); \coordinate (E) at (296:2); \coordinate (F) at (347:2); \coordinate (G) at (399:2);
% \foreach \pt in {(A), (B), (C), (D), (E), (F), (G)} \fill \pt circle (3pt);
% \draw (A)--(B)--(C)--(D)--(E)--(F)--(G)--(A) (A)--(C)--(E)--(G)--(B)--(D)--(F)--(A);
% \end{scope}
% \begin{scope}[xshift=210mm, yshift=17mm]
% \coordinate (A) at (90:2); \coordinate (B) at (141:2); \coordinate (C) at (193:2); \coordinate (D) at (244:2); \coordinate (E) at (296:2); \coordinate (F) at (347:2); \coordinate (G) at (399:2);
% \foreach \pt in {(A), (B), (C), (D), (E), (F), (G)} \fill \pt circle (3pt);
% \draw (A)--(B)--(C)--(D)--(E)--(F)--(G)--(A) (A)--(D)--(G)--(C)--(F)--(B)--(E)--(A);
% \end{scope}
% \end{tikzpicture}}



\ввпзадача [Формула Эйлера]
Для каждого связного плоского графа с $В$ вершинами,
$Р$ р\"ебрами и $Г$ гранями имеет место равенство: \ $В-Р+Г=2$.
Докажите это\\
\пункт
для дерева;
\пункт для графа с одним циклом;
\пункт
в общем случае.\\
\спункт Как изменится формула, если в плоском графе будет $k$ компонент связности?
\кзадача

\опр
Граф без кратных рёбер и петель называется \выд{простым}.
\копр

\задача
Докажите для простых плоских графов:
\вСтрочку
\пункт $2Р\geq3Г$ при $Г\geq2$;
\пункт $Р\leq3В-6$~при~$В\geq3$.
\кзадача



\пзадача
С помощью задачи 4 выясните, является ли плоским полный граф с пятью вершинами?
\кзадача

\пзадача
Можно ли построить три дома, вырыть три колодца и соединить тропинками каждый
дом с каждым колодцем так, чтобы тропинки не пересекались? ({\em Указание:} вам снова поможет задача 4.)
\кзадача

%\задача
%Как изменится формула Эйлера, если связный граф с непересекающимися
%р\"ебра\-ми нарисовать не на плоскости, а на \пункт сфере?
%\пункт торе (бублике, или сфере с ручкой)?
%\пункт кренделе (сфере с двумя ручками)?
%\пункт сфере с $g$ ручками?
%\кзадача

\задача
Пусть $G$ --- простой плоский граф. Докажите, что
\вСтрочку
%\сНовойСтроки
\пункт
в $G$ есть вершина степени меньше~6;
\пункт вершины графа $G$ можно %правильно
раскрасить
в 6 или менее цветов так, что никакие две вершины одного цвета
не будут соединены ребром;
\спункт можно ли так же раскрасить граф не более чем в 5 цветов?
\кзадача

%\задача
%Можно ли разбить какой-нибудь выпуклый шестиугольник на несколько
%меньших выпуклых
%шестиугольников так, чтобы любые два шестиугольника разбиения
%\кзадача


\пзадача
Внутри квадрата отметили несколько точек и соединили непересекающимися отрезками друг с другом и с вершинами квадрата так, что квадрат разбился на треугольники.
%Квадрат разрезан на треугольники так, что любые два треугольника либо не имеют общих точек, либо имеют ровно одну общую точку --- вершину, либо имеют общую сторону
%(такое разбиение называется {\em триангуляцией}). При этом на сторонах квадрата нет вершин треугольников, кроме вершин квадрата.
\пункт Сколько вышло треугольников, если точек внутри 179?
\пункт Докажите, что число треугольников всегда чётно.
\кзадача

\задача
Можно ли разбить какой-нибудь шестиугольник на выпуклые шестиугольники
так, чтобы выполнялось условие: границы любых двух из этих шестиугольников
(включая исходный) либо не имеют общих точек,
либо имеют только общую вершину или общую сторону?
\кзадача

% \задача
% На плоскости отмечено несколько точек, никакие три %из которых
% не лежат на одной прямой. Двое %играют в такую игру: они
% по очереди соединяют какие-то две ещ\"е не соедин\"енные
% точки отрезком так, чтобы отрезки не пересекались нигде, кроме
% отмеченных точек. Кто не может сделать ход --- проиграл.
% Зависит ли исход от того, как играют соперники?
% \кзадача

\задача Докажите формулу Эйлера
\вСтрочку
\пункт для произвольного
связного графа с непересекающимися р\"ебра\-ми,
нарисованного на сфере;
\пункт для произвольного
выпуклого многогранника.
\кзадача

\пзадача
\пункт Дан выпуклый многогранник, грани которого являются $n$-угольниками,
и в каж\-дой вершине сходится $k$ граней. Докажите, что
$1/n+1/k=1/2+1/r$, где $r$ --- число его р\"ебер.
\пункт Проверьте это равенство для приведённых ниже многогранников.
\кзадача



\vspace*{.1truecm}

\putpicture{0mm}{-22mm}{pct_graph_tetr}

\putpicture{40mm}{-17mm}{pct_graph_cube}

\putpicture{80mm}{-12mm}{pct_graph_oct}

\putpicture{120mm}{-7mm}{pct_graph_dod}

\putpicture{160mm}{-2mm}{pct_graph_icos}

\vspace*{.2truecm}

\задача
Выпуклый многогранник называют \выд{правильным}, если
все его грани ---~\hbox{правильные} $n$-угольники, и
в каждой его вершине сходится $k$ граней. Докажите,
что любой такой многогранник комбинаторно устроен как
тетраэдр, куб, октаэдр, додекаэдр или икосаэдр (см.~рис.).
\кзадача

\задача
У выпуклого многогранника все грани — правильные пятиугольники или правильные шестиугольники.
\пункт Докажите, что в каждой вершине этого многогранника сходятся ровно три грани.
\пункт Сколько пятиугольных граней у данного многоугольника?
\кзадача

\задача
Докажите, что связный плоский граф {\em эйлеров} (то есть у него ещё и степень каждой вершины чётна)
если и только если его грани можно раскрасить в два цвета так,
чтобы любое ребро принадлежало границам двух граней разного цвета.
\кзадача

\сзадача
На плоскости расположены $n$ непересекающихся отрезков и $n+2$ точки,
не лежащие на этих отрезках. Докажите, что какие-то две точки \лк видят
друг друга\пк\ (то есть если соединить эти две точки отрезком, он не пересечёт
ни одного из данных $n$ отрезков).
\кзадача


\ЛичныйКондуит{0mm}{6mm}

% \GenXMLW
%\СделатьКондуит{7mm}{7mm}


\end{document}

% \задача
% В институте $N$ учёных, работающих в 500 отраслях науки. По каждой отрасли есть ровно 10 специалистов (один и тот же человек может быть специалистом в любом числе областей). Докажите, что можно разбить учёных на две группы, в каждой из которых будут специалисты по всем отраслям науки.
% \кзадача

%\задача
%\кзадача


%\раздел{Разные задачи}

%\задача
%Имеется сеть дорог (см.~рис. 1). Из вершины выходят $2^{100}$ человек.
%Половина ид\"ет направо, половина --- налево. Дойдя до первого
%перекр\"естка, каждая группа делится: половина идет направо,
%половина --- налево. Такое же разделение происходит на каждом
%перекр\"естке. Сколько людей прид\"ет в каждый из перекр\"естков
%сотого ряда?
%\кзадача




%\раздел{Разбиения чисел}

% \сзадача
% Докажите, что число разбиений\footnote[2]{
% Разбиения, отличающиеся только порядком слагаемых,
% считаются одинаковыми.}
% натурального $n$
% на $k$ натуральных слагаемых равно числу разбиений $n$
% в сумму натуральных слагаемых, наибольшее из которых равно $k$.
% \кзадача
%
%
%
% \ссзадача Какие $n$ столькими же способами
% представимы$^2$ в виде суммы~ч\"ет\-ного числа различных
% натуральных слагаемых,
% сколькими способами они представимы в виде суммы неч\"етного числа
% различных натуральных слагаемых? Что можно сказать об остальных $n$?
% \кзадача
%
% \сзадача
% Докажите, что число разбиений$^{2}$ натурального $n$
% на неч\"етные натуральные слагаемые равно числу разбиений $n$ на попарно
% различные натуральные слагаемые.
% \кзадача
%

\раздел{Метод траекторий и числа Каталана}

\задача Возле кассы собралось $n+m$ человек; $n$ из них
имеют по купюре 100 руб., а другие $m$ ---  по купюре
50 руб. Сначала в кассе нет денег, билет стоит 50 руб.
Сколько есть способов размещения всех покупателей
в очереди так, чтобы никто не ждал сдачи?
\кзадача

\задача [Метод траекторий]
Будем рассматривать на клетчатой плоскости пути
с началом и концом в узлах клеток,
состоящие из диагоналей клеток, где каждая диагональ
ид\"ет либо вправо вверх, либо вправо вниз (если двигаться по пути
от начала к концу).
%Число диагоналей в пути называется его длиной.
\сНовойСтроки
\пункт Сколько существует путей, выходящих из начала координат, в которых
$m$ диагоналей идут вправо вверх, а $n$ диагоналей идут вправо вниз?
\пункт Сколько существует путей, соединяющих узел $(0,0)$ с узлом $(x,y)$
(где $x,y\geq0$)?
\пункт [Принцип отражения]
Узлы $A$ и $B$ лежат над осью абцисс, $B$ лежит правее $A$.
Докажите, что число путей, идущих из $A$ в $B$, которые касаются
оси абцисс или пересекают е\"е, равно числу всех путей из $A'$
в $B$, где $A'$ --- узел, симметричный $A$ относительно оси абцисс.
\кзадача

\задача [Теорема о баллотировке] Кандидат $A$ собрал на выборах $a$ голосов,
кандидат $B$ собрал $b$ голосов~\hbox{$(a>b)$.}
%Избиратели голосовали последовательно.
Сколько существует способов последовательного подсч\"ета голосов, при
которых $A$ все время будет впереди $B$ по количеству голосов?
\кзадача

%\txt{Числа Каталана можно определить многоми разными способами.
%}

\сзадача %[Числа Каталана]
Докажите, что следующие величины совпадают с числами Каталана, и найдите их:\\
$\bullet$ Число путей из точки $(0,0)$ в точку $(n,n)$, идущих по линиям
клетчатой бумаги вверх и вправо, не поднимаясь
выше прямой $y=x$;\\ \\ \\ \\ \\
\rightpicture{-40mm}{20mm}{110mm}{cat-2}
\vspace*{1cm}
% \пункт
$\bullet$ Число способов соединить данные $2n$ точек на окружности $n$ непересекающимися хордами.\\ \\ \\
\rightpicture{-40mm}{20mm}{110mm}{cat-3}
$\bullet$
Число способов провести $2n$-звенную ломаную из
левого нижнего угла таблички $n\times2n$ в правый нижний угол.
(Ломаная не может выходить за границы таблички, каждое звено
ломаной --- диагональ клетки, идущая вправо вверх или вправо вниз,
если двигаться по ломаной слева направо.)
% Разберите два случая: когда ломаная может касаться нижней стороны
% таблички в точках, отличных от углов, и когда не может.
% \пункт
% На окружности отметили $2n$ точек.
% Сколькими способами их можно соединить $n$ непересекающимися хордами?
% \пункт Найдите явную формулу для последовательности $C_n$, заданной
% начальным условием\break
% $C_0=1$
% и рекуррентной формулой $C_n=C_0C_{n-1}+C_1C_{n-2}+\ldots+C_{n-1}C_0$
% (при $n\geq1$).
\кзадача



% \vspace*{-2mm}
%\ЛичныйКондуит{0mm}{6mm}
% \vspace*{-3mm}


\раздел{От Шашкова}


\задача
Докажите, что 1 — это единственное число треугольника Паскаля, которое встречается в нём бесконечно число раз.
\кзадача


\задача
Встречается ли в треугольнике Паскаля число $n=2^2\cdot2179\cdot6179$?
\кзадача


\задача
Сколькими способами можно переставить буквы в слове \texttt{НЕПОНИМАНИЕ} так, чтобы и гласные, и согласные (по отдельности) появлялись в алфавитном порядке?
\кзадача



\задача
Найдите коэффициент при $abcde$ после раскрытия скобок в $(a+b+c+d+e)^5$.
\кзадача


\задача
Пусть $n$, $k$, $l$ — натуральные числа.
Рассмотрим коэффициент перед $a^n\cdot b^k \cdot c^l$ после раскрытия скобок в $(a+b+c)^{n+k+l}$.
\пункт
Выразите его через несколько $C_{*}^{*}$;
\пункт
Вычислите его без $C_{*}^{*}$.
\кзадача


\задача
На клетчатой бумаге нарисован прямоугольник $200\times100$.
Сколько существует путей из левого нижнего угла в правый верхний таких, что путь
\\\пункт проходит через точки с координатами $(100,50)$ и $(190, 90)$;
\\\пункт не проходит ни через точку $(100,50)$, ни через точку $(190, 90)$?
\кзадача


\задача
В шкафу 5 красных и 7 синих носков. Каким числом способов 3 человека могут одеть носки?
(носки одного цвета считаются одинаковыми, однако левая нога отличается от правой)
\кзадача


\задача
Коля и Петя пришли в булочную. Там продаются булки 8 видов. Сколькими способами они
могут купить себе по две булочки?
\кзадача

\задача
\пункт
Докажите, что $n\cdot C_{n-1}^{k-1} = k\cdot  C_n^k$.
\пункт
Докажите это тождество комбинаторными методами, не используя явную формулу для $C_n^k$.
\кзадача

\задача
Мизеров, Распасной и Шестирной играют в преферанс.
Раздаются по 10 карт на руки и две в прикуп.
Какое количество раздач существует?
\кзадача

\задача
Докажите, что для любых натуральных $n$ и $k\le n$ выполнено неравенство: $C_n^k<C_{n+1}^k$.
\кзадача

\задача
Ведущий игры \лк Русское лото\пк Михаил Борисов достаёт 86 бочонков из 99, но он --- знатный жулик.
В тех случаях, когда число 49 --- год его рождения --- не выпало, он подменяет бочонок с минимальным номером на 49.
Во сколько раз меньше комбинаций становится от таких вот махинаций? (порядок бочонков не имеет значения)
\кзадача


\задача
Докажите, что для любых натуральных $n>1$ и $k<n$ выполнено неравенство: $C_{2n}^k<C_{2n}^n$.
\кзадача

\задача
В коробке имеется 100 различных бусин. Мы собираем из них бусы по 40 бусин (очевидно, что бусины по-прежнему различны).
Какое количество разных бус можно собрать?
\кзадача



\задача
Каким числом способов можно 20 школьников разбить на 5 групп по 4 человека?
\кзадача


\задача
На плоскости нарисовали $n$ различных прямых так, что никакие три не пересекаются в одной точке.
Сколько всего треугольников можно увидеть на картинке?
\кзадача


\задача
В дружном классе из $n=26$ человек $k=100$ пар подрались.
Докажите, что можно выбрать команду из 3 человек, которые друг с другом не дрались.
% (Троек всего $C_n^3=n(n-1)(n-2)/6$, каждая драка «убивает» не больше $k(n-2)$ троек.
% Значит, остались $n(n-1)(n-2)/6 - k(n-2)$ дружных троек.
% При этом $k<n(n-1)/6
\кзадача




\end{document}

\vspace*{-0.2truecm}

\раздел{$***$}

\vspace*{-0.2truecm}

\задача
%\пункт
В НИИ работают 67 человек. Из них
47 знают английский язык, 35 --- немецкий, и 23 --- оба языка.
Сколько человек в НИИ
не знают ни английского, ни немецкого языков?
\пункт Пусть кроме этого  польский
знают 20 человек, английский и польский --- 12, немецкий и
польский --- 11, все три языка --- 5.
Сколько человек не знают ни одного из этих языков?
%\спункт [Формула включений и исключений]
%Решите задачу в общем случае: имеется $m$ языков,
%и для каждого набора языков известно, сколько человек знают все языки
%из этого набора.
\кзадача

\задача
В ряд записали 105 единиц, поставив перед каждой знак \лк$+$\пк.
Сначала изменили знак на противоположный перед каждой третьей единицей,
затем --- перед каждой пятой, а затем --- перед
каждой седьмой. Найдите значение полученного выражения.
\кзадача

\задача \вСтрочку
\пункт
На полке стоят 10 книг. Сколькими способами их можно переставить
так,~чтобы ни одна книга не осталась на месте?
\пункт А если на месте должны остаться ровно 3 книги?
\кзадача
