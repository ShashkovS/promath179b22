% !TEX encoding = Windows Cyrillic
\documentclass[a4paper,12pt]{article}
\usepackage[mag=990]{newlistok}
\usepackage{tikz}
\usetikzlibrary{calc}

\УвеличитьШирину{.9cm}
\УвеличитьВысоту{2cm}



\Заголовок{Вероятностные тесты на простоту}
\НомерЛистка{17д}
\renewcommand{\spacer}{\vspace{3.5pt}}
\ДатаЛистка{19.10.2020}
\Оценки{25/18/11}

\begin{document}
	\СоздатьЗаголовок

\раздел{Шифрование и задача дискретного логарифмирования}

{\footnotesize Предположим, что Алиса и Боб хотят обменяться секретной информацией (например, паролем). При этом все сообщения, которые они друг другу передают, становится видны постороннему наблюдателю Еве. Могут ли Алиса и Боб осуществить задуманное?

Для решения этой задачи можно использовать следующий трюк. Сначала Алиса и Боб находят большое простое число $p$ и первообразный корень $g$ по модулю $p$. После этого Алиса загадывает большое число $A$ и сообщает Бобу $g^A$, а Боб загадывает большое число $B$ и сообщает Алисе $g^B$.

После этого Алиса, имея $A$ и $g^B$, может получить $(g^B)^A=g^{A\cdot B}$. Боб, имея $B$ и $g^A$, также может получить $(g^A)^B = g^{A \cdot B}$. Используя $g^{A \cdot B}$, они могут кодировать сообщения по модулю $g^{A\cdot B}$.

А что делать Еве? Она знает $p$, $g$, $g^A = x$, $g^B=y$. Но для вычисления $g^{A\cdot B}$ этих данных недостаточно, нужно найти $A$, то есть дискретный логарифм числа $x$ по основанию $g$. Так как на данный момент нет алгоритмов, которые бы позволили быстро решить данную задачу, Ева не сможет найти число $A$, а значит, и $g^{A \cdot B}$.

Дискретный логарифм нельзя быстро найти перебором из-за того, что $g$ имеет очень большой порядок по модулю~$p$. Поэтому в данном случае вместо простых чисел можно рассматривать также числа, по модулю которых есть элементы достаточно большого порядка. Например, произведение двух простых или квадрат простого числа также будут годиться.

Чтобы использовать данный метод, надо уметь находить достаточно большие простые или почти простые числа. И~тут часто используют так называемый вероятностный тест на простоту. Это означает, что мы берём число и проводим несколько тестов. Чем больше тестов мы проведём, тем с большей вероятностью число простое. Мы не можем быть уверены, что найденное нами число простое, но число с небольшим количеством делителей нас тоже устроит.

Листок состоит из трёх частей. В первой мы обсудим, какие максимальные порядки могут быть у элементов $\Z/m\Z$. Вторая и третья посвящена двум различным вероятностным методам на простоту. При желании можно пропустить первую часть, при этом можно использовать теорему из задачи 6.}

\задача Дайте определение порядка $d_m(a)$ элемента $a \in (\Z/m\Z)^*$, покажите, что $d_m(a)$ делит $\varphi(m)$.
\кзадача

\задача Пусть $k = d_m(a)$, $l = d_m(b)$. Верно ли, что \сНовойСтроки
\пункт если $(k,l)=1$, то $d_m(a\cdot b)=kl$;
\пункт если $(k,l)\neq 1$, то $d_m(a \cdot b) =[k,l]$?
\пункт
Пусть теперь $(n,k)=1$, $a \in (\Z/nk\Z)^*$, $a_n$ и $a_k$ ---
остатки при делении $a$ на $n$ и $k$ соответственно. Докажите, что $d_{nk}(a) = [d_n(a_n),d_k(a_k)]$.
\кзадача

\задача \пункт Используя задачу 2в), найдите максимальный порядок элемента по модулю $14$; $15$.
\пункт Докажите, что первообразный корень может существовать только по модулю чисел вида $2^k$, $p^m$, $2p^m$, где $p$ --- простое число, $k,m$ --- произвольные положительные степени.
\кзадача

\задача \пункт Докажите, что для любого $l \geq 0$ выполняется равенство $5^{2^l} = 1+2^{l+2}\cdot u$ для некоторого нечётного числа $u$, зависящего от $l$.\\
\пункт Найдите $d_{2^l}(5)$.\\
\пункт Докажите, что первообразный корень cуществует по модулям $2$ и $4$ и не существует по модулю $2^k$, $k \geq 3$.
\кзадача

\задача Зафиксируем простое число $p$, и первообразный корень $g$ по модулю $p$. Будем рассматривать $g$ как натуральное число.

\пункт Докажите, что если $g^{p-1} \equiv 1 \pmod{p^2}$, то $(g+p)^{p-1} \not \equiv 1 \pmod{p^2}$. Выведите отсюда, что можно считать, заменив $g$ на $g+p$ при необходимости, что $g^{p-1} \not \equiv 1 \pmod{p^2}$.

Положим $g^{p-1}=1+pu$, где $u$ не делится на $p$.

\пункт Докажите, что $g^{(p-1)p^k} = 1+p^{k+1}u_k$ для некоторого $u_k$, не делящегося на $p$.

\пункт Докажите, что $g$ --- первообразный корень по модулю $p^n$ для любого $n$.

\кзадача

\задача[Теорема о существовании первообразного корня] Докажите, что первообразный корень по модулю $n$ существует тогда и только тогда, когда $n \in \{2, 4,
p^{\alpha},$ $ 2p^{\alpha}\}$, где $p$ --- нечётное простое, $\alpha
$  --- положительное целое.
\кзадача

%{\footnotesize  В этом листке везде, где в условии задачи сформулировано утверждение, его требуется доказать.}	
	
\раздел{Тест Ферма на простоту}

{\footnotesize Тест Ферма устроен так. Берём любой остаток $a$ по модулю $n$. Проверяем, что $(a,n)=1$. Если $a^{n-1} \not\equiv 1$, то $n$ --- не простое. Если $a^{n-1} \equiv 1$ (то есть, $a$ {\it прошёл тест Ферма по модулю $n$}), берём любой новый остаток по модулю~$n$, и т.д.

}

\noindent
{\bf Обозначение 1.} Пусть $B_n = \{ a \in (\Z/n\Z)^* \mid a^{n-1} = 1\}$ --- множество остатков по модулю $n$, которые проходят тест Ферма.

\задача Докажите, что \пункт либо $B_n = (\Z/n\Z)^*$, либо $|B_n| \leq \frac{1}{2} |(\Z/n\Z)^*|$;\\
\пункт если $n$ --- простое, то все остатки проходят тест Ферма.
\пункт Найдите $|B_n|$ для $n \leq 10$.
\кзадача

\задача Докажите, что если мы провели тест Ферма для числа $n$ ровно $k$ раз, и каждый раз остаток проходил тест Ферма, то с вероятностью $1-\frac{1}{2^k}$ все остатки проходят тест Ферма по модулю $n$ (мы считаем, что каждый раз мы независимо выбираем новый остаток). \кзадача

{\footnotesize Тем самым мы довольно быстро с большой вероятностью можем заключить, что $B_n = (\Z/n\Z)^*$. Это хорошая новость. Плохая новость --- есть составные числа, для которых это тоже верно.}

\опр {\it Числом Кармайкла} называется такое составное число $n > 1$, что $B_n = (\Z/n\Z)^*$.\копр

\теорема\label{Carm} Составное число $n$ является числом Кармайкла тогда и только тогда, когда выполнено два условия:

(1) $n$ свободно от квадратов, то есть $n$ не делится на $p^2$ для любого простого $p$;

(2) для любого простого делителя $p \mid n$ верно, что $n-1$ делится на $p-1$.

\ктеорема

\задача[Доказательство Теоремы \ref{Carm} <<в обратную сторону>>] Докажите, что если для составного числа $n$ выполнены свойства (1) и (2), то $n$ --- число Кармайкла.
\кзадача

\задача Докажите, что $3\cdot 11 \cdot 17$ и $5 \cdot 13 \cdot 17$ --- числа Кармайкла.
\кзадача

\задача Пусть $n=p^k\cdot d$, где $(d,p)=1$, $p$ --- простое и $k \geq 2$. Докажите, что\\
\пункт найдётся такое число $a$, что $a \equiv 1+p \pmod{p^k}$ и $ a \equiv 1 \pmod{d}$; \пункт  $n$ --- не число Кармайкла.
\кзадача

\задача Докажите Теорему \ref{Carm} <<в прямую сторону>>.
\кзадача

{\footnotesize Вывод из Теоремы 1: числа Кармайкла существуют, но их не очень много, что позволяет использовать тест Ферма.}

\задача Докажите, что $n$ --- число Кармайкла, если и только если $a^n \equiv a \pmod{n}$ для всех $a \in \Z$.
\кзадача

\задача Докажите, что все числа Кармайкла нечётные.
\кзадача

\задача Пусть для натурального $k$ числа $6k+1$, $12k+1$, $18k+1$ --- простые. Докажите, что тогда $(6k+1)\cdot (12k+1) \cdot (18k+1)$ --- число Кармайкла.
\кзадача

\раздел{Тест Миллера-Рабина на простоту}

\noindent
{\bf Обозначение 2.} Пусть $n$ нечётно,  $n-1=2^l \cdot m$ для некоторой степени $l$ и нечётного числа $m$.

{\footnotesize Попробуем улучшить наш тест. Как мы знаем, простые числа $n$ отличаются от составных тем, что многочлен $x^{n-1}-1$ имеет ровно $n-1$ корней в $\Z/n\Z$. Если $n$ --- нечётное простое, то данный многочлен можно разложить как $(x^{\frac{n-1}{2}}-1)(x^{\frac{n-1}{2}}+1)$, и каждая скобка будет иметь ровно $\frac{n-1}{2}$ корней. Если $\frac{n-1}{2}$ снова чётно, то многочлен можно разложить на три скобки, и.т.д. Пользуясь этой идеей, опишем тест Миллера-Рабина на простоту.

Для каждого остатка $a$ по модулю $n$ сначала вычислим $a^m$. Если оно равно 1 или $-1$, то $a$ {\em прошёл тест Миллера-Рабина}. Иначе будем возводить $a^m$ в квадрат $k$ раз подряд, вычисляя $a^{2m}$, $a^{4m}$, $\ldots$, $a^{2^{l-1}m}$. Если в какой-то момент получим $-1$, то $a$ прошёл тест Миллера-Рабина. Если не получим, то $a$ не прошёл тест Миллера-Рабина.}

\noindent
{\bf Обозначение 3.} Положим $B_{MR}(n) = \{a \in (\Z/n\Z)^* \mid a^m = 1 \mbox{ или  } a^{m2^k} = -1 \mbox{ для некоторого } 0 \leq k < l\}.$


\теорема[Миллер-Рабин]  Если $n > 9$ --- нечётное составное число, то $|B_{MR}(n)| \leq \frac{1}{4} |(\Z/n\Z)^*|$.
\ктеорема


\задача \пункт Докажите, что элементы $B_{MR}(n)$ проходят тест Ферма.\\
\пункт Опишите $B_{MR}(n)$ для $n \leq 15$. Проверьте теорему Миллера-Рабина для данных $n$.
\кзадача


\noindent
{\bf Обозначение 4.} Пусть $M_0$ --- множество остатков $a \in \Z/n\Z$, для которых $a^m \equiv 1 \pmod{n}$, и пусть\break $M'_k$ --- множество остатков $a \in \Z/n\Z$, для которых $a^{m \cdot 2^k} \equiv -1 \pmod{n}$, $k=0,\ldots,l-1$.

\задача
\пункт Докажите, что $|M_0|=|M'_0|$, $B_{MR} = M_0 \sqcup M'_0 \sqcup M'_1 \sqcup \ldots \sqcup M'_k$.\\
\пункт Пусть $p_1,\ldots p_s$ --- простые делители числа $n$. Обозначим через $N_k$ множество остатков $a \in \Z/n\Z$, для которых $a^{m \cdot 2^k} \equiv \pm 1 \pmod{p_i}$ для всех $i=1,\ldots,s$, причём хотя бы по одному модулю в равенстве будет $1$, и хотя бы по одному модулю будет $-1$. Докажите, что $|N_k| = 2^s |M'_k|$, $N_k \cap M'_k =\emptyset$.\\
\пункт Докажите, что в случае $s \geq 3$ выполнено неравенство $|B_{MR}| \leq \frac{1}{6}|(\Z/n\Z)^*|$.\\
\пункт Докажите, что в случае $s =2$ выполнено неравенство $|B_{MR}| \leq \frac{1}{4}|(\Z/n\Z)^*|$.\\
\пункт Закончите доказательство теоремы Миллера-Рабина, разобрав оставшийся случай $s=1$.
\кзадача


	\ЛичныйКондуит{0mm}{5mm}
% \GenXMLW

\end{document}
	